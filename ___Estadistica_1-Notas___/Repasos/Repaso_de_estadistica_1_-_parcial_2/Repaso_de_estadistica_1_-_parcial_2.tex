\documentclass{article}

\usepackage{generalsnips}
\usepackage{calculussnips}
\usepackage[margin = 1in]{geometry}
\usepackage{pdfpages}
\usepackage[spanish]{babel}
\usepackage{amsmath}
\usepackage{amsthm}
\usepackage[utf8]{inputenc}
\usepackage{titlesec}
\usepackage{xpatch}
\usepackage{fancyhdr}
\usepackage{tikz}
\usepackage{hyperref}
\usepackage{xfp}
\usepackage{xkeyval}

\title{Repaso de estadística 1 - parcial 2}
\date{2020 April 13, 04:24PM}
\author{David Gabriel Corzo Mcmath}

\begin{document}
\maketitle
%%%%%%%%%%%%%%%%%%%%%%%%%%%%%%%%%%%%%%%%%%%%%%%%%%%%%%%%%%%%%%%%%%%%%%%%%%%%%%%%%%%%%%%%%%%%%%%%%%%%%%%%%%%%%%%%%%%%%%%%%%%%%%%%%%%%%%%%%%%%%%
\hrulefill
\section{Variables aleatorias}
\begin{itemize}
    \item \termdefinition{Variable aleatoria}{Auna variable aleatoria que asuma ya sea un número finito de valores o una sucesión infinita de valores tales como 0, 1, 2, . . ., se le llama variable aleatoria discreta} 
    \item \termdefinition{Variable aleatoria continua}{Auna variable que puede tomar cualquier valor numérico dentro de un intervalo o colección de intervalos se le llama variable aleatoria continua. Los resultados experimentales basados en escalas de medición tales como tiempo, peso, distancia y temperatura pueden ser descritos por variables aleatorias continuas} 
\end{itemize}


%----------------------------------------------------------------------------------------
\hrulefill
\section{Distribuciones de probabilidad discreta}
\begin{itemize}
    \item \termdefinition{Distribución de probabilidad de una variable aleatoria discreta}{La disribución de probabilidad de una variable aleatoria describe cómo se distribuyen las probabilidades entre los valores de la variable aleatoria, esta se describe por una función de probabilidad. } 
    \item Ejemplo: \textbf{Venta de vehículos}
        \begin{center}
            \begin{tabular}{ |c|c| }
                \hline
                    Días & Cantidad de autos vendidos \\
                \hline
                    54 & 0 \\ 
                    117 &1 \\ 
                    72&2 \\ 
                    42&3 \\ 
                    12&4 \\ 
                    3&5 \\ 
                \hline
                    Total: 300 & \\ 
                \hline
            \end{tabular}
        \end{center}

    \item $x=$ número de automóviles vendidos en un día.  
        \begin{center}
            \begin{tabular}{ |c|l| }
                \hline
                    f(x) &  Probabilidad   \\
                \hline
                    f(0) & 54/300 = \num{\fpeval{54/300}} \\ 
                    f(1) & 117/300 = \num{\fpeval{117/300}}  \\ 
                    f(2) & 72/300 = \num{\fpeval{72/300}}\\ 
                    f(3) & 42/300 = \num{\fpeval{42/300}} \\  
                    f(4) & 12/300 = \num{\fpeval{12/300}} \\ 
                    f(5) & 3/300 = \num{\fpeval{3/300}} \\ 
                \hline
            \end{tabular}
        \end{center}
    
    \item Una función de probabilidad:
        \begin{itemize}
            \item Es: $\displaystyle f(x) \geq 0 $ 
            \item Es: $\displaystyle \sum f(x)=1$ 
        \end{itemize}
\end{itemize}


%----------------------------------------------------------------------------------------
\section{Función de probabilidad uniforme discreta}
%----------------------------------------------------------------------------------------
\subsection{Fórmula} 
        \[
          f(x) = \frac{1}{n} 
        \] Donde $n$ es el número de valores que puede tomar la variable aleatoria. 
    
%----------------------------------------------------------------------------------------
\subsection{Valor esperado}
\[
    E(x) = \mu = \sum x \cdot f(x) 
\] Es equivalente al promedio. 

%----------------------------------------------------------------------------------------
\subsection{Varianza} 
\[
    Var(x) = \sigma ^2 = \sum \p{x-\underbrace{\mu}_{\text{ Valor esperado} } } ^2 f(x)
\]
    
%----------------------------------------------------------------------------------------
\subsection{Desviación estándar}
\[
    \sigma = \sqrt{\sigma^2} 
\]

%%%%%%%%%%%%%%%%%%%%%%%%%%%%%%%%%%%%%%%%%%%%%%%%%%%%%%%%%%%%%%%%%%%%%%%%%%%%%%%%%%%%%%%%%%%%%%%%%%%%%%%%%%%%%%%%%%%%%%%%%%%%%%%%%%%%%%%%%%%%%%%%%%%%%%%%%%%%%%%%%%%%%%%%%%%%%%%%%%%%

\hrulefill
\section{Distribución de probabilidad binomial}
\begin{itemize}
    \item Propiedades de un experimento binomial:
        \begin{enumerate}
            \item Consiste en una serie de $n$ ensayos \textbf{idénticos}.
            \item En cada ensayo hay dos resultados posibles. Auno de estos resultados se le llama \emph{éxito} y al otro se le llama \emph{fracaso}.
            \item La probabilidad de éxito $p$, no cambia de ensayo a otro. Por ende, la probabilidad de fracaso, que se denota $1-p$, tampoco cambia de un ensayo a otro. 
            \item Los ensayos son independientes. 
        \end{enumerate}
    
    \item Componentes:
        \begin{itemize}
            \item $n$ número de ensayos 
            \item $x$ variable aleatoria 
            \item $p$ probabilidad de éxito 
            \item $1-p$ probabilidad de fracaso 
        \end{itemize}
    
    \item El número de resultados experimentales es una combinación:
        \[
          \binom{n}{x} = \frac{n!}{x!\p{n-x} !} 
        \] donde $n! = n(n-1)(n-2)...(2)(1)(0)$ y $0!=1$
    
    \item  Martin clothing store:
        \begin{itemize}
            \item Componentes:
                \begin{itemize}
                    \item $n=3$, tres ensayos
                    \item $p=0.30$, probabilidad de que compren de $0.30$ (éxito)
                    \item $1-p=0.70$, probabilidad que no compren (fracaso)
                \end{itemize}
            
            \item Calcular el número de resultados experimentales en los que hay dos éxitos:
                \[
                  \binom{n}{x} = \binom{3}{2} = \frac{3!}{2!(3-2)!} = \frac{3\cdot 2\cdot 1}{2\cdot 1 \cdot 1} = \frac{6}{2} = 3 
                \]            
        \end{itemize}
    
    \item Función de probabilidad binomial:
        \[
          f(x) = \binom{n}{x}p^x(1-p)^{\p{n-x}}
        \] donde:
        \begin{itemize}
            \item $f(x)$: Probabilidad de $x$ éxitos en $n$ ensayos.
            \item $n$: número de ensayos 
            \item $\binom{n}{x}$: combinaciónes de $x$ en $n$ elementos. 
            \item $p$: probabilidad de éxito 
            \item $1-p$: probabilidad de fracaso 
        \end{itemize}
    
    \item Ejemplo de Martin clothing store con probabilidades binomiales:
        \begin{center}
            \begin{tabular}{ |c|c| }
                \hline
                    $x$ & $f(x)$  \\
                \hline
                    0 & $\p{3!/0!\cdot 3!}\p{0.03}^0\p{0.70}^3 = 0.343 $ \\ 
                    1 & $\p{3!/1!\cdot 2!}\p{0.03}^1\p{0.70}^2 = 0.441 $ \\ 
                    2 & $\p{3!/2!\cdot 1!}\p{0.03}^2\p{0.70}^1 = 0.189 $ \\ 
                    3 & $\p{3!/3!\cdot 0!}\p{0.03}^3\p{0.70}^0 = 0.027 $ \\ 
                \hline
            \end{tabular}
        \end{center}
    
    \item Valor esperado o promedio de la distribución binomial:
        \[
          E(x) = \mu = n \cdot p 
        \]
    
    \item Varianza de la distribución binomial:
        \[
          Var(x) = \sigma^2 = n\cdot p \p{1-p} 
        \]
    
\end{itemize}

\subsection{Argumentos en excel de las probabilidades binomiales}
\begin{center}
    \verb|=DISTR.BINOM.N(|$\displaystyle \overbrace{x}^{\text{ variable aleatoria }};\underbrace{n}_{\text{ número de ensayos }};\overbrace{p}^{\text{ probabilidad de éxito }};\text{ FALSO }$)
\end{center}


%%%%%%%%%%%%%%%%%%%%%%%%%%%%%%%%%%%%%%%%%%%%%%%%%%%%%%%%%%%%%%%%%%%%%%%%%%%%%%%%%%%%%%%%%%%%%%%%%%%%%%%%%%%%%%%%%%%%%%%%%%%%%%%%%%%%%%%%%%%%%%%%%%%%%%%%%%%%%%%%%%%%%%%%%%%%%%%%%%%%

\hrulefill
\section{Distribución Poisson}
\subsection{Se usa cuando...}
En esta sección estudiará una variable aleatoria discreta que se suele usar para estimar el número de veces que sucede un hecho determinado (ocurrencias) en un intervalo de tiempo o de espacio. $\displaystyle \text{ Ocurrencias }/ \text{ tiempo ó espacio }$

\subsection{Condiciones de uso de Poisson}
Si se satisfacen las condiciones siguientes, el número de ocurrencias es una variable aleatoria discreta, descrita por la distribución de probabilidad de Poisson:
\begin{enumerate}
    \item La probabilidad de ocurrencia es la misma para cualesquiera dos intervalos de la misma longitud. 
    \item La ocurrencia o no-ocurrencia en cualquier intervalo es independiente de la ocurrencia o no-ocurrencia en cualquier otro intervalo. 
\end{enumerate}
    
\subsection{Función de probabilidad de Poisson}
Función de probabilidad de Poisson:
\[
    f(x) = \frac{\mu ^x e ^{-\mu}}{x!}     
\] en donde:
\begin{itemize}
    \item $f(x)$: probabilidad de x ocurrencias en un intervalo 
    \item $\mu$: valor esperado o número medio de ocurrencias 
    \item $e$ el número de Euler, 2.71...
\end{itemize}

\subsection{Formula en excel}
\begin{center}
    \verb|=POISSON.DIST(|$\underbrace{x}_{\text{ Variable aleatoria }},\overbrace{\mu}^{\text{ media }},\underbrace{\text{ FALSO }}_{\text{ Acumulado }}    $\verb|)| \newline 
    
\end{center}



%%%%%%%%%%%%%%%%%%%%%%%%%%%%%%%%%%%%%%%%%%%%%%%%%%%%%%%%%%%%%%%%%%%%%%%%%%%%%%%%%%%%%%%%%%%%%%%%%%%%%%%%%%%%%%%%%%%%%%%%%%%%%%%%%%%%%%%%%%%%%%%%%%%%%%%%%%%%%%%%%%%%%%%%%%%%%%%%%%%%
\hrulefill
\section{Probabilidad hipergeométrica}
\subsection{La distribución hipergeométrica es...}
La distribución de probabilidad hipergeométrica está estrechamente relacionada con la distribución binomial. Pero difieren en dos puntos: \textbf{en la distribución hipergeométrica los ensayos no son independientes y la probabilidad de éxito varía de ensayo a ensayo.}

%----------------------------------------------------------------------------------------
\subsection{Componentes }
\begin{itemize}
    \item $r$: número éxitos 
    \item $N$: población total 
    \item $N-r$: número de fracasos 
    \item $n$: muestra 
    \item $x$: variable aleatoria 
\end{itemize}


%----------------------------------------------------------------------------------------
\subsection{Función de probabilidad geométrica}
\[
  f(x) = \frac{\binom{r}{n}\binom{N-r}{n-x}}{\binom{N}{n}} 
\] donde $0\leq x \leq r$ 
\begin{itemize}
    \item $f(x)$: probabilidad de $x$ éxitos en $n$ ensayos.
    \item $n$: número de ensayos 
    \item $N$: número de elementos en la población 
    \item $r$: número de elementos en la población considerados éxitos. 
\end{itemize}

%----------------------------------------------------------------------------------------
\subsection{Consideraciones}
\begin{itemize}
    \item $\displaystyle \binom{N}{n}$: número de maneras que es posible tomar $n$ de $N$ 
    \item $\displaystyle \binom{r}{x}$: número de maneras que es posible tomar $x$ éxitos de un total de $r$ éxitos en una población.
    \item $\displaystyle \binom{N-r}{n-x}$: número de veces que se pueden tomar $n-x$ de $N-r$.
\end{itemize}


%----------------------------------------------------------------------------------------
\subsection{Ejemplo}
Se tienen empaques de 12 unidades, in inspector quiere seleccionar aleatoriamente tres de estas 12 unidades, si la caja tiene cinco fusibles defectuosos, \pregunta{Cuál es la probabilidad que le salga uno de los tres fusibles sdefectuoso}. Considerar:
\begin{itemize}
    \item $N=12$ el tamaño de la población 
    \item $n=3$ el tamaño de la muestra del inspector
    \item $r=5$ consideramos defecto como éxito 
    \item $x=1$ se quiere saber la probabilidad de encontrar sólo uno 
\end{itemize}
\[
  f(1) = \frac{\binom{5}{1}\binom{7}{2}}{\binom{12}{3}} = 0.4773  
\]
\pregunta{Cuál es la posibilidad de encontrar \textbf{por lo menos uno defectuoso}} 
\begin{itemize}
    \item Defecto es $r$, entonces queremos que $r=0$ 
\end{itemize}
\[
  f(0) = \frac{\binom{5}{0}\binom{7}{3}}{\binom{12}{3}} = 0.1591
\]
$f(x=0)$ es  la probabilidad de tener 0 defectos, la probabilidad es que encontremos uno, dos o incluso tres defectos. Entonces $1-f(x=0)$ por complemento encontramos $f(x<0)=0.8409$

\subsection{Fórmula con excel}
\begin{center}
   \verb|=DISTR.HIPERGEOM(|$\underbrace{x}_{\text{ Variable aleatoria }};\overbrace{n}^{\text{ Muestra }};\underbrace{r}_{\text{ Población de éxito }};\overbrace{N}^{\text{ Población }}$\verb|)|
\end{center}


%%%%%%%%%%%%%%%%%%%%%%%%%%%%%%%%%%%%%%%%%%%%%%%%%%%%%%%%%%%%%%%%%%%%%%%%%%%%%%%%%%%%%%%%%%%%%%%%%%%%%%%%%%%%%%%%%%%%%%%%%%%%%%%%%%%%%%%%%%%%%%%%%%%%%%%%%%%%%%%%%%%%%%%%%%%%%%%%%%%%

\hrulefill
\section{Distribución de probabilidad uniforme}
\subsection{La distribución es...}
Hace uso de variables aleatorias contínuas. Se asume que todos los valores son igualmete probables. 
\begin{enumerate}
    \item No se habla de una variable aleatoria tomando un valor si no un intervalo. 
    \item Si se toma un valor esa probabilidad tiende a $0$.
\end{enumerate}

\subsection{Función de densidad de probabilidad uniforme}
\[
  f(x) = \begin{cases}
    \frac{1}{b-a} & \text{ Para }\; a \leq x \leq b \\ 
    0 & \text{ En cualquier otro caso } \\ 
  \end{cases} 
\]

\subsection{Fórmulas para esta distribución}
\subsubsection{Valor esperado}
\[
  E(x) = \frac{a+b}{2} 
\]
\subsubsection{Varianza}
\[
  \frac{\p{b-a} ^2}{12} 
\]

\subsection{Fórmulas con excel}
No hay.


%%%%%%%%%%%%%%%%%%%%%%%%%%%%%%%%%%%%%%%%%%%%%%%%%%%%%%%%%%%%%%%%%%%%%%%%%%%%%%%%%%%%%%%%%%%%%%%%%%%%%%%%%%%%%%%%%%%%%%%%%%%%%%%%%%%%%%%%%%%%%%%%%%%%%%%%%%%%%%%%%%%%%%%%%%%%%%%%%%%%
\hrulefill
\section{Distribución de probabilidad normal}
\subsection{La distribución es...}
Consiste de variables aleatorias contínuas.

\subsection{Función de densidad de probabilidad normal}
\[
  f(x) = \frac{1}{\sigma \sqrt{2\pi}} e^{-\p{x-\mu}^2 / 2\mu^2 }
\] donde:
\begin{itemize}
    \item $\mu$: media 
    \item $\sigma$: desviación estándar 
    \item $\pi$: número de circunferencia unitaria (3.14...)
    \item $e$: número de Euler (2.71...)  
\end{itemize}

\subsection{Propiedades de las curvas normales}
\begin{enumerate}
    \item Las distribuciones normales se diferencian por medio de dos parámetros: $\mu$ (media) \& $\sigma$ (desviación estándar).
    \item El punto más alto de la curva normal se encuentra sobre la media, la cuenta coincide con la mediana y la moda.
    \item La media de una distribución normal puede tener cualquier valor, de $-\infty$ a $\infty$.
    \item La distribución normal es simétrica. Su sesgo es cero. 
    \item La desviación estándar $\sigma$ determina qué tan plana o ancha sea la curva, mientras más grande la desviación más plana. 
    \item El área de $-\infty$ a la media($\mu$) es siempre 0.5, al igual que el área entre la media ($\mu$) a $\infty$ es siempre 0.5.
    \item Los porcentajes de los valores que se encuentran en algunos intervalos comúnmente usados son: 
        \begin{enumerate}
            \item 68.3\% de los valores pertinentes a la variable aleatoria se encuentran a $\pm$ una desviación estándar de la media. 
            \item 95.4\% de los valores pertinentes a la variable aleatoria se encuentran a $\pm$ dos desviaciones estándar de la media. 
            \item 99.7\% de los valore pertinentes a la variable aleatoria se encuentran $\pm$ tres desviaciones estándar de la media.
        \end{enumerate}
\end{enumerate}


%%%%%%%%%%%%%%%%%%%%%%%%%%%%%%%%%%%%%%%%%%%%%%%%%%%%%%%%%%%%%%%%%%%%%%%%%%%%%%%%%%%%%%%%%%%%%%%%%%%%%%%%%%%%%%%%%%%%%%%%%%%%%%%%%%%%%%%%%%%%%%%%%%%%%%%%%%%%%%%%%%%%%%%%%%%%%%%%%%%%
\hrulefill
\section{Distribución de probabilidad normal estándar}
Es una distribución de probabilidad normal solo que con la media ($\mu$) igual a 0 y la desviación estándar igual a 1; $\mu=0$, $\sigma=1$ 

\subsection{Función de probabilidad normal estándar}
\[
  f(z) = \frac{1}{\sqrt{2\pi}} e^{z^2/2}
\]

\subsection{Tres probabilidades necesarias para calcular}
\begin{enumerate}
    \item Probabilidad que la variable aleatoria normal estándar $z$ sea menor o igual que un valor dado. 
    \item Probabilidad de $z$ esté entre dos valores dados. 
    \item Probabilidad que $z$ sea mayor o igual que un valor dado. 
\end{enumerate}

\subsection{Conversión de variables aleatorias normales a variables aleatorias estándares}
\[
  z = \frac{x-\mu}{\sigma} 
\]

\subsection{Fórmulas en excel}
\begin{center}
   \verb|=DISTR.NORM.ESTAND.N(|$\underbrace{z}_{\text{ Punto z }};\overbrace{\text{ FALSO }}^{\text{ Acumulado }}$\verb|)|
\end{center}

%%%%%%%%%%%%%%%%%%%%%%%%%%%%%%%%%%%%%%%%%%%%%%%%%%%%%%%%%%%%%%%%%%%%%%%%%%%%%%%%%%%%%%%%%%%%%%%%%%%%%%%%%%%%%%%%%%%%%%%%%%%%%%%%%%%%%%%%%%%%%%%%%%%%%%%%%%%%%%%%%%%%%%%%%%%%%%%%%%%%
\hrulefill
\section{Aproximación normal de las probabilidades binomiales}
\subsection{Se usa para...}
Tener una aproximación a las probabilidades sacadas de la binomial.

\subsection{Fórmulas}
\begin{itemize}
    \item La media: $\mu=n\cdot p$ donde $n$ es número de ensayos y $p$ es probabilidad de éxitos. 
    \item La desviación estándar: $\sigma=\sqrt{np(1-p)}$ 
\end{itemize}



%%%%%%%%%%%%%%%%%%%%%%%%%%%%%%%%%%%%%%%%%%%%%%%%%%%%%%%%%%%%%%%%%%%%%%%%%%%%%%%%%%%%%%%%%%%%%%%%%%%%%%%%%%%%%%%%%%%%%%%%%%%%%%%%%%%%%%%%%%%%%%%%%%%%%%%%%%%%%%%%%%%%%%%%%%%%%%%%%%%%
\hrulefill
\section{Distribución de probabilidad exponencial}
\subsection{Propiedades de la distribución exponencial}
\begin{itemize}
    \item La media de la distribución y la desviación estándar de la distribución son iguales. 
    \item La distribución exponencial está sesgada a la derecha.
\end{itemize}

\subsection{Función de densidad de probabilidad exponencial}
\[
  f(x) = \frac{1}{\mu} e^{-x/\mu }
\] para $x\geq 0, \mu > 0$ \newline 
Donde $\mu=$ valor esperado o media 


\subsection{Relación entre la distribución de Poisson y la exponencial}
\begin{itemize}
    \item Poisson: una distribución de probabilidad discreta que se usa para examinar el número de ocurrencias de un evento en un determinado intervalo de tiempo o espacio. 
    \item Distribución exponencial da una descripción de la longitud de los intervalos entre las ocurrencias. 
\end{itemize}
\begin{itemize}
    \item La media en Poisson = 10 $\rightarrow$ la media exponencial es $\frac{1}{10} $. 
\end{itemize}
\subsubsection{Media Poisson y media exponencial}
\[
    \overbrace{\mu}^{\text{ Media Poisson }} = k \qq \qq \overbrace{\lambda}^{\text{ Media exponencial }} = \frac{1}{\mu} 
\]


%%%%%%%%%%%%%%%%%%%%%%%%%%%%%%%%%%%%%%%%%%%%%%%%%%%%%%%%%%%%%%%%%%%%%%%%%%%%%%%%%%%%%%%%%%%%%%%%%%%%%%%%%%%%%%%%%%%%%%%%%%%%%%%%%%%%%%%%%%%%%%
\end{document}

