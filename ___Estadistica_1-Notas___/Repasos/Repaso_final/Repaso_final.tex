\documentclass{article}

\usepackage{generalsnips}
\usepackage{calculussnips}
\usepackage[margin = 1in]{geometry}
\usepackage{pdfpages}
\usepackage[spanish]{babel}
\usepackage{amsmath}
\usepackage{amsthm}
\usepackage[utf8]{inputenc}
\usepackage{titlesec}
\usepackage{xpatch}
\usepackage{fancyhdr}
\usepackage{tikz}
\usepackage{hyperref}
\title{Repaso final}
\date{2020 April 20, 10:30PM}
\author{David Gabriel Corzo Mcmath}

\begin{document}
\maketitle
%%%%%%%%%%%%%%%%%%%%%%%%%%%%%%%%%%%%%%%%%%%%%%%%%%%%%%%%%%%%%%%%%%%%%%%%%%%%%%%%%%%%%%%%%%%%%%%%%%%%%%%%%%%%%%%%%%%%%%%%%%%%%%%%%%%%%%%%%%%%%%

\section{Distribución muestral de $\bar{x}$}
\begin{itemize}
    \item La distribución muestral de $\bar{x}$ es la distribución de probabilidad de todos los valores de la media muestral. 
    \item Como toda distribución, tiene valor esperado, desviación estándar, etcétera. 
\end{itemize}
% 
\subsection{Valore esperado de $\bar{x}$ }
\[
  E(\bar{x}) = \mu 
\]donde:
\begin{itemize}
    \item $E(\bar{x})$: valor esperado 
    \item $\mu $: media poblacional 
\end{itemize}
%%
\subsubsection{Observaciones}
Cuando el valor esperado de un estimador puntual es igual al parámetro poblacional, se dice que el estimador puntual es insesgado.

% 
\subsection{Desviación estándar de $\bar{x}$}
Para una población finita: 
\[
    \sigma_{\bar{x}} = \sqrt{\frac{N-n}{N-1} }\p{\frac{\sigma}{\sqrt{n}} } 
\]
Para una población infinita ó La población sea finita y el tamaño de la muestra sea menor o igual a 5\% del tamaño de la población; es decir, $n/N \leq 0.05$: 
\[
  \sigma_{\bar{x}} = \frac{\sigma}{\sqrt{n}} 
\]
donde
\begin{itemize}
    \item $\sigma_{\bar{x}}$: desviación estándar de $\bar{x}$.
    \item $\sigma$: desviación estándar de la población. 
    \item $n$: tamaño de la muestra.
    \item $N$: tamaño de la población. 
\end{itemize}
\subsubsection{El factor de corrección para poblaciones finitas}
\[
  \text{ Factor de corrección:  }\; = \sqrt{\frac{N-n}{N-1} }
\]

\subsubsection{Error estándar}
Se refiere a la desviación estándar de un estimador puntual.


%----------------------------------------------------------------------------------------
\subsection{Forma de la distribución muestral de $\bar{x}$ }
Dos casos, tiene distribución normal o no: 
\begin{itemize}
    \item Tiene distribución normal: Cuando la población tiene distribución normal, la distribución muestral de está distribuida normalmente sea cual sea el tamaño de la muestra.
    \item No tiene distribución normal: Cuando la población de la que se tomó la muestra aleatoria simple no tiene distribución normal, el teorema del límite central ayuda a determinar la forma de la distribución muestral de $\bar{x}$.
        \begin{itemize}
            \item Teorema del límite central: Cuando se seleccionan muestras aleatorias simples de tamaño $n$ de una población, la distribución muestral de la media muestral puede aproximarse mediante una distribución normal a medida que el tamaño de la muestra se hace grande. 
        \end{itemize}
\end{itemize}



%%%%%%%%%%%%%%%%%%%%%%%%%%%%%%%%%%%%%%%%%%%%%%%%%%%%%%%%%%%%%%%%%%%%%%%%%%%%%%%%%%%%%%%%%%%%%%%%%%%%%%%%%%%%%%%%%%%%%%%%%%%%%%%%%%%%%%%%%%%%%%
\end{document}

