\section*{Preliminares}
\begin{itemize}
    \item RAPORT
\end{itemize}

%%%%%%%%%%%%%%%%%%%%%%%%%%%%%%%%%%%%%%%%%%%%%%%%%%%%%%%%%%%%%%%%%%%%%%%%%%%%%%%%%%%%%%%%%%%%%%%%

\section{Continuación }
\begin{itemize}
    \item Puntos z : 
        \[
          \text{Punto} Z_{i} = \frac{x_{i} - \bar{x}}{s} 
        \]
    
    \item \emph{\textbf{Ejemplo: }Las estaturas de la UFM \& UVG.}
        \begin{center}
           \begin{tabular}{ | p{5cm} | p{5cm} | }
               \hline
                    UFM & UVG    \\
               \hline
                    $\bar{x} = 172cm$ MAX: 194cm & $\bar{x} = 1.68$ MAX: 197cm\\
                    $S = 9.71cm$ MIN: 156cm & $S = 10.9CM$ MIN: 145cm \\ 
                \hline   
           \end{tabular}
        \end{center}
    
    \item Punto z del ejemplo:
        \begin{align*}
            Z_{\text{MAX UFM}} = \frac{194 - 172}{9.71}   &= 2.26 \text{Desviación estándar del promedio} \\ 
            Z_{\text{MIN UFM}} = \frac{156 - 172}{9.71} &= -1.65   \text{Son negativos por que son menores a la media}               \\  
            Z_{\text{MAX UVG}} = \frac{197 - 168}{10.9} &= 2.66  \text{Desviaciones estándar de su media} \\  
            Z_{\text{MIN UVG}} = \frac{145 - 168}{10.9} &= 2.11 \text{Está a ciertas desviaciones estándar de la media}                   \\  

            Z_{172} = \frac{172 - 172}{9.74} &= \frac{0}{9.71}                  \\   
        \end{align*}
    
    \item \emph{\textbf{Definición de ``punto Z":} es cuando }
        \begin{itemize}
            \item La gráfica Z está centrada en el número cero, el cero lo que quiere decir es que de ahí parte la distribución Z, es decir el punto Z es el que esta exactamente a 0 de desviación estándar.
            \item Volver un número un punto Z se le llama estandarizar. 
            \item Interpretemos: La gran mayoría de datos están concentrados en el centro. Ver el punto crítico y el punto de inflexión.
                \begin{figure}[htbp]
                    \centering
                    % \includegraphics[width=8cm]{./../__Imagenes__/2020-01-21-01-Estadistica.jpg}
                    \caption{}
                    \label{}
                \end{figure}
                \[
                  \pm z = 1 60\% \\ 
                  \pm z = 2 95\% \\ 
                \]
            
            \item 
        \end{itemize}
    \item Distribución en la forma de montaña o de campana: la distribución presentada de una manera línea continua contempla una cantidad infinitesimal de clases.
\end{itemize}
