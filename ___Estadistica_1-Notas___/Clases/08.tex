\section{}
\begin{itemize}
    \item Las probabilidades marginales están en los márgenes o en lso grand totales.
    \item Las probabilidades conjuntas, son las que están agrupadas.
    \item Ejemplo:
        \begin{align*}
            A= \text{  Reggaeton  } \\ 
            P(A)=0.4231 \quad \text{  \#Cuál es la probabilidad que ocurra el evento ``que le gusta el reggaeton''  } \\ 
            P(A^c)=1-P(A) = 0.5769 \quad \text{  \#Cual es la probabilidad que ocurra el complemento del evento A.  } \\ 
        \end{align*}
\end{itemize}

\subsection{Ley de adición}
\begin{itemize}
    \item La ley de la adición:
        \[
          P(A\cup B)= P(A)+P(B)-P(A\cap B)
        \]
        \begin{itemize}
            \item Se tiene que restar $P(A\cap B)$ por que hay instancias que la probabilidad está por arriba del 100\%.
        \end{itemize}
        
\end{itemize}

\subsection{Probabilidad condicional}
\begin{itemize}
    \item $P(Rock\vert Fut)$ dado que les gusta el fut, a cuántos les gusta el Rock.
    \item $P(Fut\vert Rock)$ dado que les gusta el rock, a cuántos les gusta el fut.
    \item \[
      P(A\vert B) = \frac{P(A\cap B)}{P(B)}
    \]
    
    \item Si la P(A) == P(B) los eventos son independientes, cuando P(A) < P(B) ó P(A) > P(B) son dependientes.
    \begin{itemize}
        \item Ser dependiente significa que las probabilidades de la música respecto de, fut varía proporcionalmente.
        \item Si alguien le gusta el fútbol tiene que ver con la música que le gusta. 
    \end{itemize}
\end{itemize}
