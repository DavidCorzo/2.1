\section{Otras medidas de localización}
\begin{itemize}
    \item Media ponderada: 
        \begin{itemize}
            \item \[
              \bar{x} = \frac{\sum_{i=1}^{n}W_i\cdotX_i} {\sum_{i=1}^{n}W_i} 
            \]
            \begin{itemize}
                \item Donde $W$ es ``weight'' o el peso que le vamos a delegar a cada clase.
                \item Toma en cuenta del peso de lo que estamos midiendo y a base de eso resulta la media ponderada. La media aritmética no toma en cuenta el peso.
                \item $X_i$ es la observación.
            \end{itemize}            
            \item \emph{\textbf{Definición de ``media ponderada":} }
        \end{itemize}
    \item Media a partir de datos agrupados: 
        \begin{itemize}
            \item \[
              \bar{x} = \frac{\sum_{i=1}^{n}f_iM_i}{n} 
            \]
            
            \item La media a partir de datos agrupados se tiene un resultado con un nivel moderado de incertidumbre. 
            \item \emph{Citación:``Cuando agrupamos datos perdemos información."}         
            \item La media aritmética no va a ser exactamente igual que la media a partir de datos agrupados.   
        \end{itemize}
    
    \item Varianza a partir de datos agrupados: 
        \begin{itemize}
            \item \[
              S^2= \frac{\sum_{}^{}f_i(M_i-\bar{x})^2}{n-1} 
            \]
                \begin{itemize}
                    \item $M_i$ es la media de las clases, si tengo un clase de 80-90 $M_i = \frac{(80+90)}{2}$.
                    \item  
                \end{itemize}
            \item 
        \end{itemize}
\end{itemize}

\section{Excel steps}
\begin{itemize}
    \item To show empty data fields:
        \begin{itemize}[label=$\downarrow$]
           \item right click en row labels (random) 
           \item Configuración de campo ...  
           \item Diseño e impresión 
           \item Mostrar elementos sin datos (cheque)
        \end{itemize}
\end{itemize}
