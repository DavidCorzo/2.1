\section{Probabilidad de Poisson}
\begin{itemize}
    \item Función de probabilidad de Poisson:
        \[
          f(x)= \frac{\mu e^x}{x!} 
        \]
        \begin{itemize}
            \item f(d): probabilidad de x ocurrencias en un intervalo 
            \item $\mu$: valor esperado o número medio de ocurrencias en un intervalo.
            \item $e$: número de Euler.
            \item Pensar $\mu$ como $\mu = \frac{\text{Eventos}}{\text{Tiempo}} $
        \end{itemize}
    
    \item Sirve para predecir cosas como:
        \begin{itemize}
            \item Las muertes de los magistrados en los estados unidos.
            \item Patrones en los huracanes.
        \end{itemize}

    \item Ejemplo de cajeros automáticos de un banco:  
        \begin{itemize}
            \item $\mu = \frac{10 \; \text{ carros }}{15 \;\text{minutos}} = 0.\overline{66} = \frac{2}{3} $
                % \plotfunction}{e^-x / x!}{-10:10}{x}{y} 
        \end{itemize}
\end{itemize}


