\section{Solución del parcial}
\subsection{1}
\begin{itemize}
    \item Prevalencia:
        \[
          \text{  Prevalencia  } = \frac{1,500}{100,000} = \underbrace{0.015}_{\text{  Probabilidades apriori  }}
        \]
    
    \item De una muestra, resultó positiva para 965 de 1,000 personas que se sabe que sí tenían la enfermedad:
        \[
           P(+|E) = 0.965  
        \]
    
    \item De una muestra resultó ser negativa para 97 de 100 personas que se sabe que no tenían la enfermedad:
        \[
          P(-|NE) = 0.97 
        \]
    
    \item \begin{center}
       \begin{tabular}{ | p{5cm} | p{5cm} | p{5cm} | p{5cm} | }
                Probabilidades apriori & Probabilidades condicioneles & Probabilidades posteriores \\ 
           \hline
                                & $P(+|E) = 0.965$ & $P(E)P(+|E) = 0.014$     &\\ 
                $P(E) = 0.015$  &                  &                          & $P(E|+)= \frac{P(E)P(+|E)}{\underbrace{P(E)P(+|E)+P(NE)P(+|NE)}_{P(+)}} = \frac{0.014}{0.04403}  $  \\
                                & $P(-|E) = 0.035$ & $P(E)P(-|E) = 0.000525 $ &\\ 
                
            \hline
                                & $P(+|NE) = 0.03$ & $P(NE)P(+|NE) = 0.02955$ &\\ 
                $P(NE) = 0.985$ &                  &                          &\\ 
                                & $P(-|NE) = 0.97$ & $P(NE)P(-|NE) = 0.95545$ &\\ 
           \hline
       \end{tabular}
    \end{center}
    
    \item Probabilidad que sea un falso negativo:
        \begin{itemize}
            \item $P(-|E) = 0.035$
        \end{itemize}
    
    \item Probabilidad de salir positivo:
        \begin{center}
           \begin{align*}
               P(+)= \frac{P(E)P(+|E)}{P(E)P(+|E) + P(NE)P(+|NE)} = \frac{0.014}{0.014+0.02955} = 0.04403 \\  
           \end{align*}
           \begin{itemize}[label=\#]
               \item Tomar en cuenta que para sacar $P(+)$ tomo todos los escenarios en los que se calcula la probabilidad de salir positivo.
           \end{itemize}
        \end{center}
    
    \item Probabilidad de salir negativo:
        \begin{center}
           \begin{align*}
               P(-) =\frac{P(E)P(-|E)}{P(E)} 
           \end{align*}
        \end{center}
\end{itemize}


%%%%%%%%%%%%%%%%%%%%%%%%%%%%%%%%%%%%%%%%%%%%%%%%%%%%%%%%%%%%%%%%%%%%%%%%%%%%%%%%%%%%%%%%%%
\subsection{2}
\begin{itemize}
    \item Permutaciones: 
        \begin{center}
           \begin{tabular}{ | p{5cm} | p{5cm} | p{5cm} |  p{5cm} |  p{5cm} | p{5cm} |}
               \hline
                    Letra 1 & Letra 2 & Letra 3             & Dígito 1 &  Dígito 2 & Dígito 3 \\
               \hline
                    26 opciones & 25 opciones & 24 opciones &  
               \hline
           \end{tabular}
           \begin{itemize}[label=\#]
               \item Una permutación es un proceso sin remplazo. No se repiten.
               \item Fórmula:
                    \[
                      _NP_n = \frac{N!}{(N-n)!} 
                    \]
                    Para el inciso en letras:
                    \[
                        _{26}P_3 = \frac{26!}{(26-3)!} = 15,600
                    \]
                    Para el inciso en números:
                    \[
                      _{10}P_3= \frac{10!}{(10-3)!} = 720 
                    \]
                    Si multiplico las dos permutaciones:
                    \[
                      15,600\cdot720 = 11,232,000 
                    \]
                    Saco la cantidad de permutaciones para sacar el total de posibilidades con repetición: 
                    \begin{center}
                       \begin{align*}
                           (26)^3 &= 17,576 \\ 
                           (10)^3 &= 1,000 \\ 
                           \text{  \# Multiplico:  } \quad 17,576 \times 1,000 = 17,576,000 \\ 
                       \end{align*}
                    Calculo la probabilidad dividiendo:
                    \[
                      \frac{_{26}P_3 \times _{10}P_3}{(26)^3\times(10)^3} \approx 0.6390
                    \]
                    \end{center}
           \end{itemize}
        \end{center}
\end{itemize}



%----------------------------------------------------------------------------------------
\subsection{3}
\begin{itemize}
    \item Para calcular independencia:
        \[
          P(A) \times P(B) = P(A\cap B)
        \]
\end{itemize}
