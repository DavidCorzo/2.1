\section{Combinaciones y permutaciones}
%%%%%%%%%%%%%%%%%%%%%%%%%%%%%%%%%%%%%%%%%%%%%%%%%%%%%%%%%%%%%%%%%%%%%%%%%%%%%%%%%%%%%%%%%%%%%%%%
\subsection{En general}
\begin{itemize}
    \item $N$ es el número total de resultados posibles, y $n$ la muestra.
    \item Combinaciones: $\left(\begin{matrix} N \\ n \\ \end{matrix}\right)$: de un conjunto de $N$ elementos cuántos
\end{itemize}

%%%%%%%%%%%%%%%%%%%%%%%%%%%%%%%%%%%%%%%%%%%%%%%%%%%%%%%%%%%%%%%%%%%%%%%%%%%%%%%%%%%%%%%%%%%%%%%%
\subsection{Combinaciones}
\begin{itemize}
    \item Combinaciones:
        \[
          C_N^n = _NC_n = \left( \begin{matrix}
              N \\ 
              n \\ 
          \end{matrix} \right) = \frac{N!}{n!(N-n)!} 
        \]
    
    \item La más usual es $_NC_n$ o $\left(\begin{matrix}
        N \\ 
        n \\ 
    \end{matrix}\right)$ 

    
    \item Cuenta la cantidad de las resultados muestrales tomando en cuenta cada combinación sin orden.
    \item Ejemplo:
        \[
          456-\text{  HBJ  }
        \]
        \begin{align*}
            C_3^7 = \frac{10!}{3!7!} = \frac{10\times 9 \times 8 \times 7 \times 6 \times 5 \times 4 \times 3 \times 2 \times 1}{(3\times 2 \times  1 )(7 \times 6 \times 5 \times 4 \times 3 \times 2 \times 1)} = \frac{720}{6}  = 120 \\   
        \end{align*}
\end{itemize}


%%%%%%%%%%%%%%%%%%%%%%%%%%%%%%%%%%%%%%%%%%%%%%%%%%%%%%%%%%%%%%%%%%%%%%%%%%%%%%%%%%%%%%%%%%%%%%%%
\subsection{Permutaciones}
\begin{itemize}
    \item \[
        P_N^n = _NP_n = n!\left( \begin{matrix}
            N \\ 
            n \\ 
        \end{matrix} \right) = \frac{N!}{(N-n)!} 
      \]
    
    \item Ejemplo:
        \begin{align*}
            n! \left(\begin{matrix}
                N \\ n \\ 
            \end{matrix}\right) = \frac{10!}{7!}  = \frac{10 \times 9 \times ... times 1}{7 \times 6 \times  ...  \times 1} = 720 \\  
        \end{align*}
\end{itemize}



%%%%%%%%%%%%%%%%%%%%%%%%%%%%%%%%%%%%%%%%%%%%%%%%%%%%%%%%%%%%%%%%%%%%%%%%%%%%%%%%%%%%%%%%%%%%%%%%
\section{Explicación de número de placas en GT }
\subsection{Combinaciones de letras}
\begin{align*}
    \frac{21!}{3!(21-3)!} = \frac{21!}{3!18!} = 1330 \\  
\end{align*}

\subsection{Permutaciones de letras}
\begin{align*}
    \frac{21!}{18!} = 21 \times 20 \times 19 = 7,980 \\ 
    720 \times 7980 = 5,745,600 \\ 
\end{align*}


%%%%%%%%%%%%%%%%%%%%%%%%%%%%%%%%%%%%%%%%%%%%%%%%%%%%%%%%%%%%%%%%%%%%%%%%%%%%%%%%%%%%%%%%%%%%%%%%
