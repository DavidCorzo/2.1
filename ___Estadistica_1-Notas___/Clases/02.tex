\section{Notas}
\begin{itemize}
    \item Tabla de frecuencias: Con todos los datos, la suma de todo es lo que se pone. 
    \item Tabla de frecuencias relativas: cuando la suma de todo es uno. 
    \item Tabla de frecuencias porcentual: cuando la suma de todo es 100\%.
\end{itemize}

%%%%%%%%%%%%%%%%%%%%%%%%%%%%%%%%%%%%%%%%%%%%%%%%%%%%%%%%%%%%%%%%%%%%%%%%%%%%%%%%%%%%%%%%%%%%%%%%

\section{Audit.xlsx}
\begin{itemize}
    \item Las diferentes categorías que se agrupan se se les da el nombre de clase, mientras más peculiaridades se tengan por clase se tendrán más clases.
    \item La cantidad total de datos $\equiv$ número de observaciones.
    \item El número de observaciones se le llama ``n''.
    \item Si queremos 5 clases cada clase debe de tener el mismo ancho, esta para dar uniformidad a todos los intervalos para ``comparar peras con peras''.
    \item Al ancho de clase que salga de la fórmula hay \textbf{que redondearlo para arriba}.
    \item Los histogramas:
        \begin{itemize}
            \item Sólo se pueden hacer para variables cuantitativas, para números.
            \item Cuando en el eje-x están intervalos son números.
            \item Las barras estarán pegadas sin ningún gap entre ellas.
            \item Son números enteros osea \textbf{discretos}.
            \item En Excel: Seleccionar una barra $\rightarrow$ click derecho $\rightarrow$ Dar formato de serie de datos $\rightarrow$ Ancho de rango $\rightarrow$ bajarlo a 0\%.
        \end{itemize}
\end{itemize}
