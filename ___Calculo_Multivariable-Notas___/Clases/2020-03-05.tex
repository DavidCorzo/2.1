\section{14.5 Derivadas direccionales y gradiente}
\begin{itemize}
    \item \termdefinition{El gradiente de $i$}{Denotado como $Df$ ó $gradf$ es la función vectorial.}
    \item Fórmula de gradiente: 
        \[
          Df = f_x \hat{i} + f_y \hat{j} = \left\langle \dervpar{f}{x},\dervpar{f}y{} \right\rangle 
        \]
    
    \item Función de tres variables: $f(x,y,z)$
        \[
          Df = \left\langle \dervpar{f}{x},\dervpar{f}{y},\dervpar{f}{z} \right\rangle 
        \]
    
    \item $f$ es una función escalar, $Df$ es una función vectorial
    \item Derivadas parciales en la dirección de $x$ ó en la dirección de $y$.
        \begin{itemize}
            \item La razón de cambio promedio entre $z_1$ y $z_0$ si uno se desplaza en la dirección del vector $\vec{u}$ el cual es unitario.
            \item 
            \[
              \frac{\Delta z}{\Delta d} = \frac{z_1-z_0}{h\sqrt{a^2+b^2}} ...  
            \]
        \end{itemize}
    
    \item Derivada direccional: es la razón de cambio instantánea (qué sucede cuando doy un pequeño paso) de la razón de cambio $z$ en la dirección de $\vec{u}$, unitario en el punto $(x_0,y_0)$
        \[
            Du\; f(x_0,y_0) = \lim_{h \to 0} \left(\frac{f(x_0+ha,y_0+hb)-f(x_0,y_0)}{h}\right)
        \]
        \begin{itemize}
            \item En 3-D :
                \begin{center}
                   \begin{align*}
                       \omega_0 = f(x_0,y_0,z_0) \\ 
                       \omega_1 0 f(x_0+ha,y_0+hb,z_0+hc) \\ 
                       Du\; f(x_0,y_0,z_0) = \lim_{h \to 0} \left(\frac{f(\vec{x_0}+h\vec{u})-f(\vec{x_0})}{h} )\right)
                       x_0 = \left\langle x_0,y_0,z_0 \right\rangle \\ 
                       \vec{u} = \left\langle a,b,c \right\rangle \\ 
                   \end{align*}
                   \begin{itemize}
                       \item \pregunta{Cómo se calcula $Du\;f(\vec{x_0})$} 
                   \end{itemize}
                   \begin{align*}
                       u = \left\langle 0,0,0 \right\rangle  \qqq D\hat{i}f(x_0,y_0) = \lim_{h \to 0} \frac{f(x_0+h,y_0,z_0)-f(x_0,y_0,z_0)}{h} \\ 
                       D\hat{i}f(x_0,y_o,z_0) = \dervpar{f}{x} 
                       \\
                       u = \left\langle 0,1,0 \right\rangle \\ 
                       D\hat{j}\;f(x_0,y_0,z_0) = \dervpar{f}{y} \\ 
                       \text{ Sea }\; g(h) = f(x_0+ha,y_0+hb,z_0+hc) \\ 
                       g'(h) = \lim_{h \to 0} \frac{f(\vec{x_0}+h\vec{u}-f(\vec{x_0}))}{h} = Du\;f(x_0,y_0,z_0) \\ 
                       \vec{u} = \left\langle a,b,c \right\rangle \\ \qqq \left| \vec{u} \right| =1 \\ 
                       g(h) = f(x,y,z) \qqq \begin{matrix}
                           x = x_0+ha \\ 
                           y = y_0+hb \\ 
                           z = z_0+hc \\ 
                       \end{matrix}  \\ 
                       \therefore \qquad g'(h) = \dervpar{f}{x} a + \dervpar{f}{y} b + \dervpar{f}{z} c \\ 
                       Df = \left\langle \dervpar{f}{x} , \dervpar{f}{y} , \dervpar{f}{z}  \right\rangle \qquad \vec{u} = \left\langle a,b,c \right\rangle \\ 
                   \end{align*}
                \end{center}
        \end{itemize}
    
    \item \textbf{Theorem: } Si $f$ es diferenciable en $x,y,z$ la derivada direccional de $f$ en el punto $\vec{x_o}$ en la dirección del vector unitario $\vec{u}$ es :
        \[
          D\vec{u} f(\vec{x_0}) = Df \cdot \vec{u} 
        \]
        Producto punto entre el gradiente y $\vec{u}$
\end{itemize}


%----------------------------------------------------------------------------------------
\subsection{Ejercicios}
\begin{enumerate}
    \item Encuentre la derivada direccional de $f(x,y)=e^x\sin(y)$ en el punto $(0,\frac{\pi }{3} )$ en la dirección de $\vec{v}= \left\langle -6,8 \right\rangle $.
        \begin{center}
           \begin{align*}
               D\vec{u}\; f(0,\frac{\pi }{3} ) = Dff(0,\frac{\pi }{3} )\cdot \vec{u} \\ 
               \vec{v} \; \text{ no es unitario } \;  \left| \vec{v} \right| = \sqrt[36+64] = 10 \\ 
               \vec{u} = \frac{\vec{v}}{\left| \vec{v} \right| } = \frac{1}{10} \left\langle -6,8 \right\rangle \\ 
               Df = \left\langle f_x,f_y \right\rangle = \left\langle e^x\sin(y),e^x\cos(y) \right\rangle \\ 
               Df(0,\frac{\pi }{3}) 0 \left\langle \frac{\sqrt{3}}{2},\frac{1}{2} \right\rangle \\ 
               D\vec{u} f(o,\frac{\pi }{3} ) = \frac{1}{10} \left\langle -6,8 \right\rangle \cdot \left\langle \frac{\sqrt{3}}{2} , \frac{1}{2}  \right\rangle \\ 
               = \frac{1}{10} \left(-3\sqrt{3}+4\right)\approx -0.119615 \\ 
               \text{ Interpretación:  } \qq f(0,\frac{\pi }{3} ) = e^0 \frac{\sqrt{3}}{2} = 0.8660 \\ 
           \end{align*}
           \begin{itemize}
               \item Resúmen: $\displaystyle Df = \left\langle \dervpar{f}{x} , \dervpar{f}{y} , \dervpar{f}{z}  \right\rangle $ \\ 
               \item Derivada direccional: $\displaystyle Du f(\vec{x_0}) = Df\cdot \vec{u} $
           \end{itemize}
        \end{center}
    
    \item Considere la función $g(r,t)=r^2\tan\left(\frac{\pi}{4}\right)$:
        \begin{itemize}
            \item Encuentre el gradiente de $g(r,t)$ en el punto $(2,1)$:
            \begin{center}
               \begin{align*}
                   Dg = \left\langle gr,gt \right\rangle = \left\langle 2r\tan\left(\frac{\pi}{4}, \frac{\pi r^2}{4}\sec^2\left(\frac{\pi t}{4}\right)\right) \right\rangle \\ 
                   Dg(2,1) = \left\langle 4\tan\p{\frac{\pi}{4}}, \pi \sec ^2 \p{\frac{\pi}{4}} \right\rangle \\ 
                   Dg(2,1)= \left\langle 4,2\pi \right\rangle \\ 
               \end{align*}
            \end{center}
            
            \item Encuentre la razón de cambio instantánea de $g$ en el punto $(2,1)$ en la dirección de $v = \left\langle -1,2 \right\rangle $:
                \begin{center}
                   \begin{align*}
                       Du \;g(2,1) = Dg(2,1)\cdot \vec{u} \\ 
                       \text{ Encontrar el vector unitario:  } \qqq \vec{v} = \frac{\vec{v}}{\left| \vec{v} \right| } = \frac{1}{\sqrt{5}}\left\langle -1,2 \right\rangle  \\ 
                       Du = \; g(2,1) = \frac{\left\langle 4,2\pi \right\rangle \cdot \left\langle -1,2 \right\rangle }{} = \frac{-4+4\pi}{} \approx 2.  
                   \end{align*}
                \end{center}        
        \end{itemize}
    
    \item Considere la función $f(x,y,z)=x^3y^2z$:
        \begin{itemize}
            \item Encientre el gradiente de $f$ en el punto $(1,2,4)$
                \begin{center}
                   \begin{align*}
                       Df(1,2,4) = \left\langle 3\cdot 4 \cdot 4 ,2 \cdot 1 \cdot 2 \cdot 4, 1 \cdot 4 \right\rangle \\ 
                       Df(1,2,4) = \left\langle 48,16,4 \right\rangle \\ 
                   \end{align*}
                \end{center}
            
            \item Encuentre $Du\; f(x,y,z)$ en $(1,2,4)$ en la dirección de $\vec{u} = \left\langle \frac{1}{3}, -\frac{2}{3} , \frac{2}{3}  \right\rangle $:
                \begin{center}
                   \begin{align*}
                       \left| \vec{u} \right|  = \frac{1}{3} \left| \left\langle 1,-2,2 \right\rangle  \right| = \frac{1}{3} \sqrt{1+4+4} = 1 \\ 
                       D\vec{u} \; f(1,2,4) = \nabla f(1,2,4) \cdot \vec{u} \\ 
                       = \frac{48}{3} - \frac{32}{3} + \frac{8}{3} = 16 - \frac{24}{3} = 18 \\ 
                       \text{ La razón instantánea de cambio es de 8 unidades } \\ 
                   \end{align*}
                   \begin{itemize}
                       \item Interpretación del gradiente: 
                        \begin{center}
                           \begin{align*}
                               D_{\vec{u}} f = \nabla f \cdot \vec{u} = |\nabla f||\vec{u}| \cos (\theta) \\ 
                               D_(\vec{u}) f = |\nabla f|\cos (\theta)  \qqq \theta \;\; \text{ Es el ángulo entre } \; Df \; \text{ y } \; \vec{u}\\ 
                           \end{align*}
                        \end{center}
                        
                        \item $D_{\vec{u}}$ es el valor máximo de $D_{\vec{u}} f$
                   \end{itemize}
                \end{center}
        \end{itemize}
\end{enumerate}



%%%%%%%%%%%%%%%%%%%%%%%%%%%%%%%%%%%%%%%%%%%%%%%%%%%%%%%%%%%%%%%%%%%%%%%%%%%%%%%%%%%%%%%%%%
\section{Valor máximo y mínimo}
\begin{itemize}
    \item El valor máximo de la derivada direccional o máxima razón de cambio, es $|\nabla f |$ y ocurre en la dirección $u = \frac{\nabla f }{|\nabla f |} $: steepest ascent
        \[
          \text{ Cuando  }\; \theta = \pi \qqq (D_{\vec{u}})_{\text{ min }} = -|\nabla f|
        \]
    
    \item El valor mínimo de la derivada direccional es $-|Df|$ y ocurre en la dirección $-Df$: Steepest descent 
\end{itemize}

\subsection{Ejercicios}
\begin{enumerate}
    \item Encuentre la máxima razón de cambio de $f(x,y)=(y^2+1)e^x$ en el punto $(0,2)$ y la dirección en que ocurre este cambio, 
        \begin{itemize}
            \item Valor máximo $|Df|$ en la dirección de $Df$
        \end{itemize}
        \begin{center}
           \begin{align*}
               Df = \left\langle e^x(y^2+1), 2ye^x \right\rangle \\ 
               Df (0,2) = \left\langle 4+1,2 \cdot 2 \cdot 1 \right\rangle = \left\langle 5,4 \right\rangle  \\ 
           \end{align*}
        \end{center}
        \begin{itemize}
            \item Valor máximo raxón de cambio 
        \end{itemize}
\end{enumerate}


