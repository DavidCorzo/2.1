\section{14.8 Multiplicadores de Lagrange}
Una función de dos variables puede estar sujeta a una restricción:
\[
  \text{ Máximo } \qq  z= f(x,y) \qq \text{ Sujeto A } \qq g(x,y) = c
\]
Si no es posible resolver para $y$ ó $x$ en la restricción, el problema no se puede reducir a una sola variable. \newline 
Se introduce una nueva variable, el \textbf{multiplicador de Lagrange} $\lambda $ para incorporar la restricción en la función objetivo. 
\[
  c - g(x,y) = 0 
\]

\[
  \underbrace{F(x,y,\lambda )}_{\text{ Función objetivo y restricción }} = \underbrace{f(x,y)}_{\text{ Objetivo }} + \lambda (\underbrace{c-g(x,y)}_{\text{ Restricción }})
\]
Extremos relativos: $\displaystyle F_x = F_y = F_\lambda = 0$
\begin{center}
   \begin{align*}
       F_x = f_x + \lambda g_x = 0 \\ 
       F_y = f_y  \lambda g_y = 0 \\ 
       F_\lambda = c - g(x,y) = 0 \\ 
   \end{align*}
   \[
       \begin{rcases}
            \nabla f = \lambda \nabla g\\
            g(x,y) = c \\ 
       \end{rcases} \text{ Condiciones necesarias para un extremo relativo }
   \]
\end{center} 

Problema $\displaystyle w = f(x,y,z)$ sujeta a $\displaystyle g(x,y,z) = c$ 
\begin{center}
   \begin{align*}
       F(x,y,z,x) = f(x,y,z) - \underbrace{\lambda}_{\text{ Variable artificial }} (c-g(x,y,z)) \\ 
    F_x=F_y=F_z=F_x=0 \\ 
    \text{ Condiciones: } \nabla f = \lambda \nabla  g \\ 
   \end{align*}
\end{center}


\subsection{Ejercicios}
\begin{enumerate}
    \item Encuentre los extremos relativos de $\displaystyle w=x^2+y^2+z^2$ sujeta a $\displaystyle 2x+y-z=18 \}$restricción. 
        \begin{center}
            Método 1: Resolver para z
           \begin{align*}
               z = 2x+y-18 \\
           \end{align*}
           Sustituya en $\displaystyle w$  para obtener una función de 2 variables.
        %    \begin{align*}
            %    w = x^2+y^2+(2x+y-18)^2 \qq \qq \nabla w = \vec{o} \\ 
            %    \begin{matrix}
            %        w_x = 2x + 4(2x+y-18) = 10x+4y-72=0 \qq R_1\\
            %        w_y = 2y + 2(2x+y-18) = 4x+4y-36=0 \qq R_2\\

            %        R_1 - R_2: \qq 6x-36 \qq \rightarrow \qq x = 6 \\ 
            %        R_2: \qq  4y=36-4x=12 \qq \rightarrow y = 3 \\ 
            %    \end{matrix}
        %    \end{align*}
           \pregunta{Cómo se encuentra $\displaystyle z$ } 
           \begin{align*}
               z = 2(6)+3-18=-3 \\ 
               \text{ Punto crítico: } (6,3,-3) \\ 
               \text{ Prueba de la segunda derivada } \qq D(x,y) = \begin{vmatrix}
                   W_{xx} & w_{xy}  \\ 
                   W_{} 
               \end{vmatrix}
           \end{align*}
           
           Método 2: multiplicadores de Lagrange
           \begin{itemize}
               \item La grangiano $\displaystyle F = w + \lambda(c-g)$ 
           \end{itemize}
           \begin{align*}
                F(x,y,z,\lambda)=x^2+y^2+z^2+\lambda(18-2x-y+z) \\ 
                F_x = 2x+2\lambda = 0 \qq \implies \qq x = \lambda =6\\ 
                F_y = 2y-\lambda = 0 \qq \implies \qq y=\lambda/2 =3\\ 
                F_z=2z+\lambda = 0 \qq \implies \qq z = -\lambda/2 = -3 \\ 
                F_\lambda = 18-2x-y+z=0 \qq \implies \qq 2x+y-z = 18 \\ 
           \end{align*}
           Sustituya $\displaystyle x,y$ \& $\displaystyle z$  en la restricción.
           \begin{align*}
               2\lambda+\frac{\lambda }{2} + \frac{\lambda}{2} = 3\lambda = 18 \qq \implies \qq \lambda=6 \\ 
           \end{align*}
        %    El punto crítico es $\displaystyle (6,3,-3)$ $\displaystyle \lambda=6$ 
        \end{center}
    
    \item Una caja sin tapa tiene un volúmen de 32,000 $\displaystyle cm^3$ . Encuentre las dimensiones de la caja que minimizan el costo.
    \begin{tikzpicture}[node distance = 2cm, auto]
        \node[] at (4,-1) (1) {};
        \node[] at (-4,-1) (1) {};
        \node[] at (4,0) (1) {};
        \node[] at (4,0) (1) {};
    \end{tikzpicture}
    \begin{center}
       \begin{align*}
           \text{ Volúmen } \qq V=xyz = 32,000 \\ 
            \text{ Área Sup. } \qq 
            A=2zy+2zx+yx \\ 
            F = A+\lambda (c-v) = 2zy+2zx+yx+\lambda (32,000-xyz) \\ 
            F_x = 2z+y-\lambda yz = 0 \qq \implies \qq \lambda yz = y+2z \qq (1) \\ 
            F_y = 2z+x-\lambda xz= 0 \qq \implies \qq \lambda xz = x+2z \qq (2) \\ 
            F_z = 2y+2x-\lambda xy = 0 \qq \implies \qq \lambda \lambda xy = 2x+2y \qq (3)\\ 
            F_\lambda = 32,000-xyz = 0 \qq \implies \qq xyz=32,000 \qq (4) \\ 
       \end{align*}
       Dividimos entonces $\displaystyle \frac{(1)}{(2)} $:
        \begin{align*}
            \frac{(1)}{(2)}: \qquad  \frac{y}{x} = \frac{y+2z}{x+2z}  \\ 
            \cancel{yx}+2zy=\cancel{xy}+2zx \qq \implies \qq y = \frac{2zx}{2z} = \frac{x}{2} \\ 
            \therefore \qq x = y \\ 
        \end{align*}
        Dividimos también $\displaystyle \frac{(1)}{(3)} $ :
        \begin{align*}
            \frac{(x)}{(3)}: \qquad \frac{z}{x} = \frac{y+2z}{2x+2y} \\ 
            \cancel{2xz} + 2yz=xy+\cancel{2zx} \\ 
            z = \frac{xy}{2y} = \frac{x}{2} 
            \therefore \qq  y = x, \qq z = x/2 \\ 
        \end{align*}
        Se sustituye en la restricción:
        \begin{align*}
            x\cdot x\cdot \frac{x}{2} = 32,000 \\ 
            x^3 = 64\cdot1000 \\ 
            x = \sqrt[3]{64}\cdot \sqrt[3]{1,000} = 4\cdot 10 = 19 \\  
        \end{align*}
        Punto crítico: $\displaystyle x=40$ , $\displaystyle y=40$ , $\displaystyle z=20$ \newline 
        Área mínima:
        \begin{align*}
            A = 2yz+2xz+xy \\ 
            A = 2(800) + 2(800) + 1,600 \\ 
            A = 3(1,600) = 4,800 cm^2 \\ 
        \end{align*}
    \end{center}
\end{enumerate}


%%%%%%%%%%%%%%%%%%%%%%%%%%%%%%%%%%%%%%%%%%%%%%%%%%%%%%%%%%%%%%%%%%%%%%%%%%%%%%%%%%%%%%%%%%
\subsection{Aplicaciónes a la economía y negocios}
\begin{enumerate}
    \item Para sustituir una orden de 100 unidades de un producto, la empresa desea distribuir la producción entre sus dos plantas. La función de costo total es:
    \[
      C(x,y) = 0.1x^2+7x+15y+1,000
    \]
    \begin{center}
        Donde $\displaystyle x$ es la planta 1 y $\displaystyle y$ es la planta 2. \newline 
        \pregunta{Cómo debe distribuirse la producción para minimizar los costos} C(0,0)=1,000.
        \begin{align*}
            \text{ Objetivo mínimizar C(x,y) } \qq \text{ Sujeta a } \qq x+y=100 \\ 
            \text{ Lagrange: } \qq F= C+\lambda(100-x-y) \\ 
            F=0.1x^2+7x+15y+1,000+100\lambda-\lambda x-\lambda y \\ 
            F_x = 0.2x+7-\lambda = 0 \qq \implies \qq 0.2x=\lambda - 7 = 8 \qq \implies \qq x = 40 \\ 
            F_y = 15 - \lambda = 0 \qq \implies \qq \lambda=15 \\ 
            F_\lambda = 100-x-y = 0 \qq \implies \qq y = 100-x = 60 \\ 
        \end{align*}
        Punto crítico en $\displaystyle (40,60)$ $\displaystyle x=15$  
        \begin{align*}
          \text{ Costo mínimo }\qq C(x,y) = 0.1(1600) +280 + 900 + 1,000 \\ 
          C(x,y) = 2,340 \\ 
        \end{align*}        
    \end{center}
    
    \item Una empresa tiene la función de producción 
    \[
        Q(C,K) = 12L+20K-L^2-2K^2   
    \] 
    La empresa tiene un presupuesto de \$88 mil para contratar trabajadores y maquinaria. Cada trabajadory cada máquina tienen un costo de \$5 y \$8 mil, resp. \newline Encuentre la producción máxima.
    \begin{center}
        \begin{itemize}[label=\#]
            \item La restricción presupuestaria es la tangente a la curva de nivel en ese punto.
        \end{itemize}
       \begin{align*}
            \text{ Restricción:  } \qq 4L +8K = 88 \\ 
            \text{ Maximizar Q } \qq F(L,K,\lambda) \\ 
            F(L,K,\lambda) = 12L+20K-L^2-2K^2+\lambda(22-L-2K) \\ 
            F_L = 12-2L-\lambda = 0 \qq \implies \qq 2L = 12-\lambda \qq \implies \qq L = 6 - \lambda/2 \\ 
            F_K = 20-4K-2\lambda = 0 \qq \implies \qq 4K = 20-2\lambda \qq \implies \qq K = 5 - \lambda/2 \\ 
            F_\lambda = 22-L-2\lambda = 0\\
            L+2K=22 \qquad 6-\frac{\lambda}{2} +10-\lambda = 22 \\ 
            -\frac{3\lambda}{2}  = 22-16 = 6 \\ 
            \lambda=-4 \qquad L  
       \end{align*}
    \end{center}
\end{enumerate}
