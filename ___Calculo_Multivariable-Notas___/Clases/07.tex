% \date{2020-Feb-06 10:23:27}
%%%%%%%%%%%%%%%%%%%%%%%%%%%%%%%%%%%%%%%%%%%%%%%%%%%%%%%%%%%%%%%%%%%%%%%%%%%%%%%%%%%%%%%%%%%%%%%%
\section{13.2 Cálculo con funciones vectoriales, pg.55}
\begin{itemize}
    \item Derivadas:
        \[
          \vec{r}\,'(t) \quad \text{  Respecto a t  }
        \]
    
    \item Integrales:
        \[
          \int_{}^{}\vec{r}\,'(t)dt \quad \text{  Respecto a t  }
        \]
\end{itemize}


%%%%%%%%%%%%%%%%%%%%%%%%%%%%%%%%%%%%%%%%%%%%%%%%%%%%%%%%%%%%%%%%%%%%%%%%%%%%%%%%%%%%%%%%%%%%%%%%
\subsection{Derivadas}
\begin{itemize}
    \item \[
      \vec{r}\,'(t) = \lim_{h \to 0} \frac{r(t+h)-r(t)}{h} 
    \]
    
    \item Como la función $\vec{r}(t)$ está definida por tres funciones componentes se puede hacer: 
        \[
          \vec{r}\,'(t) = \lim_{h \to 0} \left\langle 
          \underbrace{\lim_{h \to 0} \frac{f(t+h)-f(t)}{h}}_{f'(t)}, 
          \underbrace{\lim_{h \to 0} \frac{g(t+h)-g(t)}{h}, }_{g'(t)}
          \underbrace{\lim_{h \to 0} \frac{h(t+h)-h(t)}{h}  }_{h('t)}
          \right\rangle 
        \]
    
    \item Derivada entonces es : 
        \[
          \vec{r}\,'(t) = \left\langle f'(t),g'(t),h'(t) \right\rangle 
        \]
\end{itemize}


%%%%%%%%%%%%%%%%%%%%%%%%%%%%%%%%%%%%%%%%%%%%%%%%%%%%%%%%%%%%%%%%%%%%%%%%%%%%%%%%%%%%%%%%%%%%%%%%
\subsection{Integrales}
\begin{itemize}
    \item Integral:
        \[
          \int_{}^{}\vec{r}(t)dt = = \int_{}^{}(f \hat{i} + g \hat{j} + h \hat{k} )dt
        \]
        \[
          \hat{i} \int_{}^{}fdt + \hat{j} \int_{}^{}gdt + \hat{k} \int_{}^{}hdt 
        \]
        Integrar la función componente.
\end{itemize}


%%%%%%%%%%%%%%%%%%%%%%%%%%%%%%%%%%%%%%%%%%%%%%%%%%%%%%%%%%%%%%%%%%%%%%%%%%%%%%%%%%%%%%%%%%%%%%%%
\section{Ejercicios}
\begin{enumerate}
    \item Encuentre la 1era y segunda derivada de las siguientes funciones:
        \begin{center}
            \begin{align*}
                \vec{r}(t) = \left\langle \sin(4t),t^2,ln(\sin(t)) \right\rangle \\ 
                \vec{r}\,'(t) = \left\langle 4\cos(4t),2t,\frac{\cos(t)}{\sin(t)}  \right\rangle \\ 
                \vec{r}\,'(t) = \left\langle 4\cos(4t),2t,\cot(t) \right\rangle \\ 
            \end{align*}
            \begin{align*}
                \vec{r}\,''(t) = \left\langle f''(t),g''(t),h''(t) \right\rangle \\ 
                \therefore \quad \vec{r}\,''(t) = \left\langle -16\sin(4t),2,-\csc^2(t) \right\rangle \\ 
            \end{align*}
        \end{center}
    
    \item Derive: $\vec{s}(t) = \hat{i} \tan(4t) + hat{j}ln(4t+1) + \hat{k} (5-2t)^{\frac{1}{2} }$
        \begin{center}
            \begin{align*}
                \vec{s}\,'(t) = 4 \hat{i} (\sec(4t))^2 + \hat{j} 4(4t+1)^{-1} -\hat{k} (5-2t)^{-\frac{1}{2} } \\ 
                \vec{s}\,''(t) = 8 \hat{i} \times \sec(4t)\times  \sec(4t)\times  \tan (4t) \times 4  - 16 \hat{j} (4t-1)^{-2} - \frac{\hat{k}}{2} (5-2t)^{-\frac{3}{2} } \times (-2) \\  
                \vec{s}\,''(t) = 32 \hat{i} \times \sec^2(4t)\times  \tan (4t)  - 16 \hat{j} (4t-1)^{-2} - \frac{\hat{k}}{2} (5-2t)^{-\frac{3}{2} } \times (-2) \\ 
            \end{align*}
        \end{center}
\end{enumerate}


\section{Recordatorios \& rectas tangentes de funciones vectoriales}
\begin{itemize}
    \item \emph{\textbf{Recordar lo siguiente: }$f'(a)$ es igual a la pendiente de la drecta tangeente a $f(x)$ en $x=a$}.
    \item \emph{\textbf{Recordar lo siguiente: }La recta tangente}.
        \[
          L_1: \quad y = f(a)+f'(a)(x-a) \quad \quad \text{  Ec. Recta Tangente  }
        \]
    
    \item Con una función vectorial:
        \begin{center}
            \begin{align*}
                \vec{r}= \left\langle f,g,h \right\rangle , \quad \quad x = f(t), \, y=g(t), \, z=h(t) \\ 
                \text{  Hay ecuaciones paramétricas para cada variable:  } \\ 
                \vec{r}\,'(a) = \left\langle f'(a),g'(a),h'(a) \right\rangle \\ 
                \text{  Vector de pendientes de rectas tangentes a la curva  } \, \vec{r}(t).
            \end{align*}
        \end{center}
    
    \item La derivada de una función vectorial se le da elnombre de \textbf{``vector tangente''} $\vec{r}(t):\vec{r}\,'(a)$.
    \item Recta tangente: es ahora una función vectorial.
        \[
          \vec{r}(t) = \vec{r}(a) + \vec{r}\,'(a)t 
        \]
    
    \item Ecs. Paramétricas:
        \begin{center}
            \begin{align*}
                \begin{matrix}
                    x=f(a)+f'(a)t \\ 
                    y=g(a)+g'(a)t \\ 
                    z=h(a)+h'(t)t \\ 
                \end{matrix}
            \end{align*}
        \end{center}
    
    \item Vector tangente: $r'(a)$ en $t=a$
    \item Vector tangente unitario: $\frac{r'(a)}{\left| r'(a) \right| } = \vec{T}(a)$
\end{itemize}



%%%%%%%%%%%%%%%%%%%%%%%%%%%%%%%%%%%%%%%%%%%%%%%%%%%%%%%%%%%%%%%%%%%%%%%%%%%%%%%%%%%%%%%%%%%%%%%%
\section{Ejercicios}
\begin{itemize}
    \item Encuentre las ecs. paramétricas de la recta tangente a la curva : $s(t)=\left\langle 2\cos(t),2\sin(t),4\cos(2t) \right\rangle $ en el punto $(\sqrt{3},1,2)$:
        \begin{center}
            \begin{align*}
                \text{  Recta tangente:  } \quad \vec{r}_T(t) = \vec{r}(a)+t \vec{r}\,'(a) \\ 
                \vec{r}_T(a) = \left\langle \sqrt{3},1,2 \right\rangle \\ 
                \text{  Derivada: } \quad \vec{r}\,'(t) = \left\langle -2\sin(t),2\cos(t),-8\sin(2t) \right\rangle \\ 
                \text{  \textbf{Nos preguntamos:} ¿Cómo encuentro ``a'' ? igualamos   } \quad r(t) = \left\langle \sqrt{3},1,2 \right\rangle  \\ 
                \begin{matrix}
                    2\cos(t) = \sqrt[]{3} \implies \cos(t) = \frac{\sqrt{3}}{2} \implies t = \frac{\pi }{6}  \\ 
                    2\sin(t) = 1 \implies 2\sin(\frac{\pi }{6} ) = 2 \times \frac{1}{2} = 1\\ 
                    4 \cos(2t)=2 \implies 4\cos(\frac{\pi }{3} ) = 4 \times \frac{1}{2} = 2\\ 
                \end{matrix} \\ 
                \text{  Vector tangente:   } \quad \vec{r}\,'(\frac{\pi }{6} )= \left\langle -2\sin(\frac{\pi }{6}),2cos(\frac{\pi}{6} ), -8\sin(\frac{\pi }{3} )  \right\rangle \\ 
                \vec{r}_T(t)= \left\langle \sqrt{3},1,2 \right\rangle + t \left\langle -1,\sqrt{3},-4\sqrt{3} \right\rangle \\ 
                \therefore  \\ 
                \begin{matrix}
                    x= \sqrt{3}-1t \\ 
                    y=1+\sqrt{3}t \\ 
                    z=2-4\sqrt{3}t \\ 
                \end{matrix}\\ 
            \end{align*}
        \end{center}
\end{itemize}

















