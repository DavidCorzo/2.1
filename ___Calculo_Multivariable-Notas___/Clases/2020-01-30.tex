% \date{2020-Jan-30 10:22:58}
%%%%%%%%%%%%%%%%%%%%%%%%%%%%%%%%%%%%%%%%%%%%%%%%%%%%%%%%%%%%%%%%%%%%%%%%%%%%%%%%%%%%%%%%%%%%%%%%
\section{Resolución de corto}
\begin{itemize}
    \item Determine el área del triángulo entre los puntos P(), Q(), R():
        \begin{align*}
            \vec{a} = \vv{PQ} = \left\langle 4,3,-2 \right\rangle \\ 
            \vec{b} = \vv{PR} = \left\langle 5,5,1 \right\rangle \\ 
            \text{  Área  } = \frac{1}{2} \left| \vec{a} \times  \vec{b} \right| \\ 
            \begin{vmatrix}
                \hat{i} & \hat{j} & \hat{k} \\ 
                4 & 3 & -2 \\ 
                3 & 5 & 1 \\ 
            \end{vmatrix} = 13 \hat{i} - 14 \hat{j} + 5 \hat{k} \\  
            \text{  Área  } = \frac{1}{2} \sqrt[]{} \\ 
        \end{align*}

\end{itemize}
    



%%%%%%%%%%%%%%%%%%%%%%%%%%%%%%%%%%%%%%%%%%%%%%%%%%%%%%%%%%%%%%%%%%%%%%%%%%%%%%%%%%%%%%%%%%%%%%%%
%%%%%%%%%%%%%%%%%%%%%%%%%%%%%%%%%%%%%%%%%%%%%%%%%%%%%%%%%%%%%%%%%%%%%%%%%%%%%%%%%%%%%%%%%%%%%%%%
\section{Rectas y planos}
\begin{itemize}
    \item Ecs. Rectas: $\vec{r}= \vec{r_0} + t \vec{v}$ 
        \begin{align*}
            \text{  si  } a \neq b \neq c \neq 0    \quad \frac{x-x_0}{a} = \frac{y-y_0}{b} = \frac{z-z_0}{c}  \\ 
        \end{align*}
    
    \item Paramétricas:
        \begin{align*}
            x = x_0 +at \\ 
            y = y_0 +bt \\ 
            z = z_0 +ct \\ 
        \end{align*}
    
    \item Ecuación de plano:
        \begin{align*}
            \hat{n} = \cdot \vec{r}-\vec{r_0} \\ 
            a(x-x_0)+b(y-y_0) + c(z-z_0) = 0 \\ 
            \hat{n}= \vec{a} \times  \vec{b} \\ 
        \end{align*}
\end{itemize}

\subsection{Ejercicios}
\begin{enumerate}
    \item Considere los planos $x+y=0$ \& $x+2y+z=1$.
        \begin{enumerate}
            \item Determine si los planos son paralelos so no lo son encuentre el ángulo entr ellos:
                \begin{align*}
                    \hat{n_1} = \left\langle 1,1,0 \right\rangle \\ 
                    \hat{n_2} = \left\langle 1,2,1 \right\rangle \\ 
                    \therefore \text{  Los dos planos no son paralelos  } \\ 
                \end{align*}
                \begin{itemize}
                    \item El $\hat{n_1}$ \& $\hat{n_2}$ no son necesariamente ortogonales.
                \end{itemize}
                \begin{align*}
                    \cos \theta = \frac{\hat{n_1}\cdot\hat{n_2}}{\left| \hat{n_1} \right|\left| \hat{n_2} \right| } = \frac{3}{\sqrt[]{2}} \\ 
                    \cos \theta = \frac{3}{2\sqrt[]{3}}=\frac{\sqrt[]{3}}{2} \qquad \theta = \frac{\pi}{2} \\   
                \end{align*}
        \end{enumerate}
    
    \item Encuentre la ec. de la recta que interseca a ambos planos $x+y=0$ \& $x+2y+z=1$: 
        \begin{align*}
            r = \vec{r_0} + t \vec{v} \\ 
            \text{  Dos puntos sobre la recta  } \\ 
            \text{  Como la recta esta en ambos planos, se debe resolver el sig. sistema de ecuaciones  } \\ 
            x + y = 0 \implies x=-y \\ 
            x+2y+z=1 \implies y = z-1 \\ 
            \text{  z tiene cualquier valor, ahora encontrar escogiendo cualquier punto sobre la recta, en este caso 0   } \\ 
            \text{  Primer punto   } \quad z & = 0 \\
            y &= 1\\ 
            x & = -1 \\ 
            \therefore \left\langle -1,1,0 \right\rangle \\ 
            \text{  Segundo punto  } \quad z &= 1 \\ 
            y & = 0 \\ 
            x & = 0 \\ 
            \therefore \left\langle 0,0,1 \right\rangle \\ 
        \end{align*}
    
    \item Encuentre la ecuación de la recta que pasa por P(-1,1,0) y Q$\underbrace{(0,0,1)}_{r_0}$:
        \begin{align*}
            \vec{r_0} = \left\langle 0,0,1 \right\rangle \! \left\langle -1,1,0 \right\rangle \\ 
            \vec{v} = \vv{QP} 0 \left\langle -1,1,-1 \right\rangle \\ 
            \text{  Ecuaciones paramétricas de la recta:  } \\ 
            x = 0-t \quad y = 0 + t \quad z = 1-t \\ 
        \end{align*}
    
    \item Solución alterna:
        \begin{align*}
            x = -y \quad y = 1-z \quad \text{  Más incognitas que ecuaciones.  }\\
            x,y \; \text{  ó   }\; z \text{  \quad pueden tener cualquier valor  } \quad z=t \\ 
            \begin{matrix}
                x = -1 + t \\ 
                y = 1 - t \\ 
                t = t \\ 
            \end{matrix} \therefore v_2 = \left\langle 1,-1,1 \right\rangle  \quad \vec{r_0} = \left\langle -1,1-0 \right\rangle \\ 
        \end{align*}
    
    \item Solución geométrica:
        \begin{itemize}
            \item Encuentre un punto en ambos planos (0,0,1).
            \item L arecta está en el plano I, entonces la recta es perpendicular al vector normal del plano I.
            \item Está en el plano z, entonces también es perpendicular al segundo vector normal.
            \item $\therefore $ la recta es perpendicular a ambos $\hat{n_1}$ \& $\hat{n_2}$
                \begin{align*}
                    \vec{v} = \hat{n_1} \times \hat{n_2} = \begin{vmatrix}
                        \hat{i} & \hat{j} & \hat{k} \\ 
                        1 & 1 & 0 \\ 
                        1 & 2 & 1 \\ 
                    \end{vmatrix} = \hat{i} - \hat{j} + \hat{k} \\ 
                    \text{  Ecuación de la recta:  } \quad r = \left\langle 0,0,1 \right\rangle + t \left\langle 1,-1,0 \right\rangle \\ 
                \end{align*}
        \end{itemize}
    
    \item Ejercicio 3: Encuentre el punto en el que la línea recta $x=1+2t$, $y=4t$, $z=5t$ interseca al plano. $x-y+2z=17$.
        \begin{align*}
            \begin{matrix}
                x = 1+2t \\ 
                y = 4t \\ 
                z= 5t \\ 
            \end{matrix} \\ 
            \text{  Plano  } \\ 
            x-y+2z = 17 \quad 1+2t-4t+10t = 17 \\ 
            8t = 16 \implies \therefore  t = 2 \\ 
        \end{align*}
        El punto de intersección es (5,8,10).
    
    \item Ejercicio 4: Encuentre una ec. del plano que contiene  la recta $x=1+t$, $y=2-t$, $z=4-3t$ y es paralela a plano $5x+2y+z=1$.
        \begin{itemize}
            \item Cualquier punto sobre la recta que también esté sobre el plano, t= 0. 
        \end{itemize}
        \begin{align*}
            \text{  Evaluemos en t=0  } \quad x=1, \, y=2, \, z=4\\
            \vec{r_0}= \left\langle 1,2,4 \right\rangle  \\ 
        \end{align*}
        \begin{itemize}
            \item \textbf{Nos preguntamos:} ¿Cómo se encuentra $\hat{n}$?
            \item El vectos de dirección de la recta $v=\left\langle 1,-1,-2 \right\rangle$ es paralelo al plano.
            \item Como es paralelo al seguno plano, entonces tiene que ser perpendicular $\hat{n_2} = \left\langle 5,2,1 \right\rangle$
            \item Lo que ocurre entonces es:
        \end{itemize}
            \begin{align*}
                \vec{r_0} = \left\langle 1,2,4 \right\rangle \quad \hat{n}= \left\langle 5,2,1 \right\rangle \\ 
                \text{  Ec. Plano:  } \, \implies \, 5(x-1) + 2(y-2)+1(z-4)=0 \\ 
            \end{align*}
    \item Ejercicio 5: Encuentre los números directores para la recta de intersección entre los planos $x+y+z=1$ \& $x+2y+3z=1$.
            \begin{itemize}
                \item \emph{\textbf{Definición de ``numeros directores":} a,b,c del vector de dirección $\left\langle a,b,c \right\rangle $}
                \item La recta es ortogonal a ambos vectores normales:
            \end{itemize}
            \begin{align*}
                \hat{n_1} = \left\langle 1,1,1 \right\rangle \quad text{ \& } \hat{n_2} = \left\langle 1,2,3 \right\rangle \quad \text{  de ambos planos  }\\
                \vec{v} = \hat{n_1} \times \hat{n_2} = \begin{vmatrix}
                    \hat{i} & \hat{j} & \hat{k} \\
                    1 & 1 & 1 \\ 
                    1 & 2 & 1 \\ 
                \end{vmatrix}   = \hat{i} -2\hat{j} +\hat{k} \\ 
                \text{  Los números directores:   } \quad a=1, b=2, c= 1 \\ 
            \end{align*}
    
    \item Ejercicio 6: Encuentre las ecs. aparamétricas de la recta que pasa por el punto (0,1,2), que es paralelo al plano $x+y+z=2$ y es perpendicular a la recta $r = \left\langle -2t,0,3t \right\rangle $.
            \begin{align*}
               L_1 r= \vec{r_0}+t \vec{v} \quad r_0 = \left\langle 0,1,2 \right\rangle 
            \end{align*}
            \begin{itemize}
                
                \item Aclaraciones: $L_1$ es la incógnita que tenemos que encontrar.
                \item \textbf{Nos preguntamos:} ¿Cómo se encuentra $r$?
                
                \item Plano I: $\hat{n} = \left\langle 1,1,1 \right\rangle $ es perpendicular al plano, es paralelo a $L_1$.
                \item Recta II: $\hat{v_2}= \left\langle -2,0,3 \right\rangle $ es perpendicular a $L_1$
                \item La recta es perpendiculae a $\hat{n}$ y a $\vec{v_2}$
            \end{itemize}
            \begin{align*}
                v=\hat{n} \times \vec{v_2} = \begin{vmatrix}
                    \hat{i} & \hat{j} & \hat{k} \\ 
                    1 & 1 & 1 \\ 
                    -2 & 0 & 3 \\ 
                \end{vmatrix} = 3 \hat{i} - 5 \hat{j} + 2 \hat{k} \\ 
                r_0 = \left\langle 0,1,2  \right\rangle \\ 
                v = \hat{v_2} \times \hat{n} \quad \text{  Ecuaciones paramétricas:   } \\ 
                \begin{matrix}
                    x=0-3t \\ 
                    y = 1-5t \\ 
                    z = 2 +2t \\ 
                \end{matrix} \\ 
            \end{align*}
            
\end{enumerate}

