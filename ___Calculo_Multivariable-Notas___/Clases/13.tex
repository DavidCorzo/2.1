\documentclass{article}

\usepackage{davidcorzo}
\preliminaries{}{}{}

\begin{document}
\section{ 14.5 Regla de la cadena}
\begin{itemize}
    \item Explicación: 
    \begin{center}
       \begin{align*}
           y &= f(g(t)) \qqq y =f(x) \qq x=g(t) \\ 
           y &\implies x \implies t \qquad \dervpar{y}{t} = \dervpar{y}{x} \cdot \dervpar{x}{t} \\   
           \text{ Caso 1: } \; z&= f(x,y) \qq x = g(t) \qq y=h(t) \\ 
       \end{align*}
    \end{center}
    
    \item  Caso 1: \textbf{¿}Cómo se encuentra $\dervpar{z}{t}$ \textbf{?}:
        \begin{center}
           \begin{align*}
            z = f(x(t),y(t)) \qq \qq \\ 
           \end{align*}
           \begin{itemize}
               \item Variable independiente $z$ 
               \item Variable intermedia $x,y$
               \item Variable independiente $t$
           \end{itemize}
           \[
             \dervpar{z}{t} = \dervpar{z}{x} \dervpar{x}{t} + \dervpar{z}{y} \dervpar{y}{t} 
           \]
        \end{center}
    
    \item Caso 2: $z = f(x,y)$, $x = g(s,t)$, $y = h(s,t)$: 
        \begin{center}
           \begin{align*}
               \dervpar{z}{s} &= \dervpar{z}{x} \dervpar{x}{s} + \dervpar{z}{y} \dervpar{y}{s} \\ 
               \dervpar{z}{t} &= \dervpar{z}{x} \dervpar{x}{t} + \dervpar{z}{y} \dervpar{y}{t} \\ 
           \end{align*}
        \end{center}
\end{itemize}


%----------------------------------------------------------------------------------------
\subsection{Ejercicios}
\begin{enumerate}
    \item Suponga que el costo de producir $x$ unidades de A y de $y$ unidades de B es:
        \[
          C(x,y) = \p{3x^2+y^3+4}^{\frac{1}{3} }
        \]
        Las funciones de producción para cada producto es:
        \[
          x = 10KL \qquad \qquad y = 5k^2+4L
        \]
        Encuentre la razón de cambio de $C$ respecto al capital y al trabajo.
        \begin{center}
           \begin{align*}
               \dervpar{C}{K} &= \dervpar{C}{x} \dervpar{x}{K} + \dervpar{C}{y}\dervpar{y}{K}  \\ 
               \dervpar{C}{L} & \dervpar{C}{x} \dervpar{X}{L} + \dervpar{C}{y} \dervpar{y}{L} \\ 
               \dervpar{C}{K} &= \frac{1}{3} 6x \p{3x^2+y^3+4}^{-\frac{2}{3}} \cdot 10 + \frac{1}{3} \frac{3y^2}{\p{3x^2+y^3+4}^\frac{2}{3}} \\ 
               \dervpar{C}{K} &= \frac{2x}{\p{3x^2+y^3+4}^{\frac{2}{4}}} \cdot 10K + \frac{y^2}{\p{3x^2+y^2+4}} \p{4} \\  
           \end{align*}
        \end{center}
    
    \item Suponga que $z=f(u,v,w)$ y que $u,v,w$ son funciones de $t$. Encuentre $\dervpar{z}{t}$:
        % \begin{center}
            % \begin{tikzpicture}[node distance = 2cm, auto]
            %     \node (1) {$z$}; 
            %     \node [draw, below of=1, left of=1](2) {$u$}; 
            %     \node [draw, below of=1](3) {$v$}; 
            %     \node [draw, below of=1, right of=1](4) {$w$}; 
            %     \node [draw, below of=2](5) {$t$}; 
            %     \node [draw, below of=3](6) {$t$}; 
            %     \node [draw, below of=4](7) {$t$}; 
            % \end{tikzpicture}
            % \begin{tikzpicture}[level distance=1.5cm,
            %     level 1/.style={sibling distance=3cm},
            %     level 2/.style={sibling distance=1.5cm}]
            %     \node {$z$}
            %       child {node {left}
            %         child {node {lleft}}
            %         child {node {rleft}}
            %       }
            %       child {node {right}
            %       child {node {lright}}
            %         child {node {rright}}
            %       };
            %   \end{tikzpicture}
        % \end{center}
\end{enumerate}

\subsection{Ejercicios varios}
\begin{enumerate}
    \item Encuentre las derivadas parciales indicadas:
        \begin{center}
        \begin{align*}
            % w &= \sqrt{x^2+y^2} \\ 
            % x &= p^2+q^3+r-1 \\ 
            % y &= \ln(p)+e^{q}+e^{\ln (r)} \\ 
            % \dervpar{w}{p} & \evaluate{(p=1,q=0,r=3)}{}  \\

            \dervpar{w}{p} &= \dervpar{w}{x} \dervpar{x}{p} + \dervpar{w}{y} \dervpar{y}{p} \\ 
            \dervpar{w}{p} &= \frac{x}{\p{x^2+y^2}^{\frac{1}{2} }}\cdot 2p + \frac{y}{\p{x^2+y^2}^{\frac{1}{2}}} \frac{1}{p}   \\ 
            x(1,0,3) &= 1^2-0^3+3-1=3 \\ 
            y (1,0,3) &= \ln(1)+e^0+e^{\ln(3)}=0+1+3=4 \\ 
            \\ 
            \dervpar{w}{p} \evaluate{(1,0,3)}{} &= \frac{3}{\sqrt{9+16}} \cdot 2 + \frac{4}{5} \cdot \frac{1}{1} = \frac{6}{5} + \frac{4}{5} = 2 \\  
        \end{align*}
        \end{center}

    \item $h=4-t^2$, $t=2a+3b+4c$, $\dervpar{h}{b} \evaluate{(4,2,3)}{} $:
    \begin{center}
       \begin{align*}
           \dervpar{h}{b} &= -2 \p{2a+3b+4c}\cdot 3 \\ 
       \end{align*}
    \end{center}
    
    \item $w=\ln(x,y,z)$, $x=r^2-s^2$, $y=rs$, $z=r^2+s^2$:
        \begin{center}
           \begin{align*}
                \dervpar{w}{r} &= \dervpar{w}{x} \dervpar{x}{r} + \dervpar{w}{y} \dervpar{y}{r} + \dervpar{w}{z} \dervpar{z}{r} \\  
                w_x = \frac{yz}{xyz} &= \frac{1}{x} \\ 
                \dervpar{w}{r} = \frac{2r}{x} + \dervpar{s}{y} + \dervpar{2r}{z} \\ 
           \end{align*}
        \end{center}
\end{enumerate}



%%%%%%%%%%%%%%%%%%%%%%%%%%%%%%%%%%%%%%%%%%%%%%%%%%%%%%%%%%%%%%%%%%%%%%%%%%%%%%%%%%%%%%%%%%
\section{Derivación implícita, planos y rectas tangentes}
\begin{enumerate}I
    \item Encuentre las ecs. paramétricas de las rectas tangentes a $z=\sin (x) \tan (x)  $ en la dirección de $x$ \& $y$ en el punto $\p{\frac{\pi}{6} , \frac{\pi}{4} }$:
        \begin{center}
           \begin{align*}
               \text{ En la dirección de x:  } \; m_x &= z_x\p{\frac{\pi}{6},\frac{\pi}{4}  } \\ 
               \text{ de y:  }\; m_y &= z_y\p{\frac{\pi}{6} ,\frac{\pi}{4} } \\ 
               z_x = \cos \argp{x}  \tan \argp{y} & z_x\p{\frac{\pi}{6} , \frac{\pi}{4} } = \frac{\sqrt{3}}{2} \\
               z_x &= \cos  \argp{}  \\ 
           \end{align*}
        \end{center}
\end{enumerate}



\end{document}
