% \documentclass{article}
\title{Temporary}
\author{David Gabriel Corzo Mcmath}
\date{\today}
%%%%%%%%%%%%%%%%%%%%%%%%%%%%%%%%%%%%%%%%%%%%%%%%%%%%%%%%%%%%%%%%%%%%%%%%%%%%%%%%%%%%%%%%%%%%%%%%%%%%%%%%%%%%%%%%%%%%%%%%%%%%%%%%%%%%%%%%%%%%%%%
\usepackage[margin = 1in]{geometry}
\usepackage{graphicx}
\usepackage{fontenc}
\usepackage{pdfpages}
\usepackage[spanish]{babel}
\usepackage{amsmath}
\usepackage{amsthm}
\usepackage[utf8]{inputenc}
\usepackage{enumitem}
\usepackage{mathtools}
\usepackage{import}
\usepackage{xifthen}
\usepackage{pdfpages}
\usepackage{transparent}
\usepackage{color}
\usepackage{fancyhdr}
\usepackage{lipsum}
\usepackage{sectsty}
\usepackage{titlesec}
\usepackage{calc}
\usepackage{lmodern}
\usepackage{xpatch}
\usepackage{blindtext}
\usepackage{bookmark}
\usepackage{fancyhdr}
\usepackage{xcolor}
\usepackage{tikz}
\usepackage{blindtext}
\usepackage{hyperref}
\usepackage{listing}
\usepackage{spverbatim}
\usepackage{fancyvrb}
\usepackage{fvextra}
\usepackage{amssymb}
\usepackage{pifont}
\usepackage{longtable}
\usetikzlibrary{arrows,shapes}
%%%%%%%%%%%%%%%%%%%%%%%%%%%%%%%%%%%%%%%%%%%%%%%%%%%%%%%%%%%%%%%%%%%%%%%%%%%%%%%%%%%%%%%%%%%%%%%%%%%%%%%%%%%%%%%%%%%%%%%%%%%%%%%%%%%%%%%%%%%%%%%






% \begin{document}


\section{Resolución de corto}
\begin{enumerate}
    \item Analice la función $r=\left\langle 3e^{-t},ln(2t^2-1),\tan(2\pi ) \right\rangle $ en $t=1$:
        \begin{center}
           \begin{align*}
               \lim_{t \to 1} \vec{r}(t) = \left\langle \lim_{t \to 1} 3e^{-t},\lim_{t \to 1} ln(2t^2-1),\lim_{t \to 1} \tan(2\pi ) \right\rangle \\ 
               \vec{r} = \left\langle 3e^{-1},ln(1),\tan(2\pi) \right\rangle  = \left\langle 3e^{-1},0,0 \right\rangle \\
               \therefore \quad \text{  r es contínua en t=1  } \\   
           \end{align*}
           \begin{itemize}[label=\#]
               \item Si la pregunta hubiese sido en cuándo se indefine, se saca el dominio de cada función.
           \end{itemize}
        \end{center}
    
    \item Encuentre la ec. de la recta tatente a $r(t)= \left\langle te^{t-1},\frac{8}{\pi } \arctan(t), 2\ln(t) \right\rangle $ en $t=1$.
        \begin{center}
           \begin{align*}
               \vec{r}(0) = \left\langle 1\times e^0,\frac{8}{\pi } \arctan(1), 2ln(0) \right\rangle = \left\langle 1,2,0 \right\rangle \\ 
            %    \vec{r}\,'(t) = \left\langle e^{t-1} \right\rangle 
            \text{  Terminar de copiar  } \\ 
           \end{align*}
        \end{center}
\end{enumerate}


%%%%%%%%%%%%%%%%%%%%%%%%%%%%%%%%%%%%%%%%%%%%%%%%%%%%%%%%%%%%%%%%%%%%%%%%%%%%%%%%%%%%%%%%%%%%%%%%%%%
\section{14.1 Funciones de varias variables}
\begin{itemize}
    \item Cuando teníamos sólo una función de una variable no había tanta complicación, las gráficas eran curvas en el plano. Cuando empezaba y terminaba la curva en $x$ nos daba el dominio. Había una variable independiente $x$ y la variable dependiente $y$, los dominios eran intervalos, y cada $x$ sólo podía tener \textbf{un} sólo valor de $y$.
    \item En funciones de 2 variables se va a describir como:  
        \begin{center}
           \begin{align*}
                z = f(x,y) \quad &\text{  Dos variables independientes x,y  } \\ 
                & \text{  Variable dependiente z  } \\ 
           \end{align*}
        \end{center}
        \begin{itemize}
            \item Entonces $f$ es una regla que asigna a cada punto $(x,y)$ a lo sumo un valor de $z$.
        \end{itemize}
        \[
          f: \, \underbrace{\mathbb{R}^2}_{\text{  Dominio  }} \rightarrow \underbrace{\mathbb{R}}_{\text{  Rango  }}
        \]
        \begin{itemize}
            \item Estamos pasando de una región por medio de una función $z$ llego a tener $f(x,y)$ en la dimensión correspondiente.
            \item Los dominios en estas funciones se vuelven superficies.
        \end{itemize}
    
    \item El dominio de una función de dos variables: un conjunto que consiste de todos los puntos o pares ordenados $(x,y)$ para los cuales $f(x,y)$ para los cuales $f(x,y)$ está definida.
        \[
          \mathbb{D}: \quad \text{En una dimensión: Todos los números x para los cuales f(x) está definida  }
        \]
        \begin{itemize}
            \item Evite la división por cero.
            \item Raíces pares de números negativos.
            \item Logaritmos de números negativos o cero.
        \end{itemize}
    
    \item El dominio de $f$ en una función de dos variables es una región:
        \begin{itemize}
            \item Las regiones que estén sombreadas son partes del dominio.
        \end{itemize}
        \begin{itemize}[label=\#]
            \item Para graficar funciones de dos variables son más fáciles de graficar que de una sola variable.
        \end{itemize}
\end{itemize}


%%%%%%%%%%%%%%%%%%%%%%%%%%%%%%%%%%%%%%%%%%%%%%%%%%%%%%%%%%%%%%%%%%%%%%%%%%%%%%%%%%%%%%%%%%%%%%%%%%%
\section{Ejercicios}
Encuentre y bosqueje el dominio de las sigs. funciones. \newline 
Sombree la región dque es parte del $\mathbb{D}$ y utilice líneas discontínuas para denotar a curvas que no son parte del $\mathbb{D}$
\begin{enumerate}
    \item $c(x,y)=10x+20y$ : 
        \begin{center}
           \begin{align*}
               \text{  Nunca se indefine.  } \\ 
               \mathbb{D}: \underbrace{(-\infty,\infty )}_{x} \underbrace{\times}_{\text{  Producto cartesiano  }} \underbrace{(-\infty ,\infty )}_{y} = \mathbb{R}^2 \\ 
           \end{align*}
           \begin{itemize}[label=\#]
               \item Producto cartesiano denota \textbf{todas las combinaciones posibles en un conjunto de $n$ elementos}.
               \item Explicaciones de productos cartesianos:
                \[
                  \mathbb{R} \cup  \mathbb{R} = \mathbb{R} \quad \mathbb{R} \times \mathbb{R} = \mathbb{R}^2     
                \]
                
                \item Definición de producto cartesiano:
                    \[
                      x \times y = \{(x,y) \; \text{  tal que  } \; x \in X, \, y \in Y \}
                    \]
                
                \item Producto cartesiano vs. unión:
                    \begin{center}
                       \begin{align*}
                           x \times y = \{(1,1),(1,2),(1,3),(2,1),(2,2),(2,3),(3,1),(3,2),(3,3)\} \\ 
                           x \cup y = \{(1),(2),(3)\}
                       \end{align*}
                    \end{center}
           \end{itemize}
        \end{center}
    
    \item $z = \frac{8}{x^2-y^2} $:
        \begin{center}
           \begin{align*}
               \text{  Definida si   } \; x^2 \neq y^2 \\ 
               \mathbb{R}^2 - \{x^2\neq y^2\} \\ 
               y \neq \sqrt{x^2} \\ 
               y \neq \pm x \\ 
           \end{align*}
        \end{center}
    
    \item $R(x,y)= \sqrt{9-x^2-y^2}$ : 
        \begin{center}
           \begin{align*}
               \text{  Definida  } \; \begin{matrix}
                   9-x^2-y^2 \geq 0 \\ 
                   9 \geq x^2 + y^2 \\ 
                   \mathbb{D}: x^2+y^2 \neq 9 \\ 
               \end{matrix} \\ 
               \text{  Círculo de radio 3 centrado en el orígen  } \\ 
               \mathbb{D} = \{(x,y) \; \text{  tal que  } \; x^2+y^2 \leq 9  \} \\ 
           \end{align*}
        \end{center}
    
    \item $Q(x,y)=\frac{1}{\sqrt{x^2+y^2-9}} $ : 
        \begin{center}
           \begin{align*}
               \mathbb{D}: \; \begin{matrix}
                   x^2+y^2 > 0 \\ 
                   x^2+y^2 > 9 \\ 
               \end{matrix} \\ 
               \therefore \text{  Afuera del círculo o disco de radio 3  } \\ 
           \end{align*}
        \end{center}
    
    \item $ z = \frac{(x+4)}{(y-2)(x-4)(y+2)} $ : 
        \begin{center}
           \begin{align*}
               \text{  Definida si  }: \quad y \neq \pm 2, \; x\neq 4 \\ 
               \mathbb{D}: \quad \mathbb{R}^2-  \{y\neq \pm 2, x\neq 4\} \\ 
           \end{align*}
        \end{center}
    
    \item $h(x,y)=\ln(2-yx)$ : 
        \begin{center}
           \begin{align*}
               \text{  Definida si  }: \quad \begin{matrix}
                   2-yx &> 0 \\ 
                   2 &> yx \\ 
                   y &< \frac{2}{x} \\ 
               \end{matrix} \\ 
               \therefore \mathbb{D}: \; y < \frac{2}{x} \\ 
           \end{align*}
        \end{center}
\end{enumerate}


%%%%%%%%%%%%%%%%%%%%%%%%%%%%%%%%%%%%%%%%%%%%%%%%%%%%%%%%%%%%%%%%%%%%%%%%%%%%%%%%%%%%%%%%%%%%%%%%%%%
\subsection{Gráfica de $z=f(x,y)$}
\begin{itemize}
    \item Gráfica de $z=f(x,y)$: Son superficies y consisten de todas las \emph{triplas} ordenadas $(x,y,z)$ donde $z$.
\end{itemize}



%%%%%%%%%%%%%%%%%%%%%%%%%%%%%%%%%%%%%%%%%%%%%%%%%%%%%%%%%%%%%%%%%%%%%%%%%%%%%%%%%%%%%%%%%%%%%%%%%%%
\section{Curva de nivel o traza horizontal}
\begin{itemize}
    \item En $f(x,y)=k$ k es una constante, rebane la superficie con los planos horizontales $z=k$ y grafique cada curva en el plano.
\end{itemize}



















    
% \end{document}
