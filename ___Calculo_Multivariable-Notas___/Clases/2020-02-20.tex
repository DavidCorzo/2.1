% \documentclass{article}
\title{Temporary}
\author{David Gabriel Corzo Mcmath}
\date{\today}

\usepackage{amsmath}

% \usepackage{davidcorzo}

\newcommand{\derpar}[2]{
    \ensuremath{
        \frac{\partial {#1}}{{\partial {#2}}
    }
}






% \begin{document}
    


\section{14.3 Derivadas parciales}
\begin{itemize}
    \item Derivada en una dimensión:
        \[
          \lim_{h \to 0} \frac{f(x+h)-f(x)}{h} = f'(x)  
        \]
    
    \item En una función con dos variables independientes:
        \begin{align*}
            f(x,y) = \begin{rcases}
                \begin{matrix}
                    f_x(x,y) \\ 
                    f_y(x,y) \\ 
                \end{matrix}  
            \end{rcases} \text{  Derivadas parciales  } \\
        \end{align*}
    
    \item Al derivarse parcialmente respecto a una variable, la otra se mantiene constante:
        \begin{align*}
            f_x(x,y) =& \lim_{h \to 0} \frac{f(x+h,y)-f(x,y)}{h} \quad \quad \text{  \# y se mantiene constante  }\\  
            f_y(x,y) =& \lim_{h \to 0} \frac{f(x,y+h)-f(x,y)}{h} \quad \quad \text{ \# x se mantiene constante  }\\  
        \end{align*}
    
    \item Se pueden utilizar todas las reglas de derivación para funciones de 1 variable:
        \begin{itemize}
            \item Suma 
            \item Producto 
            \item Cociente 
            \item Cadena 
        \end{itemize}
    
    \item $1^{\text{  eras  }}$ derivadas parciales de $f(x,y)$: encuentre todas las derivadas parciales posibles de $f_x$ \& $f_y$
        \begin{itemize}
            \item Notación:
                \[
                  f_x=\frac{\delta f}{\delta x} = \frac{\delta z}{\delta x}  
                \]
                \[
                    f_x=\frac{\delta f}{\delta y} = \frac{\delta z}{\delta y}  
                \]
            
            \item Evite $f'(x,y)$ para evitar ambigüedad.
        \end{itemize}
\end{itemize}



%%%%%%%%%%%%%%%%%%%%%%%%%%%%%%%%%%%%%%%%%%%%%%%%%%%%%%%%%%%%%%%%%%%%%%%%%%%%%%%%%%%%%%%%%%
\subsection{Ejercicios}
Encuentre las derivadas parciales de las siguientes funciones.
\begin{enumerate}
    \item $f(x,y)=2x^2+3xy\,$ : \emph{\textbf{Recordar lo siguiente: }$f_x(x,y)$ \& $f_y(x,y)$} 
        \begin{center}
           \begin{align*}               
                f_x = 4x+3y \quad \quad f_y=0+3x \\ 
           \end{align*}
        \end{center}
    
    \item $g(x,y)=y(x^2+1)^3+x^2(y^4-4)^4+5x^2y^3\,$ :
        \begin{center}
           \begin{align*}
                g_x&=3y(x^2+1)^22x+2x(y^4-4)^4+10xy^3 \\ 
                g_y&=1 \cdot (x^2+1)^3 + 16y^3x^2(y^4-4)^3+15x^2y^2 \\ 
           \end{align*}
        \end{center}        
    
    \item $h(s,t)=(s^2+10t)^2\cdot(t^4+s^3)^3\,$: \# Regla del producto y de la cadena.
        \begin{center}
           \begin{align*}
               h_s &= 4s(s^2+10t)^1\cdot(t^4+s^3)^3+3\cdot3s^2(s^2+10t)^2\cdot(t^4+s^3)^2 \\ 
               h_t&= 20(s^2+10t)^1\cdot(t^4+s^3)^3+12t^3(s^2+10t)^2,.
               (t^4+s^3)^2 \\  
           \end{align*}
        \end{center}
\end{enumerate}
%----------------------------------------------------------------------------------------
\begin{itemize}[label=\#]
    \item Evalúe la derivada en punto $(a,b)$:            
        \[
            f_x(a,b)=\frac{\delta f}{\delta x} \Big|_{(a,b)}^{} 
        \]
\end{itemize}

%----------------------------------------------------------------------------------------
\begin{enumerate}
    \item $w(r,\theta)=r^2\sin(2\theta)+e^{\pi r - \theta}\,$, encuentre $\frac{\delta w}{\delta \theta} \Big|_{(2,\pi)}^{}$
        \begin{center}
           \begin{align*}
               \frac{\delta w}{\delta \theta } &= 2r^2\cos(2\theta)-e^{\pi r -\theta} \\ 
               \frac{\delta w}{\delta \theta } \Big|_{(2,\pi)}^{} &= w_\theta(2,\pi) = 2\cdot 4\cos(2\pi)-e^{2\pi-\pi} \\ 
               &= 8-e^\pi
           \end{align*}
        \end{center}
\end{enumerate}


%%%%%%%%%%%%%%%%%%%%%%%%%%%%%%%%%%%%%%%%%%%%%%%%%%%%%%%%%%%%%%%%%%%%%%%%%%%%%%%%%%%%%%%%%%
\section{Derivadas parciales par funciones de 2 o más variables}
\begin{itemize}
    \item Se deriva respecto a una variable y el resto se mantienen constantes.
        \[
            w = f(x,y,z) 
        \]
        3 $1^{\text{  eras  }}$ derivadas parciales: $f_x,f_y,f_z$. 
        \[
          u = f(x_1,x_2,\dots,x_n)
        \]
        n derivadas parciales: 
        \[
          \frac{\delta u }{\delta x} , \dots \frac{\delta u}{\delta x_n} 
        \]
\end{itemize}


%----------------------------------------------------------------------------------------
\subsection{Ejercicio}
Encuentre todas las primeras derivadas pariales de las sigentes funciones:
\begin{itemize}
    \item $f(x,y,z)=\sqrt[4]{x^4+8xz+2y^2}$
        \begin{center}
           \begin{align*}
               f_x &= \frac{1}{4} (x^4+8xz+2y^2)^{-\frac{3}{4} }\cdot(4x^3+8z+0) \\ 
               f_y &= \frac{1}{4} (x^4+8xz+2y^2)^{-\frac{3}{4} }\cdot(4y) \\ 
               f_z &= \frac{1}{4} (x^4+8xz+2y^2)^{-\frac{3}{4} }\cdot(8x) \\ 
           \end{align*}
        \end{center}
    
    \item $p(r,\theta,\phi)=r\cdot \tan (\phi^2-4\theta)\,$: 
        \begin{center}
           \begin{align*}
               p_r &= \tan (\phi^2-4^\theta) \\ 
               p_{\theta} &= -4r \sec ^2 (\phi^2-4\theta) \\ 
               p_{\phi} &= 2\phi r \sec ^2(\phi^2-4\theta) \\ 
           \end{align*}
           \begin{itemize}[label=\#]
               \item Funciones vectoriales 1 variable: $\vec{r}\,' (t), \dots $
           \end{itemize}
        \end{center}
\end{itemize}


%%%%%%%%%%%%%%%%%%%%%%%%%%%%%%%%%%%%%%%%%%%%%%%%%%%%%%%%%%%%%%%%%%%%%%%%%%%%%%%%%%%%%%%%%%
\section{Derivadas parciales de orden superior (pág. 100)}
\begin{itemize}
    \item Orden superior: Segundas, terceras, cuartas, etc. derivadas.
    \item Como $f_x(x,y)$ \& $f_y(x,y)$ son también funciones en dos variables, pueden tener derivadas parciales.
        \begin{tikzpicture}[node distance = 2.5cm, auto]
            \node [block] (1) {$f_x$};
            \node [block,right of=1] (2) {$f_{xx}$}; 
            \node [block,right of=1,below of=1] (3) {$f_{xy}$};
            \path [line] (1) -- (2);
            \path [line] (1) -- (3); 
        \end{tikzpicture}
        \begin{tikzpicture}[node distance = 2.5cm, auto]
            \node [block] (1) {$f_y$};
            \node [block,right of=1] (2) {$f_{yy}$}; 
            \node [block,right of=1,below of=1] (3) {$f_{yx}$};
            \path [line] (1) -- (2);
            \path [line] (1) -- (3); 
        \end{tikzpicture}
    
    \item Las segundas derivadas parciales, éstas también tienen sus derivadas parciales, terceras derivadas parciales.
        \begin{center}
           \begin{tabular}{  p{1cm}  p{1cm}  p{1cm}  p{1cm}  }
                    $f_{xxx}$ & $f_{xxy}$ & $f_{yyy}$ & $f_{yxy}$  \\
                    $f_{xxy}$ & $f_{xyx}$ & $f_{yyx}$ & $f_{yxx}$ \\ 
           \end{tabular}
        \end{center}
    
    \item Las derivadas parciales cruzadas $f_{xy}$ \& $f_{yx}$ son iguales si la función es diferenciable.
        \[
          f_{xy} = f_{yx} \quad \quad f_{xyy} = f_{yyx}= f_{yxy}
        \]
    
    \item Notación delta:
        \begin{center}
           \begin{align*}
               f_{xx} &= \frac{\delta }{\delta x} \left(\frac{\delta f}{\delta x}\right) = \frac{\delta^2f}{\delta x^2}  \quad  \quad  f_{yy} = \frac{\delta^2f}{\delta y^2} \\ 
               f_{xy} &= \frac{\delta}{\delta y} \left(\frac{\delta f}{\delta x}\right) = \frac{\delta ^2 f}{\delta y \delta x}  \quad \quad f_{yx} = \frac{\delta^2f}{\delta y \delta y} \\ 
           \end{align*}
        \end{center}
\end{itemize}


%----------------------------------------------------------------------------------------
\subsection{Ejercicios}
Encuentre todas las 2das derivadas parciales:
\begin{enumerate}
    \item $f(x,y)=\sin (mx+ny)\quad m,n \in \mathbb{R}\,$:  
        \begin{align*}
                \text{  Primeras derivadas parciales  :} \\ 
                f_x =& m \cos (mx+ny) \\ 
                f_y &= n \cos (mx+ny ) \\ 
                \text{  Segundas derivadas parciales:   } \\ 
                f_{xx} &= -m^2 \sin (mx+ny ) \\ 
                f_{yy} &= -n ^2 \sin (mx+ny) \\ 
                \begin{rcases}
                    f_{xy} &= -mn \sin (mx+ny) \\ 
                    f_{yx} &= -mn \sin (mx+ny) \\ 
                \end{rcases} \text{  Iguales  } \\ 
        \end{align*}
    
    \item $z = \cos (2xy)\,$ : 
        \begin{center}
           \begin{align*}
               1^{\text{  eras  }}: \quad \frac{\delta z }{\delta x} &= -2 \sin (2xy) 
               ,\quad \frac{\delta z}{\delta y } = -2x \sin (2xy) 
               \\  
               2^{\text{  das  }}: \quad \frac{\delta ^2 z}{\delta x^2} &= -4 y^2 \cos (2xy) ,\quad \frac{\delta^2 z }{\delta y^2} = -4x^2 \cos (2xy) \\  
           \end{align*}
        \end{center}
\end{enumerate}



























% \end{document}
