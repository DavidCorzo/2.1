%%%%%%%%%%%%%%%%%%%%%%%%%%%%%%%%%%%%%%%%%%%%%%%%%%%%%%%%%%%%%%%%%%%%%%%%%%%%%%%%%%%%%%%%%%%%%%%%
\section{12.1 Sistema tridimensional de coordenadas}
Para localizar un punto en un plano, se necesitan dos números. 
\begin{itemize}
    \item $a$ la coordenada $x$ 
    \item $b$ la coordenada $y$ 
\end{itemize}
En el plano $\displaystyle \mathbb{R}^2$ 

\begin{itemize}
    \item Los ejes de coordenadas son perpendiculares entre sí.
\end{itemize}

\begin{center}
    \begin{tabular}{ c }
        \multicolumn{1}{l}{\textbf{\large Sistema de coordenadas en dos dimensiones: }} \\
    \begin{tikzpicture}
        \begin{pgfonlayer}{nodelayer}
            \node [style=none] (0) at (-5.75, 5) {};
            \node [style=none] (1) at (-5.75, -1.75) {};
            \node [style=none] (2) at (1.75, -1.75) {};
            \node [style=none] (8) at (1.05, 0.2) {$(a,b)$};
            \node [style=none] (9) at (-3.675, 3.675) {$(b,a)$};
            \node [style=none] (11) at (-6.5, 5) {$y$};
            \node [style=none] (12) at (1.75, -2.25) {$x$};
            \node [style=none] (13) at (-5.50, -1.50) {};
            \node [style=none] (14) at (-5.50, -1.725) {};
            \node [style=none] (15) at (-5.75, -1.50) {};
            \node [style=darkdot] (16) at (-4.25, 3.5) {};
            \node [style=darkdot] (17) at (0.75, -0.25) {};
            \node [style=none] (18) at (-4.25, -1.75) {};
            \node [style=none] (19) at (-5.75, -0.25) {};
            \node [style=none] (20) at (-5.75, 3.5) {};
            \node [style=none] (21) at (0.75, -1.75) {};
            \node [style=none] (26) at (-4.25, 3.75) {};
            \node [style=none] (27) at (-5.75, 3.75) {};
            \node [style=none] (28) at (-6, -1.75) {};
            \node [style=none] (29) at (-6, 3.5) {};
            \node [style=none] (30) at (-5.725, -2) {};
            \node [style=none] (31) at (0.775, -2) {};
            \node [style=none] (32) at (0.90, -0.25) {};
            \node [style=none] (33) at (0.90, -1.75) {};
        \end{pgfonlayer}
        \begin{pgfonlayer}{edgelayer}
            \draw [style=line with arrow] (1.center) to (0.center);
            \draw [style=line with arrow] (1.center) to (2.center);
            \draw (14.center) to (13.center);
            \draw (13.center) to (15.center);
            \draw [style=dashed line] (18.center) to (16.center);
            \draw [style=dashed line] (16.center) to (20.center);
            \draw [style=dashed line] (19.center) to (17.center);
            \draw [style=dashed line] (17.center) to (21.center);
            \rbraceontikz{(27.center)}{(26.center)}{$f(y)$};
            \lbraceontikz{(29.center)}{(28.center)}{$y$}{xshift=-1cm};
            \lbraceontikz{(30.center)}{(31.center)}{$x$}{xshift=0cm,yshift=-1cm};
            \lbraceontikz{(33.center)}{(32.center)}{$f(x)$}{xshift=1cm};
        \end{pgfonlayer}
    \end{tikzpicture} \\
    \end{tabular}
\end{center}
%----------------------------------------------------------------------------------------


\begin{itemize}
    \item En el sistema tridimensional de coordenadas rectangulares, cada punto en el espacio es una terna ordenada $\displaystyle \p{x,y,z} $.
        \begin{center}
           \begin{align*}
               \text{ Espacio:  } \qq \mathbb{R}^3 &= \cb{\p{x,y,z} \qq \text{ talque } \qq x,y,z \; \in \mathbb{R}} \\ 
               \mathbb{R}^3 &= \mathbb{R}^2 \times \mathbb{R} \\  
           \end{align*}
           \begin{itemize}
               \item $x$ transversal 
               \item $y$ horizontal 
               \item $z$ vertical 
               \item $\displaystyle z=f\p{x,y} $ 
           \end{itemize}
        \end{center}

        \begin{center}
            \begin{tabular}{ c }
                \multicolumn{1}{l}{\textbf{\large Sistema tridimensional de coordenadas: }} \\ 
                \begin{tikzpicture}
                        \begin{pgfonlayer}{nodelayer}
                            \node [style=none] (0) at (-3, 5) {};
                            \node [style=none] (1) at (-3, -1.25) {};
                            \node [style=none] (2) at (2.5, -1.25) {};
                            \node [style=none] (3) at (-6.5, -4.75) {};
                            \node [style=none] (4) at (-3.5, 5) {};
                            \node [style=none] (5) at (-3.5, 5) {$z$};
                            \node [style=none] (6) at (-6.5, -5.25) {$x$};
                            \node [style=none] (7) at (2.75, -1.75) {$y$};
                            \node [style=none] (8) at (-5.25, -3.5) {};
                            \node [style=none] (9) at (-2, -3.5) {};
                            \node [style=none] (10) at (0, -1.25) {};
                            \node [style=none] (12) at (-2.075, 1) {};
                            \node [style=none] (13) at (-3, 2.5) {};
                            \node [style=none] (14) at (-3, 2.5) {};
                            \node [style=dot] (15) at (-2.075, 1) {};
                            \node [style=none] (17) at (-1.5, 1.5) {$(a,b,c)$};
                            \node [style=none] (18) at (-5.6, -2.975) {$(a,0,0)$};
                            \node [style=none] (19) at (-2, -4) {$(a,b,0)$};
                        \end{pgfonlayer}
                        \begin{pgfonlayer}{edgelayer}
                            \draw [style=line with arrow] (1.center) to (3.center);
                            \draw [style=line with arrow] (1.center) to (0.center);
                            \draw [style=line with arrow] (1.center) to (2.center);
                            \draw (8.center) to (9.center);
                            \draw [style=dashed line] (9.center) to (10.center);
                            \draw [style=dashed line] (12.center) to (9.center);
                            \draw [style=dashed line] (14.center) to (12.center);
                            \draw [style=dasharr] (1.center) to (8.center);
                            \draw [style=dasharr] (8.center) to (9.center);
                            \draw [style=dasharr] (9.center) to (15);
                        \end{pgfonlayer}
                \end{tikzpicture} \\ 
            \end{tabular}
        \end{center}
        \begin{itemize}
            \item Las líneas punteadas se utilizan para simbolizar las partes de abajo, izquierda y detrás.
        \end{itemize}
    \item Las líneas punteadas se usan para simbolizar las partes debajo, izquierda y detrás.
\end{itemize}



%----------------------------------------------------------------------------------------
\subsection{Ejercicios}
Identifique las siguientes puntos: 
\begin{enumerate}
    \item $\displaystyle \p{0,0,0} $ 
        \begin{center}
            \begin{tikzpicture}
                \begin{pgfonlayer}{nodelayer}
                    \node [style=none] (0) at (-3, 2.25) {};
                    \node [style=none] (1) at (-3, -1.25) {};
                    \node [style=none] (2) at (0.75, -1.25) {};
                    \node [style=none] (3) at (-5, -3.25) {};
                    \node [style=none] (4) at (-3.5, 2.25) {};
                    \node [style=none] (5) at (-3.5, 2.25) {$z$};
                    \node [style=none] (6) at (-5, -3.75) {$x$};
                    \node [style=none] (7) at (1, -1.75) {$y$};
                    \node [style=darkdot] (8) at (-3, -1.25) {};
                    \node [style=none] (9) at (-2.25, -0.75) {$(0,0,0)$};
                \end{pgfonlayer}
                \begin{pgfonlayer}{edgelayer}
                    \draw [style=line with arrow] (1.center) to (3.center);
                    \draw [style=line with arrow] (1.center) to (0.center);
                    \draw [style=line with arrow] (1.center) to (2.center);
                \end{pgfonlayer}
            \end{tikzpicture}            
        \end{center}
    
    \item $\displaystyle \p{0,1,0} $ 
        \begin{center}
            \begin{tikzpicture}
                \begin{pgfonlayer}{nodelayer}
                    \node [style=none] (0) at (-3, 2.25) {};
                    \node [style=none] (1) at (-3, -1.25) {};
                    \node [style=none] (2) at (0.75, -1.25) {};
                    \node [style=none] (3) at (-5, -3.25) {};
                    \node [style=none] (4) at (-3.5, 2.25) {};
                    \node [style=none] (5) at (-3.5, 2.25) {$z$};
                    \node [style=none] (6) at (-5, -3.75) {$x$};
                    \node [style=none] (7) at (1, -1.75) {$y$};
                    \node [style=none] (9) at (-2.5, -2) {$(0,1,0)$};
                    \node [style=none] (10) at (-2.25, -1.25) {};
                    \node [style=darkdot] (11) at (-2.25, -1.25) {};
                \end{pgfonlayer}
                \begin{pgfonlayer}{edgelayer}
                    \draw [style=line with arrow] (1.center) to (3.center);
                    \draw [style=line with arrow] (1.center) to (0.center);
                    \draw [style=line with arrow] (1.center) to (2.center);
                    \draw [style=arrowws] (1.center) to (11.center);
                \end{pgfonlayer}
            \end{tikzpicture}   
        \end{center}
    
    \item $\displaystyle \p{-1,1,0}$
        \begin{center}
            \begin{tikzpicture}
                \begin{pgfonlayer}{nodelayer}
                    \node [style=none] (0) at (-3, 2.25) {};
                    \node [style=none] (1) at (-3, -1.25) {};
                    \node [style=none] (2) at (0.75, -1.25) {};
                    \node [style=none] (3) at (-5, -3.25) {};
                    \node [style=none] (4) at (-3.5, 2.25) {};
                    \node [style=none] (5) at (-3.5, 2.25) {$z$};
                    \node [style=none] (6) at (-5, -3.75) {$x$};
                    \node [style=none] (7) at (1, -1.75) {$y$};
                    \node [style=none] (9) at (-1, -0.5) {$(-1,1,0)$};
                    \node [style=none] (10) at (-1.25, 0.25) {};
                    \node [style=none] (11) at (-1.5, 0.25) {};
                    \node [style=none] (12) at (-2.5, -0.75) {};
                    \node [style=none] (13) at (-1.75, -0.75) {};
                    \node [style=darkdot] (14) at (-1.75, -0.75) {};
                \end{pgfonlayer}
                \begin{pgfonlayer}{edgelayer}
                    \draw [style=line with arrow] (1.center) to (3.center);
                    \draw [style=line with arrow] (1.center) to (0.center);
                    \draw [style=line with arrow] (1.center) to (2.center);
                    \draw [style=line] (1.center) to (11.center);
                    \draw [style=arrowws] (1.center) to (12.center);
                    \draw [style=arrowws] (12.center) to (13.center);
                \end{pgfonlayer}
            \end{tikzpicture}
        \end{center}
    
    \item $\displaystyle \p{1,3,-1} $  
        \begin{tikzpicture}
            \begin{pgfonlayer}{nodelayer}
                \node [style=none] (0) at (-3, 2.25) {};
                \node [style=none] (1) at (-3, -1.25) {};
                \node [style=none] (2) at (0.75, -1.25) {};
                \node [style=none] (3) at (-5, -3.25) {};
                \node [style=none] (4) at (-3.5, 2.25) {};
                \node [style=none] (5) at (-3.5, 2.25) {$z$};
                \node [style=none] (6) at (-5, -3.75) {$x$};
                \node [style=none] (7) at (1, -1.75) {$y$};
                \node [style=none] (9) at (-2.25, -2.5) {$(1,3,-1)$};
                \node [style=none] (10) at (-3.325, -1.575) {};
                \node [style=none] (11) at (-1.75, -1.575) {};
                \node [style=none] (12) at (-1.75, -1.975) {};
                \node [style=darkdot] (13) at (-1.75, -1.975) {};
            \end{pgfonlayer}
            \begin{pgfonlayer}{edgelayer}
                \draw [style=line with arrow] (1.center) to (3.center);
                \draw [style=line with arrow] (1.center) to (0.center);
                \draw [style=line with arrow] (1.center) to (2.center);
                \draw [style=arrowws] (1.center) to (10.center);
                \draw [style=arrowws] (10.center) to (11.center);
                \draw [style=arrowws] (11.center) to (12.center);
            \end{pgfonlayer}
        \end{tikzpicture}
\end{enumerate}



%%%%%%%%%%%%%%%%%%%%%%%%%%%%%%%%%%%%%%%%%%%%%%%%%%%%%%%%%%%%%%%%%%%%%%%%%%%%%%%%%%%%%%%%%%
\section{Planos coordenados}
\begin{itemize}
    \item Planos-$\displaystyle xy$:
        % 10,11
        % 12,13
        % 14,15
        
        \begin{center}
            \begin{tikzpicture}
                \begin{pgfonlayer}{nodelayer}
                    \node [style=none] (0) at (-3, 3.25) {};
                    \node [style=none] (1) at (-3, -1.25) {};
                    \node [style=none] (2) at (1.75, -1.25) {};
                    \node [style=none] (3) at (-5.5, -3.75) {};
                    \node [style=none] (5) at (-3.5, 3.25) {$z$};
                    \node [style=none] (6) at (-5.3, -4.125) {$x$};
                    \node [style=none] (7) at (2, -1.5) {$y$};
                    \node [style=none] (8) at (-5, -3.25) {};
                    \node [style=none] (9) at (-3, -1.25) {};
                    \node [style=none] (10) at (-3, 2.25) {};
                    \node [style=none] (11) at (-5, 0.5) {};
                    \node [style=none] (12) at (1.025, 2.275) {};
                    \node [style=none] (13) at (-0.25, -3.25) {};
                    \node [style=none] (14) at (-4.25, -3.25) {};
                    \node [style=none] (15) at (-3.45, -3.25) {};
                    \node [style=none] (16) at (-2.7, -3.25) {};
                    \node [style=none] (17) at (-1.9, -3.25) {};
                    \node [style=none] (18) at (-1.075, -3.25) {};
                    \node [style=none] (20) at (-4.5, 0.925) {};
                    \node [style=none] (21) at (-4.05, 1.325) {};
                    \node [style=none] (22) at (-3.65, 1.675) {};
                    \node [style=none] (23) at (-3.3, 1.975) {};
                    \node [style=none] (24) at (-4.55, -2.75) {};
                    \node [style=none] (25) at (-4.075, -2.325) {};
                    \node [style=none] (26) at (-3.675, -1.925) {};
                    \node [style=none] (27) at (-3.325, -1.6) {};
                    \node [style=none] (29) at (-2.4, -1.25) {};
                    \node [style=none] (30) at (-1.7, -1.25) {};
                    \node [style=none] (31) at (-0.975, -1.25) {};
                    \node [style=none] (32) at (-0.35, -1.25) {};
                    \node [style=none] (33) at (0.3, -1.25) {};
                    \node [style=none] (34) at (-2.425, 2.25) {};
                    \node [style=none] (35) at (-1.675, 2.25) {};
                    \node [style=none] (36) at (-0.95, 2.25) {};
                    \node [style=none] (37) at (-0.275, 2.25) {};
                    \node [style=none] (38) at (0.325, 2.25) {};
                    \node [style=none] (39) at (1, -1.25) {};
                    \node [style=none] (40) at (-1.25, 0.5) {Plano-yz};
                    \node [style=none] (41) at (-4, -0.5) {Plano-xz};
                    \node [style=none] (42) at (-1.75, -2.25) {Plano-xy};
                \end{pgfonlayer}
                \begin{pgfonlayer}{edgelayer}
                    \draw [style=line with arrow] (1.center) to (3.center);
                    \draw [style=line with arrow] (1.center) to (0.center);
                    \draw [style=line with arrow] (1.center) to (2.center);
                    \draw [style=dashed line] (8.center) to (11.center);
                    \draw [style=dashed line] (11.center) to (10.center);
                    \draw [style=dashed line] (8.center) to (13.center);
                    \draw [style=dashed line] (12.center) to (10.center);
                    \draw [style=dashed line] (29.center) to (34.center);
                    \draw [style=dashed line] (35.center) to (30.center);
                    \draw [style=dashed line] (36.center) to (31.center);
                    \draw [style=dashed line] (37.center) to (32.center);
                    \draw [style=dashed line] (33.center) to (38.center);
                    \draw [style=dashed line] (27.center) to (23.center);
                    \draw [style=dashed line] (22.center) to (26.center);
                    \draw [style=dashed line] (25.center) to (21.center);
                    \draw [style=dashed line] (20.center) to (24.center);
                    \draw [style=dashed line] (14.center) to (29.center);
                    \draw [style=dashed line] (15.center) to (30.center);
                    \draw [style=dashed line] (16.center) to (31.center);
                    \draw [style=dashed line] (17.center) to (32.center);
                    \draw [style=dashed line] (18.center) to (33.center);
                    \draw [style=dashed line] (12.center) to (39.center);
                    \draw [style=dashed line] (39.center) to (13.center);
                \end{pgfonlayer}
            \end{tikzpicture}
        \end{center}
        
    
    \item Los planos:  
        \begin{itemize}
            \item Plano-$\displaystyle yz$:  $\displaystyle x=0$ 
            \item Plano-$\displaystyle xz$: $\displaystyle y=0$ 
            \item Plano-$\displaystyle yx$: $\displaystyle z=0$ 
        \end{itemize}
    
    \item El primer octante:
        \begin{center}
            \begin{tikzpicture}
                \begin{pgfonlayer}{nodelayer}
                    \node [style=none] (0) at (-3, 3.25) {};
                    \node [style=none] (1) at (-3, -1.25) {};
                    \node [style=none] (2) at (1.75, -1.25) {};
                    \node [style=none] (3) at (-5.5, -3.75) {};
                    \node [style=none] (5) at (-3.5, 3.25) {$z$};
                    \node [style=none] (6) at (-5.3, -4.125) {$x$};
                    \node [style=none] (7) at (2, -1.5) {$y$};
                    \node [style=none] (8) at (-5, -3.25) {};
                    \node [style=none] (9) at (-3, -1.25) {};
                    \node [style=none] (10) at (-3, 2.25) {};
                    \node [style=none] (11) at (-5, 0.5) {};
                    \node [style=none] (12) at (1.025, 2.275) {};
                    \node [style=none] (13) at (-0.25, -3.25) {};
                    \node [style=none] (14) at (-4.25, -3.25) {};
                    \node [style=none] (15) at (-3.45, -3.25) {};
                    \node [style=none] (16) at (-2.7, -3.25) {};
                    \node [style=none] (17) at (-1.9, -3.25) {};
                    \node [style=none] (18) at (-1.075, -3.25) {};
                    \node [style=none] (20) at (-4.5, 0.925) {};
                    \node [style=none] (21) at (-4.05, 1.325) {};
                    \node [style=none] (22) at (-3.65, 1.675) {};
                    \node [style=none] (23) at (-3.3, 1.975) {};
                    \node [style=none] (24) at (-4.55, -2.75) {};
                    \node [style=none] (25) at (-4.075, -2.325) {};
                    \node [style=none] (26) at (-3.675, -1.925) {};
                    \node [style=none] (27) at (-3.325, -1.6) {};
                    \node [style=none] (29) at (-2.4, -1.25) {};
                    \node [style=none] (30) at (-1.7, -1.25) {};
                    \node [style=none] (31) at (-0.975, -1.25) {};
                    \node [style=none] (32) at (-0.35, -1.25) {};
                    \node [style=none] (33) at (0.3, -1.25) {};
                    \node [style=none] (34) at (-2.425, 2.25) {};
                    \node [style=none] (35) at (-1.675, 2.25) {};
                    \node [style=none] (36) at (-0.95, 2.25) {};
                    \node [style=none] (37) at (-0.275, 2.25) {};
                    \node [style=none] (38) at (0.325, 2.25) {};
                    \node [style=none] (39) at (1, -1.25) {};
                    \node [style=none] (43) at (-0.25, 0.5) {};
                    \node [style=none] (44) at (-1.875, -2.25) {Arriba del suelo};
                    \node [style=none] (45) at (-4.675, -0.55) {Derecha de esta pared};
                    \node [style=none] (46) at (-1.375, 1.425) {Delante de esta pared};
                \end{pgfonlayer}
                \begin{pgfonlayer}{edgelayer}
                    \draw [style=line with arrow] (1.center) to (3.center);
                    \draw [style=line with arrow] (1.center) to (0.center);
                    \draw [style=line with arrow] (1.center) to (2.center);
                    \draw [style=dashed line] (8.center) to (11.center);
                    \draw [style=dashed line] (11.center) to (10.center);
                    \draw [style=dashed line] (8.center) to (13.center);
                    \draw [style=dashed line] (12.center) to (10.center);
                    \draw [style=dashed line] (29.center) to (34.center);
                    \draw [style=dashed line] (35.center) to (30.center);
                    \draw [style=dashed line] (36.center) to (31.center);
                    \draw [style=dashed line] (37.center) to (32.center);
                    \draw [style=dashed line] (33.center) to (38.center);
                    \draw [style=dashed line] (27.center) to (23.center);
                    \draw [style=dashed line] (22.center) to (26.center);
                    \draw [style=dashed line] (25.center) to (21.center);
                    \draw [style=dashed line] (20.center) to (24.center);
                    \draw [style=dashed line] (14.center) to (29.center);
                    \draw [style=dashed line] (15.center) to (30.center);
                    \draw [style=dashed line] (16.center) to (31.center);
                    \draw [style=dashed line] (17.center) to (32.center);
                    \draw [style=dashed line] (18.center) to (33.center);
                    \draw [style=dashed line] (12.center) to (39.center);
                    \draw [style=dashed line] (39.center) to (13.center);
                    \draw (43.center) to (12.center);
                    \draw (43.center) to (11.center);
                    \draw (43.center) to (13.center);
                \end{pgfonlayer}
            \end{tikzpicture}
        \end{center}
    
    \item Planos en el espacio:  
        \begin{itemize}
            \item En dos dimensiones cuando se proponía $\displaystyle x=a$ ó $\displaystyle y=b$ se sabía que se hablaba de una recta horizontal o vertical. 
        \end{itemize}
        \begin{tikzpicture}
            \begin{pgfonlayer}{nodelayer}
                \node [style=none] (0) at (-2.5, 2.75) {};
                \node [style=none] (1) at (-2.475, -6) {};
                \node [style=none] (2) at (2.975, -1.75) {};
                \node [style=none] (3) at (-8, -1.75) {};
                \node [style=none] (4) at (3.25, -2.25) {$x$};
                \node [style=none] (5) at (-3, 3) {$y$};
                \node [style=none] (6) at (-8.25, 0.5) {};
                \node [style=none] (7) at (3, 0.5) {};
                \node [style=none] (8) at (1.25, 2.75) {};
                \node [style=none] (9) at (1.25, -6) {};
                \node [style=none] (10) at (1.25, 3.25) {$x=5$};
                \node [style=none] (11) at (3.75, 0.5) {$y=4$};
            \end{pgfonlayer}
            \begin{pgfonlayer}{edgelayer}
                \draw [style=double arrow] (3.center) to (2.center);
                \draw [style=double arrow] (0.center) to (1.center);
                \draw [style=double arrow] (7.center) to (6.center);
                \draw [style=double arrow] (9.center) to (8.center);
            \end{pgfonlayer}
        \end{tikzpicture}
    
    \item En tres dimensiones $\displaystyle x=a, y=b, z=c$ son gráficas de planos.
        \begin{center}
            \begin{tikzpicture}
                \begin{pgfonlayer}{nodelayer}
                    \node [style=none] (0) at (-3.5, 4) {};
                    \node [style=none] (1) at (-3.5, -2.5) {};
                    \node [style=none] (2) at (3.75, -2.5) {};
                    \node [style=none] (3) at (-7.5, -6.75) {};
                    \node [style=none] (4) at (-7.25, -7.25) {$x$};
                    \node [style=none] (5) at (-4.25, 4.25) {$z$};
                    \node [style=none] (6) at (3.75, -3) {$y$};
                    \node [style=none] (7) at (-3, -7.5) {$y=3$};
                    \node [style=none] (8) at (-2.75, -7) {};
                    \node [style=none] (9) at (0.75, -2.75) {};
                    \node [style=none] (10) at (0.75, 5.75) {};
                    \node [style=none] (11) at (-2.75, 1.5) {};
                    \node [style=none] (12) at (-2.325, 2) {};
                    \node [style=none] (13) at (-1.925, 2.5) {};
                    \node [style=none] (14) at (-1.5, 3) {};
                    \node [style=none] (15) at (-1, 3.5) {};
                    \node [style=none] (16) at (-0.75, 4) {};
                    \node [style=none] (17) at (-0.25, 4.5) {};
                    \node [style=none] (18) at (0.15, 5) {};
                    \node [style=none] (19) at (0.475, 5.4) {};
                    \node [style=none] (20) at (-2.25, -6.5) {};
                    \node [style=none] (21) at (-1.925, -6.025) {};
                    \node [style=none] (22) at (-1.5, -5.5) {};
                    \node [style=none] (23) at (-1, -5) {};
                    \node [style=none] (24) at (-0.75, -4.5) {};
                    \node [style=none] (25) at (-0.25, -4) {};
                    \node [style=none] (26) at (0.15, -3.5) {};
                    \node [style=none] (27) at (0.475, -3.1) {};
                    \node [style=darkdot] (28) at (-0.75, -2.5) {};
                \end{pgfonlayer}
                \begin{pgfonlayer}{edgelayer}
                    \draw [style=line with arrow] (1.center) to (0.center);
                    \draw [style=line with arrow] (1.center) to (2.center);
                    \draw [style=line with arrow] (1.center) to (3.center);
                    \draw [style=dashed line] (11.center) to (10.center);
                    \draw [style=dashed line] (10.center) to (9.center);
                    \draw [style=dashed line] (9.center) to (8.center);
                    \draw [style=dashed line] (8.center) to (11.center);
                    \draw [style=dashed line] (12.center) to (20.center);
                    \draw [style=dashed line] (21.center) to (13.center);
                    \draw [style=dashed line] (14.center) to (22.center);
                    \draw [style=dashed line] (23.center) to (15.center);
                    \draw [style=dashed line] (16.center) to (24.center);
                    \draw [style=dashed line] (25.center) to (17.center);
                    \draw [style=dashed line] (18.center) to (26.center);
                    \draw [style=dashed line] (27.center) to (19.center);
                \end{pgfonlayer}
            \end{tikzpicture}
        \end{center}
    
    \item Ejemplo: $\displaystyle z=2$ el ``techo'' en $\displaystyle z=2$ 
        \begin{center}
            \begin{tikzpicture}
                \begin{pgfonlayer}{nodelayer}
                    \node [style=none] (0) at (-3.5, 4) {};
                    \node [style=none] (1) at (-3.5, -2.5) {};
                    \node [style=none] (2) at (3.75, -2.5) {};
                    \node [style=none] (3) at (-7.5, -6.75) {};
                    \node [style=none] (4) at (-7.25, -7.25) {$x$};
                    \node [style=none] (5) at (-4.25, 4.25) {$z$};
                    \node [style=none] (6) at (3.75, -3) {$y$};
                    \node [style=none] (7) at (-10.75, -3.25) {$z=2$};
                    \node [style=none] (8) at (-10.25, -2.75) {};
                    \node [style=none] (9) at (-1.35, -2.75) {};
                    \node [style=none] (10) at (2.25, 0.75) {};
                    \node [style=none] (11) at (-6.25, 0.75) {};
                    \node [style=none] (12) at (-9.75, -2.75) {};
                    \node [style=none] (13) at (-9.25, -2.75) {};
                    \node [style=none] (14) at (-8.6, -2.75) {};
                    \node [style=none] (15) at (-8, -2.75) {};
                    \node [style=none] (16) at (-7.425, -2.75) {};
                    \node [style=none] (17) at (-6.75, -2.75) {};
                    \node [style=none] (18) at (-6.175, -2.75) {};
                    \node [style=none] (19) at (-5.525, -2.75) {};
                    \node [style=none] (20) at (-4.75, -2.75) {};
                    \node [style=none] (21) at (-4, -2.75) {};
                    \node [style=none] (22) at (-3.25, -2.75) {};
                    \node [style=none] (23) at (-2.5, -2.75) {};
                    \node [style=none] (24) at (-2, -2.75) {};
                    \node [style=none] (27) at (-5.75, 0.75) {};
                    \node [style=none] (28) at (-5.25, 0.75) {};
                    \node [style=none] (29) at (-4.675, 0.75) {};
                    \node [style=none] (30) at (-4, 0.75) {};
                    \node [style=none] (31) at (-3.5, 0.75) {};
                    \node [style=none] (32) at (-2.75, 0.75) {};
                    \node [style=none] (33) at (-2.075, 0.75) {};
                    \node [style=none] (34) at (-1.45, 0.75) {};
                    \node [style=none] (35) at (-0.675, 0.75) {};
                    \node [style=none] (36) at (0.025, 0.75) {};
                    \node [style=none] (37) at (0.675, 0.75) {};
                    \node [style=none] (38) at (1.25, 0.75) {};
                    \node [style=none] (39) at (1.75, 0.75) {};
                    \node [style=darkdot] (40) at (-3.45, -0.9) {};
                \end{pgfonlayer}
                \begin{pgfonlayer}{edgelayer}
                    \draw [style=line with arrow] (1.center) to (0.center);
                    \draw [style=line with arrow] (1.center) to (2.center);
                    \draw [style=line with arrow] (1.center) to (3.center);
                    \draw [style=dashed line] (8.center) to (11.center);
                    \draw [style=dashed line] (11.center) to (10.center);
                    \draw [style=dashed line] (10.center) to (9.center);
                    \draw [style=dashed line] (9.center) to (8.center);
                    \draw [style=dashed line] (27.center) to (12.center);
                    \draw [style=dashed line] (13.center) to (28.center);
                    \draw [style=dashed line] (29.center) to (14.center);
                    \draw [style=dashed line] (30.center) to (15.center);
                    \draw [style=dashed line] (16.center) to (31.center);
                    \draw [style=dashed line] (32.center) to (17.center);
                    \draw [style=dashed line] (18.center) to (33.center);
                    \draw [style=dashed line] (34.center) to (19.center);
                    \draw [style=dashed line] (20.center) to (35.center);
                    \draw [style=dashed line] (21.center) to (36.center);
                    \draw [style=dashed line] (22.center) to (37.center);
                    \draw [style=dashed line] (23.center) to (38.center);
                    \draw [style=dashed line] (39.center) to (24.center);
                \end{pgfonlayer}
            \end{tikzpicture}                      
        \end{center}
    
    \item Ecuación lineal en 3-D va a graficar un plano:
        \[
          ax+by+cz=d
        \]
        \begin{itemize}
            \item Generalmente se grafican sólo en el primer octante se cada $\displaystyle a,b,c$ y $\displaystyle d$ es positiva.
        \end{itemize}
\end{itemize}


%----------------------------------------------------------------------------------------
\subsection{Ejercicios}
\begin{enumerate}
    \item Bosqueje el plano $\displaystyle 2x+4y+3z=12$ sólo en el primer octante: 
        \begin{center}
            \begin{align*}
                \text{ Intersección-$x$  }: \qquad 2x=12 \qimplies (6,0,0) \\ 
                \text{ Intersección-$x$  }: \qquad 4y=12 \qimplies (0,3,0) \\ 
                \text{ Intersección-$x$  }: \qquad 3z=12 \qimplies (0,0,4) \\ 
            \end{align*}
        \end{center}
        \begin{center}
            \begin{tikzpicture}
                \begin{pgfonlayer}{nodelayer}
                    \node [style=none] (0) at (-3.5, 4) {};
                    \node [style=none] (1) at (-3.5, -2.5) {};
                    \node [style=none] (2) at (3.75, -2.5) {};
                    \node [style=none] (3) at (-8.5, -8.25) {};
                    \node [style=none] (4) at (-8.25, -8.75) {$x$};
                    \node [style=none] (5) at (-4.25, 4.25) {$z$};
                    \node [style=none] (6) at (3.75, -3) {$y$};
                    \node [style=none] (7) at (-7.45, -7.025) {};
                    \node [style=none] (8) at (-3.5, 1) {};
                    \node [style=none] (9) at (0.5, -2.5) {};
                    \node [style=none] (10) at (0.75, -1.75) {$(0,3,0)$};
                    \node [style=none] (11) at (-4.5, 1.25) {$(0,0,4)$};
                    \node [style=none] (12) at (-8.75, -6.5) {$(6,0,0)$};
                    \node [style=none] (13) at (2.5, 1) {Una los tres puntos para obtener un segmento del plano};
                    \node [style=none] (14) at (-3.875, 0.25) {};
                    \node [style=none] (15) at (-4.25, -0.5) {};
                    \node [style=none] (16) at (-4.625, -1.25) {};
                    \node [style=none] (17) at (-5.025, -2.1) {};
                    \node [style=none] (18) at (-5.4, -2.85) {};
                    \node [style=none] (19) at (-5.725, -3.45) {};
                    \node [style=none] (20) at (-6.075, -4.175) {};
                    \node [style=none] (21) at (-6.65, -5.375) {};
                    \node [style=none] (22) at (-7, -6.1) {};
                    \node [style=none] (23) at (-0.05, -2.825) {};
                    \node [style=none] (24) at (-0.75, -3.2) {};
                    \node [style=none] (25) at (-1.525, -3.65) {};
                    \node [style=none] (26) at (-2.25, -4.05) {};
                    \node [style=none] (27) at (-2.975, -4.475) {};
                    \node [style=none] (28) at (-3.7, -4.9) {};
                    \node [style=none] (29) at (-4.4, -5.275) {};
                    \node [style=none] (30) at (-5.125, -5.7) {};
                    \node [style=none] (31) at (-5.825, -6.075) {};
                    \node [style=none] (32) at (-6.5, -6.475) {};
                    \node [style=none] (34) at (-6.35, -4.75) {};
                \end{pgfonlayer}
                \begin{pgfonlayer}{edgelayer}
                    \draw [style=line with arrow] (1.center) to (0.center);
                    \draw [style=line with arrow] (1.center) to (2.center);
                    \draw [style=line with arrow] (1.center) to (3.center);
                    \draw (7.center) to (8.center);
                    \draw (8.center) to (9.center);
                    \draw [in=30, out=-150] (9.center) to (7.center);
                    \draw (14.center) to (23.center);
                    \draw (24.center) to (15.center);
                    \draw (16.center) to (25.center);
                    \draw (26.center) to (17.center);
                    \draw (27.center) to (18.center);
                    \draw (29.center) to (20.center);
                    \draw (19.center) to (28.center);
                    \draw (34.center) to (30.center);
                    \draw (31.center) to (21.center);
                    \draw (22.center) to (32.center);
                \end{pgfonlayer}
            \end{tikzpicture}                   
        \end{center}
\end{enumerate}
