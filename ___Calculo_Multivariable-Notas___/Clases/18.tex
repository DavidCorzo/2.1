\section{15.2 y 15.3 Integrales dobles}
\begin{itemize}
    \item El volúmen del sólido entre las dos superficies $\displaystyle z=f\p{x,y} $ y la región $R$ en el plano $xy$ es:
        \[
          V = \iint_{D}^{} f\p{x,y} dA 
        \]
    
    \item Región rectangular $\displaystyle R = [a,b] \times [c,d]$ 
        \[
          \iint_{D}^{f\p{x,y} }dA = \int_{a}^{b} \p{\int_{c}^{d} f\p{x,y}dy  }dx = \int_{c}^{d}\int_{a}^{b}f\p{x,y} dxdy 
        \]
    
    \item En algunos problemas un orden de integración simplifica considerablemente la evaluación de la integral doble.
\end{itemize}


%----------------------------------------------------------------------------------------
\subsection{Ejercicios}
\begin{enumerate}
    \item Evalúe $\displaystyle I = \int_{0}^{2}\p{\int_{0}^{3}ye^{-xy}dy} dx$ 
        \begin{center}
           \begin{align*}
               I_1 = \int_{0}^{3}yc^{-xy}dy = \frac{-y}{x} e^{-xy} \evaluate{y=0}{y=3} + \int_{0}^{3}e^{-xy}dy \\ 
               \\
               \text{ Cambiar los ordenes de integración } \\ 
               \begin{matrix}
                   u = -xy \\ 
                   du = -ydx \\ 
               \end{matrix}
               I = \int_{0}^{3}\p{\int_{0}^{2}e^{-xy}ydx} dy \\ 
               I = \int_{0}^{3}-e^{-xy}\evaluate{x=0}{x=2} dy = \int_{0}^{3}\p{-e^{-2y}+1}dy \\ 
               I = \frac{1}{2} e^{-2y} + y \evaluate{y=0}{y=3} = \frac{1}{2} e^{-6}+3-\frac{1}{2} \\  
           \end{align*}
        \end{center}
\end{enumerate}


%%%%%%%%%%%%%%%%%%%%%%%%%%%%%%%%%%%%%%%%%%%%%%%%%%%%%%%%%%%%%%%%%%%%%%%%%%%%%%%%%%%%%%%%%%
\section{5.3 Integrales dobles en regiones generales}
\begin{itemize}
    \item Considere la región $D$ $\displaystyle G(X)\leq y \leq f(x)$ ó $\displaystyle a\leq x \leq b$.
        \[
          \iint_{D}^{}H(x,y)dA = \int_{a}^{b}\p{\int_{g(x)}^{f(x)}H(x,y)dy} dx 
        \]
\end{itemize} 

