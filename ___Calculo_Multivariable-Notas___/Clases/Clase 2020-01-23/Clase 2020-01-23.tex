\documentclass{article}
\title{Clase 2020-01-23}
\author{David Gabriel Corzo Mcmath}
\date{2020-Jan-23 10:33:52}
%%%%%%%%%%%%%%%%%%%%%%%%%%%%%%%%%%%%%%%%%%%%%%%%%%%%%%%%%%%%%%%%%%%%%%%%%%%%%%%%%%%%%%%%%%%%%%%%%%%%%%%%%%%%%%%%%%%%%%%%%%%%%%%%%%%%%%%%%%%%%%%
\usepackage[margin = 1in]{geometry}
\usepackage{graphicx}
\usepackage{fontenc}
\usepackage{pdfpages}
\usepackage[spanish]{babel}
\usepackage{amsmath}
\usepackage{amsthm}
\usepackage[utf8]{inputenc}
\usepackage{enumitem}
\usepackage{mathtools}
\usepackage{import}
\usepackage{xifthen}
\usepackage{pdfpages}
\usepackage{transparent}
\usepackage{color}
\usepackage{fancyhdr}
\usepackage{lipsum}
\usepackage{sectsty}
\usepackage{titlesec}
\usepackage{calc}
\usepackage{lmodern}
\usepackage{xpatch}
\usepackage{blindtext}
\usepackage{bookmark}
\usepackage{fancyhdr}
\usepackage{xcolor}
\usepackage{tikz}
\usepackage{blindtext}
\usepackage{hyperref}
\usepackage{listing}
\usepackage{spverbatim}
\usepackage{fancyvrb}
\usepackage{fvextra}
\usepackage{amssymb}
\usepackage{pifont}
\usepackage{longtable}
\usepackage{bm}
%%%%%%%%%%%%%%%%%%%%%%%%%%%%%%%%%%%%%%%%%%%%%%%%%%%%%%%%%%%%%%%%%%%%%%%%%%%%%%%%%%%%%%%%%%%%%%%%%%%%%%%%%%%%%%%%%%%%%%%%%%%%%%%%%%%%%%%%%%%%%%%
\begin{document}
\maketitle


\section{12.4 Producto Cruz}
\begin{itemize}
    \item \emph{\textbf{Definición de ``Determinantes":} Matriz (arreglo rectangular de números).}
    \item \emph{\textbf{Definición de ``Cuadrada":} Mismo número de filas y columnas. }
    \item \begin{equation*}
        \begin{vmatrix}
            a & b \\ 
            c & d \\ 
        \end{vmatrix}
        = ad - bc \\ 
    \end{equation*}
    \# Determinante de orden 2. Matriz de 2x2 
    
    \item pie:
        \begin{equation*}
            \begin{vmatrix}
                3 & 4 \\ 
                -1 & 2 \\ 
            \end{vmatrix}
            = 6 - (-1) (4) = 6 + 4 = 10 \\ 
        \end{equation*}
    
    \item Determinante de orden 3: Matriz 3x3 suma de tres determinantes de orden 2:
        \begin{equation*}
            \begin{vmatrix}
                a_{1} & a_{2} & a_{3} \\ 
                b_{1} & b_{2} & b_{3} \\ 
                c_{1} & c_{2} & c_{3} \\ 
            \end{vmatrix}
            = a_{1}\begin{vmatrix}
                b_{2} & b_{3} \\ 
                c_{2} & c_{3} \\ 
            \end{vmatrix} + a_{2}
            \begin{vmatrix}
                b_1 &  b_3 \\ 
                c_1 & c_3 \\  
            \end{vmatrix} + a_3 \begin{vmatrix}
                b_1 & b_2 \\ 
                c_1 & c_2 \\ 
            \end{vmatrix}
        \end{equation*}
        3 matrices de 2x2.
    
    \item p.e. \begin{equation*}
        \begin{vmatrix}
            2&0&2\\ 
            1&3&0\\
            1&-1&2\\
        \end{vmatrix} = 2 \begin{vmatrix}
            3 & 0 \\ 
            -1 & 2 \\ 
        \end{vmatrix} - 0 \begin{vmatrix}
            1 & 0 \\ 
            1 & 2 \\ 
        \end{vmatrix} + 2 \begin{vmatrix}
            1 & 3 \\ 
            1 & -1 \\ 
        \end{vmatrix}
    \end{equation*}
    \[
      2(6-0)-0 + 2(-1-3) = 12 - 8 = 4 
    \]


\end{itemize}

\section{Producto Cruz}
\begin{itemize}
    \item Dados dos vectores :
        \begin{align*}
            \vec{a} & = a_1 \hat{i}  + a_2 \hat{j} + a_3 \hat{k} \\ 
            \vec{b} & = b_1 \hat{i} + b_2 \hat{j} + b_3 \hat{k} \\ 
        \end{align*}
    
    \item \textbf{Nos preguntamos:} ¿Cómo se encuentra un vector $\vec{c}$  que es perpendicular a $\vec{a}$ y a $\vec{b}$?
        \[
          \vec{c} \cdot \vec{a} = 0 
        \]
        \[
          \vec{c} \cdot \vec{b} = 0 
        \]
    
    \item Resuelva para $c_1,c_2,c_3$ :
        \begin{align*}
            c_1a_1+c_2a_2+c_3a_3=0\\ 
            c_1b_1+c_2b_2+c_3b_3=0\\ 
        \end{align*}
    
    \item El producto cruz $\vec{c}= \vec{a} \times \vec{b} = 0$ es un vector perpendicular a ambos vectores $\vec{a}$ \& $\vec{b}$.
        \begin{equation*}
            \vec{a}\times \vec{b} = \begin{vmatrix}
                \hat{i} & \hat{j} & \hat{k} \\ 
                a_1 & a_2 & a_3 \\ 
                b_1 & b_2 & b_3 \\ 
            \end{vmatrix} = \hat{i}(a_2b_3 - a_3b_2) - \hat{j}(a_1b_3-a_3b_1)+\hat{k}(a_1b_2-a_2b_1) 
        \end{equation*}
    
    \item Observaciones:
        \begin{itemize}
            \item El producto cruz es un vector, mientras que el producto es un número o escalar.
            \item El producto cruz \textbf{no} es conmutativo $\vec{a}\times \vec{b} \neq \vec{b}\times \vec{a}$.
        \end{itemize}
        \begin{equation*}
            \vec{b}\times \vec{a} = \begin{vmatrix}
                \hat{i} & \hat{j} & \hat{k} \\ 
                b_1 & b_2 & b_3 \\ 
                a_1 & a_2 & a_3 \\ 
            \end{vmatrix} = \hat{i}(b_2a_3-a_2b_3)+\hat{j}(a_1b_3-a_3b_1)+\hat{k}(a_2b_1-a_1b_2)
        \end{equation*}
    
    \item Por ejemplo:
        \begin{equation*}
            \begin{vmatrix}
                \hat{i} & \hat{j} & \hat{k} \\ 
                2 & 3 & 0 \\ 
                1 & 0 & 5 \\ 
            \end{vmatrix} = \hat{i}\begin{vmatrix}
                3 & 0 \\ 
                0 & 5 \\ 
            \end{vmatrix} - \hat{j}\begin{vmatrix}
                2 & 0 \\ 
                1 & 5 \\ 
            \end{vmatrix} + \hat{k} \begin{vmatrix}
                2 & 3 \\ 
                1 & 0 \\ 
            \end{vmatrix}
        \end{equation*}
        \[
          \therefore \vec{a}\times \vec{b} = 15\hat{i}-10\hat{j}-3\hat{k}
        \]
    
    \item Verifique $\vec{a}\times \vec{b}$ es ortogonal a $\vec{a}$ \& a $\vec{b}$.
        \begin{align*}
            (\vec{a}\times \vec{b})\cdot \vec{a} & = \langle 15,-10,-3 \rangle \cdot \langle 2,3,0 \rangle = 30-30+0 = 0 \therefore \text{  son ortogonales  } \\ 
            (\vec{a}\times \vec{b}) \cdot \vec{b} & = \langle 15, -10,-3\rangle \cdot \langle 1,0,5 \rangle = 15+0-15 = 0 \therefore \text{  son ortogonales  } \\ 
            \\ 
            & \vec{a}\times \vec{b} \perp a_1b \\ 
        \end{align*}
    
    \item Aclaración: en dos dimensiones $\vec{a}\times \vec{b} = \begin{vmatrix}
        \hat{i} & \hat{j} \\ 
        a_1 & a_2 \\ 
        b_1 & b_2 \\ 
    \end{vmatrix} $  No es posible evaluarlo.
    
    \item Existen en tres dimensiones pero si se intenta evaluar en cuatro dimensiones la siguiente matriz no es posible:
    \[
        \text{  En 3-D: } \exists \text{  En 4-D:  } \nexists 
    \]
        \[
           \vec{a}\times \vec{b} = \begin{vmatrix}
              \hat{i} & \hat{j} & \hat{k} & \hat{l} \\ 
              1 & 0 & 2 & 3 \\ 
              4 & 1 & 5 & -2 \\ 
          \end{vmatrix}
        \]
        No es posible evaluarlo.
    
    \item Ejemplo:
          \begin{align*}
            \begin{vmatrix}
                \hat{i} & \hat{j} & \hat{k} \\ 
                1 & 0 & 5 \\ 
                2 & 3 & 0 \\ 
            \end{vmatrix} = \hat{i}\begin{vmatrix}
                0 & 5 \\ 
                3 & 0 \\ 
            \end{vmatrix} - \hat{j}\begin{vmatrix}
                1 & 5 \\ 
                2 & 0 \\ 
            \end{vmatrix} + \hat{k}\begin{vmatrix}
                1 & 0 \\ 
                2 & 3 \\ 
            \end{vmatrix} \\ 
            = 15\hat{i}+10\hat{j}+3\hat{k}\\ 
          \end{align*}
          Entonces... en general: 
          \[
                \vec{a}\times \vec{b} = -(\vec{b}\times \vec{a})
          \]
\end{itemize}























\end{document}
