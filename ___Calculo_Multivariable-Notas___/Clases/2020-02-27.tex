
\section{Derivadas parciales, rectas tangentes y planos tangentes}

%----------------------------------------------------------------------------------------
\subsection{Interpretación de la derivada parcial}
\begin{itemize}
    \item $\mathbb{C}$ curva de intersección entre $z=f(x,y)$ y $y=b$.
    \item Recta tantente a eta curva en el punto $(a,b,f(a,b))$:
        \[
          \text{  Derivada : }\; f_x(x,b) \quad \quad \text{  Pendiente:  }\; f_x(a,b)
        \]
    
    \item Derivadas parciales: $f_x(a,b)$ resulta ser la pendiente de la recta tangente a la curva $f(x,b)$ en la dirección de $x$.
        \[
          L = \left\langle a,b,f(a,b) \right\rangle + t \left\langle 1,0,f(a,b) \right\rangle \quad \quad \text{  donde:  } \; x =t, y=b,z=f(t,b) 
        \]
    
    \item  Para encontrar $L_2$ $x=a$:
        \begin{center}
           \begin{align*}
               x&=a, y=t, x=f(z,y) \;\implies\; z_y=f_y(a,y) \; \implies \; z_y=f_y(a,b) \\ 
               z_y&=f(a,b) \; \text{  es la pendiente de la tangente  a la curva   }\; f(a,y) \; \text{  en la dirección de   }\; y \\   
               L_2 &= \left\langle a,b,f(a,y) \right\rangle + t \left\langle 0,1,f_y(a,b) \right\rangle \\ 
           \end{align*}
        \end{center}
    
    \item Estas dos rectas se utilizan para construir un plano tangente a la superficie.
    \item La ecuación del plano es un plano que es paralelo a $L_1$ \& $L_2$.
        \begin{center}
           \begin{align*}
               L_1&= \left\langle a,b,f(a,b) \right\rangle + t \overbrace{\left\langle 1,0,f(a,b) \right\rangle}^{v_1} \\ 
               L_2&= \left\langle a,b,f(a,b) \right\rangle + t \underbrace{\left\langle 0,1,f(a,b) \right\rangle}_{v_2} \\ 
           \end{align*}
        \end{center}
        La ec. vectorial:
            \[
              \hat{n} \cdot (-r_0)=0 \quad \vec{r}_0  = \left\langle a,b,f(a,b) \right\rangle 
            \]
        % \begin{center}
        % %    \begin{align*}
        % %        \hat{n}= v_1 \times v_2 = \begin{matrix}
                   
        % %        \end{matrix}
        % %    \end{align*}
        % \end{center}
        terminar excursión.
\end{itemize}


%----------------------------------------------------------------------------------------
\subsection{Ejercicios}
\begin{itemize}
    \item Encuentre el plano tangenge a la superficie $z=\ln(x-2y)$ en el punto $(3,1,0)$:
        \begin{center}
           \begin{align*}
               f(a,b) \quad f_x(a,b) \quad f_y(a,b) \quad a=3,\; b=1 \\ 
               f(3,1) = \ln(3-2) = \ln(1) = 0 \\ 
               \frac{\partial f}{\partial x } = \frac{1}{x-2y}  \quad \frac{\partial f}{\partial x } \Big|_{}^{(3,1)} = \frac{1}{3-2} = 1 \\ 
               \frac{\partial f}{\partial y } = \frac{-2}{x-2y} \quad \frac{\partial f}{\partial y } \Big|_{}^{(3,1)} = \frac{-2}{3-2} = -2 \\ 
               \text{  La ecuación del plano tangente:  } \quad \begin{matrix}
                    z = f(3,1)+f_x(x-3)+f_y(y-1) \\ 
                    z = 0 + x-3-2y+2 \\ 
                    \therefore \quad  z = x-2y-1 \\ 
               \end{matrix} \\ 
           \end{align*}
        \end{center}
\end{itemize}


%%%%%%%%%%%%%%%%%%%%%%%%%%%%%%%%%%%%%%%%%%%%%%%%%%%%%%%%%%%%%%%%%%%%%%%%%%%%%%%%%%%%%%%%%%
\section{Aproximaciones lineales}
\begin{itemize}
    \item La aproximación lienal de $z=f(x,y)$, linearización.
    \item La aproximación lineal de $z$ en $(a,b)$ es el plano tangente a la superficie.
        \[
          L(x,y)= f(a,b) + \frac{\partial }{} 
        \]
\end{itemize}


%----------------------------------------------------------------------------------------
\subsection{Ejercicios}
Considere la función $f(x,y)=\sqrt[]{2x+2e^y}$: 
\begin{itemize}
    \item Encuentre la aproximación lineal de $f$ en el punto $(7,0)$: \newline Encuentre $f(7,0) \quad f_y(7,0)$
        \begin{center}
           \begin{align*}
               f(7,0)&= \sqrt{14+2} = 4 \\ 
               f_x(x,y)&= (2x+2e^y)^{-\frac{1}{2} } \quad \quad f_x(0,7)= \frac{1}{\sqrt{14+2}} ? \frac{1}{4} \\ 
               f_y(x,y) &= \frac{e^y}{\sqrt{2x+2e^y}} \quad \quad f_y(7,0) = \frac{1}{\sqrt{14+2}} ? \frac{1}{4} \\ 
               \therefore &\; \text{  La aproximación lineal o plano tangente:   }\; L= 4+\frac{1}{4} (x-7) + \frac{1}{4} y \\ 
               \text{  Cerca de (7,0): }\; &\sqrt[]{2x+2e^y} \approx \frac{9}{4} + \frac{1}{4} x + \frac{1}{4} y \\ 
           \end{align*}
        \end{center}
    
    \item Utilice la aproximación lineal para aproximar el valor de $\sqrt{8+2e}$ : 
        \begin{center}
           \begin{align*}
               f(4,1) = \sqrt{8+2e} \approx 3.5 \approx L(4,1) \\ 
               L(4,1) = \frac{9}{4} + \frac{4}{4} + \frac{1}{4} = \frac{7}{2} = 3.5 \\ 
               \text{  En realidad  : }\; \sqrt{8+2e} \approx 3.665592 \\ 
           \end{align*}
        \end{center}
    
    \item Ejercicio 3: Encuentre la aproximación lineal de $g(x,y)=1+\ln (xy-5)$ en el punto $(2,3)$:
        \begin{center}
           \begin{align*}
               g(2,3)&=1+2 \ln (6-5) = 1+0 = 1 \\ 
               g_x(x,y) &= 0 + 1\cdot \ln (xy-5) + \frac{xy}{xy-5}  \\ 
               g_x(2,3) &= \ln (1) + \frac{6}{6-5} = 0 + \frac{6}{1} = 6 \\ 
               g_y(x,y) &= 0 + \frac{x\cdot x}{xy-5} \\ 
               g_y(2,3) &= \frac{4}{6-5} = 4 \\  
           \end{align*}
           La aproximación lineal entonces es:  
            \begin{align*}
                \therefore \\ 
                L(x,y) &= 1 + 6(x-2) + 4(y-3) \\ 
                L(x,y) &= -23 + 6x + 4y \\ 
            \end{align*}
        \end{center}
\end{itemize}



%%%%%%%%%%%%%%%%%%%%%%%%%%%%%%%%%%%%%%%%%%%%%%%%%%%%%%%%%%%%%%%%%%%%%%%%%%%%%%%%%%%%%%%%%%
\section{12.4 Derivadas implicitas y 12.5 Regla de la cadena }
\begin{itemize}
    \item Funciones 2 variables $z=f(x,y)$
    \item Explícita: $z$ no está sólo en función de $x$ \& $y$.
    \item Ejemplos: $x^2+y^2+z^2=16$, $\sqrt[]{z^2-x^2}=y+z$
    \item \textbf{¿}Cómo se encuetnran $\frac{\partial z}{\partial x }$ \& $\frac{\partial z}{\partial y} $ \textbf{?}:
        \begin{itemize}
            \item Implicita $x^2+y^2+z^2=16$ es una esfera de 4 (rango [-4,4]) en dos hemisferios:
                \[
                  z = +\sqrt{16-x^2-y^2}
                \]
        \end{itemize}
        \begin{center}
           \begin{align*}
               \frac{\partial z}{\partial x} &= \frac{1}{2} (16-x^2-y^2)^{-\frac{1}{2} }(-2x) = \frac{-x}{\sqrt{16-x^2-y^2}} = - \frac{x}{z} \\ 
               \frac{\partial z}{\partial y} &= -\frac{y}{z} \\  
           \end{align*}
        \end{center}
    
    \item Derivación implicita, se pueden encontrar $z_x$ \& $z_y$ sin necesidad de resolver para $z$.
        \begin{center}
           \begin{align*}
               x^2+y^2+z^2=16 \quad \text{  z \& y son independientes  } \\ 
               \frac{\partial }{\partial x}(x^2+y^2+z^2(x,y)) = \frac{\partial }{\partial x} (16) \\ 
               2x+0+2z \frac{\partial z}{\partial x} = 0 \\ 
               2z \frac{\partial z}{\partial x} = -2x  \quad  \implies \quad  \frac{\partial z}{\partial x} ? \frac{-x}{z} \\ 
                \frac{\partial }{\partial y} (x^2+y^2+z^2) = \frac{\partial }{\partial y}(0) \\ 
                0+2y+2z \frac{\partial z}{\partial y} = 0 \quad \implies \quad \frac{\partial z}{\partial y} = \frac{-y}{z} \\  
           \end{align*}
        \end{center}
\end{itemize}


\subsection{Derivación parcial implícita abreviada}
\begin{itemize}
    \item $x^2+y^2+z^2=16$ como $x \ln (y) + x^2 \sqrt[]{1+x+z} = k$
    \item Forma implícita: $F(x,y,z(x,y))=$ constante. $\frac{\partial z}{\partial x} $ use la regla de la cadena.
        \begin{center}
           \begin{align*}
               \frac{\partial F}{\partial x} &+ \frac{\partial F}{\partial z} \frac{\partial z}{\partial x}  = 0  \quad \implies \quad z_x= - \frac{f_x}{f_z} \\ 
               \frac{\partial f}{\partial y} &+ \frac{\partial f}{\partial z} \frac{\partial z}{\partial y} = 0 \quad \implies  \quad z_y = - \frac{f_y}{f_z} \\ 
           \end{align*}
        \end{center}
\end{itemize}


%----------------------------------------------------------------------------------------
\subsection{Ejercicios}
Encuentre las primeras derivadas parciales de $z$.
\begin{enumerate}
    \item $\ln(zy)+9z-xyz = 1 $:
        \begin{center}
           \begin{align*}
               \begin{matrix}
                   F_x = -yz \\ 
                   F_y = y^{-1}+0-xy \\ 
                   F_z = z^{-1}+9-xy \\ 
               \end{matrix}
               \begin{matrix}
                   \frac{\partial z}{\partial x} = -\frac{F_x}{F_y} = \frac{yz}{z^{-1}+9-xy} \\ 
                   \frac{\partial z}{\partial y} = \frac{xz-y^{-1}}{z^{-1}+9-xy} \\    
               \end{matrix} \\ 
           \end{align*}
        \begin{itemize}[label=\#]
            \item Sin derivación parcial implícita
            \item $z(x,y)$ agregue $z_x$ cada vez que aparece $z$.
        \end{itemize}
        \begin{align*}
            \frac{yz_x}{z_y} + 9 z_x - y_x
        \end{align*}
        \end{center}
\end{enumerate}


















