% \documentclass{article}
\title{Temporary}
\author{David Gabriel Corzo Mcmath}
\date{\today}
%%%%%%%%%%%%%%%%%%%%%%%%%%%%%%%%%%%%%%%%%%%%%%%%%%%%%%%%%%%%%%%%%%%%%%%%%%%%%%%%%%%%%%%%%%%%%%%%%%%%%%%%%%%%%%%%%%%%%%%%%%%%%%%%%%%%%%%%%%%%%%%
\usepackage[margin = 1in]{geometry}
\usepackage{graphicx}
\usepackage{fontenc}
\usepackage{pdfpages}
\usepackage[spanish]{babel}
\usepackage{amsmath}
\usepackage{amsthm}
\usepackage[utf8]{inputenc}
\usepackage{enumitem}
\usepackage{mathtools}
\usepackage{import}
\usepackage{xifthen}
\usepackage{pdfpages}
\usepackage{transparent}
\usepackage{color}
\usepackage{fancyhdr}
\usepackage{lipsum}
\usepackage{sectsty}
\usepackage{titlesec}
\usepackage{calc}
\usepackage{lmodern}
\usepackage{xpatch}
\usepackage{blindtext}
\usepackage{bookmark}
\usepackage{fancyhdr}
\usepackage{xcolor}
\usepackage{tikz}
\usepackage{blindtext}
\usepackage{hyperref}
\usepackage{listing}
\usepackage{spverbatim}
\usepackage{fancyvrb}
\usepackage{fvextra}
\usepackage{amssymb}
\usepackage{pifont}
\usepackage{longtable}
\usetikzlibrary{arrows,shapes}
%%%%%%%%%%%%%%%%%%%%%%%%%%%%%%%%%%%%%%%%%%%%%%%%%%%%%%%%%%%%%%%%%%%%%%%%%%%%%%%%%%%%%%%%%%%%%%%%%%%%%%%%%%%%%%%%%%%%%%%%%%%%%%%%%%%%%%%%%%%%%%%






% \date{2020-Feb-11 10:05:03}
% \begin{document}

\section{13.2 Cálculo de funciones vectoriales}
\begin{itemize}
    \item Derivadas:
        \[
            \vec{r}\,'(t)=\left\langle f'(t),g'(t),h'(t) \right\rangle \\ 
        \]
    
    \item Vector Tangente:
        \[
          \vec{r}\, ' (t) 
        \]
    
    \item Tangente unitario:
        \[
          \vec{T}(t)=\frac{r'(t)}{\left| r'(t) \right| } 
        \]
    
    \item Integrales indefinidas:
        \begin{align*}
          \int_{}^{}\left\langle f,g,h \right\rangle dt = \left\langle F+C_1,G+C_2,H+C_3 \right\rangle \\
          \int_{}^{}\vec{r}(t) dt  = \vec{R}(t) + \vec{C} \\
          \vec{R} \quad \text{ vector de Antiderivadas  } \\ 
          \vec{C} \quad \text{  Vector de constantes  } \\   
        \end{align*}
    
    \item Integrales definidas:
        \[
          \int_{a}^{b}\vec{r}(t)dt = \hat{i} \int_{a}^{b}f(t)dt+ \hat{j} \int_{a}^{b}g(t)dt + \hat{k} \int_{a}^{b}h(t)dt
        \]
\end{itemize}


%%%%%%%%%%%%%%%%%%%%%%%%%%%%%%%%%%%%%%%%%%%%%%%%%%%%%%%%%%%%%%%%%%%%%%%%%%%%%%%%%%%%%%%%%%%%%%%%%%%
\section{Ejercicios de integración}
\begin{enumerate}
    \item $\int_{0}^{1}\left[\frac{4}{1+t^2}\hat{i} + sec^2(\frac{\pi t}{4} )\right] dt$:
        \begin{align*}
            4 \hat{i} \times \tan^{-1}(t) \Big|_0^1 + \hat{k} \times \tan(\frac{\pi t}{4} ) \Big|_{0}^{1} \\ 
            I_i = 4 \hat{i}  \frac{\pi}{4} + \hat{k} \frac{4}{\pi } = pi \hat{i} + \hat{k} \frac{4}{\pi } = \left\langle \pi,0,\frac{4}{\pi } \right\rangle \\ 
        \end{align*}
    
    \item $\int_{}^{}\left\langle te^{t^2},te^t,\frac{q}{\sqrt[]{1-t^2}} \right\rangle dt$ :
        \begin{align*}
            x: \quad \int_{}^{}e^{t^2}t\,dt =& \frac{1}{2} \int_{}^{}e^{u}du = \frac{1}{2} e^{t^2} +C_1 \\ 
                &\begin{matrix}
                    u = t^2 \\ 
                    du = 2tdt \\ 
                \end{matrix} \\ 
            y: \quad \int_{}^{} te^{t} dt = te^{t} - \int_{}^{} te^t-e^t+C_2 \\ 
                \begin{matrix}
                    u = t \quad dv = e^t dt \\ 
                    du = dt \quad v = e^t \\ 
                \end{matrix}
            t: \quad \int_{}^{}\frac{1}{1-t^2} dt = \frac{\cos(\theta)}{\sin(\theta)}d\theta = \int_{}^{} d\theta = \underbrace{\theta + C_3}_{\sin^{-1}(t)+C_3} = \sin^{-1}(t)+C_3 \\
            \therefore \quad \int_{}^{} \left\langle te^{t^2},te^t,\frac{1}{\sqrt{1-t^2}} \right\rangle dt = \frac{1}{2} et^{t^2}+C_1,te^t-e^t+C_2,\sin^1(t) + C_3 \\  
        \end{align*}
\end{enumerate}


%%%%%%%%%%%%%%%%%%%%%%%%%%%%%%%%%%%%%%%%%%%%%%%%%%%%%%%%%%%%%%%%%%%%%%%%%%%%%%%%%%%%%%%%%%%%%%%%%%%
\section{Movimiento en el espacio}
Dado el vector posición $\vec{r}(t)$ de un objeto: 
\begin{itemize}
    \item Vector velocidad:
        \[
          \vec{c}(t) = \vec{r}\, ' (t)
        \]
    
    \item Vector aceleración:
        \[
          \vec{a}(t) = \vec{v}\, (t) = \vec{r}\, '' (t)
        \]
    
    \item Rapidez:
        \[
          \left| \vec{v}(t) \right| 
        \]
    
    \item Distancia:
        \[
          \left| \vec{r}(t) \right| 
        \]    
\end{itemize}

Dado el vector de aceleración $\vec{a}(t)$ :
\begin{itemize}
    \item Velocidad:
        \[
          \vec{v}(t) = \int_{}^{}\vec{a}(t)dt + \vec{C}_1
        \]
    
    \item Desplazamiento o posición:
        \[
          \vec{r}(t) = \int_{}^{} \vec{v}(t)dt + C_2 
        \]
\end{itemize}


\subsection{Ejercicios}
\begin{enumerate}
    \item Encuentre la velocidad, aceleración y rapidez dada la posición del objeto:
        \begin{align*}
            \vec{r}(t) = \hat{i} t  + 2 \hat{j} \cosh(4t) + 3 \hat{k} \sinh(3t) \\ 
            \text{  Encontramos velocidad:  } \quad \vec{r}\, ' (t) = \vec{v}(t) = \hat{i} + 8 \hat{j} \sinh(4t)+9 \hat{k} \cosh(3t) \\ 
            \text{  Encontramos la aceleración:  } \quad \vec{r}\, ''(a) = \vec{a}(t) + 32 \hat{j} \cosh(4t) + 27 \hat{k} \sinh(3t) \\ 
            \text{  Encontramos la rapidez:   } \quad \left| \vec{v}(t) \right| = \sqrt{1+64\sinh(4t)+81\sinh^2(3t)} \\ 
            \text{  Encontramos la distancia:   } \quad \left| \vec{r}(t) \right| = \sqrt{t^2+4\cosh^2(4t)+9\sinh^2(3t)} \\ 
        \end{align*}
        \begin{itemize}[label=\#]
            \item Tarea \# 6: Integrales func. vectoriales 14.1 Funciones en varias variables.
            \item Tarea opcional consolidado: 12,13,14.1 
        \end{itemize}
    
    \item Encuentre la velocidad y posición del objeto dada $\vec{a}(t)$ y las condiciones iniciales:
        \begin{center}
            \begin{align*}
                \vec{a}(t)= 6 t \hat{i}+ \hat{j}  \cos(t)- \hat{k} \sin(2t), \quad \vec{v}(0) = \begin{matrix}
                    \hat{i} + \hat{k} \\ 
                    \vec{r}(0) = 2 \hat{j}  - \hat{k} \\ 
                \end{matrix} \\ 
                \text{  Velocidad:   } \quad \int_{}^{}\vec{a}(t)dt \\ 
                \vec{v}(t) = \left\langle 3t^2+C_1, \sin(t)+C_2, \frac{1}{2}\cos(2t)+C_3 \right\rangle \\ 
                \text{  Encuentro   } \; \vec{v}(0) = \left\langle C_1,C_2,\frac{1}{2} + C_3 \right\rangle  = \left\langle 1,0,1 \right\rangle \\
                \text{  Resolver para las constantes:   } \quad \begin{matrix}
                    C_1 = 1, \\  C_2 = 0, \\  \frac{1}{2} + C_3 = 1 \quad \implies  \quad C_3 = \frac{1}{2} \\ 
                \end{matrix} \\ 
                \text{  Posición:   } \quad \int_{}^{}\vec{v}(t)dt \\ 
                \vec{r}(t) \left\langle t^3+t+d_1, -\cos(t)+d_2, \frac{1}{4}\sin(2t) + \frac{t}{2} + d_3 \right\rangle \\ 
                \vec{r}(0) = \left\langle \underbrace{d_1,-1+d_2,d_3}_{\begin{matrix}
                    d_1 = 0 \\ 
                    -1+d_2=2 \implies d_2 = 3 \\ 
                    d_3 = -1 \\ 
                \end{matrix}} \right\rangle  \\
                \text{  Posición:   }\quad \vec{r}(t)= \left\langle t^3+t,3-\cos(t),\frac{1}{4}\sin(2t)+\frac{t}{2}-1 \right\rangle \\ 
            \end{align*}
        \end{center}
    
    \item $\vec{a}(t)= 8t \hat{i} + \sinh(t)\hat{j} - \hat{k} e^{\frac{t}{2} }$ : 
        \begin{center}
            \begin{align*}
                \underbrace{\vec{v}(0) = \vec{0}}_{\text{  Está en reposo  }} \quad \quad \vec{s}(0) = 2 \hat{i} + \hat{j} - 3 \hat{k}  \\ 
                \text{  Velocidad:   }\quad \vec{v}(t)= \left\langle 4t^2+C_1,\cosh(t)+C_2,-2e^{\frac{t}{2}}+C_3 \right\rangle \\ 
                \vec{v}(0) = \left\langle \underbrace{C_1,1+C_2,-2+C_3}_{\begin{matrix}
                    C_1 = 0, \\  C_2 = -1 \\ C_3 =2 \\ 
                \end{matrix}} \right\rangle = \left\langle 0,0,0 \right\rangle \\ 
                \vec{v}(t) = \left\langle 4t^2,\cosh(t)-1,-2e^{\frac{t}{2} +2} \right\rangle \\ 
                \text{  Posición:   } \quad \vec{r}(t) = \left\langle \frac{4}{3}t^3+C_1, \sinh(t)-t+C_2,-4e^{\frac{t}{2}}+2t+C_3 \right\rangle  \\ 
                \vec{r}(0) = \left\langle C_1,C_2,-4+C_3 \right\rangle = \underbrace{\left\langle 2,1,-3 \right\rangle }_{\begin{matrix}
                    C_2=1 \\ 
                    C_3 = -3+4 = 1 \\ 
                \end{matrix}} \\ 
                \vec{r}(t) = \left\langle \frac{4}{3}t^3+2,\sinh(t)-t+1,-4e^{\frac{t}{2}}+2t+1 \right\rangle \\  
            \end{align*}    
            \begin{itemize}[label=\#]
                \item Se evalúa el vector en 0 por que se quiere saber el valor de las constantes cuando están en reposo.
                \item Por defecto siempre evaluar en 0 para encontrar $C_1,C_2$\&$C_3$.
            \end{itemize}
        \end{center}
\end{enumerate}


%%%%%%%%%%%%%%%%%%%%%%%%%%%%%%%%%%%%%%%%%%%%%%%%%%%%%%%%%%%%%%%%%%%%%%%%%%%%%%%%%%%%%%%%%%%%%%%%%%%
\section{13.3 lOGN}
10.4 Ecs. Paramétricas de una curva en el plano de dos dimensiones era:
\begin{center}
   \begin{align*}
       \begin{matrix}
        x=f(t) \\ 
        y = g(t) \\ 
       \end{matrix}
   \end{align*}
\end{center}
\begin{itemize}
    \item La longitud de arco:
        \begin{align*}
            L = \int_{a}^{b} \sqrt{(x')^2+(y')^2+(z')^2} dt
        \end{align*}
    
    \item Función vectorial:
        \[
          \vec{r} = \left\langle f,g,h \right\rangle = \left\langle x,y,z \right\rangle \\ 
        \]
    
    
    \item Derivada de función vectorial:
        \[
          \vec{r}\,'= \left\langle x',y',z' \right\rangle 
        \]
    
    \item Magnitud: 
        \[
          \left| \vec{r}\,' \right| = \sqrt{(x')^2+(y')^2+(z')^2}
        \]
    
    \item En general: 
        \[
          L = \int_{a}^{b} = \left| \vec{r}\,'(t) \right|dt  
        \]
\end{itemize}



%%%%%%%%%%%%%%%%%%%%%%%%%%%%%%%%%%%%%%%%%%%%%%%%%%%%%%%%%%%%%%%%%%%%%%%%%%%%%%%%%%%%%%%%%%%%%%%%%%%
\section{Ejercicios}
Encuentre la longitud de las siguientes curvas:
\begin{enumerate}
    \item $\vec{r}(t)= \left\langle \cos(t),\sin(t),\ln(\cos) \right\rangle $ en $0 \le t \le \frac{\pi }{4} $
        \begin{center}
            \begin{align*}
                L = \int_{0 }^{\frac{\pi }{4}} \left| \vec{r}\, ' (t) \right| dt \\ 
                \vec{r}\,'(t) = \left\langle -\sin(t),\cos(t),\tan^2(t) \right\rangle \\ 
                \left| \vec{r}\,'(t) \right| = \sqrt{\sin^2(t)+\cos^2(t)+\tan^2(t)} = \sqrt{1+\tan^2(t)} = \sec^2(t) = \sec^2(t) \\ 
                L = \int_{0}^{\frac{\pi }{4}}\sec(t)dt = \ln\left| \sec(t)+\tan(t) \right| \Big|_{0}^{\frac{\pi }{4}} = \ln \left| \sec(\frac{\pi }{4}) + \tan(\frac{\pi }{4} ) \right| -  \ln \left| \sec(0) + \tan(0) \right|\\ 
                L = \ln \left| \frac{2}{\sqrt{2}} + 1  \right|  - \ln \left| 1 \right| = \ln \left| \sqrt{2}+1 \right| \\ 
            \end{align*}
        \end{center}
    
    \item $\vec{r}(t)= \left\langle 12t,8t^{\frac{3}{2}},3t^2 \right\rangle $ en $0 \le t \le 1$ :
        \begin{center}
           \begin{align*}
               \vec{r}\,'(t) = \left\langle 12,12t^{\frac{1}{2}},6t \right\rangle = 6 \left\langle 2,2t^{\frac{1}{2}},t \right\rangle  \\ 
               \left| \vec{r}\,'(t) \right| = 6 \sqrt{4+4t+t^2} = 6 \sqrt{(t+2)^2}= 6(t+2) \\ 
               L = \int_{0}^{1} (6t+12) dt = 3t^2+12t \Big|_{0}^{1} = 3+12 = 15 \\ 
           \end{align*}
        \end{center}
\end{enumerate}









% \end{document}
