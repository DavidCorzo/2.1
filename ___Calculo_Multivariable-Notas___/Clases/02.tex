\section{12.1.2 Distancias y superficies básicas}
\begin{itemize}
    \item En 2-D, La distancia entre $P_1(x_1,y_i)$ \& $P_2(x_2,y_2)$, se encontraba la distancia entre dos puntos estaba dada por el teorema de pitágoras. 
        \begin{center}
            \begin{itemize}
                \item Para encontrar la distancia entre $\displaystyle P_1$ y $\displaystyle P_2$, se utiliza la fórmula:
                    \[
                      d = \sqrt{\p{x_2-x_1} ^2+\p{y_2-y_1} ^2} \qq \text{ ó } \qq \p{x_2-x_1} ^2+\p{y_2-y_1} ^2 = d^2 
                    \]
            \end{itemize}
            \begin{tikzpicture}
                \begin{pgfonlayer}{nodelayer}
                    \node [style=darkdot] (0) at (-6.25, -0.5) {};
                    \node [style=darkdot] (1) at (-2.5, 5) {};
                    \node [style=none] (2) at (-2.5, -0.5) {};
                    \node [style=none] (3) at (-1.5, 2) {$\displaystyle y_2-y_1$};
                    \node [style=none] (4) at (-4.5, -1.25) {$\displaystyle x_2-x_1$};
                    \node [style=none] (5) at (-7, -0.75) {$P_1$};
                    \node [style=none] (6) at (-2.25, 5.5) {$P_2$};
                    \node [style=none] (19) at (-5, 2.5) {$d$};
                    \node [style=none] (20) at (3.5, 3.5) {$\displaystyle d=\sqrt{(x_2-x_1)^2+(y_2-y_1)^2}$};
                    \node [style=none] (21) at (3.5, 1) {$\displaystyle (x_2-x_1)^2+(y_2-y_1)^2=d^2$};
                \end{pgfonlayer}
                \begin{pgfonlayer}{edgelayer}
                    \draw (0) to (2.center);
                    \draw (2.center) to (1);
                    \draw (0) to (1);
                \end{pgfonlayer}
            \end{tikzpicture}
            \begin{itemize}
                \item La ecuación de circunferencia de radio $d$ centrada en $\displaystyle (x_1,y_1)$.
            \end{itemize}
         
            \begin{tikzpicture}
                \begin{pgfonlayer}{nodelayer}
                    \node [style=none] (0) at (-7.5, 0.75) {};
                    \node [style=none] (1) at (-2.625, 0.75) {};
                    \node [style=none] (2) at (-7.5, 0.75) {};
                    \node [style=none] (3) at (-2.625, 0.75) {};
                    \node [style=none] (4) at (-5.075, 0.825) {};
                    \node [style=none] (5) at (-3.625, 2.75) {};
                    \node [style=darkdot] (6) at (-5.075, 0.825) {};
                    \node [style=darkdot] (7) at (-3.625, 2.75) {};
                    \node [style=none] (8) at (-4.625, 2) {$d$};
                    \node [style=none] (9) at (-3.125, 3) {$(x,y)$};
                    \node [style=none] (10) at (-5.325, 0.325) {$(x_1,y_1)$};
                    \node [style=none] (11) at (2.125, 1) {Ec. circunferencia de radio $d$ centrada en $(x_1,y_1)$};
                \end{pgfonlayer}
                \begin{pgfonlayer}{edgelayer}
                    \draw [in=90, out=90, looseness=1.75] (0.center) to (1.center);
                    \draw [bend right=90, looseness=1.75] (2.center) to (3.center);
                    \draw (4.center) to (5.center);
                \end{pgfonlayer}
            \end{tikzpicture}            
        \end{center}    
        
        
    \item En 3-D, la diferencia entre $P_1(x_1,y_1,z_1)$ \& $P_2(x_2,y_2,z_2)$, calcule la diferencia entre $z_2$ y $z_1$.
        \[
          d = + \sqrt{(x_2-x_1)^2+(y_2-y_1)^2+(z_2-z_1)^2} 
        \]
        \begin{itemize}
            \item Tomar siempre el resultado positivo.
            \item Notación de $d=\left| p_2 \, p_1 \right| $.
            \item Si elevamos al cuadrado ambos lados, resultamos con la ecuación de una esfera y ya no una circunferencia.  
        \end{itemize}
        \[
            (x-x_1)^2+(y-y_1)^2+(z-z_1)^2 = d^2 
        \]
        \begin{itemize}
            \item La ecuación de una esfera de radio r centrada en $(x_1,y_1,z_1)$.
        \end{itemize}
    
    \item La esfera más utilizada es la que está centrada en el origen $(0,0,0)$ :
        \[
          x^2+y^2+z^2=r^2
        \]
        \begin{itemize}
            \item Radio: $r$
        \end{itemize}
        \begin{center}
            \begin{tikzpicture}
                \begin{pgfonlayer}{nodelayer}
                    \node [style=none] (0) at (-3, 0.25) {};
                    \node [style=none] (1) at (-3, 8.25) {};
                    \node [style=none] (2) at (5, 0.25) {};
                    \node [style=none] (3) at (-7, -5) {};
                    \node [style=none] (4) at (-7, -5) {};
                    \node [style=none] (5) at (5.25, -0.25) {$y$};
                    \node [style=none] (6) at (-7.25, -5.5) {$x$};
                    \node [style=none] (7) at (-3.75, 8.5) {$z$};
                    \node [style=none] (8) at (-4.25, 2.75) {};
                    \node [style=none] (9) at (3.75, 2.75) {};
                    \node [style=none] (10) at (-0.275, 6.325) {};
                    \node [style=none] (11) at (-0.25, -0.85) {};
                    \node [style=darkdot] (12) at (-0.25, 2.75) {};
                    \node [style=darkdot] (13) at (2.275, 5.6) {};
                    \node [style=none] (14) at (1, 4.75) {$d$};
                    \node [style=none] (15) at (3, 6) {$(x_2,y_2,z_2)$};
                    \node [style=none] (16) at (-0.25, 2.25) {$(x_1,y_1,z_1)$};
                    \node [style=none] (17) at (0, -2.25) {$\displaystyle \p{x-x_1}^2+\p{y-y_1}^2+\p{z-z_1}^2=d^2$};
                    \node [style=none] (18) at (-0.25, -3.5) {Ec. de una esfera de radio $r$ centrada en $(x_1,y_1,z_1)$};
                \end{pgfonlayer}
                \begin{pgfonlayer}{edgelayer}
                    \draw [style=line with arrow] (0.center) to (4.center);
                    \draw [style=line with arrow] (0.center) to (2.center);
                    \draw [style=line with arrow] (0.center) to (1.center);
                    \draw [style=dashed line, bend left=60, looseness=0.50] (8.center) to (9.center);
                    \draw [bend right, looseness=0.75] (8.center) to (9.center);
                    \draw [bend right=90, looseness=1.50] (8.center) to (9.center);
                    \draw [bend left=90, looseness=1.50] (8.center) to (9.center);
                    \draw [bend right] (10.center) to (11.center);
                    \draw [style=dashed line, bend right] (11.center) to (10.center);
                    \draw (12) to (13);
                \end{pgfonlayer}
            \end{tikzpicture}                        
        \end{center}
\end{itemize}


%%%%%%%%%%%%%%%%%%%%%%%%%%%%%%%%%%%%%%%%%%%%%%%%%%%%%%%%%%%%%%%%%%%%%%%%%%%%%%%%%%%%%%%%%%%%%%%%
\section{Ejercicios}
\begin{itemize}
    \item Encuentre el centro y radio de la esfera cuya ecuación es: 
        \[
            x^2+y^2+z^2+8x-6y+4z+4=0
        \]
        \begin{itemize}
            \item Tener en cuenta que es como que si estuviesen desarrollando la ecuación $x^2+y^2+z^2=r^2$ y agregando constantes.
            \item Hay que \textbf{completar al cuadrado}.
        \end{itemize}

        \begin{center}
            \begin{align*}
                x^2+y^2+z^2+8x-6y+4z+4&=0 \\ 
                x^2+8x+\square+y^2-6y+\square+z^2+4z+\square&=-4 \\ 
                \begin{matrix}
                    \text{  Para x:  }\, \left(\frac{8}{2}\right)^2 &= 16 \\ 
                    \text{  Para y:  }\, \left( \frac{6}{2}  \right)^2 &= 9 \\ 
                    \text{  Para z:  }\, \left(\frac{4}{2}\right)^2 &= 4 \\ 
                \end{matrix} \\ 
                x^2+8x+16+y^2-6y+9+z^2+4z+4&= -4 +16+9+4\\
                (x+4)^2+(y-3)^2+(z+2)^2&= \underbrace{25}_{r^2} \\ 
                \therefore \, \text{  La esfera se enfoca en centro:  }: (-4,3,-2) \\ 
                \therefore \text{  Radio:  } \, \sqrt[]{25}=5 \\ 
            \end{align*}
        \end{center}
        
        \begin{itemize}
            \item Tener en cuenta que $z=x^2+y^2$ no es una esfera, es una paraboloide.
        \end{itemize}
    
    \item Encontrar la distancia entre un punto y un plano coordenado, encuentre la distancia entre el punto $(1,3,5)$ y el plano $xz$.
        \begin{itemize}
            \item Vamos a estrellar ese punto contra el eje $xz$, la proyección del punto $P$ sobre el plano.
        \end{itemize}
        \begin{align*}
            \text{  Distancia entre   }\,\, P_1,P_2 \quad d=\sqrt[]{(1-0)^2+(3-0)^2+(5-5)^2} \\ 
            \text{  La proyección del punto (a,b,c) sobre el plano xz es el punto (a,0,c).  }\\ 
            \therefore \quad \text{  La distancia mínima entre p y el plano es:  } \\ 
            d=\left| 0+b^2+0 \right| = \left| b \right| \\ 
        \end{align*}
    
    
    \item ¿Cuál es la distancia entre el punto (1,3,5) y el plano xy?
        \begin{itemize}
            \item Asumo z=0
        \end{itemize}
        \begin{center}
           \begin{align*}
               d_{min}=&\sqrt[]{0+0+5^2}\\ 
               d_{min}=&5 \\ 
           \end{align*}
        \end{center}
    
    \item Ejercicio 6: Considere los puntos $A(3,0,-4),B(9,0,0)$ Y $C(0,1,\sqrt[]{15})$:    
        \begin{itemize}
            \item ¿Cuáles de los siguientes puntos está más cercano al origen?
            \item Hay que calcular la distancia de cada punto respecto del origen $(0,0,0)$.
            \item El origen se denota como $O(0,0,0)$
        \end{itemize}
        \begin{center}
           \begin{align*}
               d_{AO} =& \left| AO \right| \sqrt[]{9+0+16}=\sqrt[]{25}=5 \\ 
               d_{BO} =& \left| BO \right|= \sqrt[]{81+0+0} = \sqrt[]{81} = 9 \\ 
                d_{CO} 0 \left| CO \right| =& \sqrt[]{0+1+15} = \sqrt[]{16}=4 \\ 
           \end{align*}
        \end{center}
        \begin{itemize}
            \item El punto $C$ es el más cercano al origen.
        \end{itemize}
        \begin{itemize}
            \item ¿Cuáles de los puntos están sobre el plano $yz$? 
            \item Se asume x: 0
            \item $A$ y $B$ no están sobre el plano $yz$ $x\neq 0$.
            \item El punto $C$ $(0,1,\sqrt[]{15})$ si están sobre el plano $yz$.
            \item ¿Cuáles de los puntos está más cercano al plano $yz$? x=0:
                \begin{itemize}
                    \item Dado a que el punto C está en el plano yz su distancia es 0 entonces ese es el más cercano.
                \end{itemize}
        \end{itemize}
\end{itemize}





