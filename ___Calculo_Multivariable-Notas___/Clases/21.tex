\section{}
Caja rectangular    
\begin{figure}[H]
    \centering
    % \includegraphics[]{Clases/figs/2020-04-16_01} 
\end{figure}
\begin{center}
   \begin{align*}
       B = \sqb{a,b} \times \sqb{c,d} \\ 
       \iiint_{B}^{}f\p{x,y,z} dV = \int_{a}^{b}\int_{c}^{d}\int_{r}^{s} f(x,y,z) dz dy dx \\  
   \end{align*}
   Se intercambia el orden de integración, esto resulta en una permutación
\end{center}

\subsection{Ejercicios}
\begin{enumerate}
    \item $\displaystyle \iiint_{B}^{}\p{xy+3z^2} dV$ $\displaystyle B = \sqb{0,2} \times \sqb{1,0} \times \sqb{0,3} $ 
        \begin{center}
           \begin{align*}
               I_a &= \int_{0}^{1}\int_{0}^{3}\int_{0}^{2}\p{x,y+3z^2} dx dz dy \\ 
               &= \int_{0}^{1}\int_{0}^{3}\p{\frac{x^2y}{2} +3z^2x\evaluate{x=0}{2} } dz dy \\ 
                &= \int_{0}^{1}\int_{0}^{3}\p{2y+6z^2} dz dy \\ 
                &= \int_{0}^{1}\p{2yz+2z^3\evaluate{z=0}{z=3}   dy} \\ 
                &= \int_{0}^{1}\p{6y+54} dy \\ 
                &= \\ 
           \end{align*}
        \end{center}
    
    \item $\displaystyle \iiint_{B}^{}e^{x+y+z}dV$ $\displaystyle B = \sqb{0,\ln\p{ 2 }  } \times \sqb{0,\ln\p{ 3 }  } \times \sqb{0,\ln\p{ 4 }  } $  
        \begin{center}
           \begin{align*}
                I_b &= \int_{0}^{2}\int_{0}^{\ln\p{ 3 }  }\int_{0}^{\ln\p{ 4 }  } e^{x+y+z} dz dy dx \\ 
                I_b &= \int_{0}^{\ln\p{ 2 }  }\int_{0}^{\ln\p{ 3 }  }e^{x+y} \\ 
           \end{align*}
        \end{center}
\end{enumerate}


%%%%%%%%%%%%%%%%%%%%%%%%%%%%%%%%%%%%%%%%%%%%%%%%%%%%%%%%%%%%%%%%%%%%%%%%%%%%%%%%%%%%%%%%%%
\section{Integrales triples sobre unu sólido general}
\[
  u_1(x,y) \leq z \leq u_2(x,y) 
\]
D es la región de proyección del sólido $E$ sobre el plano $xy$.
\[
    \iiint_{E}^{}f(x,y,z)dV = \iint_{D}^{}\p{\int_{u_1(x,y)}^{u_2(x,y)} f(x,y,z)dz } dA  
\]
\[
  (x,y) \in D
\]

D como una región tipo I, tipo II o en polares.
\begin{center}
   \begin{align*}
       a \leq x \leq b \qq \qq a_1(x) \leq y \leq a_2(x) \\ 
    %    \iiint_{E}^{}f\; dV &= \iint_{D}^{}\p{int
    %    }  \\ 
   \end{align*}
\end{center}


\subsection{Sólido tipo II}
\begin{center}
   \begin{align*}
       u_1(y,z) \leq x \leq u_2(y,z) \\ 
   \end{align*}
   \begin{figure}[H]
       \centering
    %    \includegraphics[]{Clases/figs/} 
   \end{figure}
\end{center}
