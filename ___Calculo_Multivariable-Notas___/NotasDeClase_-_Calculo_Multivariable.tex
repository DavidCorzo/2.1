\documentclass{book}
\title{Cálculo Multivariable - Clases y apuntes de clase}
\author{David Gabriel Corzo Mcmath}
\date{2020-01-06}

%%%%%%%%%%%%%%%%%%%%%%%%%%%%%%%%%%%%%%%%%%%%%%%%%%%%%%%%%%%%%%%%%%%%%%%%%%%%%%%%%%%%%%%%%%%%%%%%%%%

\usepackage{grffile}
\usepackage[margin = 1in]{geometry}
\usepackage{graphicx}
\usepackage{fontenc}
\usepackage{pdfpages}
\usepackage[spanish]{babel}
\usepackage{amsmath}
\usepackage{amsthm}
\usepackage[utf8]{inputenc}
\usepackage{enumitem}
\usepackage{mathtools}
\usepackage{import}
\usepackage{xifthen}
\usepackage{pdfpages}
\usepackage{transparent}
\usepackage{color}
\usepackage{fancyhdr}
\usepackage{lipsum}
\usepackage{sectsty}
\usepackage{titlesec}
\usepackage{calc}
\usepackage{lmodern}
\usepackage{xpatch}
\usepackage{blindtext}
\usepackage{bookmark}
\usepackage{fancyhdr}
\usepackage{xcolor}
\usepackage{tikz}
\usepackage{blindtext}
\usepackage{hyperref}
\usepackage{listing}
\usepackage{spverbatim}
\usepackage{fancyvrb}
\usepackage{fvextra}
\usepackage{amssymb}
\usepackage{pifont}
\usepackage{longtable}
\usepackage{multirow}
\usepackage{esvect}
\usetikzlibrary{arrows,shapes}
%%%%%%%%%%%%%%%%%%%%%%%%%%%%%%%%%%%%%%%%%%%%%%%%%%%%%%%%%%%%%%%%%%%%%%%%%%%%%%%%%%%%%%%%%%%%%%%%
%%%%%%%%%%%%%%%%%%%%%%%%%%%%%%%%%%%%%%%%%%%%%%%%%%%%%%%%%%%%%%%%%%%%%%%%%%%%%%%%%%%%%%%%%%%%%%%%
% THIS IS TO CENTER AND DEDICATE A CHAPTER NAME AN ENTIRE PAGE 
\titleformat{\chapter}[display]{\vfill\filcenter}{{\filcenter\fontsize{48pt}{48pt}\usefont{T1}{cm}{m}{n}{\centering\chaptername}\fontsize{80pt}{80pt}\selectfont\thechapter}}{5pt}{\Huge\usefont{T1}{cm}{b}{n}\parbox{\textwidth-\widthof{\LARGE\sffamily{\centering\chaptername}}}}[\vfill\clearpage]\titlespacing*{\chapter}{0pt}{0pt}{50pt}
% TO SEPATE WHOLE PAGE DEDICATION IN THE TABLE OF CONTENTS 
\titleformat{name=\chapter,numberless}[display]{\filcenter}{{}}{5pt}{\Huge\usefont{T1}{cm}{b}{n}\centering}
% MAKE THE TITLE OF THE CHAPTER CENTER!!
\makeatletter\xpatchcmd{\@makeschapterhead}{\Huge \bfseries  #1\par\nobreak}{\Huge \bfseries\centering #1\par\nobreak}{\typeout{Patched makeschapterhead}}{\typeout{patching of @makeschapterhead failed}}\xpatchcmd{\@makechapterhead}{\huge\bfseries \@chapapp\space \thechapter}{\huge\bfseries\centering \@chapapp\space \thechapter}{\typeout{Patched @makechapterhead}}{\typeout{Patching of @makechapterhead failed}}\makeatother
% Clear the header and footer
\fancyhead{}\fancyfoot{} %
% Set the right side of the footer to be the page number 
{\fontfamily{mc}\selectfont\fancyfoot[L]{\thepage}\fancypagestyle{plain}{\renewcommand{\headrulewidth}{0pt}\fancyhf{}\fancyfoot[L]{\thepage}}} 
% SHADED CIRCLE IN PAGE NUMBER 
 \newcommand*\circled[1]{\tikz[baseline=(char.base)]{\node[shape=circle,fill=gray!50,inner sep=2pt] (char) {#1};}}\pagestyle{fancy}\fancyhf{}\fancyhead[EL]{\nouppercase\leftmark}\fancyhead[OR]{\nouppercase\rightmark}\fancyfoot[C]{\circled{\thepage}}\fancypagestyle{plain}{\fancyhf{}\fancyfoot[C]{\circled{\thepage}}\renewcommand{\headrulewidth}{0pt}}
%%%%%%%%%%%%%%%%%%%%%%%%%%%%%%%%%%%%%%%%%%%%%%%%%%%%%%%%%%%%%%%%%%%%%%%%%%%%%%%%%%%%%%%%%%%%%%%%
%%%%%%%%%%%%%%%%%%%%%%%%%%%%%%%%%%%%%%%%%%%%%%%%%%%%%%%%%%%%%%%%%%%%%%%%%%%%%%%%%%%%%%%%%%%%%%%%

% Here the document starts
\begin{document}
\maketitle
\tableofcontents

% Begin block styles
\tikzstyle{decision}=[diamond, draw, fill=white!20,text width=4.5em, text badly centered, node distance=3cm, inner sep=0pt]
\tikzstyle{block}=[rectangle, draw, fill=white!20,text width=5em, text centered, rounded corners, minimum height=4em]
\tikzstyle{line}=[draw, -latex']
\tikzstyle{cloud}=[draw, ellipse,fill=white!20, node distance=3cm,minimum height=2em]
% End block styles 


\chapter{Clase - 2020-01-07}
% \date{2020-Feb-06 10:23:27}
% \documentclass{article}
\title{Temporary}
\author{David Gabriel Corzo Mcmath}
\date{\today}
%%%%%%%%%%%%%%%%%%%%%%%%%%%%%%%%%%%%%%%%%%%%%%%%%%%%%%%%%%%%%%%%%%%%%%%%%%%%%%%%%%%%%%%%%%%%%%%%%%%%%%%%%%%%%%%%%%%%%%%%%%%%%%%%%%%%%%%%%%%%%%%
\usepackage[margin = 1in]{geometry}
\usepackage{graphicx}
\usepackage{fontenc}
\usepackage{pdfpages}
\usepackage[spanish]{babel}
\usepackage{amsmath}
\usepackage{amsthm}
\usepackage[utf8]{inputenc}
\usepackage{enumitem}
\usepackage{mathtools}
\usepackage{import}
\usepackage{xifthen}
\usepackage{pdfpages}
\usepackage{transparent}
\usepackage{color}
\usepackage{fancyhdr}
\usepackage{lipsum}
\usepackage{sectsty}
\usepackage{titlesec}
\usepackage{calc}
\usepackage{lmodern}
\usepackage{xpatch}
\usepackage{blindtext}
\usepackage{bookmark}
\usepackage{fancyhdr}
\usepackage{xcolor}
\usepackage{tikz}
\usepackage{blindtext}
\usepackage{hyperref}
\usepackage{listing}
\usepackage{spverbatim}
\usepackage{fancyvrb}
\usepackage{fvextra}
\usepackage{amssymb}
\usepackage{pifont}
\usepackage{longtable}
\usetikzlibrary{arrows,shapes}
%%%%%%%%%%%%%%%%%%%%%%%%%%%%%%%%%%%%%%%%%%%%%%%%%%%%%%%%%%%%%%%%%%%%%%%%%%%%%%%%%%%%%%%%%%%%%%%%%%%%%%%%%%%%%%%%%%%%%%%%%%%%%%%%%%%%%%%%%%%%%%%






% \begin{document}


%%%%%%%%%%%%%%%%%%%%%%%%%%%%%%%%%%%%%%%%%%%%%%%%%%%%%%%%%%%%%%%%%%%%%%%%%%%%%%%%%%%%%%%%%%%%%%%%
\section{12.1 Sistema tridimensional de coordenadas}
Para localizar un punto en un plano, se necesitan dos números. 
\begin{itemize}
    \item $a$ la coordenada $x$ 
    \item $b$ la coordenada $y$ 
\end{itemize}
En el plano $\displaystyle \mathbb{R}^2$ 

\begin{itemize}
    \item Los ejes de coordenadas son perpendiculares entre sí.
\end{itemize}

\begin{center}
    \begin{tabular}{ c }
        \multicolumn{1}{l}{\textbf{\large Sistema de coordenadas en dos dimensiones: }} \\
    \begin{tikzpicture}
        \begin{pgfonlayer}{nodelayer}
            \node [style=none] (0) at (-5.75, 5) {};
            \node [style=none] (1) at (-5.75, -1.75) {};
            \node [style=none] (2) at (1.75, -1.75) {};
            \node [style=none] (8) at (1.05, 0.2) {$(a,b)$};
            \node [style=none] (9) at (-3.675, 3.675) {$(b,a)$};
            \node [style=none] (11) at (-6.5, 5) {$y$};
            \node [style=none] (12) at (1.75, -2.25) {$x$};
            \node [style=none] (13) at (-5.50, -1.50) {};
            \node [style=none] (14) at (-5.50, -1.725) {};
            \node [style=none] (15) at (-5.75, -1.50) {};
            \node [style=dot] (16) at (-4.25, 3.5) {};
            \node [style=dot] (17) at (0.75, -0.25) {};
            \node [style=none] (18) at (-4.25, -1.75) {};
            \node [style=none] (19) at (-5.75, -0.25) {};
            \node [style=none] (20) at (-5.75, 3.5) {};
            \node [style=none] (21) at (0.75, -1.75) {};
            \node [style=none] (26) at (-4.25, 3.75) {};
            \node [style=none] (27) at (-5.75, 3.75) {};
            \node [style=none] (28) at (-6, -1.75) {};
            \node [style=none] (29) at (-6, 3.5) {};
            \node [style=none] (30) at (-5.725, -2) {};
            \node [style=none] (31) at (0.775, -2) {};
            \node [style=none] (32) at (0.90, -0.25) {};
            \node [style=none] (33) at (0.90, -1.75) {};
        \end{pgfonlayer}
        \begin{pgfonlayer}{edgelayer}
            \draw [style=line with arrow] (1.center) to (0.center);
            \draw [style=line with arrow] (1.center) to (2.center);
            \draw (14.center) to (13.center);
            \draw (13.center) to (15.center);
            \draw [style=dashed line] (18.center) to (16.center);
            \draw [style=dashed line] (16.center) to (20.center);
            \draw [style=dashed line] (19.center) to (17.center);
            \draw [style=dashed line] (17.center) to (21.center);
            \rbraceontikz{(27.center)}{(26.center)}{$f(y)$};
            \lbraceontikz{(29.center)}{(28.center)}{$y$}{xshift=-1cm};
            \lbraceontikz{(30.center)}{(31.center)}{$x$}{xshift=0cm,yshift=-1cm};
            \lbraceontikz{(33.center)}{(32.center)}{$f(x)$}{xshift=1cm};
        \end{pgfonlayer}
    \end{tikzpicture} \\
    \end{tabular}
\end{center}
%----------------------------------------------------------------------------------------


\begin{itemize}
    \item En el sistema tridimensional de coordenadas rectangulares, cada punto en el espacio es una terna ordenada $\displaystyle \p{x,y,z} $.
        \begin{center}
           \begin{align*}
               \text{ Espacio:  } \qq \mathbb{R}^3 &= \cb{\p{x,y,z} \qq \text{ talque } \qq x,y,z \; \in \mathbb{R}} \\ 
               \mathbb{R}^3 &= \mathbb{R}^2 \times \mathbb{R} \\  
           \end{align*}
           \begin{itemize}
               \item $x$ transversal 
               \item $y$ horizontal 
               \item $z$ vertical 
               \item $\displaystyle z=f\p{x,y} $ 
           \end{itemize}
        \end{center}

        \begin{center}
            \begin{tabular}{ c }
                \multicolumn{1}{l}{\textbf{\large Sistema tridimensional de coordenadas: }} \\ 
                \begin{tikzpicture}
                        \begin{pgfonlayer}{nodelayer}
                            \node [style=none] (0) at (-3, 5) {};
                            \node [style=none] (1) at (-3, -1.25) {};
                            \node [style=none] (2) at (2.5, -1.25) {};
                            \node [style=none] (3) at (-6.5, -4.75) {};
                            \node [style=none] (4) at (-3.5, 5) {};
                            \node [style=none] (5) at (-3.5, 5) {$z$};
                            \node [style=none] (6) at (-6.5, -5.25) {$x$};
                            \node [style=none] (7) at (2.75, -1.75) {$y$};
                            \node [style=none] (8) at (-5.25, -3.5) {};
                            \node [style=none] (9) at (-2, -3.5) {};
                            \node [style=none] (10) at (0, -1.25) {};
                            \node [style=none] (12) at (-2.075, 1) {};
                            \node [style=none] (13) at (-3, 2.5) {};
                            \node [style=none] (14) at (-3, 2.5) {};
                            \node [style=dot] (15) at (-2.075, 1) {};
                            \node [style=none] (17) at (-1.5, 1.5) {$(a,b,c)$};
                            \node [style=none] (18) at (-5.6, -2.975) {$(a,0,0)$};
                            \node [style=none] (19) at (-2, -4) {$(a,b,0)$};
                        \end{pgfonlayer}
                        \begin{pgfonlayer}{edgelayer}
                            \draw [style=line with arrow] (1.center) to (3.center);
                            \draw [style=line with arrow] (1.center) to (0.center);
                            \draw [style=line with arrow] (1.center) to (2.center);
                            \draw (8.center) to (9.center);
                            \draw [style=dashed line] (9.center) to (10.center);
                            \draw [style=dashed line] (12.center) to (9.center);
                            \draw [style=dashed line] (14.center) to (12.center);
                            \draw [style=dasharr] (1.center) to (8.center);
                            \draw [style=dasharr] (8.center) to (9.center);
                            \draw [style=dasharr] (9.center) to (15);
                        \end{pgfonlayer}
                \end{tikzpicture} \\ 
            \end{tabular}
        \end{center}
        \begin{itemize}
            \item Las líneas punteadas se utilizan para simbolizar las partes de abajo, izquierda y detrás.
        \end{itemize}
    \item Las líneas punteadas se usan para simbolizar las partes debajo, izquierda y detrás.
\end{itemize}



%----------------------------------------------------------------------------------------
\subsection{Ejercicios}
Identifique las siguientes puntos: 
\begin{enumerate}
    \item $\displaystyle \p{0,0,0} $ 
        \begin{center}
            \begin{tikzpicture}
                \begin{pgfonlayer}{nodelayer}
                    \node [style=none] (0) at (-3, 2.25) {};
                    \node [style=none] (1) at (-3, -1.25) {};
                    \node [style=none] (2) at (0.75, -1.25) {};
                    \node [style=none] (3) at (-5, -3.25) {};
                    \node [style=none] (4) at (-3.5, 2.25) {};
                    \node [style=none] (5) at (-3.5, 2.25) {$z$};
                    \node [style=none] (6) at (-5, -3.75) {$x$};
                    \node [style=none] (7) at (1, -1.75) {$y$};
                    \node [style=darkdot] (8) at (-3, -1.25) {};
                    \node [style=none] (9) at (-2.25, -0.75) {$(0,0,0)$};
                \end{pgfonlayer}
                \begin{pgfonlayer}{edgelayer}
                    \draw [style=line with arrow] (1.center) to (3.center);
                    \draw [style=line with arrow] (1.center) to (0.center);
                    \draw [style=line with arrow] (1.center) to (2.center);
                \end{pgfonlayer}
            \end{tikzpicture}            
        \end{center}
    
    \item $\displaystyle \p{0,1,0} $ 
        \begin{center}
            \begin{tikzpicture}
                \begin{pgfonlayer}{nodelayer}
                    \node [style=none] (0) at (-3, 2.25) {};
                    \node [style=none] (1) at (-3, -1.25) {};
                    \node [style=none] (2) at (0.75, -1.25) {};
                    \node [style=none] (3) at (-5, -3.25) {};
                    \node [style=none] (4) at (-3.5, 2.25) {};
                    \node [style=none] (5) at (-3.5, 2.25) {$z$};
                    \node [style=none] (6) at (-5, -3.75) {$x$};
                    \node [style=none] (7) at (1, -1.75) {$y$};
                    \node [style=none] (9) at (-2.5, -2) {$(0,1,0)$};
                    \node [style=none] (10) at (-2.25, -1.25) {};
                    \node [style=darkdot] (11) at (-2.25, -1.25) {};
                \end{pgfonlayer}
                \begin{pgfonlayer}{edgelayer}
                    \draw [style=line with arrow] (1.center) to (3.center);
                    \draw [style=line with arrow] (1.center) to (0.center);
                    \draw [style=line with arrow] (1.center) to (2.center);
                    \draw [style=arrowws] (1.center) to (11.center);
                \end{pgfonlayer}
            \end{tikzpicture}   
        \end{center}
    
    \item $\displaystyle \p{-1,1,0}$
        \begin{center}
            \begin{tikzpicture}
                \begin{pgfonlayer}{nodelayer}
                    \node [style=none] (0) at (-3, 2.25) {};
                    \node [style=none] (1) at (-3, -1.25) {};
                    \node [style=none] (2) at (0.75, -1.25) {};
                    \node [style=none] (3) at (-5, -3.25) {};
                    \node [style=none] (4) at (-3.5, 2.25) {};
                    \node [style=none] (5) at (-3.5, 2.25) {$z$};
                    \node [style=none] (6) at (-5, -3.75) {$x$};
                    \node [style=none] (7) at (1, -1.75) {$y$};
                    \node [style=none] (9) at (-1, -0.5) {$(-1,1,0)$};
                    \node [style=none] (10) at (-1.25, 0.25) {};
                    \node [style=none] (11) at (-1.5, 0.25) {};
                    \node [style=none] (12) at (-2.5, -0.75) {};
                    \node [style=none] (13) at (-1.75, -0.75) {};
                    \node [style=darkdot] (14) at (-1.75, -0.75) {};
                \end{pgfonlayer}
                \begin{pgfonlayer}{edgelayer}
                    \draw [style=line with arrow] (1.center) to (3.center);
                    \draw [style=line with arrow] (1.center) to (0.center);
                    \draw [style=line with arrow] (1.center) to (2.center);
                    \draw [style=line] (1.center) to (11.center);
                    \draw [style=arrowws] (1.center) to (12.center);
                    \draw [style=arrowws] (12.center) to (13.center);
                \end{pgfonlayer}
            \end{tikzpicture}
        \end{center}
    
    \item $\displaystyle \p{1,3,-1} $  
        \begin{tikzpicture}
            \begin{pgfonlayer}{nodelayer}
                \node [style=none] (0) at (-3, 2.25) {};
                \node [style=none] (1) at (-3, -1.25) {};
                \node [style=none] (2) at (0.75, -1.25) {};
                \node [style=none] (3) at (-5, -3.25) {};
                \node [style=none] (4) at (-3.5, 2.25) {};
                \node [style=none] (5) at (-3.5, 2.25) {$z$};
                \node [style=none] (6) at (-5, -3.75) {$x$};
                \node [style=none] (7) at (1, -1.75) {$y$};
                \node [style=none] (9) at (-2.25, -2.5) {$(1,3,-1)$};
                \node [style=none] (10) at (-3.325, -1.575) {};
                \node [style=none] (11) at (-1.75, -1.575) {};
                \node [style=none] (12) at (-1.75, -1.975) {};
                \node [style=darkdot] (13) at (-1.75, -1.975) {};
            \end{pgfonlayer}
            \begin{pgfonlayer}{edgelayer}
                \draw [style=line with arrow] (1.center) to (3.center);
                \draw [style=line with arrow] (1.center) to (0.center);
                \draw [style=line with arrow] (1.center) to (2.center);
                \draw [style=arrowws] (1.center) to (10.center);
                \draw [style=arrowws] (10.center) to (11.center);
                \draw [style=arrowws] (11.center) to (12.center);
            \end{pgfonlayer}
        \end{tikzpicture}
\end{enumerate}



%%%%%%%%%%%%%%%%%%%%%%%%%%%%%%%%%%%%%%%%%%%%%%%%%%%%%%%%%%%%%%%%%%%%%%%%%%%%%%%%%%%%%%%%%%
\section{Planos coordenados}
\begin{itemize}
    \item Planos-$\displaystyle xy$:
        % 10,11
        % 12,13
        % 14,15
        
        \begin{center}
            \begin{tikzpicture}
                \begin{pgfonlayer}{nodelayer}
                    \node [style=none] (0) at (-3, 3.25) {};
                    \node [style=none] (1) at (-3, -1.25) {};
                    \node [style=none] (2) at (1.75, -1.25) {};
                    \node [style=none] (3) at (-5.5, -3.75) {};
                    \node [style=none] (5) at (-3.5, 3.25) {$z$};
                    \node [style=none] (6) at (-5.3, -4.125) {$x$};
                    \node [style=none] (7) at (2, -1.5) {$y$};
                    \node [style=none] (8) at (-5, -3.25) {};
                    \node [style=none] (9) at (-3, -1.25) {};
                    \node [style=none] (10) at (-3, 2.25) {};
                    \node [style=none] (11) at (-5, 0.5) {};
                    \node [style=none] (12) at (1.025, 2.275) {};
                    \node [style=none] (13) at (-0.25, -3.25) {};
                    \node [style=none] (14) at (-4.25, -3.25) {};
                    \node [style=none] (15) at (-3.45, -3.25) {};
                    \node [style=none] (16) at (-2.7, -3.25) {};
                    \node [style=none] (17) at (-1.9, -3.25) {};
                    \node [style=none] (18) at (-1.075, -3.25) {};
                    \node [style=none] (20) at (-4.5, 0.925) {};
                    \node [style=none] (21) at (-4.05, 1.325) {};
                    \node [style=none] (22) at (-3.65, 1.675) {};
                    \node [style=none] (23) at (-3.3, 1.975) {};
                    \node [style=none] (24) at (-4.55, -2.75) {};
                    \node [style=none] (25) at (-4.075, -2.325) {};
                    \node [style=none] (26) at (-3.675, -1.925) {};
                    \node [style=none] (27) at (-3.325, -1.6) {};
                    \node [style=none] (29) at (-2.4, -1.25) {};
                    \node [style=none] (30) at (-1.7, -1.25) {};
                    \node [style=none] (31) at (-0.975, -1.25) {};
                    \node [style=none] (32) at (-0.35, -1.25) {};
                    \node [style=none] (33) at (0.3, -1.25) {};
                    \node [style=none] (34) at (-2.425, 2.25) {};
                    \node [style=none] (35) at (-1.675, 2.25) {};
                    \node [style=none] (36) at (-0.95, 2.25) {};
                    \node [style=none] (37) at (-0.275, 2.25) {};
                    \node [style=none] (38) at (0.325, 2.25) {};
                    \node [style=none] (39) at (1, -1.25) {};
                    \node [style=none] (40) at (-1.25, 0.5) {Plano-yz};
                    \node [style=none] (41) at (-4, -0.5) {Plano-xz};
                    \node [style=none] (42) at (-1.75, -2.25) {Plano-xy};
                \end{pgfonlayer}
                \begin{pgfonlayer}{edgelayer}
                    \draw [style=line with arrow] (1.center) to (3.center);
                    \draw [style=line with arrow] (1.center) to (0.center);
                    \draw [style=line with arrow] (1.center) to (2.center);
                    \draw [style=dashed line] (8.center) to (11.center);
                    \draw [style=dashed line] (11.center) to (10.center);
                    \draw [style=dashed line] (8.center) to (13.center);
                    \draw [style=dashed line] (12.center) to (10.center);
                    \draw [style=dashed line] (29.center) to (34.center);
                    \draw [style=dashed line] (35.center) to (30.center);
                    \draw [style=dashed line] (36.center) to (31.center);
                    \draw [style=dashed line] (37.center) to (32.center);
                    \draw [style=dashed line] (33.center) to (38.center);
                    \draw [style=dashed line] (27.center) to (23.center);
                    \draw [style=dashed line] (22.center) to (26.center);
                    \draw [style=dashed line] (25.center) to (21.center);
                    \draw [style=dashed line] (20.center) to (24.center);
                    \draw [style=dashed line] (14.center) to (29.center);
                    \draw [style=dashed line] (15.center) to (30.center);
                    \draw [style=dashed line] (16.center) to (31.center);
                    \draw [style=dashed line] (17.center) to (32.center);
                    \draw [style=dashed line] (18.center) to (33.center);
                    \draw [style=dashed line] (12.center) to (39.center);
                    \draw [style=dashed line] (39.center) to (13.center);
                \end{pgfonlayer}
            \end{tikzpicture}
        \end{center}
        
    
    \item Los planos:  
        \begin{itemize}
            \item Plano-$\displaystyle yz$:  $\displaystyle x=0$ 
            \item Plano-$\displaystyle xz$: $\displaystyle y=0$ 
            \item Plano-$\displaystyle yx$: $\displaystyle z=0$ 
        \end{itemize}
    
    \item El primer octante:
        \begin{center}
            \begin{tikzpicture}
                \begin{pgfonlayer}{nodelayer}
                    \node [style=none] (0) at (-3, 3.25) {};
                    \node [style=none] (1) at (-3, -1.25) {};
                    \node [style=none] (2) at (1.75, -1.25) {};
                    \node [style=none] (3) at (-5.5, -3.75) {};
                    \node [style=none] (5) at (-3.5, 3.25) {$z$};
                    \node [style=none] (6) at (-5.3, -4.125) {$x$};
                    \node [style=none] (7) at (2, -1.5) {$y$};
                    \node [style=none] (8) at (-5, -3.25) {};
                    \node [style=none] (9) at (-3, -1.25) {};
                    \node [style=none] (10) at (-3, 2.25) {};
                    \node [style=none] (11) at (-5, 0.5) {};
                    \node [style=none] (12) at (1.025, 2.275) {};
                    \node [style=none] (13) at (-0.25, -3.25) {};
                    \node [style=none] (14) at (-4.25, -3.25) {};
                    \node [style=none] (15) at (-3.45, -3.25) {};
                    \node [style=none] (16) at (-2.7, -3.25) {};
                    \node [style=none] (17) at (-1.9, -3.25) {};
                    \node [style=none] (18) at (-1.075, -3.25) {};
                    \node [style=none] (20) at (-4.5, 0.925) {};
                    \node [style=none] (21) at (-4.05, 1.325) {};
                    \node [style=none] (22) at (-3.65, 1.675) {};
                    \node [style=none] (23) at (-3.3, 1.975) {};
                    \node [style=none] (24) at (-4.55, -2.75) {};
                    \node [style=none] (25) at (-4.075, -2.325) {};
                    \node [style=none] (26) at (-3.675, -1.925) {};
                    \node [style=none] (27) at (-3.325, -1.6) {};
                    \node [style=none] (29) at (-2.4, -1.25) {};
                    \node [style=none] (30) at (-1.7, -1.25) {};
                    \node [style=none] (31) at (-0.975, -1.25) {};
                    \node [style=none] (32) at (-0.35, -1.25) {};
                    \node [style=none] (33) at (0.3, -1.25) {};
                    \node [style=none] (34) at (-2.425, 2.25) {};
                    \node [style=none] (35) at (-1.675, 2.25) {};
                    \node [style=none] (36) at (-0.95, 2.25) {};
                    \node [style=none] (37) at (-0.275, 2.25) {};
                    \node [style=none] (38) at (0.325, 2.25) {};
                    \node [style=none] (39) at (1, -1.25) {};
                    \node [style=none] (43) at (-0.25, 0.5) {};
                    \node [style=none] (44) at (-1.875, -2.25) {Arriba del suelo};
                    \node [style=none] (45) at (-4.675, -0.55) {Derecha de esta pared};
                    \node [style=none] (46) at (-1.375, 1.425) {Delante de esta pared};
                \end{pgfonlayer}
                \begin{pgfonlayer}{edgelayer}
                    \draw [style=line with arrow] (1.center) to (3.center);
                    \draw [style=line with arrow] (1.center) to (0.center);
                    \draw [style=line with arrow] (1.center) to (2.center);
                    \draw [style=dashed line] (8.center) to (11.center);
                    \draw [style=dashed line] (11.center) to (10.center);
                    \draw [style=dashed line] (8.center) to (13.center);
                    \draw [style=dashed line] (12.center) to (10.center);
                    \draw [style=dashed line] (29.center) to (34.center);
                    \draw [style=dashed line] (35.center) to (30.center);
                    \draw [style=dashed line] (36.center) to (31.center);
                    \draw [style=dashed line] (37.center) to (32.center);
                    \draw [style=dashed line] (33.center) to (38.center);
                    \draw [style=dashed line] (27.center) to (23.center);
                    \draw [style=dashed line] (22.center) to (26.center);
                    \draw [style=dashed line] (25.center) to (21.center);
                    \draw [style=dashed line] (20.center) to (24.center);
                    \draw [style=dashed line] (14.center) to (29.center);
                    \draw [style=dashed line] (15.center) to (30.center);
                    \draw [style=dashed line] (16.center) to (31.center);
                    \draw [style=dashed line] (17.center) to (32.center);
                    \draw [style=dashed line] (18.center) to (33.center);
                    \draw [style=dashed line] (12.center) to (39.center);
                    \draw [style=dashed line] (39.center) to (13.center);
                    \draw (43.center) to (12.center);
                    \draw (43.center) to (11.center);
                    \draw (43.center) to (13.center);
                \end{pgfonlayer}
            \end{tikzpicture}
        \end{center}
    
    \item Planos en el espacio:  
        \begin{itemize}
            \item En dos dimensiones cuando se proponía $\displaystyle x=a$ ó $\displaystyle y=b$ se sabía que se hablaba de una recta horizontal o vertical. 
        \end{itemize}
        \begin{tikzpicture}
            \begin{pgfonlayer}{nodelayer}
                \node [style=none] (0) at (-2.5, 2.75) {};
                \node [style=none] (1) at (-2.475, -6) {};
                \node [style=none] (2) at (2.975, -1.75) {};
                \node [style=none] (3) at (-8, -1.75) {};
                \node [style=none] (4) at (3.25, -2.25) {$x$};
                \node [style=none] (5) at (-3, 3) {$y$};
                \node [style=none] (6) at (-8.25, 0.5) {};
                \node [style=none] (7) at (3, 0.5) {};
                \node [style=none] (8) at (1.25, 2.75) {};
                \node [style=none] (9) at (1.25, -6) {};
                \node [style=none] (10) at (1.25, 3.25) {$x=5$};
                \node [style=none] (11) at (3.75, 0.5) {$y=4$};
            \end{pgfonlayer}
            \begin{pgfonlayer}{edgelayer}
                \draw [style=double arrow] (3.center) to (2.center);
                \draw [style=double arrow] (0.center) to (1.center);
                \draw [style=double arrow] (7.center) to (6.center);
                \draw [style=double arrow] (9.center) to (8.center);
            \end{pgfonlayer}
        \end{tikzpicture}
    
    \item En tres dimensiones $\displaystyle x=a, y=b, z=c$ son gráficas de planos.
        \begin{center}
            \begin{tikzpicture}
                \begin{pgfonlayer}{nodelayer}
                    \node [style=none] (0) at (-3.5, 4) {};
                    \node [style=none] (1) at (-3.5, -2.5) {};
                    \node [style=none] (2) at (3.75, -2.5) {};
                    \node [style=none] (3) at (-7.5, -6.75) {};
                    \node [style=none] (4) at (-7.25, -7.25) {$x$};
                    \node [style=none] (5) at (-4.25, 4.25) {$z$};
                    \node [style=none] (6) at (3.75, -3) {$y$};
                    \node [style=none] (7) at (-3, -7.5) {$y=3$};
                    \node [style=none] (8) at (-2.75, -7) {};
                    \node [style=none] (9) at (0.75, -2.75) {};
                    \node [style=none] (10) at (0.75, 5.75) {};
                    \node [style=none] (11) at (-2.75, 1.5) {};
                    \node [style=none] (12) at (-2.325, 2) {};
                    \node [style=none] (13) at (-1.925, 2.5) {};
                    \node [style=none] (14) at (-1.5, 3) {};
                    \node [style=none] (15) at (-1, 3.5) {};
                    \node [style=none] (16) at (-0.75, 4) {};
                    \node [style=none] (17) at (-0.25, 4.5) {};
                    \node [style=none] (18) at (0.15, 5) {};
                    \node [style=none] (19) at (0.475, 5.4) {};
                    \node [style=none] (20) at (-2.25, -6.5) {};
                    \node [style=none] (21) at (-1.925, -6.025) {};
                    \node [style=none] (22) at (-1.5, -5.5) {};
                    \node [style=none] (23) at (-1, -5) {};
                    \node [style=none] (24) at (-0.75, -4.5) {};
                    \node [style=none] (25) at (-0.25, -4) {};
                    \node [style=none] (26) at (0.15, -3.5) {};
                    \node [style=none] (27) at (0.475, -3.1) {};
                    \node [style=darkdot] (28) at (-0.75, -2.5) {};
                \end{pgfonlayer}
                \begin{pgfonlayer}{edgelayer}
                    \draw [style=line with arrow] (1.center) to (0.center);
                    \draw [style=line with arrow] (1.center) to (2.center);
                    \draw [style=line with arrow] (1.center) to (3.center);
                    \draw [style=dashed line] (11.center) to (10.center);
                    \draw [style=dashed line] (10.center) to (9.center);
                    \draw [style=dashed line] (9.center) to (8.center);
                    \draw [style=dashed line] (8.center) to (11.center);
                    \draw [style=dashed line] (12.center) to (20.center);
                    \draw [style=dashed line] (21.center) to (13.center);
                    \draw [style=dashed line] (14.center) to (22.center);
                    \draw [style=dashed line] (23.center) to (15.center);
                    \draw [style=dashed line] (16.center) to (24.center);
                    \draw [style=dashed line] (25.center) to (17.center);
                    \draw [style=dashed line] (18.center) to (26.center);
                    \draw [style=dashed line] (27.center) to (19.center);
                \end{pgfonlayer}
            \end{tikzpicture}
        \end{center}
    
    \item Ejemplo: $\displaystyle z=2$ el ``techo'' en $\displaystyle z=2$ 
        \begin{center}
            \begin{tikzpicture}
                \begin{pgfonlayer}{nodelayer}
                    \node [style=none] (0) at (-3.5, 4) {};
                    \node [style=none] (1) at (-3.5, -2.5) {};
                    \node [style=none] (2) at (3.75, -2.5) {};
                    \node [style=none] (3) at (-7.5, -6.75) {};
                    \node [style=none] (4) at (-7.25, -7.25) {$x$};
                    \node [style=none] (5) at (-4.25, 4.25) {$z$};
                    \node [style=none] (6) at (3.75, -3) {$y$};
                    \node [style=none] (7) at (-10.75, -3.25) {$z=2$};
                    \node [style=none] (8) at (-10.25, -2.75) {};
                    \node [style=none] (9) at (-1.35, -2.75) {};
                    \node [style=none] (10) at (2.25, 0.75) {};
                    \node [style=none] (11) at (-6.25, 0.75) {};
                    \node [style=none] (12) at (-9.75, -2.75) {};
                    \node [style=none] (13) at (-9.25, -2.75) {};
                    \node [style=none] (14) at (-8.6, -2.75) {};
                    \node [style=none] (15) at (-8, -2.75) {};
                    \node [style=none] (16) at (-7.425, -2.75) {};
                    \node [style=none] (17) at (-6.75, -2.75) {};
                    \node [style=none] (18) at (-6.175, -2.75) {};
                    \node [style=none] (19) at (-5.525, -2.75) {};
                    \node [style=none] (20) at (-4.75, -2.75) {};
                    \node [style=none] (21) at (-4, -2.75) {};
                    \node [style=none] (22) at (-3.25, -2.75) {};
                    \node [style=none] (23) at (-2.5, -2.75) {};
                    \node [style=none] (24) at (-2, -2.75) {};
                    \node [style=none] (27) at (-5.75, 0.75) {};
                    \node [style=none] (28) at (-5.25, 0.75) {};
                    \node [style=none] (29) at (-4.675, 0.75) {};
                    \node [style=none] (30) at (-4, 0.75) {};
                    \node [style=none] (31) at (-3.5, 0.75) {};
                    \node [style=none] (32) at (-2.75, 0.75) {};
                    \node [style=none] (33) at (-2.075, 0.75) {};
                    \node [style=none] (34) at (-1.45, 0.75) {};
                    \node [style=none] (35) at (-0.675, 0.75) {};
                    \node [style=none] (36) at (0.025, 0.75) {};
                    \node [style=none] (37) at (0.675, 0.75) {};
                    \node [style=none] (38) at (1.25, 0.75) {};
                    \node [style=none] (39) at (1.75, 0.75) {};
                    \node [style=darkdot] (40) at (-3.45, -0.9) {};
                \end{pgfonlayer}
                \begin{pgfonlayer}{edgelayer}
                    \draw [style=line with arrow] (1.center) to (0.center);
                    \draw [style=line with arrow] (1.center) to (2.center);
                    \draw [style=line with arrow] (1.center) to (3.center);
                    \draw [style=dashed line] (8.center) to (11.center);
                    \draw [style=dashed line] (11.center) to (10.center);
                    \draw [style=dashed line] (10.center) to (9.center);
                    \draw [style=dashed line] (9.center) to (8.center);
                    \draw [style=dashed line] (27.center) to (12.center);
                    \draw [style=dashed line] (13.center) to (28.center);
                    \draw [style=dashed line] (29.center) to (14.center);
                    \draw [style=dashed line] (30.center) to (15.center);
                    \draw [style=dashed line] (16.center) to (31.center);
                    \draw [style=dashed line] (32.center) to (17.center);
                    \draw [style=dashed line] (18.center) to (33.center);
                    \draw [style=dashed line] (34.center) to (19.center);
                    \draw [style=dashed line] (20.center) to (35.center);
                    \draw [style=dashed line] (21.center) to (36.center);
                    \draw [style=dashed line] (22.center) to (37.center);
                    \draw [style=dashed line] (23.center) to (38.center);
                    \draw [style=dashed line] (39.center) to (24.center);
                \end{pgfonlayer}
            \end{tikzpicture}                      
        \end{center}
    
    \item Ecuación lineal en 3-D va a graficar un plano:
        \[
          ax+by+cz=d
        \]
        \begin{itemize}
            \item Generalmente se grafican sólo en el primer octante se cada $\displaystyle a,b,c$ y $\displaystyle d$ es positiva.
        \end{itemize}
\end{itemize}


%----------------------------------------------------------------------------------------
\subsection{Ejercicios}
\begin{enumerate}
    \item Bosqueje el plano $\displaystyle 2x+4y+3z=12$ sólo en el primer octante: 
        \begin{center}
            \begin{align*}
                \text{ Intersección-$x$  }: \qquad 2x=12 \qimplies (6,0,0) \\ 
                \text{ Intersección-$x$  }: \qquad 4y=12 \qimplies (0,3,0) \\ 
                \text{ Intersección-$x$  }: \qquad 3z=12 \qimplies (0,0,4) \\ 
            \end{align*}
        \end{center}
        \begin{center}
            \begin{tikzpicture}
                \begin{pgfonlayer}{nodelayer}
                    \node [style=none] (0) at (-3.5, 4) {};
                    \node [style=none] (1) at (-3.5, -2.5) {};
                    \node [style=none] (2) at (3.75, -2.5) {};
                    \node [style=none] (3) at (-8.5, -8.25) {};
                    \node [style=none] (4) at (-8.25, -8.75) {$x$};
                    \node [style=none] (5) at (-4.25, 4.25) {$z$};
                    \node [style=none] (6) at (3.75, -3) {$y$};
                    \node [style=none] (7) at (-7.45, -7.025) {};
                    \node [style=none] (8) at (-3.5, 1) {};
                    \node [style=none] (9) at (0.5, -2.5) {};
                    \node [style=none] (10) at (0.75, -1.75) {$(0,3,0)$};
                    \node [style=none] (11) at (-4.5, 1.25) {$(0,0,4)$};
                    \node [style=none] (12) at (-8.75, -6.5) {$(6,0,0)$};
                    \node [style=none] (13) at (2.5, 1) {Una los tres puntos para obtener un segmento del plano};
                    \node [style=none] (14) at (-3.875, 0.25) {};
                    \node [style=none] (15) at (-4.25, -0.5) {};
                    \node [style=none] (16) at (-4.625, -1.25) {};
                    \node [style=none] (17) at (-5.025, -2.1) {};
                    \node [style=none] (18) at (-5.5, -2.75) {};
                    \node [style=none] (19) at (-5.725, -3.45) {};
                    \node [style=none] (20) at (-6.075, -4.175) {};
                    \node [style=none] (21) at (-6.65, -5.375) {};
                    \node [style=none] (22) at (-7, -6.1) {};
                    \node [style=none] (23) at (-0.05, -2.825) {};
                    \node [style=none] (24) at (-0.75, -3.2) {};
                    \node [style=none] (25) at (-1.525, -3.65) {};
                    \node [style=none] (26) at (-2.25, -4.05) {};
                    \node [style=none] (27) at (-2.975, -4.475) {};
                    \node [style=none] (28) at (-3.7, -4.9) {};
                    \node [style=none] (29) at (-4.4, -5.275) {};
                    \node [style=none] (30) at (-5.125, -5.7) {};
                    \node [style=none] (31) at (-5.825, -6.075) {};
                    \node [style=none] (32) at (-6.5, -6.475) {};
                    \node [style=none] (34) at (-6.35, -4.75) {};
                \end{pgfonlayer}
                \begin{pgfonlayer}{edgelayer}
                    \draw [style=line with arrow] (1.center) to (0.center);
                    \draw [style=line with arrow] (1.center) to (2.center);
                    \draw [style=line with arrow] (1.center) to (3.center);
                    \draw (7.center) to (8.center);
                    \draw (8.center) to (9.center);
                    \draw [in=30, out=-150] (9.center) to (7.center);
                    \draw (14.center) to (23.center);
                    \draw (24.center) to (15.center);
                    \draw (16.center) to (25.center);
                    \draw (26.center) to (17.center);
                    \draw (27.center) to (18.center);
                    \draw (29.center) to (20.center);
                    \draw (19.center) to (28.center);
                    \draw (34.center) to (30.center);
                    \draw (31.center) to (21.center);
                    \draw (22.center) to (32.center);
                \end{pgfonlayer}
            \end{tikzpicture}            
        \end{center}
\end{enumerate}


\chapter{Clase - 2020-01-23}
% \date{2020-Jan-23 10:33:52}
%%%%%%%%%%%%%%%%%%%%%%%%%%%%%%%%%%%%%%%%%%%%%%%%%%%%%%%%%%%%%%%%%%%%%%%%%%%%%%%%%%%%%%%%%%%%%%%%
\section{12.4 Producto Cruz}
\begin{itemize}
    \item \emph{\textbf{Definición de ``Determinantes":} Matriz (arreglo rectangular de números).}
    \item \emph{\textbf{Definición de ``Cuadrada":} Mismo número de filas y columnas. }
    \item \begin{center}
        \begin{align*}
            \begin{vmatrix}
                a & b \\ 
                c & d \\ 
            \end{vmatrix} = ad - bc \\ 
        \end{align*}
    \end{center}
    \# Determinante de orden 2. Matriz de 2x2 
    
    \item pie:
        \begin{center}
            \begin{align*}
                \begin{vmatrix}
                    3 & 4 \\ 
                    -1 & 2 \\ 
                \end{vmatrix}
                = 6 - (-1) (4) = 6 + 4 = 10 \\ 
            \end{align*}
        \end{center}
    
    \item Determinante de orden 3: Matriz 3x3 suma de tres determinantes de orden 2:
        \begin{center}
            \begin{align*}
                \begin{vmatrix}
                    a_{1} & a_{2} & a_{3} \\ 
                    b_{1} & b_{2} & b_{3} \\ 
                    c_{1} & c_{2} & c_{3} \\ 
                \end{vmatrix}
                = a_{1}\begin{vmatrix}
                    b_{2} & b_{3} \\ 
                    c_{2} & c_{3} \\ 
                \end{vmatrix} + a_{2}
                \begin{vmatrix}
                    b_1 &  b_3 \\ 
                    c_1 & c_3 \\  
                \end{vmatrix} + a_3 \begin{vmatrix}
                    b_1 & b_2 \\ 
                    c_1 & c_2 \\ 
                \end{vmatrix}
            \end{align*}
        \end{center}
        3 matrices de 2x2.
    
    \item p.e. \begin{center}
        \begin{align*}
            \begin{vmatrix}
                2&0&2\\ 
                1&3&0\\
                1&-1&2\\
            \end{vmatrix} = 2 \begin{vmatrix}
                3 & 0 \\ 
                -1 & 2 \\ 
            \end{vmatrix} - 0 \begin{vmatrix}
                1 & 0 \\ 
                1 & 2 \\ 
            \end{vmatrix} + 2 \begin{vmatrix}
                1 & 3 \\ 
                1 & -1 \\ 
            \end{vmatrix}
        \end{align*}
    \end{center}
    \[
      2(6-0)-0 + 2(-1-3) = 12 - 8 = 4 
    \]


\end{itemize}

\section{Producto Cruz}
\begin{itemize}
    \item Dados dos vectores :
        \begin{center}
            \begin{align*}
                \vec{a} & = a_1 \hat{i}  + a_2 \hat{j} + a_3 \hat{k} \\ 
                \vec{b} & = b_1 \hat{i} + b_2 \hat{j} + b_3 \hat{k} \\ 
            \end{align*}
        \end{center}
    
    \item \textbf{Nos preguntamos:} ¿Cómo se encuentra un vector $\vec{c}$  que es perpendicular a $\vec{a}$ y a $\vec{b}$?
        \[
          \vec{c} \cdot \vec{a} = 0 
        \]
        \[
          \vec{c} \cdot \vec{b} = 0 
        \]
    
    \item Resuelva para $c_1,c_2,c_3$ :
        \begin{center}
            \begin{align*}
                c_1a_1+c_2a_2+c_3a_3=0\\ 
                c_1b_1+c_2b_2+c_3b_3=0\\ 
            \end{align*}
        \end{center}
    
    \item El producto cruz $\vec{c}= \vec{a} \times \vec{b} = 0$ es un vector perpendicular a ambos vectores $\vec{a}$ \& $\vec{b}$.
        \begin{center}
            \begin{align*}
                \vec{a}\times \vec{b} = \begin{vmatrix}
                    \hat{i} & \hat{j} & \hat{k} \\ 
                    a_1 & a_2 & a_3 \\ 
                    b_1 & b_2 & b_3 \\ 
                \end{vmatrix} = \hat{i}(a_2b_3 - a_3b_2) - \hat{j}(a_1b_3-a_3b_1)+\hat{k}(a_1b_2-a_2b_1) 
            \end{align*}
        \end{center}
    
    \item Observaciones:
        \begin{itemize}
            \item El producto cruz es un vector, mientras que el producto es un número o escalar.
            \item El producto cruz \textbf{no} es conmutativo $\vec{a}\times \vec{b} \neq \vec{b}\times \vec{a}$.
        \end{itemize}
        \begin{center}
            \begin{align*}
                \vec{b}\times \vec{a} = \begin{vmatrix}
                    \hat{i} & \hat{j} & \hat{k} \\ 
                    b_1 & b_2 & b_3 \\ 
                    a_1 & a_2 & a_3 \\ 
                \end{vmatrix} = \hat{i}(b_2a_3-a_2b_3)+\hat{j}(a_1b_3-a_3b_1)+\hat{k}(a_2b_1-a_1b_2)
            \end{align*}
        \end{center}
    
    \item Por ejemplo:
        \begin{center}
            \begin{align*}
                \begin{vmatrix}
                    \hat{i} & \hat{j} & \hat{k} \\ 
                    2 & 3 & 0 \\ 
                    1 & 0 & 5 \\ 
                \end{vmatrix} = \hat{i}\begin{vmatrix}
                    3 & 0 \\ 
                    0 & 5 \\ 
                \end{vmatrix} - \hat{j}\begin{vmatrix}
                    2 & 0 \\ 
                    1 & 5 \\ 
                \end{vmatrix} + \hat{k} \begin{vmatrix}
                    2 & 3 \\ 
                    1 & 0 \\ 
                \end{vmatrix}
            \end{align*}
        \end{center}
        \[
          \therefore \vec{a}\times \vec{b} = 15\hat{i}-10\hat{j}-3\hat{k}
        \]
    
    \item Verifique $\vec{a}\times \vec{b}$ es ortogonal a $\vec{a}$ \& a $\vec{b}$.
        \begin{align*}
            (\vec{a}\times \vec{b})\cdot \vec{a} & = \langle 15,-10,-3 \rangle \cdot \langle 2,3,0 \rangle = 30-30+0 = 0 \therefore \text{  son ortogonales  } \\ 
            (\vec{a}\times \vec{b}) \cdot \vec{b} & = \langle 15, -10,-3\rangle \cdot \langle 1,0,5 \rangle = 15+0-15 = 0 \therefore \text{  son ortogonales  } \\ 
            \\ 
            & \vec{a}\times \vec{b} \perp a_1b \\ 
        \end{align*}
    
    \item Aclaración: en dos dimensiones $\vec{a}\times \vec{b} = \begin{vmatrix}
        \hat{i} & \hat{j} \\ 
        a_1 & a_2 \\ 
        b_1 & b_2 \\ 
    \end{vmatrix} $  No es posible evaluarlo.
    
    \item Existen en tres dimensiones pero si se intenta evaluar en cuatro dimensiones la siguiente matriz no es posible:
    \[
        \text{  En 3-D: } \exists \text{  En 4-D:  } \nexists 
    \]
        \[
           \vec{a}\times \vec{b} = \begin{vmatrix}
              \hat{i} & \hat{j} & \hat{k} & \hat{l} \\ 
              1 & 0 & 2 & 3 \\ 
              4 & 1 & 5 & -2 \\ 
          \end{vmatrix}
        \]
        No es posible evaluarlo.
    
    \item Ejemplo:
          \begin{center}
            \begin{align*}
                \begin{vmatrix}
                    \hat{i} & \hat{j} & \hat{k} \\ 
                    1 & 0 & 5 \\ 
                    2 & 3 & 0 \\ 
                \end{vmatrix} = \hat{i}\begin{vmatrix}
                    0 & 5 \\ 
                    3 & 0 \\ 
                \end{vmatrix} - \hat{j}\begin{vmatrix}
                    1 & 5 \\ 
                    2 & 0 \\ 
                \end{vmatrix} + \hat{k}\begin{vmatrix}
                    1 & 0 \\ 
                    2 & 3 \\ 
                \end{vmatrix} \\ 
                = 15\hat{i}+10\hat{j}+3\hat{k}\\ 
              \end{align*}
          \end{center}
          Entonces... en general: 
          \[
                \vec{a}\times \vec{b} = -(\vec{b}\times \vec{a})
          \]
\end{itemize}






















\chapter{Clase - 2020-01-28}
\documentclass{article}
\title{Clase -2020-01-28}
\author{David Gabriel Corzo Mcmath}
\date{2020-Jan-28 10:08:14}
%%%%%%%%%%%%%%%%%%%%%%%%%%%%%%%%%%%%%%%%%%%%%%%%%%%%%%%%%%%%%%%%%%%%%%%%%%%%%%%%%%%%%%%%%%%%%%%%%%%%%%%%%%%%%%%%%%%%%%%%%%%%%%%%%%%%%%%%%%%%%%%
\usepackage[margin = 1in]{geometry}
\usepackage{graphicx}
\usepackage{fontenc}
\usepackage{pdfpages}
\usepackage[spanish]{babel}
\usepackage{amsmath}
\usepackage{amsthm}
\usepackage[utf8]{inputenc}
\usepackage{enumitem}
\usepackage{mathtools}
\usepackage{import}
\usepackage{xifthen}
\usepackage{pdfpages}
\usepackage{transparent}
\usepackage{color}
\usepackage{fancyhdr}
\usepackage{lipsum}
\usepackage{sectsty}
\usepackage{titlesec}
\usepackage{calc}
\usepackage{lmodern}
\usepackage{xpatch}
\usepackage{blindtext}
\usepackage{bookmark}
\usepackage{fancyhdr}
\usepackage{xcolor}
\usepackage{tikz}
\usepackage{blindtext}
\usepackage{hyperref}
\usepackage{listing}
\usepackage{spverbatim}
\usepackage{fancyvrb}
\usepackage{fvextra}
\usepackage{amssymb}
\usepackage{pifont}
\usepackage{longtable}
\usepackage{tikz-3dplot}
\usepackage{esvect}
%%%%%%%%%%%%%%%%%%%%%%%%%%%%%%%%%%%%%%%%%%%%%%%%%%%%%%%%%%%%%%%%%%%%%%%%%%%%%%%%%%%%%%%%%%%%%%%%%%%%%%%%%%%%%%%%%%%%%%%%%%%%%%%%%%%%%%%%%%%%%%%
\begin{document}
\maketitle

\section{12.5 Rectas y planos}
\begin{itemize}
    \item Ecuación de una recta 
    \item Vector posición $\vec{r}_0  = \langle x_0,y_,z_0 \rangle $
    \item Vector dirección $\vec{v}_0 = \langle a,b,c \rangle$
    \item Ecuación vectorial: $\vec{r} = \vec{r}_0 + t\vec{v}$ donde t es el parámetro.
    \item Ecuaciónes paramétricas: \begin{align*}
        x = x_0 +at \\ 
        y = y_0+ at \\ 
        z = z_0+at \\ 
    \end{align*}
    
    \item Resuelva para $t$ en las tres ecuaciones:  
        \begin{align*}
            t = \frac{x-x_0}{a} 
            t = \frac{y-y_0}{b} 
            t = \frac{z-z_0}{c} 
        \end{align*}
        Estas son las ecuaciónes simétricas de la recta donde $a,b,c \neq 0$.
        
        \item Vector dirección $\vec{v}= \langle a,0,c \rangle $ las ecuaciones en la recta cambian:
            \begin{align*}
                \underbrace{\vec{r} = \vec{r}_0 + t \vec{v}}_{\text{  Vectorial  }} \\ 
                x = x_0+at \\ 
                y = y_0 \\ 
                z = z_0 +ct \\ 
                \text{  Entonces queda así:  } \\ 
                \frac{x-x_0}{a} = \frac{z-z_0}{c} \\ 
                \underbrace{y = y_0}_{\text{  Simétrica  }} \\   
            \end{align*}
\end{itemize}

\subsection{Ejercicio 3: Encuentre las ecs. simétricas de la recta que pasa por los puntos dados. Encuentre en qué punto la recta interseca al plano xz. pg.41}
\begin{itemize}
    \item P(2,8,-2) \& Q(2,6,4) 
    \begin{align*}
        \text{  Vector posición  } = \overrightarrow{OP} = R_0 = \langle 2,6,7 \rangle \\ 
        \text{  Vector dirección  } \overrightarrow{PQ} = \vec{V} = \langle 0,-2,6 \rangle \\ 
        \text{  Ec. vectorial  } = \vec{r} = \langle 2,8,-2 \rangle + t \langle 0,-2,6 \rangle \\
        \text{  Ecs. simétricas  } = x = a, \frac{y-8}{-2} = \frac{z+2}{6} \\   
    \end{align*}
    
    \item \textbf{Nos preguntamos:} ¿Cual es la intersección con el plano xz?
    \begin{align*}
        \text{  Use,  y=0} x = 2, \frac{-8}{-2} =&  \frac{z+2}{6} \\ 
        & = 6 \cdot 4 = z+ 2 \implies z= 22 \\ 
    \end{align*}

    
    \item La intersección con el plano xz es el punto $(1,0,22)$: 
    \begin{align*}
        \vec{r}_0 = \langle 4,6,10 \rangle \\ 
        \vec{v} = \vv{PQ} = \langle 2,0,0 \rangle \\ 
        \text{  Vectorial:    } \vec{r} = \langle 4,6,10 \rangle + t \langle 2,0,0 \rangle \\ 
        \text{  Paramétricas: } x = 4 + 2t, y = 6, z = 10 \\ 
        \text{  Simétricas:  }t = \frac{x-4}{2} , y = 6, z=10 \\  
    \end{align*}
    
    \item \textbf{Nos preguntamos:} ¿Cual es el punto de instersección con el plano xz?
    \begin{align*}
        \text{  Use: y=0  }
    \end{align*}
    Explicación: por la recta $y=6$ siempre será 6, nunca podrá ser $0$, no puede intersecar con el plano xz, \textbf{No hay}.
\end{itemize}


%%%%%%%%%%%%%%%%%%%%%%%%%%%%%%%%%%%%%%%%%%%%%%%%%%%%%%%%%%%%%%%%%%%%%%%%%%%%%%%%%%%%%%%%%%%%%%%%

\section{Rectas paralelas}
Dos rectas $\vec{r}_1 = \vec{r}_{01} + t \vec{v}$ \& $\vec{r}_{2} = \vec{r}_{02} + t \vec{v}_2 $ son paralelas si y solo si sus vectores de dirección $\vec{v}_1$ y $\vec{v}_2 $ son paralelas.
\begin{figure}[htbp]
    \centering
    % \includegraphics[width=cm]{}
    \caption{}
    \label{}
\end{figure} 

Entones en el espacio tenemos 3 tipos de rectas:
\begin{enumerate}
    \item Rectas paralelas  
    \item Rectas intersecan en un punto
    \item Rectas Ublicuas (no paralelas \& no intersecan)
\end{enumerate}

\subsection{Ejercicio 4: Determine si los siguientes pares de rectas son paralelas, oblicuas o se intersecan.}
\begin{itemize}
    \item \begin{align*}
        \frac{x-2}{8} = \frac{y-3}{24} = \frac{z-2}{16} , \frac{x-10}{-2} = \frac{y+15}{-6} = \frac{z+24}{-4} \\ 
        \vec{v}_1 = \langle 8,24,16 \rangle, \vec{v}_2  = \left\langle -2,-6,-4 \right\rangle \\ 
        \text{  Entoces...  }, \left\langle \frac{8}{-2}, \frac{24}{-6} , \frac{16}{-4}   \right\rangle  \\ 
        \left\langle -4,-4,-4   \right\rangle, \therefore \text{  Son paralelas  } \\   
    \end{align*}
    El vector dirección está en el denominador.
    
     \item \begin{align*}
        L_1:  x = 3-4t, y = 6-2t, z= 2+ 0t, t \in IR\\  
         L_2:  x = 3+ 8s, y = -2s, z = 8+2s, s \in IR \\ 
         \text{  Utilize una variable parámetro para cada recta  } \\ 
         v_1 = \left\langle -4,-2,0 \right\rangle , v_2 = \left\langle 8,-2,2 \right\rangle \text{  No son paralelas  } \\ 
         \text{  Analice si las rectas se intersecan  } \\ 
         x=x \rightarrow 5-4t=3+8s \\ 
         y=y \rightarrow 6-2t=-2s \\ 
         z=z \rightarrow 2 = 8 + 2s \rightarrow s = -3 \\ 
         5-4=-22 \rightarrow 4t=-27 \rightarrow -4t=-27 \rightarrow = \frac{27}{4} \\ 
         6-2t = 6 \rightarrow 2t = 0 \rightarrow t=0 \\ 
         \therefore \text{  Como no hay una $t$ única (no es posible $0 \neq \frac{27}{4} $), las dos rctas no se intersecan.  } \\ 
         L_1 \text{  \&  }L_2 \text{  Son oblicuas  } \\ 
         \text{  Eliminación Gausiana  } \\ 
         \begin{matrix}
             4t+8s = 2 \\ 
             2t+25 = 6 \\ 
             0t + 2s = -6 \\ 
         \end{matrix} = 
         \begin{vmatrix}
             4 & 8 & 2 \\ 
             2 & 2 & 6 \\ 
             0 & 2 & -6 \\ 
         \end{vmatrix} 
         0,0,\text{  número  } \implies \text{  No hay solución  } \\ 
     \end{align*}
\end{itemize}


\section{La ecuación de un plano}
Previamente en 12.1 $ax+by+cz = 0$.
\begin{figure}[htbp]
    \centering
    % \includegraphics[width=cm]{}
    \caption{}
    \label{}
\end{figure}
Para encontrar la ec. de un plano se necesita:
    \begin{enumerate}
        \item Un punuto sobre el plano $P$: $\vec{r_0}=\overrightarrow{OP}$
        \item Un vector normal u ortognoal al plano: $\hat{n}_0 \left\langle a,b,c \right\rangle $
    \end{enumerate}
\subsection{Derivación de la e. plano}
\begin{align*}
    P(x_0,y_0,z_0), Q(x_1,y_1,z_1) \text{  Son dos puntos sobre el plano  } \\ 
    \vec{r_0} = = \overrightarrow{0P} = \left\langle x_0,y_0,z_0 \right\rangle \\ 
    \vec{r} = \overrightarrow{0Q} = \left\langle x,y,z \right\rangle \\ 
\end{align*}
El vector $\vec{RP} = \vec{r} + \vec{r-0}$ está sobre el plano, por lo que tiene que ser ortogonal a $\hat{n}$.
 \begin{align*}
     \hat{n} \perp \vec{r}- \vec{r_0} \rightarrow \underbrace{\hat{n}\cdot (\vec{r}-\vec{r_0})}_{\text{  Ec. vectorial de un plano  }} \\ 
    \text{  Se puede reescribir como:  } \\ 
     \underbrace{\left\langle a,b,c \right\rangle \cdot \left\langle x+x_0,y-y_0,z-z_0 \right\rangle + c(z-z_0) = 0 }_{\text{  Ecuación escalar de un plano  }} \\ 
     ax+by+cz = \underbrace{ax_0+by_0+cz_0}_{0} \\ 
 \end{align*}

Para encontrar la ec. de un plano se necesita 3 puntos P,Q,R: \textbf{hay infinitas respuestas equivalentes } $\hat{n}=\vec{}\times \vec{}$.
\begin{align*}
    \vec{r_0} = \overrightarrow{OP}, \overrightarrow{0Q}, \overrightarrow{0R} \\ 
    \underbrace{\hat{n} = \overrightarrow{PQ} \times \overrightarrow{PR}}_{\text{  Tienen que empezar en el mismo punto  }} \\ 
    \text{  Hat infinitas respuestas:  } \\ 
    \hat{n} = \vv{PR} \times \vv{PQ} \\ 
\end{align*}


\subsection{Ejercicio 1: pg45 Encuentre la ec. del plano que pasa por los 3 puntos dados.}
\begin{enumerate}
    \item $P(3,-1,3), Q(8,2,4), R(1,2,5)$
    \begin{align*}
        \text{  Ecuación del plano :   }, \hat{n} \cdot (\vec{r}- \vec{r_0}) = 0\\
        \text{  Ecuaciónn de la recta :   }, \vec{r} = \vec{r_0}+t \vec{v} \\ 
        \vec{r_0} = \left\langle 8,2,4 \right\rangle  \\ 
    \end{align*}
    Encuentre dos vectores que están sobre el plano y que comiencen en el mismo punto.
    \begin{align*}
        \vec{u}= \overrightarrow{PQ} = \left\langle 5,3,1  \right\rangle, \vec{v}= \overrightarrow{PR}= \left\langle -2,3,2 \right\rangle \\ 
        \text{  ¡¡ $\hat{n}$ es ortogonal a ambos vectores !! } \\ 
        \hat{n} = \overrightarrow{PQ} \times \overrightarrow{PR} = \begin{vmatrix}
            \hat{i}& \hat{j}& \hat{k} \\
            5 & 3 & 1 \\ 
            -2 & 3 & 2 \\   
        \end{vmatrix} = 3 \hat{i} - 12 \hat{j} -+ 21 \hat{k} \\ 
        \text{  Ec. Plano  }, \hat{n} \cdot ( \vec{r} - \vec{r_0}) = 0 \\ 
        \text{  Ec. Vectorial  }, \left\langle 3,-12,21 \right\rangle \cdot \left\langle x-8,y-2,z-4 \right\rangle = 0 \\ 
        \text{  Escalar  }, 3(x-8)-  
    \end{align*}

    
    \item P(0,0,0), Q(1,0,2), y R(0,2,3)
    \begin{align*}
        \text{  Vector posición: } \vec{r_0} & = \left\langle 0,0,0 \right\rangle \\ 
          \text{ dos  vectoes sobre el plano:  } \begin{matrix*}
             \vec{PQ} = \left\langle 1,0,2 \right\rangle \\ 
             \vec{PR} = \left\langle 0,2,3 \right\rangle  \\ 
         \end{matrix*} \\ 
        \text{  Vector normal:  } \hat{n} & = \overrightarrow{PQ} \times  \overrightarrow{PR} \\ 
          = \begin{vmatrix}
             \hat{i} & \hat{j} & \hat{k} \\
              \text{  terminar  } \\
         \end{vmatrix} \\
    \end{align*}
    
    \item Ecuación del plano:
    \[
      -4x-3y+2z=0
    \]
\end{enumerate}

 \section{Rectas paralelas $v_1$ y $v_2$ son paralelos}
Dos planos $\hat{n_1} \cdot (\vec{r}- \vec{r_1})= 0$ y $\hat{n_2} \cdot ( \vec{r}- \vec{r_2}) = 0$ son paralelas sí y sólo si $\hat{n_1}$ y $\hat{n_2}$ son paralelas.

 En caso que no sean paralelas, se puede encontrar el ángulos de intersección entre dos planos.

























\end{document}


\chapter{Clase - 2020-01-30}
% \date{2020-Jan-30 10:22:58}
%%%%%%%%%%%%%%%%%%%%%%%%%%%%%%%%%%%%%%%%%%%%%%%%%%%%%%%%%%%%%%%%%%%%%%%%%%%%%%%%%%%%%%%%%%%%%%%%
\section{Resolución de corto}
\begin{itemize}
    \item Determine el área del triángulo entre los puntos P(), Q(), R():
        \begin{align*}
            \vec{a} = \vv{PQ} = \left\langle 4,3,-2 \right\rangle \\ 
            \vec{b} = \vv{PR} = \left\langle 5,5,1 \right\rangle \\ 
            \text{  Área  } = \frac{1}{2} \left| \vec{a} \times  \vec{b} \right| \\ 
            \begin{vmatrix}
                \hat{i} & \hat{j} & \hat{k} \\ 
                4 & 3 & -2 \\ 
                3 & 5 & 1 \\ 
            \end{vmatrix} = 13 \hat{i} - 14 \hat{j} + 5 \hat{k} \\  
            \text{  Área  } = \frac{1}{2} \sqrt[]{} \\ 
        \end{align*}

\end{itemize}
    



%%%%%%%%%%%%%%%%%%%%%%%%%%%%%%%%%%%%%%%%%%%%%%%%%%%%%%%%%%%%%%%%%%%%%%%%%%%%%%%%%%%%%%%%%%%%%%%%
%%%%%%%%%%%%%%%%%%%%%%%%%%%%%%%%%%%%%%%%%%%%%%%%%%%%%%%%%%%%%%%%%%%%%%%%%%%%%%%%%%%%%%%%%%%%%%%%
\section{Rectas y planos}
\begin{itemize}
    \item Ecs. Rectas: $\vec{r}= \vec{r_0} + t \vec{v}$ 
        \begin{align*}
            \text{  si  } a \neq b \neq c \neq 0    \quad \frac{x-x_0}{a} = \frac{y-y_0}{b} = \frac{z-z_0}{c}  \\ 
        \end{align*}
    
    \item Paramétricas:
        \begin{align*}
            x = x_0 +at \\ 
            y = y_0 +bt \\ 
            z = z_0 +ct \\ 
        \end{align*}
    
    \item Ecuación de plano:
        \begin{align*}
            \hat{n} = \cdot \vec{r}-\vec{r_0} \\ 
            a(x-x_0)+b(y-y_0) + c(z-z_0) = 0 \\ 
            \hat{n}= \vec{a} \times  \vec{b} \\ 
        \end{align*}
\end{itemize}

\subsection{Ejercicios}
\begin{enumerate}
    \item Considere los planos $x+y=0$ \& $x+2y+z=1$.
        \begin{enumerate}
            \item Determine si los planos son paralelos so no lo son encuentre el ángulo entr ellos:
                \begin{align*}
                    \hat{n_1} = \left\langle 1,1,0 \right\rangle \\ 
                    \hat{n_2} = \left\langle 1,2,1 \right\rangle \\ 
                    \therefore \text{  Los dos planos no son paralelos  } \\ 
                \end{align*}
                \begin{itemize}
                    \item El $\hat{n_1}$ \& $\hat{n_2}$ no son necesariamente ortogonales.
                \end{itemize}
                \begin{align*}
                    \cos \theta = \frac{\hat{n_1}\cdot\hat{n_2}}{\left| \hat{n_1} \right|\left| \hat{n_2} \right| } = \frac{3}{\sqrt[]{2}} \\ 
                    \cos \theta = \frac{3}{2\sqrt[]{3}}=\frac{\sqrt[]{3}}{2} \qquad \theta = \frac{\pi}{2} \\   
                \end{align*}
        \end{enumerate}
    
    \item Encuentre la ec. de la recta que interseca a ambos planos $x+y=0$ \& $x+2y+z=1$: 
        \begin{align*}
            r = \vec{r_0} + t \vec{v} \\ 
            \text{  Dos puntos sobre la recta  } \\ 
            \text{  Como la recta esta en ambos planos, se debe resolver el sig. sistema de ecuaciones  } \\ 
            x + y = 0 \implies x=-y \\ 
            x+2y+z=1 \implies y = z-1 \\ 
            \text{  z tiene cualquier valor, ahora encontrar escogiendo cualquier punto sobre la recta, en este caso 0   } \\ 
            \text{  Primer punto   } \quad z & = 0 \\
            y &= 1\\ 
            x & = -1 \\ 
            \therefore \left\langle -1,1,0 \right\rangle \\ 
            \text{  Segundo punto  } \quad z &= 1 \\ 
            y & = 0 \\ 
            x & = 0 \\ 
            \therefore \left\langle 0,0,1 \right\rangle \\ 
        \end{align*}
    
    \item Encuentre la ecuación de la recta que pasa por P(-1,1,0) y Q$\underbrace{(0,0,1)}_{r_0}$:
        \begin{align*}
            \vec{r_0} = \left\langle 0,0,1 \right\rangle \! \left\langle -1,1,0 \right\rangle \\ 
            \vec{v} = \vv{QP} 0 \left\langle -1,1,-1 \right\rangle \\ 
            \text{  Ecuaciones paramétricas de la recta:  } \\ 
            x = 0-t \quad y = 0 + t \quad z = 1-t \\ 
        \end{align*}
    
    \item Solución alterna:
        \begin{align*}
            x = -y \quad y = 1-z \quad \text{  Más incognitas que ecuaciones.  }\\
            x,y \; \text{  ó   }\; z \text{  \quad pueden tener cualquier valor  } \quad z=t \\ 
            \begin{matrix}
                x = -1 + t \\ 
                y = 1 - t \\ 
                t = t \\ 
            \end{matrix} \therefore v_2 = \left\langle 1,-1,1 \right\rangle  \quad \vec{r_0} = \left\langle -1,1-0 \right\rangle \\ 
        \end{align*}
    
    \item Solución geométrica:
        \begin{itemize}
            \item Encuentre un punto en ambos planos (0,0,1).
            \item L arecta está en el plano I, entonces la recta es perpendicular al vector normal del plano I.
            \item Está en el plano z, entonces también es perpendicular al segundo vector normal.
            \item $\therefore $ la recta es perpendicular a ambos $\hat{n_1}$ \& $\hat{n_2}$
                \begin{align*}
                    \vec{v} = \hat{n_1} \times \hat{n_2} = \begin{vmatrix}
                        \hat{i} & \hat{j} & \hat{k} \\ 
                        1 & 1 & 0 \\ 
                        1 & 2 & 1 \\ 
                    \end{vmatrix} = \hat{i} - \hat{j} + \hat{k} \\ 
                    \text{  Ecuación de la recta:  } \quad r = \left\langle 0,0,1 \right\rangle + t \left\langle 1,-1,0 \right\rangle \\ 
                \end{align*}
        \end{itemize}
    
    \item Ejercicio 3: Encuentre el punto en el que la línea recta $x=1+2t$, $y=4t$, $z=5t$ interseca al plano. $x-y+2z=17$.
        \begin{align*}
            \begin{matrix}
                x = 1+2t \\ 
                y = 4t \\ 
                z= 5t \\ 
            \end{matrix} \\ 
            \text{  Plano  } \\ 
            x-y+2z = 17 \quad 1+2t-4t+10t = 17 \\ 
            8t = 16 \implies \therefore  t = 2 \\ 
        \end{align*}
        El punto de intersección es (5,8,10).
    
    \item Ejercicio 4: Encuentre una ec. del plano que contiene  la recta $x=1+t$, $y=2-t$, $z=4-3t$ y es paralela a plano $5x+2y+z=1$.
        \begin{itemize}
            \item Cualquier punto sobre la recta que también esté sobre el plano, t= 0. 
        \end{itemize}
        \begin{align*}
            \text{  Evaluemos en t=0  } \quad x=1, \, y=2, \, z=4\\
            \vec{r_0}= \left\langle 1,2,4 \right\rangle  \\ 
        \end{align*}
        \begin{itemize}
            \item \textbf{Nos preguntamos:} ¿Cómo se encuentra $\hat{n}$?
            \item El vectos de dirección de la recta $v=\left\langle 1,-1,-2 \right\rangle$ es paralelo al plano.
            \item Como es paralelo al seguno plano, entonces tiene que ser perpendicular $\hat{n_2} = \left\langle 5,2,1 \right\rangle$
            \item Lo que ocurre entonces es:
        \end{itemize}
            \begin{align*}
                \vec{r_0} = \left\langle 1,2,4 \right\rangle \quad \hat{n}= \left\langle 5,2,1 \right\rangle \\ 
                \text{  Ec. Plano:  } \, \implies \, 5(x-1) + 2(y-2)+1(z-4)=0 \\ 
            \end{align*}
    \item Ejercicio 5: Encuentre los números directores para la recta de intersección entre los planos $x+y+z=1$ \& $x+2y+3z=1$.
            \begin{itemize}
                \item \emph{\textbf{Definición de ``numeros directores":} a,b,c del vector de dirección $\left\langle a,b,c \right\rangle $}
                \item La recta es ortogonal a ambos vectores normales:
            \end{itemize}
            \begin{align*}
                \hat{n_1} = \left\langle 1,1,1 \right\rangle \quad text{ \& } \hat{n_2} = \left\langle 1,2,3 \right\rangle \quad \text{  de ambos planos  }\\
                \vec{v} = \hat{n_1} \times \hat{n_2} = \begin{vmatrix}
                    \hat{i} & \hat{j} & \hat{k} \\
                    1 & 1 & 1 \\ 
                    1 & 2 & 1 \\ 
                \end{vmatrix}   = \hat{i} -2\hat{j} +\hat{k} \\ 
                \text{  Los números directores:   } \quad a=1, b=2, c= 1 \\ 
            \end{align*}
    
    \item Ejercicio 6: Encuentre las ecs. aparamétricas de la recta que pasa por el punto (0,1,2), que es paralelo al plano $x+y+z=2$ y es perpendicular a la recta $r = \left\langle -2t,0,3t \right\rangle $.
            \begin{align*}
               L_1 r= \vec{r_0}+t \vec{v} \quad r_0 = \left\langle 0,1,2 \right\rangle 
            \end{align*}
            \begin{itemize}
                
                \item Aclaraciones: $L_1$ es la incógnita que tenemos que encontrar.
                \item \textbf{Nos preguntamos:} ¿Cómo se encuentra $r$?
                
                \item Plano I: $\hat{n} = \left\langle 1,1,1 \right\rangle $ es perpendicular al plano, es paralelo a $L_1$.
                \item Recta II: $\hat{v_2}= \left\langle -2,0,3 \right\rangle $ es perpendicular a $L_1$
                \item La recta es perpendiculae a $\hat{n}$ y a $\vec{v_2}$
            \end{itemize}
            \begin{align*}
                v=\hat{n} \times \vec{v_2} = \begin{vmatrix}
                    \hat{i} & \hat{j} & \hat{k} \\ 
                    1 & 1 & 1 \\ 
                    -2 & 0 & 3 \\ 
                \end{vmatrix} = 3 \hat{i} - 5 \hat{j} + 2 \hat{k} \\ 
                r_0 = \left\langle 0,1,2  \right\rangle \\ 
                v = \hat{v_2} \times \hat{n} \quad \text{  Ecuaciones paramétricas:   } \\ 
                \begin{matrix}
                    x=0-3t \\ 
                    y = 1-5t \\ 
                    z = 2 +2t \\ 
                \end{matrix} \\ 
            \end{align*}
            
\end{enumerate}



\chapter{Clase - 2020-02-04}
\documentclass{article}
\title{Clase 2020-02-04}
\author{David Gabriel Corzo Mcmath}
\date{2020-Feb-04 10:18:27}
%%%%%%%%%%%%%%%%%%%%%%%%%%%%%%%%%%%%%%%%%%%%%%%%%%%%%%%%%%%%%%%%%%%%%%%%%%%%%%%%%%%%%%%%%%%%%%%%%%%%%%%%%%%%%%%%%%%%%%%%%%%%%%%%%%%%%%%%%%%%%%%
\usepackage[margin = 1in]{geometry}
\usepackage{graphicx}
\usepackage{fontenc}
\usepackage{pdfpages}
\usepackage[spanish]{babel}
\usepackage{amsmath}
\usepackage{amsthm}
\usepackage[utf8]{inputenc}
\usepackage{enumitem}
\usepackage{mathtools}
\usepackage{import}
\usepackage{xifthen}
\usepackage{pdfpages}
\usepackage{transparent}
\usepackage{color}
\usepackage{fancyhdr}
\usepackage{lipsum}
\usepackage{sectsty}
\usepackage{titlesec}
\usepackage{calc}
\usepackage{lmodern}
\usepackage{xpatch}
\usepackage{blindtext}
\usepackage{bookmark}
\usepackage{fancyhdr}
\usepackage{xcolor}
\usepackage{tikz}
\usepackage{blindtext}
\usepackage{hyperref}
\usepackage{listing}
\usepackage{spverbatim}
\usepackage{fancyvrb}
\usepackage{fvextra}
\usepackage{amssymb}
\usepackage{pifont}
\usepackage{longtable}
\usetikzlibrary{arrows}
%%%%%%%%%%%%%%%%%%%%%%%%%%%%%%%%%%%%%%%%%%%%%%%%%%%%%%%%%%%%%%%%%%%%%%%%%%%%%%%%%%%%%%%%%%%%%%%%%%%%%%%%%%%%%%%%%%%%%%%%%%%%%%%%%%%%%%%%%%%%%%%
\begin{document}
\maketitle

\section{13.1 Funciones vectoriales y curvas en el espacio}
\begin{itemize}
    \item Una función vectorial $\vec{r}: R \implies V_3$ :
    \[
      \vec{r}(t) = \left\langle f(t),g(t),z(t) \right\rangle 
    \]
    
    La variable t es un parámetro.
    
    \item Dominio: Números reales, Rango: vector 3D:
        \begin{align*}
            \vec{r} \mathbb{IR} \implies V_3 \quad \vec{r}(t) = \left\langle f(t),g(t),h(t) \right\rangle \\ 
            \text{  t es un parámetro  } \quad \vec{r} = f(t)\hat{i} + g(t)\hat{j} + h(t)\hat{k} \\
        \end{align*}
    
    \item Ejemplo de una función vectorial:
        \begin{align*}
            \vec{r} = \left\langle a,b,c \right\rangle  + t \left\langle d,e,f \right\rangle \\ 
            \vec{r} = \left\langle a+td,b+et,c+tf \right\rangle \\ 
            x = f(t), \quad y = g(t), \quad z = h(t) \\ 
        \end{align*}
    
    \item Ecs. Paramétricas de una función vectorial: 
    
    \item Dominio de ina función vectorial: encuentre el dominio de cada función componente. El dominio de $\vec{r}$ es la intersección de los dominios de cada función componente.
\end{itemize}

%%%%%%%%%%%%%%%%%%%%%%%%%%%%%%%%%%%%%%%%%%%%%%%%%%%%%%%%%%%%%%%%%%%%%%%%%%%%%%%%%%%%%%%%%%%%%%%%
\subsection{Ejercicios}
\begin{enumerate}
    \item Encuentre el dominio:
        \begin{align*}
            r(t) = \left\langle \sqrt[]{r^2-9}, e^{5ln(t)}, ln(t+5) \right\rangle \\ 
            \text{  Evadir raíces negativas, y ln(0)  } \\ 
            \begin{matrix}
                \sqrt[]{t^2-9} \quad \implies \quad \text{  Definida   } \quad t^2 \geq 9 \\ 
                e^{\sin(t)} \quad \text{  siempore definida  } \\ 
                ln(t+5) \quad \text{  Definida cuando   } \quad t +5 > 0 \quad \quad (-5, \infty ) \\ 
                \therefore \text{  El dominio es de   } \quad (-5,\infty ) \, \cup (-5,-3) \, \cup (-3,3) \, \cup [3,\infty ) \\ 
            \end{matrix} \\ 
        \end{align*}
        Recordar: [a,b] el numero si es parte del dominio a,b son partes del dominio. (a,b) los puntos a,b no son parte del dominio.
    
    \item \begin{align*}
        \vec{s}(t) = \left\langle \sin^3(t^2), \cosh(\frac{t}{t^2+1} ), \frac{1}{e^t+4}  \right\rangle \\ 
        \begin{matrix}
            sin^3(t^2), ID_{f(t)} = IR \\ 
            \cosh(\frac{t}{t^2+1} ), ID_{g(t)} = IR \\ 
            \frac{1}{e^t+4}, ID_{h(t)} = IR \\ 
        \end{matrix} \\ 
        \therefore \text{  Dominio de   } \, \vec{s}(t) = (-\infty ,\infty ) \\ 
        e^+4\neq 0 \implies e^t=-4 \implies t = \underbrace{ln(-4)}_{\text{  indefinido  }} \\ 
    \end{align*}
\end{enumerate}


%%%%%%%%%%%%%%%%%%%%%%%%%%%%%%%%%%%%%%%%%%%%%%%%%%%%%%%%%%%%%%%%%%%%%%%%%%%%%%%%%%%%%%%%%%%%%%%%
\section{Limites y continuidad}
\begin{itemize}
    \item \[
        \lim_{t \to a}\vec{r}(t) = \left\langle \lim_{t \to a} f(t),\lim_{t \to a} g(t),\lim_{t \to a} h(t) \right\rangle 
      \]
    
    \item Evalúe el límite de cada función componente.
    \item Si no existe por lo menos un límite de una función componente, entonces $\lim_{t \to a} \vec{r}(t) $ no existe.
    \item f(t) está definida en t=a
    \[
      \lim_{t \to a} f(t) = f(a)
    \]
    
    \item Si se indefine y tiene forma de $\frac{0}{0} $, $\frac{\infty}{\infty} $ usar L'H$\hat{o}$pital.
        \begin{align*}
            \lim_{t \to a} \frac{f(t)}{g(t)} \underbrace{=}_{\frac{0}{0} } \lim_{t \to a} \frac{f'(t)}{g'(t)} \quad \text{  L'Hopital  }
        \end{align*}
    
    \item Contínua en $t=a$ si $\lim_{t \to a} \vec{r}(t)=\vec{r}(a)$
    \item Evite asíntotas verticales, saltos y agujeros. Ejemplo: 
        \begin{align*} 
            \lim_{t \to a} \frac{\sin(x)}{x} \underbrace{=}_{\text{  LH  }} \lim_{t \to a} \frac{cos(x)}{1} = 1 \\ 
        \end{align*}
    
\end{itemize}


%%%%%%%%%%%%%%%%%%%%%%%%%%%%%%%%%%%%%%%%%%%%%%%%%%%%%%%%%%%%%%%%%%%%%%%%%%%%%%%%%%%%%%%%%%%%%%%%
\subsection{Ejercicios}
\begin{itemize}
    \item Sea $\vec{r}(t)=\left\langle \frac{\tan(\pi t)}{t} , e^{t-2}, \frac{ln(t-1)}{t^2-1}  \right\rangle $.
    \item Analice si la función $\vec{r}(t)$ es contínua en $t=2$.
        \begin{align*}
            \vec{r}(t) = \left\langle \frac{\tan 2\pi }{2}, e^0, \frac{ln(1)}{3}   \right\rangle \\ 
            \begin{matrix}
                \lim_{t \to 2} \underbrace{\frac{\tan \pi t}{t}}_{\frac{0}{2} } = 0 \\
                \lim_{t \to 2} e^{t-2} = 1 \\ 
                \lim_{t \to 2} \frac{ln(t-1)}{t^2-1} = 0 \\ 
            \end{matrix} \\ 
            \therefore \vec{r} \, \text{ si es contínua en t=2   } \, \lim_{t \to 1} \vec{r}(t) = \vec{r}(2)
        \end{align*}
    
    \item Encuentre $\lim_{t \to 1} \vec{r}(t)$ analice el límite de cada función componente por separado.
        \begin{align*}
            f: \, \lim_{t \to 1} \frac{\tan 2\pi }{2} = \frac{0}{1}  \\ 
            g: \, \lim_{t \to 1} e^{t-2} = e^{-1} \\ 
            h: \, \lim_{t \to 1} \frac{ln(t-1)}{t^2-1} = \, \text{  No existe, por ln(0) estar indefinido.  } \\ 
        \end{align*}
    
    \item Analice si $\vec{r}(t)$ es contínua e t=1.
        \begin{align*}
            \underbrace{\lim_{t \to 1} \vec{r}(t) = \vec{r}(1) }_{\text{  No es contínua en t=1, r(1) está indefinida.  }}\\ 
        \end{align*}
    
    \item Agujero $\vec{s}(t) = \left\langle \frac{\tan \pi t }{t-1} , e^{t-2}, \frac{ln(2t-1)}{t^2-1}  \right\rangle $ \newline 
     No es contínua en t=1, pero su límite existe.
     \begin{align*}
         \lim_{t \to 1} \frac{\tan \pi t}{t-1} \underbrace{=}_{LH} \lim_{t \to 1} \frac{\pi \sec^2 \pi t }{1} = \frac{\pi}{(\cos \pi)^2} = \pi \\ 
        \lim_{t \to 1} e^{t-2 } = e^-1 = \frac{1}{e} \\ 
        \lim_{t \to 1} \frac{\ln(2t-1)}{t^2-1} \underbrace{=}_{\frac{0}{0}} = \lim_{t \to 1} \frac{\frac{2}{2t-1} }{2t} = \lim_{t \to 1} \frac{2}{2t(2t-1)} = \frac{1}{1(2-1)} = 1 \\ 
        \therefore \lim_{t \to 1} \left\langle \pi, \frac{1}{e}, 1 \right\rangle \quad \text{  es un agujero   } \, \vec{s}(1) \, \text{  está indefinido  } \\   
     \end{align*}
\end{itemize}


%%%%%%%%%%%%%%%%%%%%%%%%%%%%%%%%%%%%%%%%%%%%%%%%%%%%%%%%%%%%%%%%%%%%%%%%%%%%%%%%%%%%%%%%%%%%%%%%
\section{Curvas en el espacio}

\begin{align*}
    x = f(t) \\ 
    y = g(t) \\ 
    z = h(t) \\ 
\end{align*}
\begin{figure}[htbp]
    \centering
    % \includegraphics[width=cm]{}
    \caption{Curvas paramétricas en el espacio}
    \label{}
\end{figure}

%%%%%%%%%%%%%%%%%%%%%%%%%%%%%%%%%%%%%%%%%%%%%%%%%%%%%%%%%%%%%%%%%%%%%%%%%%%%%%%%%%%%%%%%%%%%%%%%
\subsection{Espirales}
\begin{itemize}
    \item Grafique la curva $\vec{r}(t)$:
        \begin{align*}
            \vec{r}(t) = \underbrace{2 \hat{i} \sin (t)}_{x} + \underbrace{2 \hat{j} \cos (t) 
            }_{y} + \underbrace{\hat{k} \frac{t}{\pi}}_{z} \\ 
            \begin{matrix}
                t & x & y & z \\ 
                0 & 0 & 2 & 0.5 \\ 
                \frac{\pi}{2} & 2&  0&  0.5 \\ 
                \pi & 0 & -2&  1 \\ 
                \frac{3\pi}{2} & 2 &  0 & 1.5 \\   
                2\pi &  0 & 2 & 2 \\ 
            \end{matrix} 
        \end{align*}
        \begin{figure}[htbp]
            \centering
            % \includegraphics[width=cm]{}
            \caption{Curva paramétrica}
            \label{}
        \end{figure}
    
    \item Grafique:
        \begin{align*}
            \vec{r}(t) = \left\langle \sin \pi t , t ,\cos \pi t \right\rangle \\ 
            \text{  Graficar la circumferencia  } \, x^2+z^2= 1\, , y = 0 \\ 
            \vec{r}(0) = \left\langle 0,0,1 \right\rangle \quad \text{  El vector que nos servirá para delimitar la gráfica del espiral  } \\
            \text{  Por ejemplo:   } \, \vec{s}(t) = \left\langle \sin t, t^2 , \cos t \right\rangle \\    
        \end{align*}
\end{itemize}




















\end{document}


\chapter{Clase - 2020-02-06}
\documentclass{article}
\title{2020-02-06}
\author{David Gabriel Corzo Mcmath}
\date{2020-Feb-06 10:23:27}
%%%%%%%%%%%%%%%%%%%%%%%%%%%%%%%%%%%%%%%%%%%%%%%%%%%%%%%%%%%%%%%%%%%%%%%%%%%%%%%%%%%%%%%%%%%%%%%%%%%%%%%%%%%%%%%%%%%%%%%%%%%%%%%%%%%%%%%%%%%%%%%
\usepackage[margin = 1in]{geometry}
\usepackage{graphicx}
\usepackage{fontenc}
\usepackage{pdfpages}
\usepackage[spanish]{babel}
\usepackage{amsmath}
\usepackage{amsthm}
\usepackage[utf8]{inputenc}
\usepackage{enumitem}
\usepackage{mathtools}
\usepackage{import}
\usepackage{xifthen}
\usepackage{pdfpages}
\usepackage{transparent}
\usepackage{color}
\usepackage{fancyhdr}
\usepackage{lipsum}
\usepackage{sectsty}
\usepackage{titlesec}
\usepackage{calc}
\usepackage{lmodern}
\usepackage{xpatch}
\usepackage{blindtext}
\usepackage{bookmark}
\usepackage{fancyhdr}
\usepackage{xcolor}
\usepackage{tikz}
\usepackage{blindtext}
\usepackage{hyperref}
\usepackage{listing}
\usepackage{spverbatim}
\usepackage{fancyvrb}
\usepackage{fvextra}
\usepackage{amssymb}
\usepackage{pifont}
\usepackage{longtable}
%%%%%%%%%%%%%%%%%%%%%%%%%%%%%%%%%%%%%%%%%%%%%%%%%%%%%%%%%%%%%%%%%%%%%%%%%%%%%%%%%%%%%%%%%%%%%%%%%%%%%%%%%%%%%%%%%%%%%%%%%%%%%%%%%%%%%%%%%%%%%%%
\begin{document}
\maketitle

\section{13.2 Cálculo con funciones vectoriales, pg.55}
\begin{itemize}
    \item Derivadas:
        \[
          \vec{r}'(t) \quad \text{  Respecto a t  }
        \]
    
    \item Integrales:
        \[
          \int_{}^{}\vec{r}'(t)dt \quad \text{  Respecto a t  }
        \]
\end{itemize}


%%%%%%%%%%%%%%%%%%%%%%%%%%%%%%%%%%%%%%%%%%%%%%%%%%%%%%%%%%%%%%%%%%%%%%%%%%%%%%%%%%%%%%%%%%%%%%%%
\subsection{Deriviadas}
\begin{itemize}
    \item \[
      \vec{r}'(t) = \lim_{h \to 0} \frac{r(t+h)-r(t)}{h} 
    \]
    
    \item Como la fucnión $\vec{r}(t)$ está definida por tres funciones componentes se puede hacer: 
        \[
          \vec{r}'(t) = \lim_{h \to 0} \left\langle 
          \underbrace{\lim_{h \to 0} \frac{f(t+h)-f(t)}{h}}_{f'(t)}, 
          \underbrace{\lim_{h \to 0} \frac{g(t+h)-g(t)}{h}, }_{g'(t)}
          \underbrace{\lim_{h \to 0} \frac{h(t+h)-h(t)}{h}  }_{h('t)}
          \right\rangle 
        \]
    
    \item Derivada entonces es : 
        \[
          \vec{r}'(t) = \left\langle f'(t),g'(t),h'(t) \right\rangle 
        \]
\end{itemize}


%%%%%%%%%%%%%%%%%%%%%%%%%%%%%%%%%%%%%%%%%%%%%%%%%%%%%%%%%%%%%%%%%%%%%%%%%%%%%%%%%%%%%%%%%%%%%%%%
\subsection{Integrales}
\begin{itemize}
    \item Integral:
        \[
          \int_{}^{}\vec{r}(t)dt = = \int_{}^{}(f \hat{i} + g \hat{j} + h \hat{k} )dt
        \]
        \[
          \hat{i} \int_{}^{}fdt + \hat{j} \int_{}^{}gdt + \hat{k} \int_{}^{}hdt 
        \]
        Integrar la función componente.
\end{itemize}


%%%%%%%%%%%%%%%%%%%%%%%%%%%%%%%%%%%%%%%%%%%%%%%%%%%%%%%%%%%%%%%%%%%%%%%%%%%%%%%%%%%%%%%%%%%%%%%%
\section{Ejercicios}
\begin{enumerate}
    \item Encuentre la 1era y segunda derivada de las siguientes funciones:
        \begin{align*}
            \vec{r}(t) = \left\langle \sin(4t),t^2,ln(\sin(t)) \right\rangle \\ 
            \vec{r}'(t) = \left\langle 4\cos(4t),2t,\frac{\cos(t)}{\sin(t)}  \right\rangle \\ 
            \vec{r}'(t) = \left\langle 4\cos(4t),2t,\cot(t) \right\rangle \\ 
        \end{align*}
        \begin{align*}
            \vec{r}''(t) = \left\langle f''(t),g''(t),h''(t) \right\rangle \\ 
            \therefore \quad \vec{r}''(t) = \left\langle -16\sin(4t),2,-\csc^2(t) \right\rangle \\ 
        \end{align*}
    
    \item Derive: $\vec{s}(t) = \hat{i} \tan(4t) + hat{j}ln(4t+1) + \hat{k} (5-2t)^{\frac{1}{2} }$
        \begin{align*}
            \vec{s}'(t) = 4 \hat{i} (\sec(4t))^2 + \hat{j} 4(4t+1)^{-1} -\hat{k} (5-2t)^{-\frac{1}{2} } \\ 
            \vec{s}''(t) = 8 \hat{i} \times \sec(4t)\times  \sec(4t)\times  \tan (4t) \times 4  - 16 \hat{j} (4t-1)^{-2} - \frac{\hat{k}}{2} (5-2t)^{-\frac{3}{2} } \times (-2) \\  
            \vec{s}''(t) = 32 \hat{i} \times \sec^2(4t)\times  \tan (4t)  - 16 \hat{j} (4t-1)^{-2} - \frac{\hat{k}}{2} (5-2t)^{-\frac{3}{2} } \times (-2) \\ 
        \end{align*}
\end{enumerate}


\section{Recordatorios \& rectas tangentes de funciones vectoriales}
\begin{itemize}
    \item \emph{\textbf{Recordar lo siguiente: }$f'(a)$ es igual a la pendiente de la drecta tangeente a $f(x)$ en $x=a$}.
    \item \emph{\textbf{Recordar lo siguiente: }La recta tangente}.
        \[
          L_1: \quad y = f(a)+f'(a)(x-a) \quad \quad \text{  Ec. Recta Tangente  }
        \]
    
    \item Con una función vectorial:
        \begin{align*}
            \vec{r}= \left\langle f,g,h \right\rangle , \quad \quad x = f(t), \, y=g(t), \, z=h(t) \\ 
            \text{  Hay ecuaciones paramétricas para cada variable:  } \\ 
            \vec{r}'(a) = \left\langle f'(a),g'(a),h'(a) \right\rangle \\ 
            \text{  Vector de pendientes de rectas tangentes a la curva  } \, \vec{r}(t).
        \end{align*}
    
    \item La derivada de una función vectorial se le da elnombre de \textbf{``vector tangente''} $\vec{r}(t):\vec{r}'(a)$.
    \item Recta tangente: es ahora una función vectorial.
        \[
          \vec{r}(t) = \vec{r}(a) + \vec{r}'(a)t 
        \]
    
    \item Ecs. Paramétricas:
        \begin{align*}
            \begin{matrix}
                x=f(a)+f'(a)t \\ 
                y=g(a)+g'(a)t \\ 
                z=h(a)+h'(t)t \\ 
            \end{matrix}
        \end{align*}
    
    \item Vector tangente: $r'(a)$ en $t=a$
    \item Vector tangente unitario: $\frac{r'(a)}{\left| r'(a) \right| } = \vec{T}(a)$
\end{itemize}



%%%%%%%%%%%%%%%%%%%%%%%%%%%%%%%%%%%%%%%%%%%%%%%%%%%%%%%%%%%%%%%%%%%%%%%%%%%%%%%%%%%%%%%%%%%%%%%%
\section{Ejercicios}
\begin{itemize}
    \item Encuentre las ecs. paramétricas de la recta tangente a la curva : $s(t)=\left\langle 2\cos(t),2\sin(t),4\cos(2t) \right\rangle $ en el punto $(\sqrt{3},1,2)$:
        \begin{align*}
            \text{  Recta tangente:  } \quad \vec{r}_T(t) = \vec{r}(a)+t \vec{r}'(a) \\ 
            \vec{r}_T(a) = \left\langle \sqrt{3},1,2 \right\rangle \\ 
            \text{  Derivada: } \quad \vec{r}'(t) = \left\langle -2\sin(t),2\cos(t),-8\sin(2t) \right\rangle \\ 
            \text{  \textbf{Nos preguntamos:} ¿Cómo encuentro ``a'' ? igualamos   } \quad r(t) = \left\langle \sqrt{3},1,2 \right\rangle  \\ 
            \begin{matrix}
                2\cos(t) = \sqrt[]{3} \implies \cos(t) = \frac{\sqrt{3}}{2} \implies t = \frac{\pi }{6}  \\ 
                2\sin(t) = 1 \implies 2\sin(\frac{\pi }{6} ) = 2 \times \frac{1}{2} = 1\\ 
                4 \cos(2t)=2 \implies 4\cos(\frac{\pi }{3} ) = 4 \times \frac{1}{2} = 2\\ 
            \end{matrix} \\ 
            \text{  Vector tangente:   } \quad \vec{r}'(\frac{\pi }{6} )= \left\langle -2\sin(\frac{\pi }{6}),2cos(\frac{\pi}{6} ), -8\sin(\frac{\pi }{3} )  \right\rangle \\ 
            \vec{r}_T(t)= \left\langle \sqrt{3},1,2 \right\rangle + t \left\langle -1,\sqrt{3},-4\sqrt{3} \right\rangle \\ 
            \therefore  \\ 
            \begin{matrix}
                x= \sqrt{3}-1 \\ 
                y=1+\sqrt{3}t \\ 
                z=2-4\sqrt{3}t \\ 
            \end{matrix}\\ 
        \end{align*}
\end{itemize}

























\end{document}


\chapter{Clase - 2020-02-11}
% \documentclass{article}
\title{Temporary}
\author{David Gabriel Corzo Mcmath}
\date{\today}
%%%%%%%%%%%%%%%%%%%%%%%%%%%%%%%%%%%%%%%%%%%%%%%%%%%%%%%%%%%%%%%%%%%%%%%%%%%%%%%%%%%%%%%%%%%%%%%%%%%%%%%%%%%%%%%%%%%%%%%%%%%%%%%%%%%%%%%%%%%%%%%
\usepackage[margin = 1in]{geometry}
\usepackage{graphicx}
\usepackage{fontenc}
\usepackage{pdfpages}
\usepackage[spanish]{babel}
\usepackage{amsmath}
\usepackage{amsthm}
\usepackage[utf8]{inputenc}
\usepackage{enumitem}
\usepackage{mathtools}
\usepackage{import}
\usepackage{xifthen}
\usepackage{pdfpages}
\usepackage{transparent}
\usepackage{color}
\usepackage{fancyhdr}
\usepackage{lipsum}
\usepackage{sectsty}
\usepackage{titlesec}
\usepackage{calc}
\usepackage{lmodern}
\usepackage{xpatch}
\usepackage{blindtext}
\usepackage{bookmark}
\usepackage{fancyhdr}
\usepackage{xcolor}
\usepackage{tikz}
\usepackage{blindtext}
\usepackage{hyperref}
\usepackage{listing}
\usepackage{spverbatim}
\usepackage{fancyvrb}
\usepackage{fvextra}
\usepackage{amssymb}
\usepackage{pifont}
\usepackage{longtable}
\usetikzlibrary{arrows,shapes}
%%%%%%%%%%%%%%%%%%%%%%%%%%%%%%%%%%%%%%%%%%%%%%%%%%%%%%%%%%%%%%%%%%%%%%%%%%%%%%%%%%%%%%%%%%%%%%%%%%%%%%%%%%%%%%%%%%%%%%%%%%%%%%%%%%%%%%%%%%%%%%%






% \date{2020-Feb-11 10:05:03}
% \begin{document}

\section{13.2 Cálculo de funciones vectoriales}
\begin{itemize}
    \item Derivadas:
        \[
            \vec{r}\,'(t)=\left\langle f'(t),g'(t),h'(t) \right\rangle \\ 
        \]
    
    \item Vector Tangente:
        \[
          \vec{r}\, ' (t) 
        \]
    
    \item Tangente unitario:
        \[
          \vec{T}(t)=\frac{r'(t)}{\left| r'(t) \right| } 
        \]
    
    \item Integrales indefinidas:
        \begin{align*}
          \int_{}^{}\left\langle f,g,h \right\rangle dt = \left\langle F+C_1,G+C_2,H+C_3 \right\rangle \\
          \int_{}^{}\vec{r}(t) dt  = \vec{R}(t) + \vec{C} \\
          \vec{R} \quad \text{ vector de Antiderivadas  } \\ 
          \vec{C} \quad \text{  Vector de constantes  } \\   
        \end{align*}
    
    \item Integrales definidas:
        \[
          \int_{a}^{b}\vec{r}(t)dt = \hat{i} \int_{a}^{b}f(t)dt+ \hat{j} \int_{a}^{b}g(t)dt + \hat{k} \int_{a}^{b}h(t)dt
        \]
\end{itemize}


%%%%%%%%%%%%%%%%%%%%%%%%%%%%%%%%%%%%%%%%%%%%%%%%%%%%%%%%%%%%%%%%%%%%%%%%%%%%%%%%%%%%%%%%%%%%%%%%%%%
\section{Ejercicios de integración}
\begin{enumerate}
    \item $\int_{0}^{1}\left[\frac{4}{1+t^2}\hat{i} + sec^2(\frac{\pi t}{4} )\right] dt$:
        \begin{align*}
            4 \hat{i} \times \tan^{-1}(t) \Big|_0^1 + \hat{k} \times \tan(\frac{\pi t}{4} ) \Big|_{0}^{1} \\ 
            I_i = 4 \hat{i}  \frac{\pi}{4} + \hat{k} \frac{4}{\pi } = pi \hat{i} + \hat{k} \frac{4}{\pi } = \left\langle \pi,0,\frac{4}{\pi } \right\rangle \\ 
        \end{align*}
    
    \item $\int_{}^{}\left\langle te^{t^2},te^t,\frac{q}{\sqrt[]{1-t^2}} \right\rangle dt$ :
        \begin{align*}
            x: \quad \int_{}^{}e^{t^2}t\,dt =& \frac{1}{2} \int_{}^{}e^{u}du = \frac{1}{2} e^{t^2} +C_1 \\ 
                &\begin{matrix}
                    u = t^2 \\ 
                    du = 2tdt \\ 
                \end{matrix} \\ 
            y: \quad \int_{}^{} te^{t} dt = te^{t} - \int_{}^{} te^t-e^t+C_2 \\ 
                \begin{matrix}
                    u = t \quad dv = e^t dt \\ 
                    du = dt \quad v = e^t \\ 
                \end{matrix}
            t: \quad \int_{}^{}\frac{1}{1-t^2} dt = \frac{\cos(\theta)}{\sin(\theta)}d\theta = \int_{}^{} d\theta = \underbrace{\theta + C_3}_{\sin^{-1}(t)+C_3} = \sin^{-1}(t)+C_3 \\
            \therefore \quad \int_{}^{} \left\langle te^{t^2},te^t,\frac{1}{\sqrt{1-t^2}} \right\rangle dt = \frac{1}{2} et^{t^2}+C_1,te^t-e^t+C_2,\sin^1(t) + C_3 \\  
        \end{align*}
\end{enumerate}


%%%%%%%%%%%%%%%%%%%%%%%%%%%%%%%%%%%%%%%%%%%%%%%%%%%%%%%%%%%%%%%%%%%%%%%%%%%%%%%%%%%%%%%%%%%%%%%%%%%
\section{Movimiento en el espacio}
Dado el vector posición $\vec{r}(t)$ de un objeto: 
\begin{itemize}
    \item Vector velocidad:
        \[
          \vec{c}(t) = \vec{r}\, ' (t)
        \]
    
    \item Vector aceleración:
        \[
          \vec{a}(t) = \vec{v}\, (t) = \vec{r}\, '' (t)
        \]
    
    \item Rapidez:
        \[
          \left| \vec{v}(t) \right| 
        \]
    
    \item Distancia:
        \[
          \left| \vec{r}(t) \right| 
        \]    
\end{itemize}

Dado el vector de aceleración $\vec{a}(t)$ :
\begin{itemize}
    \item Velocidad:
        \[
          \vec{v}(t) = \int_{}^{}\vec{a}(t)dt + \vec{C}_1
        \]
    
    \item Desplazamiento o posición:
        \[
          \vec{r}(t) = \int_{}^{} \vec{v}(t)dt + C_2 
        \]
\end{itemize}


\subsection{Ejercicios}
\begin{enumerate}
    \item Encuentre la velocidad, aceleración y rapidez dada la posición del objeto:
        \begin{align*}
            \vec{r}(t) = \hat{i} t  + 2 \hat{j} \cosh(4t) + 3 \hat{k} \sinh(3t) \\ 
            \text{  Encontramos velocidad:  } \quad \vec{r}\, ' (t) = \vec{v}(t) = \hat{i} + 8 \hat{j} \sinh(4t)+9 \hat{k} \cosh(3t) \\ 
            \text{  Encontramos la aceleración:  } \quad \vec{r}\, ''(a) = \vec{a}(t) + 32 \hat{j} \cosh(4t) + 27 \hat{k} \sinh(3t) \\ 
            \text{  Encontramos la rapidez:   } \quad \left| \vec{v}(t) \right| = \sqrt{1+64\sinh(4t)+81\sinh^2(3t)} \\ 
            \text{  Encontramos la distancia:   } \quad \left| \vec{r}(t) \right| = \sqrt{t^2+4\cosh^2(4t)+9\sinh^2(3t)} \\ 
        \end{align*}
        \begin{itemize}[label=\#]
            \item Tarea \# 6: Integrales func. vectoriales 14.1 Funciones en varias variables.
            \item Tarea opcional consolidado: 12,13,14.1 
        \end{itemize}
    
    \item Encuentre la velocidad y posición del objeto dada $\vec{a}(t)$ y las condiciones iniciales:
        \begin{center}
            \begin{align*}
                \vec{a}(t)= 6 t \hat{i}+ \hat{j}  \cos(t)- \hat{k} \sin(2t), \quad \vec{v}(0) = \begin{matrix}
                    \hat{i} + \hat{k} \\ 
                    \vec{r}(0) = 2 \hat{j}  - \hat{k} \\ 
                \end{matrix} \\ 
                \text{  Velocidad:   } \quad \int_{}^{}\vec{a}(t)dt \\ 
                \vec{v}(t) = \left\langle 3t^2+C_1, \sin(t)+C_2, \frac{1}{2}\cos(2t)+C_3 \right\rangle \\ 
                \text{  Encuentro   } \; \vec{v}(0) = \left\langle C_1,C_2,\frac{1}{2} + C_3 \right\rangle  = \left\langle 1,0,1 \right\rangle \\
                \text{  Resolver para las constantes:   } \quad \begin{matrix}
                    C_1 = 1, \\  C_2 = 0, \\  \frac{1}{2} + C_3 = 1 \quad \implies  \quad C_3 = \frac{1}{2} \\ 
                \end{matrix} \\ 
                \text{  Posición:   } \quad \int_{}^{}\vec{v}(t)dt \\ 
                \vec{r}(t) \left\langle t^3+t+d_1, -\cos(t)+d_2, \frac{1}{4}\sin(2t) + \frac{t}{2} + d_3 \right\rangle \\ 
                \vec{r}(0) = \left\langle \underbrace{d_1,-1+d_2,d_3}_{\begin{matrix}
                    d_1 = 0 \\ 
                    -1+d_2=2 \implies d_2 = 3 \\ 
                    d_3 = -1 \\ 
                \end{matrix}} \right\rangle  \\
                \text{  Posición:   }\quad \vec{r}(t)= \left\langle t^3+t,3-\cos(t),\frac{1}{4}\sin(2t)+\frac{t}{2}-1 \right\rangle \\ 
            \end{align*}
        \end{center}
    
    \item $\vec{a}(t)= 8t \hat{i} + \sinh(t)\hat{j} - \hat{k} e^{\frac{t}{2} }$ : 
        \begin{center}
            \begin{align*}
                \underbrace{\vec{v}(0) = \vec{0}}_{\text{  Está en reposo  }} \quad \quad \vec{s}(0) = 2 \hat{i} + \hat{j} - 3 \hat{k}  \\ 
                \text{  Velocidad:   }\quad \vec{v}(t)= \left\langle 4t^2+C_1,\cosh(t)+C_2,-2e^{\frac{t}{2}}+C_3 \right\rangle \\ 
                \vec{v}(0) = \left\langle \underbrace{C_1,1+C_2,-2+C_3}_{\begin{matrix}
                    C_1 = 0, \\  C_2 = -1 \\ C_3 =2 \\ 
                \end{matrix}} \right\rangle = \left\langle 0,0,0 \right\rangle \\ 
                \vec{v}(t) = \left\langle 4t^2,\cosh(t)-1,-2e^{\frac{t}{2} +2} \right\rangle \\ 
                \text{  Posición:   } \quad \vec{r}(t) = \left\langle \frac{4}{3}t^3+C_1, \sinh(t)-t+C_2,-4e^{\frac{t}{2}}+2t+C_3 \right\rangle  \\ 
                \vec{r}(0) = \left\langle C_1,C_2,-4+C_3 \right\rangle = \underbrace{\left\langle 2,1,-3 \right\rangle }_{\begin{matrix}
                    C_2=1 \\ 
                    C_3 = -3+4 = 1 \\ 
                \end{matrix}} \\ 
                \vec{r}(t) = \left\langle \frac{4}{3}t^3+2,\sinh(t)-t+1,-4e^{\frac{t}{2}}+2t+1 \right\rangle \\  
            \end{align*}    
            \begin{itemize}[label=\#]
                \item Se evalúa el vector en 0 por que se quiere saber el valor de las constantes cuando están en reposo.
                \item Por defecto siempre evaluar en 0 para encontrar $C_1,C_2$\&$C_3$.
            \end{itemize}
        \end{center}
\end{enumerate}


%%%%%%%%%%%%%%%%%%%%%%%%%%%%%%%%%%%%%%%%%%%%%%%%%%%%%%%%%%%%%%%%%%%%%%%%%%%%%%%%%%%%%%%%%%%%%%%%%%%
\section{13.3 lOGN}
10.4 Ecs. Paramétricas de una curva en el plano de dos dimensiones era:
\begin{center}
   \begin{align*}
       \begin{matrix}
        x=f(t) \\ 
        y = g(t) \\ 
       \end{matrix}
   \end{align*}
\end{center}
\begin{itemize}
    \item La longitud de arco:
        \begin{align*}
            L = \int_{a}^{b} \sqrt{(x')^2+(y')^2+(z')^2} dt
        \end{align*}
    
    \item Función vectorial:
        \[
          \vec{r} = \left\langle f,g,h \right\rangle = \left\langle x,y,z \right\rangle \\ 
        \]
    
    
    \item Derivada de función vectorial:
        \[
          \vec{r}\,'= \left\langle x',y',z' \right\rangle 
        \]
    
    \item Magnitud: 
        \[
          \left| \vec{r}\,' \right| = \sqrt{(x')^2+(y')^2+(z')^2}
        \]
    
    \item En general: 
        \[
          L = \int_{a}^{b} = \left| \vec{r}\,'(t) \right|dt  
        \]
\end{itemize}



%%%%%%%%%%%%%%%%%%%%%%%%%%%%%%%%%%%%%%%%%%%%%%%%%%%%%%%%%%%%%%%%%%%%%%%%%%%%%%%%%%%%%%%%%%%%%%%%%%%
\section{Ejercicios}
Encuentre la longitud de las siguientes curvas:
\begin{enumerate}
    \item $\vec{r}(t)= \left\langle \cos(t),\sin(t),\ln(\cos) \right\rangle $ en $0 \le t \le \frac{\pi }{4} $
        \begin{center}
            \begin{align*}
                L = \int_{0 }^{\frac{\pi }{4}} \left| \vec{r}\, ' (t) \right| dt \\ 
                \vec{r}\,'(t) = \left\langle -\sin(t),\cos(t),\tan^2(t) \right\rangle \\ 
                \left| \vec{r}\,'(t) \right| = \sqrt{\sin^2(t)+\cos^2(t)+\tan^2(t)} = \sqrt{1+\tan^2(t)} = \sec^2(t) = \sec^2(t) \\ 
                L = \int_{0}^{\frac{\pi }{4}}\sec(t)dt = \ln\left| \sec(t)+\tan(t) \right| \Big|_{0}^{\frac{\pi }{4}} = \ln \left| \sec(\frac{\pi }{4}) + \tan(\frac{\pi }{4} ) \right| -  \ln \left| \sec(0) + \tan(0) \right|\\ 
                L = \ln \left| \frac{2}{\sqrt{2}} + 1  \right|  - \ln \left| 1 \right| = \ln \left| \sqrt{2}+1 \right| \\ 
            \end{align*}
        \end{center}
    
    \item $\vec{r}(t)= \left\langle 12t,8t^{\frac{3}{2}},3t^2 \right\rangle $ en $0 \le t \le 1$ :
        \begin{center}
           \begin{align*}
               \vec{r}\,'(t) = \left\langle 12,12t^{\frac{1}{2}},6t \right\rangle = 6 \left\langle 2,2t^{\frac{1}{2}},t \right\rangle  \\ 
               \left| \vec{r}\,'(t) \right| = 6 \sqrt{4+4t+t^2} = 6 \sqrt{(t+2)^2}= 6(t+2) \\ 
               L = \int_{0}^{1} (6t+12) dt = 3t^2+12t \Big|_{0}^{1} = 3+12 = 15 \\ 
           \end{align*}
        \end{center}
\end{enumerate}









% \end{document}


\chapter{Clase - 2020-02-11}
% \documentclass{article}
\title{Temporary}
\author{David Gabriel Corzo Mcmath}
\date{\today}
%%%%%%%%%%%%%%%%%%%%%%%%%%%%%%%%%%%%%%%%%%%%%%%%%%%%%%%%%%%%%%%%%%%%%%%%%%%%%%%%%%%%%%%%%%%%%%%%%%%%%%%%%%%%%%%%%%%%%%%%%%%%%%%%%%%%%%%%%%%%%%%
\usepackage[margin = 1in]{geometry}
\usepackage{graphicx}
\usepackage{fontenc}
\usepackage{pdfpages}
\usepackage[spanish]{babel}
\usepackage{amsmath}
\usepackage{amsthm}
\usepackage[utf8]{inputenc}
\usepackage{enumitem}
\usepackage{mathtools}
\usepackage{import}
\usepackage{xifthen}
\usepackage{pdfpages}
\usepackage{transparent}
\usepackage{color}
\usepackage{fancyhdr}
\usepackage{lipsum}
\usepackage{sectsty}
\usepackage{titlesec}
\usepackage{calc}
\usepackage{lmodern}
\usepackage{xpatch}
\usepackage{blindtext}
\usepackage{bookmark}
\usepackage{fancyhdr}
\usepackage{xcolor}
\usepackage{tikz}
\usepackage{blindtext}
\usepackage{hyperref}
\usepackage{listing}
\usepackage{spverbatim}
\usepackage{fancyvrb}
\usepackage{fvextra}
\usepackage{amssymb}
\usepackage{pifont}
\usepackage{longtable}
\usetikzlibrary{arrows,shapes}
%%%%%%%%%%%%%%%%%%%%%%%%%%%%%%%%%%%%%%%%%%%%%%%%%%%%%%%%%%%%%%%%%%%%%%%%%%%%%%%%%%%%%%%%%%%%%%%%%%%%%%%%%%%%%%%%%%%%%%%%%%%%%%%%%%%%%%%%%%%%%%%






% \begin{document}


\section{Resolución de corto}
\begin{enumerate}
    \item Analice la función $r=\left\langle 3e^{-t},ln(2t^2-1),\tan(2\pi ) \right\rangle $ en $t=1$:
        \begin{center}
           \begin{align*}
               \lim_{t \to 1} \vec{r}(t) = \left\langle \lim_{t \to 1} 3e^{-t},\lim_{t \to 1} ln(2t^2-1),\lim_{t \to 1} \tan(2\pi ) \right\rangle \\ 
               \vec{r} = \left\langle 3e^{-1},ln(1),\tan(2\pi) \right\rangle  = \left\langle 3e^{-1},0,0 \right\rangle \\
               \therefore \quad \text{  r es contínua en t=1  } \\   
           \end{align*}
           \begin{itemize}[label=\#]
               \item Si la pregunta hubiese sido en cuándo se indefine, se saca el dominio de cada función.
           \end{itemize}
        \end{center}
    
    \item Encuentre la ec. de la recta tatente a $r(t)= \left\langle te^{t-1},\frac{8}{\pi } \arctan(t), 2\ln(t) \right\rangle $ en $t=1$.
        \begin{center}
           \begin{align*}
               \vec{r}(0) = \left\langle 1\times e^0,\frac{8}{\pi } \arctan(1), 2ln(0) \right\rangle = \left\langle 1,2,0 \right\rangle \\ 
            %    \vec{r}\,'(t) = \left\langle e^{t-1} \right\rangle 
            \text{  Terminar de copiar  } \\ 
           \end{align*}
        \end{center}
\end{enumerate}


%%%%%%%%%%%%%%%%%%%%%%%%%%%%%%%%%%%%%%%%%%%%%%%%%%%%%%%%%%%%%%%%%%%%%%%%%%%%%%%%%%%%%%%%%%%%%%%%%%%
\section{14.1 Funciones de varias variables}
\begin{itemize}
    \item Cuando teníamos sólo una función de una variable no había tanta complicación, las gráficas eran curvas en el plano. Cuando empezaba y terminaba la curva en $x$ nos daba el dominio. Había una variable independiente $x$ y la variable dependiente $y$, los dominios eran intervalos, y cada $x$ sólo podía tener \textbf{un} sólo valor de $y$.
    \item En funciones de 2 variables se va a describir como:  
        \begin{center}
           \begin{align*}
                z = f(x,y) \quad &\text{  Dos variables independientes x,y  } \\ 
                & \text{  Variable dependiente z  } \\ 
           \end{align*}
        \end{center}
        \begin{itemize}
            \item Entonces $f$ es una regla que asigna a cada punto $(x,y)$ a lo sumo un valor de $z$.
        \end{itemize}
        \[
          f: \, \underbrace{\mathbb{R}^2}_{\text{  Dominio  }} \rightarrow \underbrace{\mathbb{R}}_{\text{  Rango  }}
        \]
        \begin{itemize}
            \item Estamos pasando de una región por medio de una función $z$ llego a tener $f(x,y)$ en la dimensión correspondiente.
            \item Los dominios en estas funciones se vuelven superficies.
        \end{itemize}
    
    \item El dominio de una función de dos variables: un conjunto que consiste de todos los puntos o pares ordenados $(x,y)$ para los cuales $f(x,y)$ para los cuales $f(x,y)$ está definida.
        \[
          \mathbb{D}: \quad \text{En una dimensión: Todos los números x para los cuales f(x) está definida  }
        \]
        \begin{itemize}
            \item Evite la división por cero.
            \item Raíces pares de números negativos.
            \item Logaritmos de números negativos o cero.
        \end{itemize}
    
    \item El dominio de $f$ en una función de dos variables es una región:
        \begin{itemize}
            \item Las regiones que estén sombreadas son partes del dominio.
        \end{itemize}
        \begin{itemize}[label=\#]
            \item Para graficar funciones de dos variables son más fáciles de graficar que de una sola variable.
        \end{itemize}
\end{itemize}


%%%%%%%%%%%%%%%%%%%%%%%%%%%%%%%%%%%%%%%%%%%%%%%%%%%%%%%%%%%%%%%%%%%%%%%%%%%%%%%%%%%%%%%%%%%%%%%%%%%
\section{Ejercicios}
Encuentre y bosqueje el dominio de las sigs. funciones. \newline 
Sombree la región dque es parte del $\mathbb{D}$ y utilice líneas discontínuas para denotar a curvas que no son parte del $\mathbb{D}$
\begin{enumerate}
    \item $c(x,y)=10x+20y$ : 
        \begin{center}
           \begin{align*}
               \text{  Nunca se indefine.  } \\ 
               \mathbb{D}: \underbrace{(-\infty,\infty )}_{x} \underbrace{\times}_{\text{  Producto cartesiano  }} \underbrace{(-\infty ,\infty )}_{y} = \mathbb{R}^2 \\ 
           \end{align*}
           \begin{itemize}[label=\#]
               \item Producto cartesiano denota \textbf{todas las combinaciones posibles en un conjunto de $n$ elementos}.
               \item Explicaciones de productos cartesianos:
                \[
                  \mathbb{R} \cup  \mathbb{R} = \mathbb{R} \quad \mathbb{R} \times \mathbb{R} = \mathbb{R}^2     
                \]
                
                \item Definición de producto cartesiano:
                    \[
                      x \times y = \{(x,y) \; \text{  tal que  } \; x \in X, \, y \in Y \}
                    \]
                
                \item Producto cartesiano vs. unión:
                    \begin{center}
                       \begin{align*}
                           x \times y = \{(1,1),(1,2),(1,3),(2,1),(2,2),(2,3),(3,1),(3,2),(3,3)\} \\ 
                           x \cup y = \{(1),(2),(3)\}
                       \end{align*}
                    \end{center}
           \end{itemize}
        \end{center}
    
    \item $z = \frac{8}{x^2-y^2} $:
        \begin{center}
           \begin{align*}
               \text{  Definida si   } \; x^2 \neq y^2 \\ 
               \mathbb{R}^2 - \{x^2\neq y^2\} \\ 
               y \neq \sqrt{x^2} \\ 
               y \neq \pm x \\ 
           \end{align*}
        \end{center}
    
    \item $R(x,y)= \sqrt{9-x^2-y^2}$ : 
        \begin{center}
           \begin{align*}
               \text{  Definida  } \; \begin{matrix}
                   9-x^2-y^2 \geq 0 \\ 
                   9 \geq x^2 + y^2 \\ 
                   \mathbb{D}: x^2+y^2 \neq 9 \\ 
               \end{matrix} \\ 
               \text{  Círculo de radio 3 centrado en el orígen  } \\ 
               \mathbb{D} = \{(x,y) \; \text{  tal que  } \; x^2+y^2 \leq 9  \} \\ 
           \end{align*}
        \end{center}
    
    \item $Q(x,y)=\frac{1}{\sqrt{x^2+y^2-9}} $ : 
        \begin{center}
           \begin{align*}
               \mathbb{D}: \; \begin{matrix}
                   x^2+y^2 > 0 \\ 
                   x^2+y^2 > 9 \\ 
               \end{matrix} \\ 
               \therefore \text{  Afuera del círculo o disco de radio 3  } \\ 
           \end{align*}
        \end{center}
    
    \item $ z = \frac{(x+4)}{(y-2)(x-4)(y+2)} $ : 
        \begin{center}
           \begin{align*}
               \text{  Definida si  }: \quad y \neq \pm 2, \; x\neq 4 \\ 
               \mathbb{D}: \quad \mathbb{R}^2-  \{y\neq \pm 2, x\neq 4\} \\ 
           \end{align*}
        \end{center}
    
    \item $h(x,y)=\ln(2-yx)$ : 
        \begin{center}
           \begin{align*}
               \text{  Definida si  }: \quad \begin{matrix}
                   2-yx &> 0 \\ 
                   2 &> yx \\ 
                   y &< \frac{2}{x} \\ 
               \end{matrix} \\ 
               \therefore \mathbb{D}: \; y < \frac{2}{x} \\ 
           \end{align*}
        \end{center}
\end{enumerate}


%%%%%%%%%%%%%%%%%%%%%%%%%%%%%%%%%%%%%%%%%%%%%%%%%%%%%%%%%%%%%%%%%%%%%%%%%%%%%%%%%%%%%%%%%%%%%%%%%%%
\subsection{Gráfica de $z=f(x,y)$}
\begin{itemize}
    \item Gráfica de $z=f(x,y)$: Son superficies y consisten de todas las \emph{triplas} ordenadas $(x,y,z)$ donde $z$.
\end{itemize}



%%%%%%%%%%%%%%%%%%%%%%%%%%%%%%%%%%%%%%%%%%%%%%%%%%%%%%%%%%%%%%%%%%%%%%%%%%%%%%%%%%%%%%%%%%%%%%%%%%%
\section{Curva de nivel o traza horizontal}
\begin{itemize}
    \item En $f(x,y)=k$ k es una constante, rebane la superficie con los planos horizontales $z=k$ y grafique cada curva en el plano.
\end{itemize}



















    
% \end{document}


\chapter{Clase - 2020-02-20}
% \documentclass{article}
\title{Temporary}
\author{David Gabriel Corzo Mcmath}
\date{\today}
%%%%%%%%%%%%%%%%%%%%%%%%%%%%%%%%%%%%%%%%%%%%%%%%%%%%%%%%%%%%%%%%%%%%%%%%%%%%%%%%%%%%%%%%%%%%%%%%%%%%%%%%%%%%%%%%%%%%%%%%%%%%%%%%%%%%%%%%%%%%%%%
\usepackage[margin = 1in]{geometry}
\usepackage{graphicx}
\usepackage{fontenc}
\usepackage{pdfpages}
\usepackage[spanish]{babel}
\usepackage{amsmath}
\usepackage{amsthm}
\usepackage[utf8]{inputenc}
\usepackage{enumitem}
\usepackage{mathtools}
\usepackage{import}
\usepackage{xifthen}
\usepackage{pdfpages}
\usepackage{transparent}
\usepackage{color}
\usepackage{fancyhdr}
\usepackage{lipsum}
\usepackage{sectsty}
\usepackage{titlesec}
\usepackage{calc}
\usepackage{lmodern}
\usepackage{xpatch}
\usepackage{blindtext}
\usepackage{bookmark}
\usepackage{fancyhdr}
\usepackage{xcolor}
\usepackage{tikz}
\usepackage{blindtext}
\usepackage{hyperref}
\usepackage{listing}
\usepackage{spverbatim}
\usepackage{fancyvrb}
\usepackage{fvextra}
\usepackage{amssymb}
\usepackage{pifont}
\usepackage{longtable}
\usetikzlibrary{arrows,shapes}
%%%%%%%%%%%%%%%%%%%%%%%%%%%%%%%%%%%%%%%%%%%%%%%%%%%%%%%%%%%%%%%%%%%%%%%%%%%%%%%%%%%%%%%%%%%%%%%%%%%%%%%%%%%%%%%%%%%%%%%%%%%%%%%%%%%%%%%%%%%%%%%







% \begin{document}
    


\section{14.3 Derivadas parciales}
\begin{itemize}
    \item Derivada en una dimensión:
        \[
          \lim_{h \to 0} \frac{f(x+h)-f(x)}{h} = f'(x)  
        \]
    
    \item En una función con dos variables independientes:
        \begin{align*}
            f(x,y) = \begin{rcases}
                \begin{matrix}
                    f_x(x,y) \\ 
                    f_y(x,y) \\ 
                \end{matrix}  
            \end{rcases} \text{  Derivadas parciales  } \\
        \end{align*}
    
    \item Al derivarse parcialmente respecto a una variable, la otra se mantiene constante:
        \begin{align*}
            f_x(x,y) =& \lim_{h \to 0} \frac{f(x+h,y)-f(x,y)}{h} \quad \quad \text{  \# y se mantiene constante  }\\  
            f_y(x,y) =& \lim_{h \to 0} \frac{f(x,y+h)-f(x,y)}{h} \quad \quad \text{ \# x se mantiene constante  }\\  
        \end{align*}
    
    \item Se pueden utilizar todas las reglas de derivación para funciones de 1 variable:
        \begin{itemize}
            \item Suma 
            \item Producto 
            \item Cociente 
            \item Cadena 
        \end{itemize}
    
    \item $1^{\text{  eras  }}$ derivadas parciales de $f(x,y)$: encuentre todas las derivadas parciales posibles de $f_x$ \& $f_y$
        \begin{itemize}
            \item Notación:
                \[
                  f_x=\frac{\delta f}{\delta x} = \frac{\delta z}{\delta x}  
                \]
                \[
                    f_x=\frac{\delta f}{\delta y} = \frac{\delta z}{\delta y}  
                \]
            
            \item Evite $f'(x,y)$ para evitar ambigüedad.
        \end{itemize}
\end{itemize}



%%%%%%%%%%%%%%%%%%%%%%%%%%%%%%%%%%%%%%%%%%%%%%%%%%%%%%%%%%%%%%%%%%%%%%%%%%%%%%%%%%%%%%%%%%
\subsection{Ejercicios}
Encuentre las derivadas parciales de las siguientes funciones.
\begin{enumerate}
    \item $f(x,y)=2x^2+3xy\,$ : \emph{\textbf{Recordar lo siguiente: }$f_x(x,y)$ \& $f_y(x,y)$} 
        \begin{center}
           \begin{align*}               
                f_x = 4x+3y \quad \quad f_y=0+3x \\ 
           \end{align*}
        \end{center}
    
    \item $g(x,y)=y(x^2+1)^3+x^2(y^4-4)^4+5x^2y^3\,$ :
        \begin{center}
           \begin{align*}
                g_x&=3y(x^2+1)^22x+2x(y^4-4)^4+10xy^3 \\ 
                g_y&=1 \cdot (x^2+1)^3 + 16y^3x^2(y^4-4)^3+15x^2y^2 \\ 
           \end{align*}
        \end{center}        
    
    \item $h(s,t)=(s^2+10t)^2\cdot(t^4+s^3)^3\,$: \# Regla del producto y de la cadena.
        \begin{center}
           \begin{align*}
               h_s &= 4s(s^2+10t)^1\cdot(t^4+s^3)^3+3\cdot3s^2(s^2+10t)^2\cdot(t^4+s^3)^2 \\ 
               h_t&= 20(s^2+10t)^1\cdot(t^4+s^3)^3+12t^3(s^2+10t)^2,.
               (t^4+s^3)^2 \\  
           \end{align*}
        \end{center}
\end{enumerate}
%----------------------------------------------------------------------------------------
\begin{itemize}[label=\#]
    \item Evalúe la derivada en punto $(a,b)$:            
        \[
            f_x(a,b)=\frac{\delta f}{\delta x} \Big|_{(a,b)}^{} 
        \]
\end{itemize}

%----------------------------------------------------------------------------------------
\begin{enumerate}
    \item $w(r,\theta)=r^2\sin(2\theta)+e^{\pi r - \theta}\,$, encuentre $\frac{\delta w}{\delta \theta} \Big|_{(2,\pi)}^{}$
        \begin{center}
           \begin{align*}
               \frac{\delta w}{\delta \theta } &= 2r^2\cos(2\theta)-e^{\pi r -\theta} \\ 
               \frac{\delta w}{\delta \theta } \Big|_{(2,\pi)}^{} &= w_\theta(2,\pi) = 2\cdot 4\cos(2\pi)-e^{2\pi-\pi} \\ 
               &= 8-e^\pi
           \end{align*}
        \end{center}
\end{enumerate}


%%%%%%%%%%%%%%%%%%%%%%%%%%%%%%%%%%%%%%%%%%%%%%%%%%%%%%%%%%%%%%%%%%%%%%%%%%%%%%%%%%%%%%%%%%
\section{Derivadas parciales par funciones de 2 o más variables}
\begin{itemize}
    \item Se deriva respecto a una variable y el resto se mantienen constantes.
        \[
            w = f(x,y,z) 
        \]
        3 $1^{\text{  eras  }}$ derivadas parciales: $f_x,f_y,f_z$. 
        \[
          u = f(x_1,x_2,\dots,x_n)
        \]
        n derivadas parciales: 
        \[
          \frac{\delta u }{\delta x} , \dots \frac{\delta u}{\delta x_n} 
        \]
\end{itemize}


%----------------------------------------------------------------------------------------
\subsection{Ejercicio}
Encuentre todas las primeras derivadas pariales de las sigentes funciones:
\begin{itemize}
    \item $f(x,y,z)=\sqrt[4]{x^4+8xz+2y^2}$
        \begin{center}
           \begin{align*}
               f_x &= \frac{1}{4} (x^4+8xz+2y^2)^{-\frac{3}{4} }\cdot(4x^3+8z+0) \\ 
               f_y &= \frac{1}{4} (x^4+8xz+2y^2)^{-\frac{3}{4} }\cdot(4y) \\ 
               f_z &= \frac{1}{4} (x^4+8xz+2y^2)^{-\frac{3}{4} }\cdot(8x) \\ 
           \end{align*}
        \end{center}
    
    \item $p(r,\theta,\phi)=r\cdot \tan (\phi^2-4\theta)\,$: 
        \begin{center}
           \begin{align*}
               p_r &= \tan (\phi^2-4^\theta) \\ 
               p_{\theta} &= -4r \sec ^2 (\phi^2-4\theta) \\ 
               p_{\phi} &= 2\phi r \sec ^2(\phi^2-4\theta) \\ 
           \end{align*}
           \begin{itemize}[label=\#]
               \item Funciones vectoriales 1 variable: $\vec{r}\,' (t), \dots $
           \end{itemize}
        \end{center}
\end{itemize}


%%%%%%%%%%%%%%%%%%%%%%%%%%%%%%%%%%%%%%%%%%%%%%%%%%%%%%%%%%%%%%%%%%%%%%%%%%%%%%%%%%%%%%%%%%
\section{Derivadas parciales de orden superior (pág. 100)}
\begin{itemize}
    \item Orden superior: Segundas, terceras, cuartas, etc. derivadas.
    \item Como $f_x(x,y)$ \& $f_y(x,y)$ son también funciones en dos variables, pueden tener derivadas parciales.
        \begin{tikzpicture}[node distance = 2.5cm, auto]
            \node [block] (1) {$f_x$};
            \node [block,right of=1] (2) {$f_{xx}$}; 
            \node [block,right of=1,below of=1] (3) {$f_{xy}$};
            \path [line] (1) -- (2);
            \path [line] (1) -- (3); 
        \end{tikzpicture}
        \begin{tikzpicture}[node distance = 2.5cm, auto]
            \node [block] (1) {$f_y$};
            \node [block,right of=1] (2) {$f_{yy}$}; 
            \node [block,right of=1,below of=1] (3) {$f_{yx}$};
            \path [line] (1) -- (2);
            \path [line] (1) -- (3); 
        \end{tikzpicture}
    
    \item Las segundas derivadas parciales, éstas también tienen sus derivadas parciales, terceras derivadas parciales.
        \begin{center}
           \begin{tabular}{  p{1cm}  p{1cm}  p{1cm}  p{1cm}  }
                    $f_{xxx}$ & $f_{xxy}$ & $f_{yyy}$ & $f_{yxy}$  \\
                    $f_{xxy}$ & $f_{xyx}$ & $f_{yyx}$ & $f_{yxx}$ \\ 
           \end{tabular}
        \end{center}
    
    \item Las derivadas parciales cruzadas $f_{xy}$ \& $f_{yx}$ son iguales si la función es diferenciable.
        \[
          f_{xy} = f_{yx} \quad \quad f_{xyy} = f_{yyx}= f_{yxy}
        \]
    
    \item Notación delta:
        \begin{center}
           \begin{align*}
               f_{xx} &= \frac{\delta }{\delta x} \left(\frac{\delta f}{\delta x}\right) = \frac{\delta^2f}{\delta x^2}  \quad  \quad  f_{yy} = \frac{\delta^2f}{\delta y^2} \\ 
               f_{xy} &= \frac{\delta}{\delta y} \left(\frac{\delta f}{\delta x}\right) = \frac{\delta ^2 f}{\delta y \delta x}  \quad \quad f_{yx} = \frac{\delta^2f}{\delta y \delta y} \\ 
           \end{align*}
        \end{center}
\end{itemize}


%----------------------------------------------------------------------------------------
\subsection{Ejercicios}
Encuentre todas las 2das derivadas parciales:
\begin{enumerate}
    \item $f(x,y)=\sin (mx+ny)\quad m,n \in \mathbb{R}\,$:  
        \begin{align*}
                \text{  Primeras derivadas parciales  :} \\ 
                f_x =& m \cos (mx+ny) \\ 
                f_y &= n \cos (mx+ny ) \\ 
                \text{  Segundas derivadas parciales:   } \\ 
                f_{xx} &= -m^2 \sin (mx+ny ) \\ 
                f_{yy} &= -n ^2 \sin (mx+ny) \\ 
                \begin{rcases}
                    f_{xy} &= -mn \sin (mx+ny) \\ 
                    f_{yx} &= -mn \sin (mx+ny) \\ 
                \end{rcases} \text{  Iguales  } \\ 
        \end{align*}
    
    \item $z = \cos (2xy)\,$ : 
        \begin{center}
           \begin{align*}
               1^{\text{  eras  }}: \quad \frac{\delta z }{\delta x} &= -2 \sin (2xy) 
               ,\quad \frac{\delta z}{\delta y } = -2x \sin (2xy) 
               \\  
               2^{\text{  das  }}: \quad \frac{\delta ^2 z}{\delta x^2} &= -4 y^2 \cos (2xy) ,\quad \frac{\delta^2 z }{\delta y^2} = -4x^2 \cos (2xy) \\  
           \end{align*}
        \end{center}
\end{enumerate}



























% \end{document}


\chapter{Clase - 2020-02-27}


\end{document}






