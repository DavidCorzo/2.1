\documentclass[openany]{book}

\title{Cálculo Multivariable - Clases y apuntes de clase}
\date{2020-01-06}
\author{David Gabriel Corzo Mcmath}

\usepackage[margin = 1in]{geometry}
\usepackage{pdfpages}
\usepackage{pgfplots}
\usepackage[spanish]{babel}
\usepackage[intlimits]{amsmath}
\usepackage{amssymb}
\usepackage{amsthm} 
\usepackage{esvect}
\usepackage{mathtools}
\usepackage[utf8]{inputenc}
\usepackage{titlesec}
\usepackage{xpatch}
\usepackage{fancyhdr} 
\usepackage{tikz}
\usepackage{enumitem}
\usepackage{generalsnips}
\usepackage{calculussnips}
\usepackage{wrapfig}
\usepackage{cancel}
\usepackage{tikzit}
\usepackage{float}
\input{sample.tikzstyles}
\usetikzlibrary{arrows.meta, calc, positioning}
\usetikzlibrary{decorations.markings}
\usetikzlibrary{decorations.pathreplacing}

\tikzset{
    dot/.style = {circle, fill, minimum size=#1,
                inner sep=0pt, outer sep=0pt},
    dot/.default = 4pt % size of the circle diameter 
}

\tikzset{dasharr/.style={
	dash pattern=on 3pt off 2pt,postaction={decorate,
    decoration={markings,
    mark=between positions 3pt and 1 step 10pt with {\arrow{>};}}}
}}


\newcommand{\lbraceontikz}[4]{
    \draw [decorate,decoration={brace,amplitude=10pt,mirror,raise=4pt},yshift=0pt]
    #1 -- #2 node [black,midway,#4] % xshift=-1cm
    {\footnotesize #3};
}

\newcommand{\rbraceontikz}[3]{
    \draw [decorate,decoration={brace,amplitude=10pt},xshift=-0.5cm,yshift=0pt]
    #1 -- #2 node [black,midway, yshift=0.6cm] %xshift=-0.6cm
    {\footnotesize #3};
}


\removeheaderspace

\pagestyle{plain}
\packagesneeded

%%%%%%%%%%%%%%%%%%%%%%%%%%%%%%%%%%%%%%%%%%%%%%%%%%%%%%%%%%%%%%%%%%%%%%%%%%%%%%%%%%%%%%%%%%
\begin{document}

\maketitle
\tableofcontents
\tikzblockdefinitions
%%%%%%%%%%%%%%%%%%%%%%%%%%%%%%%%%%%%%%%%%%%%%%%%%%%%%%%%%%%%%%%%%%%%%%%%%%%%%%%%%%%%%%%%%%

\chapter{Sistema tridimensional de coordenadas}
\section{Clase introductoria}
\begin{itemize}
    \item Hay dos tipos de datos en estadística; 
        \begin{enumerate}
            \item Cualitativo: el cualitativo es por 
            \item Cuantitativo: 
        \end{enumerate}
    
    \item Distribución de frecuencias: nos dice qué tan frecuente es la distribución de los datos en un set.
\end{itemize}


\chapter{Distancias y superficies básicas} % Clase - 2020-01-23
\section{¿Qué es el marketing?}
\begin{itemize}
    \item Todo aquello que una empresa u organización realiza con el objeto de identificar, conocer, cultivar y satisfacer el mercado que sirve y ser retribuido por él de manera sostenida.
    \item \emph{\textbf{Ejemplo: }Problema acerca de tarjeta de crédito. Las marcas problemáticas afectan su posicionamiento.}
    \item Se necesita hacer énfasis que el target tiene que ser investigado, conocido, entendido a profundidad.
    \item \emph{\textbf{Ejemplo: }Krispy Cream: la noción de ``tal empresa va a hacer que otra quibre''}.
    \item Analizar: \textbf{Nos preguntamos:} ¿por qué compramos donas? Es fácil saber por qué compran, \textbf{lo difícil es saber por NO compran.}
    \item Historia de McDonald's: 
        \begin{itemize}
            \item Cuando se fundó la franquicia, se inició la idea de la cajita feliz con juguete; se pudieron deducir estas conclusiones por que la Sr. Cofiño estaba en el restaurante dando la comida; ``el peor lugar para tomar decisiones es el escritorio, uno tiene que estar en la jugada''.
            \item Ahora Mc está teniendo muchos detractores ya que la gente valora más los saludable por ejemplo. 
        \end{itemize}    
    \item Ejemplo de madre e hija:
        \begin{itemize}
            \item Aprender a ver la escena.
            \item Aprender a absorber todo el contexto.
        \end{itemize}
\end{itemize}

\section{Ejemplo de señor leyendo:}
\begin{itemize}
    \item El ambiente se adecua al target, sería mal si se propone poner música a alto volúmen.
    \item Todos nosotros vamos a necesitar alguna cosa, estas son oportunidades de negocios.
    \item Se analiza que el target necesita un ambiente para poder juntarse a leer, \textbf{identificar las necesidades del target}; resumir el target a una necesidad, conocer los intereses del target.  
\end{itemize}

\section{Ejemplo de la moto:}
\begin{itemize}
    \item \textbf{Nos preguntamos:} ¿Qué pensamos?
    \item \textbf{Nos preguntamos:} ¿Qué necesidades tendrá esta familia? 
    \item \textbf{Nos preguntamos:} ¿Será la cultura?
    \item Hay que entender los targets, por qué quieren usar una moto con 3 pasajeros y el piloto.
    \item Aspecto relevante: tener en cuenta que una de las cosas más importantes es investigar el target.
    \item \textbf{Nos preguntamos:} ¿Será porque se vuelve económicamente imposible tener carro por la distancia entre la casa y el trabajo? \emph{\textbf{La respuesta a esta problemática es: }tenemos que entender cuál es el problema.}
\end{itemize}

%%%%%%%%%%%%%%%%%%%%%%%%%%%%%%%%%%%%%%%%%%%%%%%%%%%%%%%%%%%%%%%%%%%%%%%%%%%%%%%%%%%%%%%%%%%%%%%%
\section{Ejemplos del marketing (buenos / malos)}
\subsection{Anécdota colgate:}
\begin{itemize}
    \item Insight: Darnos cuenta que había un niño que llevaba cuatro veces al día por cositas, esto no se hubiera podido analizar en el escritorio.
    \item Marketing es un poco como trabajo de detective, como trabajo de investigador.
    \item Hay que conocer el mercado; a partir del ejemplo de la moto se puede intentar coordinar el mercado para ejercer funciones empresariales para resolver esta problemática.
\end{itemize}

\subsection{Ejemplo de maratón:}
\begin{itemize}
    \item Especulamos que es por una buena causa, les interesa su salud, les interesa el reto, les interesa para ver que podemos hacer.
    \item Ejemplo en Chile: el bicicross;  así van y vienen las tendencias.
\end{itemize}

\subsection{Ejemplo de Donald Trump:}
\begin{itemize}
    \item Él ganó su candidatura en EEUU por que supo identificar su target.
    \item En política se utiliza mucho esto de identificar su target.
\end{itemize}

\subsection{Ejemplo de Madonna:}
\begin{itemize}
    \item Madonna se ha mantenido por décadas, ella como artista genera más dinero que las grandes marcas en GT.
    \item Otros ejemplos como Shakira, generan mucho dinero.
    \item Walkman: es el abuelito del iPod, identificaron que los adolecentes llevaban su boom box e innovó a crear el walkman, donde se guardaba música, ahora no tenían que cargar la grabadora con sus baterías, el walkman ue por mucho tiempo el aparato para escuchar música. Esto un ejemplo clásico de identificar las necesidades del target.
    \item Los targets van cambiando sus intereses a través del tiempo.
\end{itemize}

\subsection{Ejemplo DudeWipes:}
\begin{itemize}
    \item Muchas veces con que se identifique la necesidad del target si no que diseñar un producto con el target en mente. Agarraron el mismo producto que era para bebés solo que lo orientaron para hombres.
    \item Las toallas humedas las modificaron un poco, como no tienen olor, son más gruesas, son más grandes, etcétera; este es un producto \textbf{muy bien enfocado} desde el color, hasta el logos, hasta el nombre de la marca.
    \item Todo el modelo de negocios está orientado al target.
    \item Se pueden introducir a nuevos sectores del mercado; posiblemente no podía usar toallitas de bebé se introduce estas nuevas toallitas, lugares como barbería.
    \item Otro ejemplo es el shampoo Ego; Se enfocó en el target de los hombres; se posiciona de tal manera que resuenan mucho mejor con el target.
    \item Video promoción de DudeWipes - \url{https://www.youtube.com/watch?v=4jMgM0pKEUw}
    \item Dude wipes se separó de los otros productos casi identicos.
    \item Otro ejemplo puntual es - \url{https://www.youtube.com/watch?v=vdWHucg900U}
    \item Carreer builder es un sitio para conseguir trabajo, en el comercial sale el mismo actor que en marketing es el target que trata de conseguir que los targets de la empresa visiten el sitio; Este video tiene ese problema, implica que todos los demas ajenos al target es un mono.
    \item Regla básica en marketing es \textbf{No hablar mal de nadie}.
    \item Ejemplo de Coca~Cola y pepsi; su estrategia de marketing es riesgosa por sacar anuncios en contra de pepsi.
    \item Ejemplo de Beneton: El ejemplo de un Sueco con un negro de África.
\end{itemize}

\subsection{\textbf{Nos preguntamos:} ¿Cómo evadir que las personas no encuentren cosas ajenas al marketing a la intención?}
\begin{itemize}
    \item Una solución es leer de todo. 
    \item Entender si ya se había intentado y qué tal funcionó.
    \item Que en tu mesa de trabajo hayan expertos; experimentar.
\end{itemize}

\subsection{Analizar: El debate presidencial de Donald Trump}
\begin{itemize}
    \item El audio que salió acerca de la charla en el bus.
\end{itemize}


\chapter{Rectas y planos} % Clase - 2020-01-28
\documentclass{article}
\title{Clase -2020-01-28}
\author{David Gabriel Corzo Mcmath}
\date{2020-Jan-28 10:08:14}
%%%%%%%%%%%%%%%%%%%%%%%%%%%%%%%%%%%%%%%%%%%%%%%%%%%%%%%%%%%%%%%%%%%%%%%%%%%%%%%%%%%%%%%%%%%%%%%%%%%%%%%%%%%%%%%%%%%%%%%%%%%%%%%%%%%%%%%%%%%%%%%
\usepackage[margin = 1in]{geometry}
\usepackage{graphicx}
\usepackage{fontenc}
\usepackage{pdfpages}
\usepackage[spanish]{babel}
\usepackage{amsmath}
\usepackage{amsthm}
\usepackage[utf8]{inputenc}
\usepackage{enumitem}
\usepackage{mathtools}
\usepackage{import}
\usepackage{xifthen}
\usepackage{pdfpages}
\usepackage{transparent}
\usepackage{color}
\usepackage{fancyhdr}
\usepackage{lipsum}
\usepackage{sectsty}
\usepackage{titlesec}
\usepackage{calc}
\usepackage{lmodern}
\usepackage{xpatch}
\usepackage{blindtext}
\usepackage{bookmark}
\usepackage{fancyhdr}
\usepackage{xcolor}
\usepackage{tikz}
\usepackage{blindtext}
\usepackage{hyperref}
\usepackage{listing}
\usepackage{spverbatim}
\usepackage{fancyvrb}
\usepackage{fvextra}
\usepackage{amssymb}
\usepackage{pifont}
\usepackage{longtable}
\usepackage{tikz-3dplot}
\usepackage{esvect}
%%%%%%%%%%%%%%%%%%%%%%%%%%%%%%%%%%%%%%%%%%%%%%%%%%%%%%%%%%%%%%%%%%%%%%%%%%%%%%%%%%%%%%%%%%%%%%%%%%%%%%%%%%%%%%%%%%%%%%%%%%%%%%%%%%%%%%%%%%%%%%%
\begin{document}
\maketitle

\section{12.5 Rectas y planos}
\begin{itemize}
    \item Ecuación de una recta 
    \item Vector posición $\vec{r}_0  = \langle x_0,y_,z_0 \rangle $
    \item Vector dirección $\vec{v}_0 = \langle a,b,c \rangle$
    \item Ecuación vectorial: $\vec{r} = \vec{r}_0 + t\vec{v}$ donde t es el parámetro.
    \item Ecuaciónes paramétricas: \begin{align*}
        x = x_0 +at \\ 
        y = y_0+ at \\ 
        z = z_0+at \\ 
    \end{align*}
    
    \item Resuelva para $t$ en las tres ecuaciones:  
        \begin{align*}
            t = \frac{x-x_0}{a} 
            t = \frac{y-y_0}{b} 
            t = \frac{z-z_0}{c} 
        \end{align*}
        Estas son las ecuaciónes simétricas de la recta donde $a,b,c \neq 0$.
        
        \item Vector dirección $\vec{v}= \langle a,0,c \rangle $ las ecuaciones en la recta cambian:
            \begin{align*}
                \underbrace{\vec{r} = \vec{r}_0 + t \vec{v}}_{\text{  Vectorial  }} \\ 
                x = x_0+at \\ 
                y = y_0 \\ 
                z = z_0 +ct \\ 
                \text{  Entonces queda así:  } \\ 
                \frac{x-x_0}{a} = \frac{z-z_0}{c} \\ 
                \underbrace{y = y_0}_{\text{  Simétrica  }} \\   
            \end{align*}
\end{itemize}

\subsection{Ejercicio 3: Encuentre las ecs. simétricas de la recta que pasa por los puntos dados. Encuentre en qué punto la recta interseca al plano xz. pg.41}
\begin{itemize}
    \item P(2,8,-2) \& Q(2,6,4) 
    \begin{align*}
        \text{  Vector posición  } = \overrightarrow{OP} = R_0 = \langle 2,6,7 \rangle \\ 
        \text{  Vector dirección  } \overrightarrow{PQ} = \vec{V} = \langle 0,-2,6 \rangle \\ 
        \text{  Ec. vectorial  } = \vec{r} = \langle 2,8,-2 \rangle + t \langle 0,-2,6 \rangle \\
        \text{  Ecs. simétricas  } = x = a, \frac{y-8}{-2} = \frac{z+2}{6} \\   
    \end{align*}
    
    \item \textbf{Nos preguntamos:} ¿Cual es la intersección con el plano xz?
    \begin{align*}
        \text{  Use,  y=0} x = 2, \frac{-8}{-2} =&  \frac{z+2}{6} \\ 
        & = 6 \cdot 4 = z+ 2 \implies z= 22 \\ 
    \end{align*}

    
    \item La intersección con el plano xz es el punto $(1,0,22)$: 
    \begin{align*}
        \vec{r}_0 = \langle 4,6,10 \rangle \\ 
        \vec{v} = \vv{PQ} = \langle 2,0,0 \rangle \\ 
        \text{  Vectorial:    } \vec{r} = \langle 4,6,10 \rangle + t \langle 2,0,0 \rangle \\ 
        \text{  Paramétricas: } x = 4 + 2t, y = 6, z = 10 \\ 
        \text{  Simétricas:  }t = \frac{x-4}{2} , y = 6, z=10 \\  
    \end{align*}
    
    \item \textbf{Nos preguntamos:} ¿Cual es el punto de instersección con el plano xz?
    \begin{align*}
        \text{  Use: y=0  }
    \end{align*}
    Explicación: por la recta $y=6$ siempre será 6, nunca podrá ser $0$, no puede intersecar con el plano xz, \textbf{No hay}.
\end{itemize}


%%%%%%%%%%%%%%%%%%%%%%%%%%%%%%%%%%%%%%%%%%%%%%%%%%%%%%%%%%%%%%%%%%%%%%%%%%%%%%%%%%%%%%%%%%%%%%%%

\section{Rectas paralelas}
Dos rectas $\vec{r}_1 = \vec{r}_{01} + t \vec{v}$ \& $\vec{r}_{2} = \vec{r}_{02} + t \vec{v}_2 $ son paralelas si y solo si sus vectores de dirección $\vec{v}_1$ y $\vec{v}_2 $ son paralelas.
\begin{figure}[htbp]
    \centering
    % \includegraphics[width=cm]{}
    \caption{}
    \label{}
\end{figure} 

Entones en el espacio tenemos 3 tipos de rectas:
\begin{enumerate}
    \item Rectas paralelas  
    \item Rectas intersecan en un punto
    \item Rectas Ublicuas (no paralelas \& no intersecan)
\end{enumerate}

\subsection{Ejercicio 4: Determine si los siguientes pares de rectas son paralelas, oblicuas o se intersecan.}
\begin{itemize}
    \item \begin{align*}
        \frac{x-2}{8} = \frac{y-3}{24} = \frac{z-2}{16} , \frac{x-10}{-2} = \frac{y+15}{-6} = \frac{z+24}{-4} \\ 
        \vec{v}_1 = \langle 8,24,16 \rangle, \vec{v}_2  = \left\langle -2,-6,-4 \right\rangle \\ 
        \text{  Entoces...  }, \left\langle \frac{8}{-2}, \frac{24}{-6} , \frac{16}{-4}   \right\rangle  \\ 
        \left\langle -4,-4,-4   \right\rangle, \therefore \text{  Son paralelas  } \\   
    \end{align*}
    El vector dirección está en el denominador.
    
     \item \begin{align*}
        L_1:  x = 3-4t, y = 6-2t, z= 2+ 0t, t \in IR\\  
         L_2:  x = 3+ 8s, y = -2s, z = 8+2s, s \in IR \\ 
         \text{  Utilize una variable parámetro para cada recta  } \\ 
         v_1 = \left\langle -4,-2,0 \right\rangle , v_2 = \left\langle 8,-2,2 \right\rangle \text{  No son paralelas  } \\ 
         \text{  Analice si las rectas se intersecan  } \\ 
         x=x \rightarrow 5-4t=3+8s \\ 
         y=y \rightarrow 6-2t=-2s \\ 
         z=z \rightarrow 2 = 8 + 2s \rightarrow s = -3 \\ 
         5-4=-22 \rightarrow 4t=-27 \rightarrow -4t=-27 \rightarrow = \frac{27}{4} \\ 
         6-2t = 6 \rightarrow 2t = 0 \rightarrow t=0 \\ 
         \therefore \text{  Como no hay una $t$ única (no es posible $0 \neq \frac{27}{4} $), las dos rctas no se intersecan.  } \\ 
         L_1 \text{  \&  }L_2 \text{  Son oblicuas  } \\ 
         \text{  Eliminación Gausiana  } \\ 
         \begin{matrix}
             4t+8s = 2 \\ 
             2t+25 = 6 \\ 
             0t + 2s = -6 \\ 
         \end{matrix} = 
         \begin{vmatrix}
             4 & 8 & 2 \\ 
             2 & 2 & 6 \\ 
             0 & 2 & -6 \\ 
         \end{vmatrix} 
         0,0,\text{  número  } \implies \text{  No hay solución  } \\ 
     \end{align*}
\end{itemize}


\section{La ecuación de un plano}
Previamente en 12.1 $ax+by+cz = 0$.
\begin{figure}[htbp]
    \centering
    % \includegraphics[width=cm]{}
    \caption{}
    \label{}
\end{figure}
Para encontrar la ec. de un plano se necesita:
    \begin{enumerate}
        \item Un punuto sobre el plano $P$: $\vec{r_0}=\overrightarrow{OP}$
        \item Un vector normal u ortognoal al plano: $\hat{n}_0 \left\langle a,b,c \right\rangle $
    \end{enumerate}
\subsection{Derivación de la e. plano}
\begin{align*}
    P(x_0,y_0,z_0), Q(x_1,y_1,z_1) \text{  Son dos puntos sobre el plano  } \\ 
    \vec{r_0} = = \overrightarrow{0P} = \left\langle x_0,y_0,z_0 \right\rangle \\ 
    \vec{r} = \overrightarrow{0Q} = \left\langle x,y,z \right\rangle \\ 
\end{align*}
El vector $\vec{RP} = \vec{r} + \vec{r-0}$ está sobre el plano, por lo que tiene que ser ortogonal a $\hat{n}$.
 \begin{align*}
     \hat{n} \perp \vec{r}- \vec{r_0} \rightarrow \underbrace{\hat{n}\cdot (\vec{r}-\vec{r_0})}_{\text{  Ec. vectorial de un plano  }} \\ 
    \text{  Se puede reescribir como:  } \\ 
     \underbrace{\left\langle a,b,c \right\rangle \cdot \left\langle x+x_0,y-y_0,z-z_0 \right\rangle + c(z-z_0) = 0 }_{\text{  Ecuación escalar de un plano  }} \\ 
     ax+by+cz = \underbrace{ax_0+by_0+cz_0}_{0} \\ 
 \end{align*}

Para encontrar la ec. de un plano se necesita 3 puntos P,Q,R: \textbf{hay infinitas respuestas equivalentes } $\hat{n}=\vec{}\times \vec{}$.
\begin{align*}
    \vec{r_0} = \overrightarrow{OP}, \overrightarrow{0Q}, \overrightarrow{0R} \\ 
    \underbrace{\hat{n} = \overrightarrow{PQ} \times \overrightarrow{PR}}_{\text{  Tienen que empezar en el mismo punto  }} \\ 
    \text{  Hat infinitas respuestas:  } \\ 
    \hat{n} = \vv{PR} \times \vv{PQ} \\ 
\end{align*}


\subsection{Ejercicio 1: pg45 Encuentre la ec. del plano que pasa por los 3 puntos dados.}
\begin{enumerate}
    \item $P(3,-1,3), Q(8,2,4), R(1,2,5)$
    \begin{align*}
        \text{  Ecuación del plano :   }, \hat{n} \cdot (\vec{r}- \vec{r_0}) = 0\\
        \text{  Ecuaciónn de la recta :   }, \vec{r} = \vec{r_0}+t \vec{v} \\ 
        \vec{r_0} = \left\langle 8,2,4 \right\rangle  \\ 
    \end{align*}
    Encuentre dos vectores que están sobre el plano y que comiencen en el mismo punto.
    \begin{align*}
        \vec{u}= \overrightarrow{PQ} = \left\langle 5,3,1  \right\rangle, \vec{v}= \overrightarrow{PR}= \left\langle -2,3,2 \right\rangle \\ 
        \text{  ¡¡ $\hat{n}$ es ortogonal a ambos vectores !! } \\ 
        \hat{n} = \overrightarrow{PQ} \times \overrightarrow{PR} = \begin{vmatrix}
            \hat{i}& \hat{j}& \hat{k} \\
            5 & 3 & 1 \\ 
            -2 & 3 & 2 \\   
        \end{vmatrix} = 3 \hat{i} - 12 \hat{j} -+ 21 \hat{k} \\ 
        \text{  Ec. Plano  }, \hat{n} \cdot ( \vec{r} - \vec{r_0}) = 0 \\ 
        \text{  Ec. Vectorial  }, \left\langle 3,-12,21 \right\rangle \cdot \left\langle x-8,y-2,z-4 \right\rangle = 0 \\ 
        \text{  Escalar  }, 3(x-8)-  
    \end{align*}

    
    \item P(0,0,0), Q(1,0,2), y R(0,2,3)
    \begin{align*}
        \text{  Vector posición: } \vec{r_0} & = \left\langle 0,0,0 \right\rangle \\ 
          \text{ dos  vectoes sobre el plano:  } \begin{matrix*}
             \vec{PQ} = \left\langle 1,0,2 \right\rangle \\ 
             \vec{PR} = \left\langle 0,2,3 \right\rangle  \\ 
         \end{matrix*} \\ 
        \text{  Vector normal:  } \hat{n} & = \overrightarrow{PQ} \times  \overrightarrow{PR} \\ 
          = \begin{vmatrix}
             \hat{i} & \hat{j} & \hat{k} \\
              \text{  terminar  } \\
         \end{vmatrix} \\
    \end{align*}
    
    \item Ecuación del plano:
    \[
      -4x-3y+2z=0
    \]
\end{enumerate}

 \section{Rectas paralelas $v_1$ y $v_2$ son paralelos}
Dos planos $\hat{n_1} \cdot (\vec{r}- \vec{r_1})= 0$ y $\hat{n_2} \cdot ( \vec{r}- \vec{r_2}) = 0$ son paralelas sí y sólo si $\hat{n_1}$ y $\hat{n_2}$ son paralelas.

 En caso que no sean paralelas, se puede encontrar el ángulos de intersección entre dos planos.

























\end{document}
 % 2020-01-30
% \date{2020-Jan-30 10:22:58}
%%%%%%%%%%%%%%%%%%%%%%%%%%%%%%%%%%%%%%%%%%%%%%%%%%%%%%%%%%%%%%%%%%%%%%%%%%%%%%%%%%%%%%%%%%%%%%%%
\section{Resolución de corto}
\begin{itemize}
    \item Determine el área del triángulo entre los puntos P(), Q(), R():
        \begin{align*}
            \vec{a} = \vv{PQ} = \left\langle 4,3,-2 \right\rangle \\ 
            \vec{b} = \vv{PR} = \left\langle 5,5,1 \right\rangle \\ 
            \text{  Área  } = \frac{1}{2} \left| \vec{a} \times  \vec{b} \right| \\ 
            \begin{vmatrix}
                \hat{i} & \hat{j} & \hat{k} \\ 
                4 & 3 & -2 \\ 
                3 & 5 & 1 \\ 
            \end{vmatrix} = 13 \hat{i} - 14 \hat{j} + 5 \hat{k} \\  
            \text{  Área  } = \frac{1}{2} \sqrt[]{} \\ 
        \end{align*}

\end{itemize}
    



%%%%%%%%%%%%%%%%%%%%%%%%%%%%%%%%%%%%%%%%%%%%%%%%%%%%%%%%%%%%%%%%%%%%%%%%%%%%%%%%%%%%%%%%%%%%%%%%
%%%%%%%%%%%%%%%%%%%%%%%%%%%%%%%%%%%%%%%%%%%%%%%%%%%%%%%%%%%%%%%%%%%%%%%%%%%%%%%%%%%%%%%%%%%%%%%%
\section{Rectas y planos}
\begin{itemize}
    \item Ecs. Rectas: $\vec{r}= \vec{r_0} + t \vec{v}$ 
        \begin{align*}
            \text{  si  } a \neq b \neq c \neq 0    \quad \frac{x-x_0}{a} = \frac{y-y_0}{b} = \frac{z-z_0}{c}  \\ 
        \end{align*}
    
    \item Paramétricas:
        \begin{align*}
            x = x_0 +at \\ 
            y = y_0 +bt \\ 
            z = z_0 +ct \\ 
        \end{align*}
    
    \item Ecuación de plano:
        \begin{align*}
            \hat{n} = \cdot \vec{r}-\vec{r_0} \\ 
            a(x-x_0)+b(y-y_0) + c(z-z_0) = 0 \\ 
            \hat{n}= \vec{a} \times  \vec{b} \\ 
        \end{align*}
\end{itemize}

\subsection{Ejercicios}
\begin{enumerate}
    \item Considere los planos $x+y=0$ \& $x+2y+z=1$.
        \begin{enumerate}
            \item Determine si los planos son paralelos so no lo son encuentre el ángulo entr ellos:
                \begin{align*}
                    \hat{n_1} = \left\langle 1,1,0 \right\rangle \\ 
                    \hat{n_2} = \left\langle 1,2,1 \right\rangle \\ 
                    \therefore \text{  Los dos planos no son paralelos  } \\ 
                \end{align*}
                \begin{itemize}
                    \item El $\hat{n_1}$ \& $\hat{n_2}$ no son necesariamente ortogonales.
                \end{itemize}
                \begin{align*}
                    \cos \theta = \frac{\hat{n_1}\cdot\hat{n_2}}{\left| \hat{n_1} \right|\left| \hat{n_2} \right| } = \frac{3}{\sqrt[]{2}} \\ 
                    \cos \theta = \frac{3}{2\sqrt[]{3}}=\frac{\sqrt[]{3}}{2} \qquad \theta = \frac{\pi}{2} \\   
                \end{align*}
        \end{enumerate}
    
    \item Encuentre la ec. de la recta que interseca a ambos planos $x+y=0$ \& $x+2y+z=1$: 
        \begin{align*}
            r = \vec{r_0} + t \vec{v} \\ 
            \text{  Dos puntos sobre la recta  } \\ 
            \text{  Como la recta esta en ambos planos, se debe resolver el sig. sistema de ecuaciones  } \\ 
            x + y = 0 \implies x=-y \\ 
            x+2y+z=1 \implies y = z-1 \\ 
            \text{  z tiene cualquier valor, ahora encontrar escogiendo cualquier punto sobre la recta, en este caso 0   } \\ 
            \text{  Primer punto   } \quad z & = 0 \\
            y &= 1\\ 
            x & = -1 \\ 
            \therefore \left\langle -1,1,0 \right\rangle \\ 
            \text{  Segundo punto  } \quad z &= 1 \\ 
            y & = 0 \\ 
            x & = 0 \\ 
            \therefore \left\langle 0,0,1 \right\rangle \\ 
        \end{align*}
    
    \item Encuentre la ecuación de la recta que pasa por P(-1,1,0) y Q$\underbrace{(0,0,1)}_{r_0}$:
        \begin{align*}
            \vec{r_0} = \left\langle 0,0,1 \right\rangle \! \left\langle -1,1,0 \right\rangle \\ 
            \vec{v} = \vv{QP} 0 \left\langle -1,1,-1 \right\rangle \\ 
            \text{  Ecuaciones paramétricas de la recta:  } \\ 
            x = 0-t \quad y = 0 + t \quad z = 1-t \\ 
        \end{align*}
    
    \item Solución alterna:
        \begin{align*}
            x = -y \quad y = 1-z \quad \text{  Más incognitas que ecuaciones.  }\\
            x,y \; \text{  ó   }\; z \text{  \quad pueden tener cualquier valor  } \quad z=t \\ 
            \begin{matrix}
                x = -1 + t \\ 
                y = 1 - t \\ 
                t = t \\ 
            \end{matrix} \therefore v_2 = \left\langle 1,-1,1 \right\rangle  \quad \vec{r_0} = \left\langle -1,1-0 \right\rangle \\ 
        \end{align*}
    
    \item Solución geométrica:
        \begin{itemize}
            \item Encuentre un punto en ambos planos (0,0,1).
            \item L arecta está en el plano I, entonces la recta es perpendicular al vector normal del plano I.
            \item Está en el plano z, entonces también es perpendicular al segundo vector normal.
            \item $\therefore $ la recta es perpendicular a ambos $\hat{n_1}$ \& $\hat{n_2}$
                \begin{align*}
                    \vec{v} = \hat{n_1} \times \hat{n_2} = \begin{vmatrix}
                        \hat{i} & \hat{j} & \hat{k} \\ 
                        1 & 1 & 0 \\ 
                        1 & 2 & 1 \\ 
                    \end{vmatrix} = \hat{i} - \hat{j} + \hat{k} \\ 
                    \text{  Ecuación de la recta:  } \quad r = \left\langle 0,0,1 \right\rangle + t \left\langle 1,-1,0 \right\rangle \\ 
                \end{align*}
        \end{itemize}
    
    \item Ejercicio 3: Encuentre el punto en el que la línea recta $x=1+2t$, $y=4t$, $z=5t$ interseca al plano. $x-y+2z=17$.
        \begin{align*}
            \begin{matrix}
                x = 1+2t \\ 
                y = 4t \\ 
                z= 5t \\ 
            \end{matrix} \\ 
            \text{  Plano  } \\ 
            x-y+2z = 17 \quad 1+2t-4t+10t = 17 \\ 
            8t = 16 \implies \therefore  t = 2 \\ 
        \end{align*}
        El punto de intersección es (5,8,10).
    
    \item Ejercicio 4: Encuentre una ec. del plano que contiene  la recta $x=1+t$, $y=2-t$, $z=4-3t$ y es paralela a plano $5x+2y+z=1$.
        \begin{itemize}
            \item Cualquier punto sobre la recta que también esté sobre el plano, t= 0. 
        \end{itemize}
        \begin{align*}
            \text{  Evaluemos en t=0  } \quad x=1, \, y=2, \, z=4\\
            \vec{r_0}= \left\langle 1,2,4 \right\rangle  \\ 
        \end{align*}
        \begin{itemize}
            \item \textbf{Nos preguntamos:} ¿Cómo se encuentra $\hat{n}$?
            \item El vectos de dirección de la recta $v=\left\langle 1,-1,-2 \right\rangle$ es paralelo al plano.
            \item Como es paralelo al seguno plano, entonces tiene que ser perpendicular $\hat{n_2} = \left\langle 5,2,1 \right\rangle$
            \item Lo que ocurre entonces es:
        \end{itemize}
            \begin{align*}
                \vec{r_0} = \left\langle 1,2,4 \right\rangle \quad \hat{n}= \left\langle 5,2,1 \right\rangle \\ 
                \text{  Ec. Plano:  } \, \implies \, 5(x-1) + 2(y-2)+1(z-4)=0 \\ 
            \end{align*}
    \item Ejercicio 5: Encuentre los números directores para la recta de intersección entre los planos $x+y+z=1$ \& $x+2y+3z=1$.
            \begin{itemize}
                \item \emph{\textbf{Definición de ``numeros directores":} a,b,c del vector de dirección $\left\langle a,b,c \right\rangle $}
                \item La recta es ortogonal a ambos vectores normales:
            \end{itemize}
            \begin{align*}
                \hat{n_1} = \left\langle 1,1,1 \right\rangle \quad text{ \& } \hat{n_2} = \left\langle 1,2,3 \right\rangle \quad \text{  de ambos planos  }\\
                \vec{v} = \hat{n_1} \times \hat{n_2} = \begin{vmatrix}
                    \hat{i} & \hat{j} & \hat{k} \\
                    1 & 1 & 1 \\ 
                    1 & 2 & 1 \\ 
                \end{vmatrix}   = \hat{i} -2\hat{j} +\hat{k} \\ 
                \text{  Los números directores:   } \quad a=1, b=2, c= 1 \\ 
            \end{align*}
    
    \item Ejercicio 6: Encuentre las ecs. aparamétricas de la recta que pasa por el punto (0,1,2), que es paralelo al plano $x+y+z=2$ y es perpendicular a la recta $r = \left\langle -2t,0,3t \right\rangle $.
            \begin{align*}
               L_1 r= \vec{r_0}+t \vec{v} \quad r_0 = \left\langle 0,1,2 \right\rangle 
            \end{align*}
            \begin{itemize}
                
                \item Aclaraciones: $L_1$ es la incógnita que tenemos que encontrar.
                \item \textbf{Nos preguntamos:} ¿Cómo se encuentra $r$?
                
                \item Plano I: $\hat{n} = \left\langle 1,1,1 \right\rangle $ es perpendicular al plano, es paralelo a $L_1$.
                \item Recta II: $\hat{v_2}= \left\langle -2,0,3 \right\rangle $ es perpendicular a $L_1$
                \item La recta es perpendiculae a $\hat{n}$ y a $\vec{v_2}$
            \end{itemize}
            \begin{align*}
                v=\hat{n} \times \vec{v_2} = \begin{vmatrix}
                    \hat{i} & \hat{j} & \hat{k} \\ 
                    1 & 1 & 1 \\ 
                    -2 & 0 & 3 \\ 
                \end{vmatrix} = 3 \hat{i} - 5 \hat{j} + 2 \hat{k} \\ 
                r_0 = \left\langle 0,1,2  \right\rangle \\ 
                v = \hat{v_2} \times \hat{n} \quad \text{  Ecuaciones paramétricas:   } \\ 
                \begin{matrix}
                    x=0-3t \\ 
                    y = 1-5t \\ 
                    z = 2 +2t \\ 
                \end{matrix} \\ 
            \end{align*}
            
\end{enumerate}



\chapter{Funciones vectoriales y curvas en el espacio, límites y continuidad} % 2020-02-04
% \date{2020-Feb-04 10:18:27}
\section{13.1 Funciones vectoriales y curvas en el espacio}
\begin{itemize}
    \item Una función vectorial $\vec{r}: R \implies V_3$ :
    \[
      \vec{r}(t) = \left\langle f(t),g(t),z(t) \right\rangle 
    \]
    
    La variable t es un parámetro.
    
    \item Dominio: Números reales, Rango: vector 3D:
        \begin{align*}
            \vec{r} \mathbb{IR} \implies V_3 \quad \vec{r}(t) = \left\langle f(t),g(t),h(t) \right\rangle \\ 
            \text{  t es un parámetro  } \quad \vec{r} = f(t)\hat{i} + g(t)\hat{j} + h(t)\hat{k} \\
        \end{align*}
    
    \item Ejemplo de una función vectorial:
        \begin{align*}
            \vec{r} = \left\langle a,b,c \right\rangle  + t \left\langle d,e,f \right\rangle \\ 
            \vec{r} = \left\langle a+td,b+et,c+tf \right\rangle \\ 
            x = f(t), \quad y = g(t), \quad z = h(t) \\ 
        \end{align*}
    
    \item Ecs. Paramétricas de una función vectorial: 
    
    \item Dominio de ina función vectorial: encuentre el dominio de cada función componente. El dominio de $\vec{r}$ es la intersección de los dominios de cada función componente.
\end{itemize}

%%%%%%%%%%%%%%%%%%%%%%%%%%%%%%%%%%%%%%%%%%%%%%%%%%%%%%%%%%%%%%%%%%%%%%%%%%%%%%%%%%%%%%%%%%%%%%%%
\subsection{Ejercicios}
\begin{enumerate}
    \item Encuentre el dominio:
        \begin{center}
            \begin{align*}
                r(t) = \left\langle \sqrt[]{r^2-9}, e^{5ln(t)}, ln(t+5) \right\rangle \\ 
                \text{  Evadir raíces negativas, y ln(0)  } \\ 
                \begin{matrix}
                    \sqrt[]{t^2-9} \quad \implies \quad \text{  Definida   } \quad t^2 \geq 9 \\ 
                    e^{\sin(t)} \quad \text{  siempore definida  } \\ 
                    ln(t+5) \quad \text{  Definida cuando   } \quad t +5 > 0 \quad \quad (-5, \infty ) \\ 
                    \therefore \text{  El dominio es de   } \quad (-5,\infty ) \, \cup (-5,-3) \, \cup (-3,3) \, \cup [3,\infty ) \\ 
                \end{matrix} \\ 
            \end{align*}
        \end{center}
        \begin{itemize}[label=\#]
            \item Recordar: [a,b] el numero si es parte del dominio a,b son partes del dominio. (a,b) los puntos a,b no son parte del dominio.
        \end{itemize}
    \item \begin{center}
        \begin{align*}
            \vec{s}(t) = \left\langle \sin^3(t^2), \cosh(\frac{t}{t^2+1} ), \frac{1}{e^t+4}  \right\rangle \\ 
            \begin{matrix}
                sin^3(t^2), ID_{f(t)} = IR \\ 
                \cosh(\frac{t}{t^2+1} ), ID_{g(t)} = IR \\ 
                \frac{1}{e^t+4}, ID_{h(t)} = IR \\ 
            \end{matrix} \\ 
            \therefore \text{  Dominio de   } \, \vec{s}(t) = (-\infty ,\infty ) \\ 
            e^+4\neq 0 \implies e^t=-4 \implies t = \underbrace{ln(-4)}_{\text{  indefinido  }} \\ 
        \end{align*}
    \end{center}
\end{enumerate}


%%%%%%%%%%%%%%%%%%%%%%%%%%%%%%%%%%%%%%%%%%%%%%%%%%%%%%%%%%%%%%%%%%%%%%%%%%%%%%%%%%%%%%%%%%%%%%%%
\section{Limites y continuidad}
\begin{itemize}
    \item \[
        \lim_{t \to a}\vec{r}(t) = \left\langle \lim_{t \to a} f(t),\lim_{t \to a} g(t),\lim_{t \to a} h(t) \right\rangle 
      \]
    
    \item Evalúe el límite de cada función componente.
    \item Si no existe por lo menos un límite de una función componente, entonces $\lim_{t \to a} \vec{r}(t) $ no existe.
    \item f(t) está definida en t=a
    \[
      \lim_{t \to a} f(t) = f(a)
    \]
    
    \item Si se indefine y tiene forma de $\frac{0}{0} $, $\frac{\infty}{\infty} $ usar L'H$\hat{o}$pital.
        \begin{center}
            \begin{align*}
                \lim_{t \to a} \frac{f(t)}{g(t)} \underbrace{=}_{\frac{0}{0} } \lim_{t \to a} \frac{f'(t)}{g'(t)} \quad \text{  L'Hopital  }
            \end{align*}
        \end{center}
    
    \item Contínua en $t=a$ si $\lim_{t \to a} \vec{r}(t)=\vec{r}(a)$
    \item Evite asíntotas verticales, saltos y agujeros. Ejemplo: 
        \begin{center}
            \begin{align*} 
                \lim_{t \to a} \frac{\sin(x)}{x} \underbrace{=}_{\text{  LH  }} \lim_{t \to a} \frac{cos(x)}{1} = 1 \\ 
            \end{align*}
        \end{center}
    
\end{itemize}


%%%%%%%%%%%%%%%%%%%%%%%%%%%%%%%%%%%%%%%%%%%%%%%%%%%%%%%%%%%%%%%%%%%%%%%%%%%%%%%%%%%%%%%%%%%%%%%%
\subsection{Ejercicios}
\begin{itemize}
    \item Sea $\vec{r}(t)=\left\langle \frac{\tan(\pi t)}{t} , e^{t-2}, \frac{ln(t-1)}{t^2-1}  \right\rangle $.
    \item Analice si la función $\vec{r}(t)$ es contínua en $t=2$.
        \begin{center}
            \begin{align*}
                \vec{r}(t) = \left\langle \frac{\tan 2\pi }{2}, e^0, \frac{ln(1)}{3}   \right\rangle \\ 
                \begin{matrix}
                    \lim_{t \to 2} \underbrace{\frac{\tan \pi t}{t}}_{\frac{0}{2} } = 0 \\
                    \lim_{t \to 2} e^{t-2} = 1 \\ 
                    \lim_{t \to 2} \frac{ln(t-1)}{t^2-1} = 0 \\ 
                \end{matrix} \\ 
                \therefore \vec{r} \, \text{ si es contínua en t=2   } \, \lim_{t \to 1} \vec{r}(t) = \vec{r}(2)
            \end{align*}
        \end{center}
    
    \item Encuentre $\lim_{t \to 1} \vec{r}(t)$ analice el límite de cada función componente por separado.
        \begin{center}
            \begin{align*}
                f: \, \lim_{t \to 1} \frac{\tan 2\pi }{2} = \frac{0}{1}  \\ 
                g: \, \lim_{t \to 1} e^{t-2} = e^{-1} \\ 
                h: \, \lim_{t \to 1} \frac{ln(t-1)}{t^2-1} = \, \text{  No existe, por ln(0) estar indefinido.  } \\ 
            \end{align*}
        \end{center}
    
    \item Analice si $\vec{r}(t)$ es contínua e t=1.
        \begin{center}
            \begin{align*}
                \underbrace{\lim_{t \to 1} \vec{r}(t) = \vec{r}(1) }_{\text{  No es contínua en t=1, r(1) está indefinida.  }}\\ 
            \end{align*}
        \end{center}
    
    \item Agujero $\vec{s}(t) = \left\langle \frac{\tan \pi t }{t-1} , e^{t-2}, \frac{ln(2t-1)}{t^2-1}  \right\rangle $ \newline 
     No es contínua en t=1, pero su límite existe.
     \begin{center}
        \begin{align*}
            \lim_{t \to 1} \frac{\tan \pi t}{t-1} \underbrace{=}_{LH} \lim_{t \to 1} \frac{\pi \sec^2 \pi t }{1} = \frac{\pi}{(\cos \pi)^2} = \pi \\ 
           \lim_{t \to 1} e^{t-2 } = e^-1 = \frac{1}{e} \\ 
           \lim_{t \to 1} \frac{\ln(2t-1)}{t^2-1} \underbrace{=}_{\frac{0}{0}} = \lim_{t \to 1} \frac{\frac{2}{2t-1} }{2t} = \lim_{t \to 1} \frac{2}{2t(2t-1)} = \frac{1}{1(2-1)} = 1 \\ 
           \therefore \lim_{t \to 1} \left\langle \pi, \frac{1}{e}, 1 \right\rangle \quad \text{  es un agujero   } \, \vec{s}(1) \, \text{  está indefinido  } \\   
        \end{align*}
     \end{center}
\end{itemize}


%%%%%%%%%%%%%%%%%%%%%%%%%%%%%%%%%%%%%%%%%%%%%%%%%%%%%%%%%%%%%%%%%%%%%%%%%%%%%%%%%%%%%%%%%%%%%%%%
\section{Curvas en el espacio}

\begin{center}
    \begin{align*}
        x = f(t) \\ 
        y = g(t) \\ 
        z = h(t) \\ 
    \end{align*}
\end{center}
\begin{figure}[htbp]
    \centering
    % \includegraphics[width=cm]{}
    \caption{Curvas paramétricas en el espacio}
    \label{}
\end{figure}

%%%%%%%%%%%%%%%%%%%%%%%%%%%%%%%%%%%%%%%%%%%%%%%%%%%%%%%%%%%%%%%%%%%%%%%%%%%%%%%%%%%%%%%%%%%%%%%%
\subsection{Espirales}
\begin{itemize}
    \item Grafique la curva $\vec{r}(t)$:
        \begin{center}
            \begin{align*}
                \vec{r}(t) = \underbrace{2 \hat{i} \sin (t)}_{x} + \underbrace{2 \hat{j} \cos (t) 
                }_{y} + \underbrace{\hat{k} \frac{t}{\pi}}_{z} \\ 
                \begin{matrix}
                    t & x & y & z \\ 
                    0 & 0 & 2 & 0.5 \\ 
                    \frac{\pi}{2} & 2&  0&  0.5 \\ 
                    \pi & 0 & -2&  1 \\ 
                    \frac{3\pi}{2} & 2 &  0 & 1.5 \\   
                    2\pi &  0 & 2 & 2 \\ 
                \end{matrix} 
            \end{align*}
            \begin{figure}[htbp]
                \centering
                % \includegraphics[width=cm]{}
                \caption{Curva paramétrica}
                \label{}
            \end{figure}
        \end{center}
    
    \item Grafique:
        \begin{center}
            \begin{align*}
                \vec{r}(t) = \left\langle \sin \pi t , t ,\cos \pi t \right\rangle \\ 
                \text{  Graficar la circumferencia  } \, x^2+z^2= 1\, , y = 0 \\ 
                \vec{r}(0) = \left\langle 0,0,1 \right\rangle \quad \text{  El vector que nos servirá para delimitar la gráfica del espiral  } \\
                \text{  Por ejemplo:   } \, \vec{s}(t) = \left\langle \sin t, t^2 , \cos t \right\rangle \\    
            \end{align*}
        \end{center}
\end{itemize}




















\chapter{Cálculo con funciones vectoriales}
\input{Clases/2020-02-06.tex} % 2020-02-06

\chapter{Funciones de varias variables} % 2020-02-11
% \documentclass{article}
\title{Temporary}
\author{David Gabriel Corzo Mcmath}
\date{\today}

\usepackage{amsmath}

% \usepackage{davidcorzo}

\newcommand{\derpar}[2]{
    \ensuremath{
        \frac{\partial {#1}}{{\partial {#2}}
    }
}





% \date{2020-Feb-11 10:05:03}
% \begin{document}

\section{13.2 Cálculo de funciones vectoriales}
\begin{itemize}
    \item Derivadas:
        \[
            \vec{r}\,'(t)=\left\langle f'(t),g'(t),h'(t) \right\rangle \\ 
        \]
    
    \item Vector Tangente:
        \[
          \vec{r}\, ' (t) 
        \]
    
    \item Tangente unitario:
        \[
          \vec{T}(t)=\frac{r'(t)}{\left| r'(t) \right| } 
        \]
    
    \item Integrales indefinidas:
        \begin{align*}
          \int_{}^{}\left\langle f,g,h \right\rangle dt = \left\langle F+C_1,G+C_2,H+C_3 \right\rangle \\
          \int_{}^{}\vec{r}(t) dt  = \vec{R}(t) + \vec{C} \\
          \vec{R} \quad \text{ vector de Antiderivadas  } \\ 
          \vec{C} \quad \text{  Vector de constantes  } \\   
        \end{align*}
    
    \item Integrales definidas:
        \[
          \int_{a}^{b}\vec{r}(t)dt = \hat{i} \int_{a}^{b}f(t)dt+ \hat{j} \int_{a}^{b}g(t)dt + \hat{k} \int_{a}^{b}h(t)dt
        \]
\end{itemize}


%%%%%%%%%%%%%%%%%%%%%%%%%%%%%%%%%%%%%%%%%%%%%%%%%%%%%%%%%%%%%%%%%%%%%%%%%%%%%%%%%%%%%%%%%%%%%%%%%%%
\section{Ejercicios de integración}
\begin{enumerate}
    \item $\int_{0}^{1}\left[\frac{4}{1+t^2}\hat{i} + sec^2(\frac{\pi t}{4} )\right] dt$:
        \begin{align*}
            4 \hat{i} \times \tan^{-1}(t) \Big|_0^1 + \hat{k} \times \tan(\frac{\pi t}{4} ) \Big|_{0}^{1} \\ 
            I_i = 4 \hat{i}  \frac{\pi}{4} + \hat{k} \frac{4}{\pi } = pi \hat{i} + \hat{k} \frac{4}{\pi } = \left\langle \pi,0,\frac{4}{\pi } \right\rangle \\ 
        \end{align*}
    
    \item $\int_{}^{}\left\langle te^{t^2},te^t,\frac{q}{\sqrt[]{1-t^2}} \right\rangle dt$ :
        \begin{align*}
            x: \quad \int_{}^{}e^{t^2}t\,dt =& \frac{1}{2} \int_{}^{}e^{u}du = \frac{1}{2} e^{t^2} +C_1 \\ 
                &\begin{matrix}
                    u = t^2 \\ 
                    du = 2tdt \\ 
                \end{matrix} \\ 
            y: \quad \int_{}^{} te^{t} dt = te^{t} - \int_{}^{} te^t-e^t+C_2 \\ 
                \begin{matrix}
                    u = t \quad dv = e^t dt \\ 
                    du = dt \quad v = e^t \\ 
                \end{matrix}
            t: \quad \int_{}^{}\frac{1}{1-t^2} dt = \frac{\cos(\theta)}{\sin(\theta)}d\theta = \int_{}^{} d\theta = \underbrace{\theta + C_3}_{\sin^{-1}(t)+C_3} = \sin^{-1}(t)+C_3 \\
            \therefore \quad \int_{}^{} \left\langle te^{t^2},te^t,\frac{1}{\sqrt{1-t^2}} \right\rangle dt = \frac{1}{2} et^{t^2}+C_1,te^t-e^t+C_2,\sin^1(t) + C_3 \\  
        \end{align*}
\end{enumerate}


%%%%%%%%%%%%%%%%%%%%%%%%%%%%%%%%%%%%%%%%%%%%%%%%%%%%%%%%%%%%%%%%%%%%%%%%%%%%%%%%%%%%%%%%%%%%%%%%%%%
\section{Movimiento en el espacio}
Dado el vector posición $\vec{r}(t)$ de un objeto: 
\begin{itemize}
    \item Vector velocidad:
        \[
          \vec{c}(t) = \vec{r}\, ' (t)
        \]
    
    \item Vector aceleración:
        \[
          \vec{a}(t) = \vec{v}\, (t) = \vec{r}\, '' (t)
        \]
    
    \item Rapidez:
        \[
          \left| \vec{v}(t) \right| 
        \]
    
    \item Distancia:
        \[
          \left| \vec{r}(t) \right| 
        \]    
\end{itemize}

Dado el vector de aceleración $\vec{a}(t)$ :
\begin{itemize}
    \item Velocidad:
        \[
          \vec{v}(t) = \int_{}^{}\vec{a}(t)dt + \vec{C}_1
        \]
    
    \item Desplazamiento o posición:
        \[
          \vec{r}(t) = \int_{}^{} \vec{v}(t)dt + C_2 
        \]
\end{itemize}


\subsection{Ejercicios}
\begin{enumerate}
    \item Encuentre la velocidad, aceleración y rapidez dada la posición del objeto:
        \begin{align*}
            \vec{r}(t) = \hat{i} t  + 2 \hat{j} \cosh(4t) + 3 \hat{k} \sinh(3t) \\ 
            \text{  Encontramos velocidad:  } \quad \vec{r}\, ' (t) = \vec{v}(t) = \hat{i} + 8 \hat{j} \sinh(4t)+9 \hat{k} \cosh(3t) \\ 
            \text{  Encontramos la aceleración:  } \quad \vec{r}\, ''(a) = \vec{a}(t) + 32 \hat{j} \cosh(4t) + 27 \hat{k} \sinh(3t) \\ 
            \text{  Encontramos la rapidez:   } \quad \left| \vec{v}(t) \right| = \sqrt{1+64\sinh(4t)+81\sinh^2(3t)} \\ 
            \text{  Encontramos la distancia:   } \quad \left| \vec{r}(t) \right| = \sqrt{t^2+4\cosh^2(4t)+9\sinh^2(3t)} \\ 
        \end{align*}
        \begin{itemize}[label=\#]
            \item Tarea \# 6: Integrales func. vectoriales 14.1 Funciones en varias variables.
            \item Tarea opcional consolidado: 12,13,14.1 
        \end{itemize}
    
    \item Encuentre la velocidad y posición del objeto dada $\vec{a}(t)$ y las condiciones iniciales:
        \begin{center}
            \begin{align*}
                \vec{a}(t)= 6 t \hat{i}+ \hat{j}  \cos(t)- \hat{k} \sin(2t), \quad \vec{v}(0) = \begin{matrix}
                    \hat{i} + \hat{k} \\ 
                    \vec{r}(0) = 2 \hat{j}  - \hat{k} \\ 
                \end{matrix} \\ 
                \text{  Velocidad:   } \quad \int_{}^{}\vec{a}(t)dt \\ 
                \vec{v}(t) = \left\langle 3t^2+C_1, \sin(t)+C_2, \frac{1}{2}\cos(2t)+C_3 \right\rangle \\ 
                \text{  Encuentro   } \; \vec{v}(0) = \left\langle C_1,C_2,\frac{1}{2} + C_3 \right\rangle  = \left\langle 1,0,1 \right\rangle \\
                \text{  Resolver para las constantes:   } \quad \begin{matrix}
                    C_1 = 1, \\  C_2 = 0, \\  \frac{1}{2} + C_3 = 1 \quad \implies  \quad C_3 = \frac{1}{2} \\ 
                \end{matrix} \\ 
                \text{  Posición:   } \quad \int_{}^{}\vec{v}(t)dt \\ 
                \vec{r}(t) \left\langle t^3+t+d_1, -\cos(t)+d_2, \frac{1}{4}\sin(2t) + \frac{t}{2} + d_3 \right\rangle \\ 
                \vec{r}(0) = \left\langle \underbrace{d_1,-1+d_2,d_3}_{\begin{matrix}
                    d_1 = 0 \\ 
                    -1+d_2=2 \implies d_2 = 3 \\ 
                    d_3 = -1 \\ 
                \end{matrix}} \right\rangle  \\
                \text{  Posición:   }\quad \vec{r}(t)= \left\langle t^3+t,3-\cos(t),\frac{1}{4}\sin(2t)+\frac{t}{2}-1 \right\rangle \\ 
            \end{align*}
        \end{center}
    
    \item $\vec{a}(t)= 8t \hat{i} + \sinh(t)\hat{j} - \hat{k} e^{\frac{t}{2} }$ : 
        \begin{center}
            \begin{align*}
                \underbrace{\vec{v}(0) = \vec{0}}_{\text{  Está en reposo  }} \quad \quad \vec{s}(0) = 2 \hat{i} + \hat{j} - 3 \hat{k}  \\ 
                \text{  Velocidad:   }\quad \vec{v}(t)= \left\langle 4t^2+C_1,\cosh(t)+C_2,-2e^{\frac{t}{2}}+C_3 \right\rangle \\ 
                \vec{v}(0) = \left\langle \underbrace{C_1,1+C_2,-2+C_3}_{\begin{matrix}
                    C_1 = 0, \\  C_2 = -1 \\ C_3 =2 \\ 
                \end{matrix}} \right\rangle = \left\langle 0,0,0 \right\rangle \\ 
                \vec{v}(t) = \left\langle 4t^2,\cosh(t)-1,-2e^{\frac{t}{2} +2} \right\rangle \\ 
                \text{  Posición:   } \quad \vec{r}(t) = \left\langle \frac{4}{3}t^3+C_1, \sinh(t)-t+C_2,-4e^{\frac{t}{2}}+2t+C_3 \right\rangle  \\ 
                \vec{r}(0) = \left\langle C_1,C_2,-4+C_3 \right\rangle = \underbrace{\left\langle 2,1,-3 \right\rangle }_{\begin{matrix}
                    C_2=1 \\ 
                    C_3 = -3+4 = 1 \\ 
                \end{matrix}} \\ 
                \vec{r}(t) = \left\langle \frac{4}{3}t^3+2,\sinh(t)-t+1,-4e^{\frac{t}{2}}+2t+1 \right\rangle \\  
            \end{align*}    
            \begin{itemize}[label=\#]
                \item Se evalúa el vector en 0 por que se quiere saber el valor de las constantes cuando están en reposo.
                \item Por defecto siempre evaluar en 0 para encontrar $C_1,C_2$\&$C_3$.
            \end{itemize}
        \end{center}
\end{enumerate}


%%%%%%%%%%%%%%%%%%%%%%%%%%%%%%%%%%%%%%%%%%%%%%%%%%%%%%%%%%%%%%%%%%%%%%%%%%%%%%%%%%%%%%%%%%%%%%%%%%%
\section{13.3 lOGN}
10.4 Ecs. Paramétricas de una curva en el plano de dos dimensiones era:
\begin{center}
   \begin{align*}
       \begin{matrix}
        x=f(t) \\ 
        y = g(t) \\ 
       \end{matrix}
   \end{align*}
\end{center}
\begin{itemize}
    \item La longitud de arco:
        \begin{align*}
            L = \int_{a}^{b} \sqrt{(x')^2+(y')^2+(z')^2} dt
        \end{align*}
    
    \item Función vectorial:
        \[
          \vec{r} = \left\langle f,g,h \right\rangle = \left\langle x,y,z \right\rangle \\ 
        \]
    
    
    \item Derivada de función vectorial:
        \[
          \vec{r}\,'= \left\langle x',y',z' \right\rangle 
        \]
    
    \item Magnitud: 
        \[
          \left| \vec{r}\,' \right| = \sqrt{(x')^2+(y')^2+(z')^2}
        \]
    
    \item En general: 
        \[
          L = \int_{a}^{b} = \left| \vec{r}\,'(t) \right|dt  
        \]
\end{itemize}



%%%%%%%%%%%%%%%%%%%%%%%%%%%%%%%%%%%%%%%%%%%%%%%%%%%%%%%%%%%%%%%%%%%%%%%%%%%%%%%%%%%%%%%%%%%%%%%%%%%
\section{Ejercicios}
Encuentre la longitud de las siguientes curvas:
\begin{enumerate}
    \item $\vec{r}(t)= \left\langle \cos(t),\sin(t),\ln(\cos) \right\rangle $ en $0 \le t \le \frac{\pi }{4} $
        \begin{center}
            \begin{align*}
                L = \int_{0 }^{\frac{\pi }{4}} \left| \vec{r}\, ' (t) \right| dt \\ 
                \vec{r}\,'(t) = \left\langle -\sin(t),\cos(t),\tan^2(t) \right\rangle \\ 
                \left| \vec{r}\,'(t) \right| = \sqrt{\sin^2(t)+\cos^2(t)+\tan^2(t)} = \sqrt{1+\tan^2(t)} = \sec^2(t) = \sec^2(t) \\ 
                L = \int_{0}^{\frac{\pi }{4}}\sec(t)dt = \ln\left| \sec(t)+\tan(t) \right| \Big|_{0}^{\frac{\pi }{4}} = \ln \left| \sec(\frac{\pi }{4}) + \tan(\frac{\pi }{4} ) \right| -  \ln \left| \sec(0) + \tan(0) \right|\\ 
                L = \ln \left| \frac{2}{\sqrt{2}} + 1  \right|  - \ln \left| 1 \right| = \ln \left| \sqrt{2}+1 \right| \\ 
            \end{align*}
        \end{center}
    
    \item $\vec{r}(t)= \left\langle 12t,8t^{\frac{3}{2}},3t^2 \right\rangle $ en $0 \le t \le 1$ :
        \begin{center}
           \begin{align*}
               \vec{r}\,'(t) = \left\langle 12,12t^{\frac{1}{2}},6t \right\rangle = 6 \left\langle 2,2t^{\frac{1}{2}},t \right\rangle  \\ 
               \left| \vec{r}\,'(t) \right| = 6 \sqrt{4+4t+t^2} = 6 \sqrt{(t+2)^2}= 6(t+2) \\ 
               L = \int_{0}^{1} (6t+12) dt = 3t^2+12t \Big|_{0}^{1} = 3+12 = 15 \\ 
           \end{align*}
        \end{center}
\end{enumerate}









% \end{document}


\chapter{Derivadas parciales} % 2020-02-11
\documentclass{article}
\title{Temporary}
\author{David Gabriel Corzo Mcmath}
\date{\today}

\usepackage{amsmath}

% \usepackage{davidcorzo}

\newcommand{\derpar}[2]{
    \ensuremath{
        \frac{\partial {#1}}{{\partial {#2}}
    }
}





\begin{document}


\section{Resolución de corto}
\begin{enumerate}
    \item Analice la función $r=\left\langle 3e^{-t},ln(2t^2-1),\tan(2\pi ) \right\rangle $ en $t=1$:
        \begin{center}
           \begin{align*}
               \lim_{t \to 1} \vec{r}(t) = \left\langle \lim_{t \to 1} 3e^{-t},\lim_{t \to 1} ln(2t^2-1),\lim_{t \to 1} \tan(2\pi ) \right\rangle \\ 
               \vec{r} = \left\langle 3e^{-1},ln(1),\tan(2\pi) \right\rangle  = \left\langle 3e^{-1},0,0 \right\rangle \\
               \therefore \quad \text{  r es contínua en t=1  } \\   
           \end{align*}
           \begin{itemize}[label=\#]
               \item Si la pregunta hubiese sido en cuándo se indefine, se saca el dominio de cada función.
           \end{itemize}
        \end{center}
    
    \item Encuentre la ec. de la recta tatente a $r(t)= \left\langle te^{t-1},\frac{8}{\pi } \arctan(t), 2\ln(t) \right\rangle $ en $t=1$.
        \begin{center}
           \begin{align*}
               \vec{r}(0) = \left\langle 1\times e^0,\frac{8}{\pi } \arctan(1), 2ln(0) \right\rangle = \left\langle 1,2,0 \right\rangle \\ 
            %    \vec{r}\,'(t) = \left\langle e^{t-1} \right\rangle 
            \text{  Terminar de copiar  } \\ 
           \end{align*}
        \end{center}
\end{enumerate}


%%%%%%%%%%%%%%%%%%%%%%%%%%%%%%%%%%%%%%%%%%%%%%%%%%%%%%%%%%%%%%%%%%%%%%%%%%%%%%%%%%%%%%%%%%%%%%%%%%%
\section{14.1 Funciones de varias variables}
\begin{itemize}
    \item Cuando teníamos sólo una función de una variable no había tanta complicación, las gráficas eran curvas en el plano. Cuando empezaba y terminaba la curva en $x$ nos daba el dominio. Había una variable independiente $x$ y la variable dependiente $y$, los dominios eran intervalos, y cada $x$ sólo podía tener \textbf{un} sólo valor de $y$.
    \item En funciones de 2 variables se va a describir como:  
        \begin{center}
           \begin{align*}
                z = f(x,y) \quad &\text{  Dos variables independientes x,y  } \\ 
                & \text{  Variable dependiente z  } \\ 
           \end{align*}
        \end{center}
        \begin{itemize}
            \item Entonces $f$ es una regla que asigna a cada punto $(x,y)$ a lo sumo un valor de $z$.
        \end{itemize}
        \[
          f: \, \underbrace{\mathbb{R}^2}_{\text{  Dominio  }} \rightarrow \underbrace{\mathbb{R}}_{\text{  Rango  }}
        \]
        \begin{itemize}
            \item Estamos pasando de una región por medio de una función $z$ llego a tener $f(x,y)$ en la dimensión correspondiente.
            \item Los dominios en estas funciones se vuelven superficies.
        \end{itemize}
    
    \item El dominio de una función de dos variables: un conjunto que consiste de todos los puntos o pares ordenados $(x,y)$ para los cuales $f(x,y)$ para los cuales $f(x,y)$ está definida.
        \[
          \mathbb{D}: \quad \text{En una dimensión: Todos los números x para los cuales f(x) está definida  }
        \]
        \begin{itemize}
            \item Evite la división por cero.
            \item Raíces pares de números negativos.
            \item Logaritmos de números negativos o cero.
        \end{itemize}
    
    \item El dominio de $f$ en una función de dos variables es una región:
        \begin{itemize}
            \item Las regiones que estén sombreadas son partes del dominio.
        \end{itemize}
        \begin{itemize}[label=\#]
            \item Para graficar funciones de dos variables son más fáciles de graficar que de una sola variable.
        \end{itemize}
\end{itemize}


%%%%%%%%%%%%%%%%%%%%%%%%%%%%%%%%%%%%%%%%%%%%%%%%%%%%%%%%%%%%%%%%%%%%%%%%%%%%%%%%%%%%%%%%%%%%%%%%%%%
\section{Ejercicios}
Encuentre y bosqueje el dominio de las sigs. funciones. \newline 
Sombree la región dque es parte del $\mathbb{D}$ y utilice líneas discontínuas para denotar a curvas que no son parte del $\mathbb{D}$
\begin{enumerate}
    \item $c(x,y)=10x+20y$ : 
        \begin{center}
           \begin{align*}
               \text{  Nunca se indefine.  } \\ 
               \mathbb{D}: \underbrace{(-\infty,\infty )}_{x} \underbrace{\times}_{\text{  Producto cartesiano  }} \underbrace{(-\infty ,\infty )}_{y} = \mathbb{R}^2 \\ 
           \end{align*}
           \begin{itemize}[label=\#]
               \item Producto cartesiano denota \textbf{todas las combinaciones posibles en un conjunto de $n$ elementos}.
               \item Explicaciones de productos cartesianos:
                \[
                  \mathbb{R} \cup  \mathbb{R} = \mathbb{R} \quad \mathbb{R} \times \mathbb{R} = \mathbb{R}^2     
                \]
                
                \item Definición de producto cartesiano:
                    \[
                      x \times y = \{(x,y) \; \text{  tal que  } \; x \in X, \, y \in Y \}
                    \]
                
                \item Producto cartesiano vs. unión:
                    \begin{center}
                       \begin{align*}
                           x \times y = \{(1,1),(1,2),(1,3),(2,1),(2,2),(2,3),(3,1),(3,2),(3,3)\} \\ 
                           x \cup y = \{(1),(2),(3)\}
                       \end{align*}
                    \end{center}
           \end{itemize}
        \end{center}
    
    \item $z = \frac{8}{x^2-y^2} $:
        \begin{center}
           \begin{align*}
               \text{  Definida si   } \; x^2 \neq y^2 \\ 
               \mathbb{R}^2 - \{x^2\neq y^2\} \\ 
               y \neq \sqrt{x^2} \\ 
               y \neq \pm x \\ 
           \end{align*}
        \end{center}
    
    \item $R(x,y)= \sqrt{9-x^2-y^2}$ : 
        \begin{center}
           \begin{align*}
               \text{  Definida  } \; \begin{matrix}
                   9-x^2-y^2 \geq 0 \\ 
                   9 \geq x^2 + y^2 \\ 
                   \mathbb{D}: x^2+y^2 \neq 9 \\ 
               \end{matrix} \\ 
               \text{  Círculo de radio 3 centrado en el orígen  } \\ 
               \mathbb{D} = \{(x,y) \; \text{  tal que  } \; x^2+y^2 \leq 9  \} \\ 
           \end{align*}
        \end{center}
    
    \item $Q(x,y)=\frac{1}{\sqrt{x^2+y^2-9}} $ : 
        \begin{center}
           \begin{align*}
               \mathbb{D}: \; \begin{matrix}
                   x^2+y^2 > 0 \\ 
                   x^2+y^2 > 9 \\ 
               \end{matrix} \\ 
               \therefore \text{  Afuera del círculo o disco de radio 3  } \\ 
           \end{align*}
        \end{center}
    
    \item $ z = \frac{(x+4)}{(y-2)(x-4)(y+2)} $ : 
        \begin{center}
           \begin{align*}
               \text{  Definida si  }: \quad y \neq \pm 2, \; x\neq 4 \\ 
               \mathbb{D}: \quad \mathbb{R}^2-  \{y\neq \pm 2, x\neq 4\} \\ 
           \end{align*}
        \end{center}
    
    \item $h(x,y)=\ln(2-yx)$ : 
        \begin{center}
           \begin{align*}
               \text{  Definida si  }: \quad \begin{matrix}
                   2-yx &> 0 \\ 
                   2 &> yx \\ 
                   y &< \frac{2}{x} \\ 
               \end{matrix} \\ 
               \therefore \mathbb{D}: \; y < \frac{2}{x} \\ 
           \end{align*}
        \end{center}
\end{enumerate}


%%%%%%%%%%%%%%%%%%%%%%%%%%%%%%%%%%%%%%%%%%%%%%%%%%%%%%%%%%%%%%%%%%%%%%%%%%%%%%%%%%%%%%%%%%%%%%%%%%%
\subsection{Gráfica de $z=f(x,y)$}
\begin{itemize}
    \item Gráfica de $z=f(x,y)$: Son superficies y consisten de todas las \emph{triplas} ordenadas $(x,y,z)$ donde $z$.
\end{itemize}



%%%%%%%%%%%%%%%%%%%%%%%%%%%%%%%%%%%%%%%%%%%%%%%%%%%%%%%%%%%%%%%%%%%%%%%%%%%%%%%%%%%%%%%%%%%%%%%%%%%
\section{Curva de nivel o traza horizontal}
\begin{itemize}
    \item En $f(x,y)=k$ k es una constante, rebane la superficie con los planos horizontales $z=k$ y grafique cada curva en el plano.
\end{itemize}



















    
\end{document}


\chapter{Derivadas parciales, rectas tangentes y planos tangentes} % 2020-02-20
% \documentclass{article}
\title{Temporary}
\author{David Gabriel Corzo Mcmath}
\date{\today}

\usepackage{amsmath}

% \usepackage{davidcorzo}

\newcommand{\derpar}[2]{
    \ensuremath{
        \frac{\partial {#1}}{{\partial {#2}}
    }
}






% \begin{document}
    


\section{14.3 Derivadas parciales}
\begin{itemize}
    \item Derivada en una dimensión:
        \[
          \lim_{h \to 0} \frac{f(x+h)-f(x)}{h} = f'(x)  
        \]
    
    \item En una función con dos variables independientes:
        \begin{align*}
            f(x,y) = \begin{rcases}
                \begin{matrix}
                    f_x(x,y) \\ 
                    f_y(x,y) \\ 
                \end{matrix}  
            \end{rcases} \text{  Derivadas parciales  } \\
        \end{align*}
    
    \item Al derivarse parcialmente respecto a una variable, la otra se mantiene constante:
        \begin{align*}
            f_x(x,y) =& \lim_{h \to 0} \frac{f(x+h,y)-f(x,y)}{h} \quad \quad \text{  \# y se mantiene constante  }\\  
            f_y(x,y) =& \lim_{h \to 0} \frac{f(x,y+h)-f(x,y)}{h} \quad \quad \text{ \# x se mantiene constante  }\\  
        \end{align*}
    
    \item Se pueden utilizar todas las reglas de derivación para funciones de 1 variable:
        \begin{itemize}
            \item Suma 
            \item Producto 
            \item Cociente 
            \item Cadena 
        \end{itemize}
    
    \item $1^{\text{  eras  }}$ derivadas parciales de $f(x,y)$: encuentre todas las derivadas parciales posibles de $f_x$ \& $f_y$
        \begin{itemize}
            \item Notación:
                \[
                  f_x=\frac{\delta f}{\delta x} = \frac{\delta z}{\delta x}  
                \]
                \[
                    f_x=\frac{\delta f}{\delta y} = \frac{\delta z}{\delta y}  
                \]
            
            \item Evite $f'(x,y)$ para evitar ambigüedad.
        \end{itemize}
\end{itemize}



%%%%%%%%%%%%%%%%%%%%%%%%%%%%%%%%%%%%%%%%%%%%%%%%%%%%%%%%%%%%%%%%%%%%%%%%%%%%%%%%%%%%%%%%%%
\subsection{Ejercicios}
Encuentre las derivadas parciales de las siguientes funciones.
\begin{enumerate}
    \item $f(x,y)=2x^2+3xy\,$ : \emph{\textbf{Recordar lo siguiente: }$f_x(x,y)$ \& $f_y(x,y)$} 
        \begin{center}
           \begin{align*}               
                f_x = 4x+3y \quad \quad f_y=0+3x \\ 
           \end{align*}
        \end{center}
    
    \item $g(x,y)=y(x^2+1)^3+x^2(y^4-4)^4+5x^2y^3\,$ :
        \begin{center}
           \begin{align*}
                g_x&=3y(x^2+1)^22x+2x(y^4-4)^4+10xy^3 \\ 
                g_y&=1 \cdot (x^2+1)^3 + 16y^3x^2(y^4-4)^3+15x^2y^2 \\ 
           \end{align*}
        \end{center}        
    
    \item $h(s,t)=(s^2+10t)^2\cdot(t^4+s^3)^3\,$: \# Regla del producto y de la cadena.
        \begin{center}
           \begin{align*}
               h_s &= 4s(s^2+10t)^1\cdot(t^4+s^3)^3+3\cdot3s^2(s^2+10t)^2\cdot(t^4+s^3)^2 \\ 
               h_t&= 20(s^2+10t)^1\cdot(t^4+s^3)^3+12t^3(s^2+10t)^2,.
               (t^4+s^3)^2 \\  
           \end{align*}
        \end{center}
\end{enumerate}
%----------------------------------------------------------------------------------------
\begin{itemize}[label=\#]
    \item Evalúe la derivada en punto $(a,b)$:            
        \[
            f_x(a,b)=\frac{\delta f}{\delta x} \Big|_{(a,b)}^{} 
        \]
\end{itemize}

%----------------------------------------------------------------------------------------
\begin{enumerate}
    \item $w(r,\theta)=r^2\sin(2\theta)+e^{\pi r - \theta}\,$, encuentre $\frac{\delta w}{\delta \theta} \Big|_{(2,\pi)}^{}$
        \begin{center}
           \begin{align*}
               \frac{\delta w}{\delta \theta } &= 2r^2\cos(2\theta)-e^{\pi r -\theta} \\ 
               \frac{\delta w}{\delta \theta } \Big|_{(2,\pi)}^{} &= w_\theta(2,\pi) = 2\cdot 4\cos(2\pi)-e^{2\pi-\pi} \\ 
               &= 8-e^\pi
           \end{align*}
        \end{center}
\end{enumerate}


%%%%%%%%%%%%%%%%%%%%%%%%%%%%%%%%%%%%%%%%%%%%%%%%%%%%%%%%%%%%%%%%%%%%%%%%%%%%%%%%%%%%%%%%%%
\section{Derivadas parciales par funciones de 2 o más variables}
\begin{itemize}
    \item Se deriva respecto a una variable y el resto se mantienen constantes.
        \[
            w = f(x,y,z) 
        \]
        3 $1^{\text{  eras  }}$ derivadas parciales: $f_x,f_y,f_z$. 
        \[
          u = f(x_1,x_2,\dots,x_n)
        \]
        n derivadas parciales: 
        \[
          \frac{\delta u }{\delta x} , \dots \frac{\delta u}{\delta x_n} 
        \]
\end{itemize}


%----------------------------------------------------------------------------------------
\subsection{Ejercicio}
Encuentre todas las primeras derivadas pariales de las sigentes funciones:
\begin{itemize}
    \item $f(x,y,z)=\sqrt[4]{x^4+8xz+2y^2}$
        \begin{center}
           \begin{align*}
               f_x &= \frac{1}{4} (x^4+8xz+2y^2)^{-\frac{3}{4} }\cdot(4x^3+8z+0) \\ 
               f_y &= \frac{1}{4} (x^4+8xz+2y^2)^{-\frac{3}{4} }\cdot(4y) \\ 
               f_z &= \frac{1}{4} (x^4+8xz+2y^2)^{-\frac{3}{4} }\cdot(8x) \\ 
           \end{align*}
        \end{center}
    
    \item $p(r,\theta,\phi)=r\cdot \tan (\phi^2-4\theta)\,$: 
        \begin{center}
           \begin{align*}
               p_r &= \tan (\phi^2-4^\theta) \\ 
               p_{\theta} &= -4r \sec ^2 (\phi^2-4\theta) \\ 
               p_{\phi} &= 2\phi r \sec ^2(\phi^2-4\theta) \\ 
           \end{align*}
           \begin{itemize}[label=\#]
               \item Funciones vectoriales 1 variable: $\vec{r}\,' (t), \dots $
           \end{itemize}
        \end{center}
\end{itemize}


%%%%%%%%%%%%%%%%%%%%%%%%%%%%%%%%%%%%%%%%%%%%%%%%%%%%%%%%%%%%%%%%%%%%%%%%%%%%%%%%%%%%%%%%%%
\section{Derivadas parciales de orden superior (pág. 100)}
\begin{itemize}
    \item Orden superior: Segundas, terceras, cuartas, etc. derivadas.
    \item Como $f_x(x,y)$ \& $f_y(x,y)$ son también funciones en dos variables, pueden tener derivadas parciales.
        \begin{tikzpicture}[node distance = 2.5cm, auto]
            \node [block] (1) {$f_x$};
            \node [block,right of=1] (2) {$f_{xx}$}; 
            \node [block,right of=1,below of=1] (3) {$f_{xy}$};
            \path [line] (1) -- (2);
            \path [line] (1) -- (3); 
        \end{tikzpicture}
        \begin{tikzpicture}[node distance = 2.5cm, auto]
            \node [block] (1) {$f_y$};
            \node [block,right of=1] (2) {$f_{yy}$}; 
            \node [block,right of=1,below of=1] (3) {$f_{yx}$};
            \path [line] (1) -- (2);
            \path [line] (1) -- (3); 
        \end{tikzpicture}
    
    \item Las segundas derivadas parciales, éstas también tienen sus derivadas parciales, terceras derivadas parciales.
        \begin{center}
           \begin{tabular}{  p{1cm}  p{1cm}  p{1cm}  p{1cm}  }
                    $f_{xxx}$ & $f_{xxy}$ & $f_{yyy}$ & $f_{yxy}$  \\
                    $f_{xxy}$ & $f_{xyx}$ & $f_{yyx}$ & $f_{yxx}$ \\ 
           \end{tabular}
        \end{center}
    
    \item Las derivadas parciales cruzadas $f_{xy}$ \& $f_{yx}$ son iguales si la función es diferenciable.
        \[
          f_{xy} = f_{yx} \quad \quad f_{xyy} = f_{yyx}= f_{yxy}
        \]
    
    \item Notación delta:
        \begin{center}
           \begin{align*}
               f_{xx} &= \frac{\delta }{\delta x} \left(\frac{\delta f}{\delta x}\right) = \frac{\delta^2f}{\delta x^2}  \quad  \quad  f_{yy} = \frac{\delta^2f}{\delta y^2} \\ 
               f_{xy} &= \frac{\delta}{\delta y} \left(\frac{\delta f}{\delta x}\right) = \frac{\delta ^2 f}{\delta y \delta x}  \quad \quad f_{yx} = \frac{\delta^2f}{\delta y \delta y} \\ 
           \end{align*}
        \end{center}
\end{itemize}


%----------------------------------------------------------------------------------------
\subsection{Ejercicios}
Encuentre todas las 2das derivadas parciales:
\begin{enumerate}
    \item $f(x,y)=\sin (mx+ny)\quad m,n \in \mathbb{R}\,$:  
        \begin{align*}
                \text{  Primeras derivadas parciales  :} \\ 
                f_x =& m \cos (mx+ny) \\ 
                f_y &= n \cos (mx+ny ) \\ 
                \text{  Segundas derivadas parciales:   } \\ 
                f_{xx} &= -m^2 \sin (mx+ny ) \\ 
                f_{yy} &= -n ^2 \sin (mx+ny) \\ 
                \begin{rcases}
                    f_{xy} &= -mn \sin (mx+ny) \\ 
                    f_{yx} &= -mn \sin (mx+ny) \\ 
                \end{rcases} \text{  Iguales  } \\ 
        \end{align*}
    
    \item $z = \cos (2xy)\,$ : 
        \begin{center}
           \begin{align*}
               1^{\text{  eras  }}: \quad \frac{\delta z }{\delta x} &= -2 \sin (2xy) 
               ,\quad \frac{\delta z}{\delta y } = -2x \sin (2xy) 
               \\  
               2^{\text{  das  }}: \quad \frac{\delta ^2 z}{\delta x^2} &= -4 y^2 \cos (2xy) ,\quad \frac{\delta^2 z }{\delta y^2} = -4x^2 \cos (2xy) \\  
           \end{align*}
        \end{center}
\end{enumerate}



























% \end{document}


\chapter{Multiplicadores de Lagrange} % 2020-02-27
\documentclass{article}
\title{Temporary}
\author{David Gabriel Corzo Mcmath}
\date{\today}

\usepackage{amsmath}

% \usepackage{davidcorzo}

\newcommand{\derpar}[2]{
    \ensuremath{
        \frac{\partial {#1}}{{\partial {#2}}
    }
}






\begin{document}


\section{Derivadas parciales, rectas tangentes y planos tangentes}

%----------------------------------------------------------------------------------------
\subsection{Interpretación de la derivada parcial}
\begin{itemize}
    \item $\mathbb{C}$ curva de intersección entre $z=f(x,y)$ y $y=b$.
    \item Recta tantente a eta curva en el punto $(a,b,f(a,b))$:
        \[
          \text{  Derivada : }\; f_x(x,b) \quad \quad \text{  Pendiente:  }\; f_x(a,b)
        \]
    
    \item Derivadas parciales: $f_x(a,b)$ resulta ser la pendiente de la recta tangente a la curva $f(x,b)$ en la dirección de $x$.
        \[
          L = \left\langle a,b,f(a,b) \right\rangle + t \left\langle 1,0,f(a,b) \right\rangle \quad \quad \text{  donde:  } \; x =t, y=b,z=f(t,b) 
        \]
    
    \item  Para encontrar $L_2$ $x=a$:
        \begin{center}
           \begin{align*}
               x&=a, y=t, x=f(z,y) \;\implies\; z_y=f_y(a,y) \; \implies \; z_y=f_y(a,b) \\ 
               z_y&=f(a,b) \; \text{  es la pendiente de la tangente  a la curva   }\; f(a,y) \; \text{  en la dirección de   }\; y \\   
               L_2 &= \left\langle a,b,f(a,y) \right\rangle + t \left\langle 0,1,f_y(a,b) \right\rangle \\ 
           \end{align*}
        \end{center}
    
    \item Estas dos rectas se utilizan para construir un plano tangente a la superficie.
    \item La ecuación del plano es un plano que es paralelo a $L_1$ \& $L_2$.
        \begin{center}
           \begin{align*}
               L_1&= \left\langle a,b,f(a,b) \right\rangle + t \overbrace{\left\langle 1,0,f(a,b) \right\rangle}^{v_1} \\ 
               L_2&= \left\langle a,b,f(a,b) \right\rangle + t \underbrace{\left\langle 0,1,f(a,b) \right\rangle}_{v_2} \\ 
           \end{align*}
        \end{center}
        La ec. vectorial:
            \[
              \hat{n} \cdot (-r_0)=0 \quad \vec{r}_0  = \left\langle a,b,f(a,b) \right\rangle 
            \]
        % \begin{center}
        % %    \begin{align*}
        % %        \hat{n}= v_1 \times v_2 = \begin{matrix}
                   
        % %        \end{matrix}
        % %    \end{align*}
        % \end{center}
        terminar excursión.
\end{itemize}


%----------------------------------------------------------------------------------------
\subsection{Ejercicios}
\begin{itemize}
    \item Encuentre el plano tangenge a la superficie $z=\ln(x-2y)$ en el punto $(3,1,0)$:
        \begin{center}
           \begin{align*}
               f(a,b) \quad f_x(a,b) \quad f_y(a,b) \quad a=3,\; b=1 \\ 
               f(3,1) = \ln(3-2) = \ln(1) = 0 \\ 
               \frac{\partial f}{\partial x } = \frac{1}{x-2y}  \quad \frac{\partial f}{\partial x } \Big|_{}^{(3,1)} = \frac{1}{3-2} = 1 \\ 
               \frac{\partial f}{\partial y } = \frac{-2}{x-2y} \quad \frac{\partial f}{\partial y } \Big|_{}^{(3,1)} = \frac{-2}{3-2} = -2 \\ 
               \text{  La ecuación del plano tangente:  } \quad \begin{matrix}
                    z = f(3,1)+f_x(x-3)+f_y(y-1) \\ 
                    z = 0 + x-3-2y+2 \\ 
                    \therefore \quad  z = x-2y-1 \\ 
               \end{matrix} \\ 
           \end{align*}
        \end{center}
\end{itemize}


%%%%%%%%%%%%%%%%%%%%%%%%%%%%%%%%%%%%%%%%%%%%%%%%%%%%%%%%%%%%%%%%%%%%%%%%%%%%%%%%%%%%%%%%%%
\section{Aproximaciones lineales}
\begin{itemize}
    \item La aproximación lienal de $z=f(x,y)$, linearización.
    \item La aproximación lineal de $z$ en $(a,b)$ es el plano tangente a la superficie.
        \[
          L(x,y)= f(a,b) + \frac{\partial }{} 
        \]
\end{itemize}


%----------------------------------------------------------------------------------------
\subsection{Ejercicios}
Considere la función $f(x,y)=\sqrt[]{2x+2e^y}$: 
\begin{itemize}
    \item Encuentre la aproximación lineal de $f$ en el punto $(7,0)$: \newline Encuentre $f(7,0) \quad f_y(7,0)$
        \begin{center}
           \begin{align*}
               f(7,0)&= \sqrt{14+2} = 4 \\ 
               f_x(x,y)&= (2x+2e^y)^{-\frac{1}{2} } \quad \quad f_x(0,7)= \frac{1}{\sqrt{14+2}} ? \frac{1}{4} \\ 
               f_y(x,y) &= \frac{e^y}{\sqrt{2x+2e^y}} \quad \quad f_y(7,0) = \frac{1}{\sqrt{14+2}} ? \frac{1}{4} \\ 
               \therefore &\; \text{  La aproximación lineal o plano tangente:   }\; L= 4+\frac{1}{4} (x-7) + \frac{1}{4} y \\ 
               \text{  Cerca de (7,0): }\; &\sqrt[]{2x+2e^y} \approx \frac{9}{4} + \frac{1}{4} x + \frac{1}{4} y \\ 
           \end{align*}
        \end{center}
    
    \item Utilice la aproximación lineal para aproximar el valor de $\sqrt{8+2e}$ : 
        \begin{center}
           \begin{align*}
               f(4,1) = \sqrt{8+2e} \approx 3.5 \approx L(4,1) \\ 
               L(4,1) = \frac{9}{4} + \frac{4}{4} + \frac{1}{4} = \frac{7}{2} = 3.5 \\ 
               \text{  En realidad  : }\; \sqrt{8+2e} \approx 3.665592 \\ 
           \end{align*}
        \end{center}
    
    \item Ejercicio 3: Encuentre la aproximación lineal de $g(x,y)=1+\ln (xy-5)$ en el punto $(2,3)$:
        \begin{center}
           \begin{align*}
               g(2,3)&=1+2 \ln (6-5) = 1+0 = 1 \\ 
               g_x(x,y) &= 0 + 1\cdot \ln (xy-5) + \frac{xy}{xy-5}  \\ 
               g_x(2,3) &= \ln (1) + \frac{6}{6-5} = 0 + \frac{6}{1} = 6 \\ 
               g_y(x,y) &= 0 + \frac{x\cdot x}{xy-5} \\ 
               g_y(2,3) &= \frac{4}{6-5} = 4 \\  
           \end{align*}
           La aproximación lineal entonces es:  
            \begin{align*}
                \therefore \\ 
                L(x,y) &= 1 + 6(x-2) + 4(y-3) \\ 
                L(x,y) &= -23 + 6x + 4y \\ 
            \end{align*}
        \end{center}
\end{itemize}



%%%%%%%%%%%%%%%%%%%%%%%%%%%%%%%%%%%%%%%%%%%%%%%%%%%%%%%%%%%%%%%%%%%%%%%%%%%%%%%%%%%%%%%%%%
\section{12.4 Derivadas implicitas y 12.5 Regla de la cadena }
\begin{itemize}
    \item Funciones 2 variables $z=f(x,y)$
    \item Explícita: $z$ no está sólo en función de $x$ \& $y$.
    \item Ejemplos: $x^2+y^2+z^2=16$, $\sqrt[]{z^2-x^2}=y+z$
    \item \textbf{¿}Cómo se encuetnran $\frac{\partial z}{\partial x }$ \& $\frac{\partial z}{\partial y} $ \textbf{?}:
        \begin{itemize}
            \item Implicita $x^2+y^2+z^2=16$ es una esfera de 4 (rango [-4,4]) en dos hemisferios:
                \[
                  z = +\sqrt{16-x^2-y^2}
                \]
        \end{itemize}
        \begin{center}
           \begin{align*}
               \frac{\partial z}{\partial x} &= \frac{1}{2} (16-x^2-y^2)^{-\frac{1}{2} }(-2x) = \frac{-x}{\sqrt{16-x^2-y^2}} = - \frac{x}{z} \\ 
               \frac{\partial z}{\partial y} &= -\frac{y}{z} \\  
           \end{align*}
        \end{center}
    
    \item Derivación implicita, se pueden encontrar $z_x$ \& $z_y$ sin necesidad de resolver para $z$.
        \begin{center}
           \begin{align*}
               x^2+y^2+z^2=16 \quad \text{  z \& y son independientes  } \\ 
               \frac{\partial }{\partial x}(x^2+y^2+z^2(x,y)) = \frac{\partial }{\partial x} (16) \\ 
               2x+0+2z \frac{\partial z}{\partial x} = 0 \\ 
               2z \frac{\partial z}{\partial x} = -2x  \quad  \implies \quad  \frac{\partial z}{\partial x} ? \frac{-x}{z} \\ 
                \frac{\partial }{\partial y} (x^2+y^2+z^2) = \frac{\partial }{\partial y}(0) \\ 
                0+2y+2z \frac{\partial z}{\partial y} = 0 \quad \implies \quad \frac{\partial z}{\partial y} = \frac{-y}{z} \\  
           \end{align*}
        \end{center}
\end{itemize}


\subsection{Derivación parcial implícita abreviada}
\begin{itemize}
    \item $x^2+y^2+z^2=16$ como $x \ln (y) + x^2 \sqrt[]{1+x+z} = k$
    \item Forma implícita: $F(x,y,z(x,y))=$ constante. $\frac{\partial z}{\partial x} $ use la regla de la cadena.
        \begin{center}
           \begin{align*}
               \frac{\partial F}{\partial x} &+ \frac{\partial F}{\partial z} \frac{\partial z}{\partial x}  = 0  \quad \implies \quad z_x= - \frac{f_x}{f_z} \\ 
               \frac{\partial f}{\partial y} &+ \frac{\partial f}{\partial z} \frac{\partial z}{\partial y} = 0 \quad \implies  \quad z_y = - \frac{f_y}{f_z} \\ 
           \end{align*}
        \end{center}
\end{itemize}


%----------------------------------------------------------------------------------------
\subsection{Ejercicios}
Encuentre las primeras derivadas parciales de $z$.
\begin{enumerate}
    \item $\ln(zy)+9z-xyz = 1 $:
        \begin{center}
           \begin{align*}
               \begin{matrix}
                   F_x = -yz \\ 
                   F_y = y^{-1}+0-xy \\ 
                   F_z = z^{-1}+9-xy \\ 
               \end{matrix}
               \begin{matrix}
                   \frac{\partial z}{\partial x} = -\frac{F_x}{F_y} = \frac{yz}{z^{-1}+9-xy} \\ 
                   \frac{\partial z}{\partial y} = \frac{xz-y^{-1}}{z^{-1}+9-xy} \\    
               \end{matrix} \\ 
           \end{align*}
        \begin{itemize}[label=\#]
            \item Sin derivación parcial implícita
            \item $z(x,y)$ agregue $z_x$ cada vez que aparece $z$.
        \end{itemize}
        \begin{align*}
            \frac{yz_x}{z_y} + 9 z_x - y_x
        \end{align*}
        \end{center}
\end{enumerate}































\end{document}


\chapter{Máximos y mínimos} % 2020-03-10
\input{Clases/2020-03-10.tex} % 2020-03-12

\chapter{Multiplicadores de Lagrange}
\section{14.8 Multiplicadores de Lagrange}
Una función de dos variables puede estar sujeta a una restricción:
\[
  \text{ Máximo } \qq  z= f(x,y) \qq \text{ Sujeto A } \qq g(x,y) = c
\]
Si no es posible resolver para $y$ ó $x$ en la restricción, el problema no se puede reducir a una sola variable. \newline 
Se introduce una nueva variable, el \textbf{multiplicador de Lagrange} $\lambda $ para incorporar la restricción en la función objetivo. 
\[
  c - g(x,y) = 0 
\]

\[
  \underbrace{F(x,y,\lambda )}_{\text{ Función objetivo y restricción }} = \underbrace{f(x,y)}_{\text{ Objetivo }} + \lambda (\underbrace{c-g(x,y)}_{\text{ Restricción }})
\]
Extremos relativos: $\displaystyle F_x = F_y = F_\lambda = 0$
\begin{center}
   \begin{align*}
       F_x = f_x + \lambda g_x = 0 \\ 
       F_y = f_y  \lambda g_y = 0 \\ 
       F_\lambda = c - g(x,y) = 0 \\ 
   \end{align*}
   \[
       \begin{rcases}
            \nabla f = \lambda \nabla g\\
            g(x,y) = c \\ 
       \end{rcases} \text{ Condiciones necesarias para un extremo relativo }
   \]
\end{center} 

Problema $\displaystyle w = f(x,y,z)$ sujeta a $\displaystyle g(x,y,z) = c$ 
\begin{center}
   \begin{align*}
       F(x,y,z,x) = f(x,y,z) - \underbrace{\lambda}_{\text{ Variable artificial }} (c-g(x,y,z)) \\ 
    F_x=F_y=F_z=F_x=0 \\ 
    \text{ Condiciones: } \nabla f = \lambda \nabla  g \\ 
   \end{align*}
\end{center}


\subsection{Ejercicios}
\begin{enumerate}
    \item Encuentre los extremos relativos de $\displaystyle w=x^2+y^2+z^2$ sujeta a $\displaystyle 2x+y-z=18 \}$restricción. 
        \begin{center}
            Método 1: Resolver para z
           \begin{align*}
               z = 2x+y-18 \\
           \end{align*}
           Sustituya en $\displaystyle w$  para obtener una función de 2 variables.
        %    \begin{align*}
            %    w = x^2+y^2+(2x+y-18)^2 \qq \qq \nabla w = \vec{o} \\ 
            %    \begin{matrix}
            %        w_x = 2x + 4(2x+y-18) = 10x+4y-72=0 \qq R_1\\
            %        w_y = 2y + 2(2x+y-18) = 4x+4y-36=0 \qq R_2\\

            %        R_1 - R_2: \qq 6x-36 \qq \rightarrow \qq x = 6 \\ 
            %        R_2: \qq  4y=36-4x=12 \qq \rightarrow y = 3 \\ 
            %    \end{matrix}
        %    \end{align*}
           \pregunta{Cómo se encuentra $\displaystyle z$ } 
           \begin{align*}
               z = 2(6)+3-18=-3 \\ 
               \text{ Punto crítico: } (6,3,-3) \\ 
               \text{ Prueba de la segunda derivada } \qq D(x,y) = \begin{vmatrix}
                   W_{xx} & w_{xy}  \\ 
                   W_{} 
               \end{vmatrix}
           \end{align*}
           
           Método 2: multiplicadores de Lagrange
           \begin{itemize}
               \item La grangiano $\displaystyle F = w + \lambda(c-g)$ 
           \end{itemize}
           \begin{align*}
                F(x,y,z,\lambda)=x^2+y^2+z^2+\lambda(18-2x-y+z) \\ 
                F_x = 2x+2\lambda = 0 \qq \implies \qq x = \lambda =6\\ 
                F_y = 2y-\lambda = 0 \qq \implies \qq y=\lambda/2 =3\\ 
                F_z=2z+\lambda = 0 \qq \implies \qq z = -\lambda/2 = -3 \\ 
                F_\lambda = 18-2x-y+z=0 \qq \implies \qq 2x+y-z = 18 \\ 
           \end{align*}
           Sustituya $\displaystyle x,y$ \& $\displaystyle z$  en la restricción.
           \begin{align*}
               2\lambda+\frac{\lambda }{2} + \frac{\lambda}{2} = 3\lambda = 18 \qq \implies \qq \lambda=6 \\ 
           \end{align*}
        %    El punto crítico es $\displaystyle (6,3,-3)$ $\displaystyle \lambda=6$ 
        \end{center}
    
    \item Una caja sin tapa tiene un volúmen de 32,000 $\displaystyle cm^3$ . Encuentre las dimensiones de la caja que minimizan el costo.
    \begin{tikzpicture}[node distance = 2cm, auto]
        \node[] at (4,-1) (1) {};
        \node[] at (-4,-1) (1) {};
        \node[] at (4,0) (1) {};
        \node[] at (4,0) (1) {};
    \end{tikzpicture}
    \begin{center}
       \begin{align*}
           \text{ Volúmen } \qq V=xyz = 32,000 \\ 
            \text{ Área Sup. } \qq 
            A=2zy+2zx+yx \\ 
            F = A+\lambda (c-v) = 2zy+2zx+yx+\lambda (32,000-xyz) \\ 
            F_x = 2z+y-\lambda yz = 0 \qq \implies \qq \lambda yz = y+2z \qq (1) \\ 
            F_y = 2z+x-\lambda xz= 0 \qq \implies \qq \lambda xz = x+2z \qq (2) \\ 
            F_z = 2y+2x-\lambda xy = 0 \qq \implies \qq \lambda \lambda xy = 2x+2y \qq (3)\\ 
            F_\lambda = 32,000-xyz = 0 \qq \implies \qq xyz=32,000 \qq (4) \\ 
       \end{align*}
       Dividimos entonces $\displaystyle \frac{(1)}{(2)} $:
        \begin{align*}
            \frac{(1)}{(2)}: \qquad  \frac{y}{x} = \frac{y+2z}{x+2z}  \\ 
            \cancel{yx}+2zy=\cancel{xy}+2zx \qq \implies \qq y = \frac{2zx}{2z} = \frac{x}{2} \\ 
            \therefore \qq x = y \\ 
        \end{align*}
        Dividimos también $\displaystyle \frac{(1)}{(3)} $ :
        \begin{align*}
            \frac{(x)}{(3)}: \qquad \frac{z}{x} = \frac{y+2z}{2x+2y} \\ 
            \cancel{2xz} + 2yz=xy+\cancel{2zx} \\ 
            z = \frac{xy}{2y} = \frac{x}{2} 
            \therefore \qq  y = x, \qq z = x/2 \\ 
        \end{align*}
        Se sustituye en la restricción:
        \begin{align*}
            x\cdot x\cdot \frac{x}{2} = 32,000 \\ 
            x^3 = 64\cdot1000 \\ 
            x = \sqrt[3]{64}\cdot \sqrt[3]{1,000} = 4\cdot 10 = 19 \\  
        \end{align*}
        Punto crítico: $\displaystyle x=40$ , $\displaystyle y=40$ , $\displaystyle z=20$ \newline 
        Área mínima:
        \begin{align*}
            A = 2yz+2xz+xy \\ 
            A = 2(800) + 2(800) + 1,600 \\ 
            A = 3(1,600) = 4,800 cm^2 \\ 
        \end{align*}
    \end{center}
\end{enumerate}


%%%%%%%%%%%%%%%%%%%%%%%%%%%%%%%%%%%%%%%%%%%%%%%%%%%%%%%%%%%%%%%%%%%%%%%%%%%%%%%%%%%%%%%%%%
\subsection{Aplicaciónes a la economía y negocios}
\begin{enumerate}
    \item Para sustituir una orden de 100 unidades de un producto, la empresa desea distribuir la producción entre sus dos plantas. La función de costo total es:
    \[
      C(x,y) = 0.1x^2+7x+15y+1,000
    \]
    \begin{center}
        Donde $\displaystyle x$ es la planta 1 y $\displaystyle y$ es la planta 2. \newline 
        \pregunta{Cómo debe distribuirse la producción para minimizar los costos} C(0,0)=1,000.
        \begin{align*}
            \text{ Objetivo mínimizar C(x,y) } \qq \text{ Sujeta a } \qq x+y=100 \\ 
            \text{ Lagrange: } \qq F= C+\lambda(100-x-y) \\ 
            F=0.1x^2+7x+15y+1,000+100\lambda-\lambda x-\lambda y \\ 
            F_x = 0.2x+7-\lambda = 0 \qq \implies \qq 0.2x=\lambda - 7 = 8 \qq \implies \qq x = 40 \\ 
            F_y = 15 - \lambda = 0 \qq \implies \qq \lambda=15 \\ 
            F_\lambda = 100-x-y = 0 \qq \implies \qq y = 100-x = 60 \\ 
        \end{align*}
        Punto crítico en $\displaystyle (40,60)$ $\displaystyle x=15$  
        \begin{align*}
          \text{ Costo mínimo }\qq C(x,y) = 0.1(1600) +280 + 900 + 1,000 \\ 
          C(x,y) = 2,340 \\ 
        \end{align*}        
    \end{center}
    
    \item Una empresa tiene la función de producción 
    \[
        Q(C,K) = 12L+20K-L^2-2K^2   
    \] 
    La empresa tiene un presupuesto de \$88 mil para contratar trabajadores y maquinaria. Cada trabajadory cada máquina tienen un costo de \$5 y \$8 mil, resp. \newline Encuentre la producción máxima.
    \begin{center}
        \begin{itemize}[label=\#]
            \item La restricción presupuestaria es la tangente a la curva de nivel en ese punto.
        \end{itemize}
       \begin{align*}
            \text{ Restricción:  } \qq 4L +8K = 88 \\ 
            \text{ Maximizar Q } \qq F(L,K,\lambda) \\ 
            F(L,K,\lambda) = 12L+20K-L^2-2K^2+\lambda(22-L-2K) \\ 
            F_L = 12-2L-\lambda = 0 \qq \implies \qq 2L = 12-\lambda \qq \implies \qq L = 6 - \lambda/2 \\ 
            F_K = 20-4K-2\lambda = 0 \qq \implies \qq 4K = 20-2\lambda \qq \implies \qq K = 5 - \lambda/2 \\ 
            F_\lambda = 22-L-2\lambda = 0\\
            L+2K=22 \qquad 6-\frac{\lambda}{2} +10-\lambda = 22 \\ 
            -\frac{3\lambda}{2}  = 22-16 = 6 \\ 
            \lambda=-4 \qquad L  
       \end{align*}
    \end{center}
\end{enumerate}
 % 2020-03-12

\chapter{Integrales dobles} % 2020-03-19
\input{Clases/2020-03-19.tex}

\chapter{Integrales dobles e integrales iteradas} % 2020-03-24
\section{15.2 y 15.3 Integrales dobles}
\begin{itemize}
    \item El volúmen del sólido entre las dos superficies $\displaystyle z=f\p{x,y} $ y la región $R$ en el plano $xy$ es:
        \[
          V = \iint_{D}^{} f\p{x,y} dA 
        \]
    
    \item Región rectangular $\displaystyle R = [a,b] \times [c,d]$ 
        \[
          \iint_{D}^{f\p{x,y} }dA = \int_{a}^{b} \p{\int_{c}^{d} f\p{x,y}dy  }dx = \int_{c}^{d}\int_{a}^{b}f\p{x,y} dxdy 
        \]
    
    \item En algunos problemas un orden de integración simplifica considerablemente la evaluación de la integral doble.
\end{itemize}


%----------------------------------------------------------------------------------------
\subsection{Ejercicios}
\begin{enumerate}
    \item Evalúe $\displaystyle I = \int_{0}^{2}\p{\int_{0}^{3}ye^{-xy}dy} dx$ 
        \begin{center}
           \begin{align*}
               I_1 = \int_{0}^{3}yc^{-xy}dy = \frac{-y}{x} e^{-xy} \evaluate{y=0}{y=3} + \int_{0}^{3}e^{-xy}dy \\ 
               \\
               \text{ Cambiar los ordenes de integración } \\ 
               \begin{matrix}
                   u = -xy \\ 
                   du = -ydx \\ 
               \end{matrix}
               I = \int_{0}^{3}\p{\int_{0}^{2}e^{-xy}ydx} dy \\ 
               I = \int_{0}^{3}-e^{-xy}\evaluate{x=0}{x=2} dy = \int_{0}^{3}\p{-e^{-2y}+1}dy \\ 
               I = \frac{1}{2} e^{-2y} + y \evaluate{y=0}{y=3} = \frac{1}{2} e^{-6}+3-\frac{1}{2} \\  
           \end{align*}
        \end{center}
\end{enumerate}


%%%%%%%%%%%%%%%%%%%%%%%%%%%%%%%%%%%%%%%%%%%%%%%%%%%%%%%%%%%%%%%%%%%%%%%%%%%%%%%%%%%%%%%%%%
\section{5.3 Integrales dobles en regiones generales}
\begin{itemize}
    \item Considere la región $D$ $\displaystyle G(X)\leq y \leq f(x)$ ó $\displaystyle a\leq x \leq b$.
        \[
          \iint_{D}^{}H(x,y)dA = \int_{a}^{b}\p{\int_{g(x)}^{f(x)}H(x,y)dy} dx 
        \]
\end{itemize} 



\chapter{Integrales dobles en coordenadas polares}
\input{Clases/2020-03-26.tex}

\chapter{Integrales triples}
\input{Clases/20.tex}
\section{}
Caja rectangular    
\begin{figure}[H]
    \centering
    % \includegraphics[]{Clases/figs/2020-04-16_01} 
\end{figure}
\begin{center}
   \begin{align*}
       B = \sqb{a,b} \times \sqb{c,d} \\ 
       \iiint_{B}^{}f\p{x,y,z} dV = \int_{a}^{b}\int_{c}^{d}\int_{r}^{s} f(x,y,z) dz dy dx \\  
   \end{align*}
   Se intercambia el orden de integración, esto resulta en una permutación
\end{center}

\subsection{Ejercicios}
\begin{enumerate}
    \item $\displaystyle \iiint_{B}^{}\p{xy+3z^2} dV$ $\displaystyle B = \sqb{0,2} \times \sqb{1,0} \times \sqb{0,3} $ 
        \begin{center}
           \begin{align*}
               I_a &= \int_{0}^{1}\int_{0}^{3}\int_{0}^{2}\p{x,y+3z^2} dx dz dy \\ 
               &= \int_{0}^{1}\int_{0}^{3}\p{\frac{x^2y}{2} +3z^2x\evaluate{x=0}{2} } dz dy \\ 
                &= \int_{0}^{1}\int_{0}^{3}\p{2y+6z^2} dz dy \\ 
                &= \int_{0}^{1}\p{2yz+2z^3\evaluate{z=0}{z=3}   dy} \\ 
                &= \int_{0}^{1}\p{6y+54} dy \\ 
                &= \\ 
           \end{align*}
        \end{center}
    
    \item $\displaystyle \iiint_{B}^{}e^{x+y+z}dV$ $\displaystyle B = \sqb{0,\ln\p{ 2 }  } \times \sqb{0,\ln\p{ 3 }  } \times \sqb{0,\ln\p{ 4 }  } $  
        \begin{center}
           \begin{align*}
                I_b &= \int_{0}^{2}\int_{0}^{\ln\p{ 3 }  }\int_{0}^{\ln\p{ 4 }  } e^{x+y+z} dz dy dx \\ 
                I_b &= \int_{0}^{\ln\p{ 2 }  }\int_{0}^{\ln\p{ 3 }  }e^{x+y} \\ 
           \end{align*}
        \end{center}
\end{enumerate}


%%%%%%%%%%%%%%%%%%%%%%%%%%%%%%%%%%%%%%%%%%%%%%%%%%%%%%%%%%%%%%%%%%%%%%%%%%%%%%%%%%%%%%%%%%
\section{Integrales triples sobre unu sólido general}
\[
  u_1(x,y) \leq z \leq u_2(x,y) 
\]
D es la región de proyección del sólido $E$ sobre el plano $xy$.
\[
    \iiint_{E}^{}f(x,y,z)dV = \iint_{D}^{}\p{\int_{u_1(x,y)}^{u_2(x,y)} f(x,y,z)dz } dA  
\]
\[
  (x,y) \in D
\]

D como una región tipo I, tipo II o en polares.
\begin{center}
   \begin{align*}
       a \leq x \leq b \qq \qq a_1(x) \leq y \leq a_2(x) \\ 
    %    \iiint_{E}^{}f\; dV &= \iint_{D}^{}\p{int
    %    }  \\ 
   \end{align*}
\end{center}


\subsection{Sólido tipo II}
\begin{center}
   \begin{align*}
       u_1(y,z) \leq x \leq u_2(y,z) \\ 
   \end{align*}
   \begin{figure}[H]
       \centering
    %    \includegraphics[]{Clases/figs/} 
   \end{figure}
\end{center}


\end{document}
