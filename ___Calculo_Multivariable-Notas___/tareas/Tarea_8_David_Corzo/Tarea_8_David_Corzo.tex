\documentclass{article}
\usepackage[margin = 1in]{geometry}\usepackage{pdfpages}
\usepackage[spanish]{babel}
\usepackage{amsmath}
\usepackage{amsthm}
\usepackage[utf8]{inputenc}
\usepackage{titlesec}
\usepackage{xpatch}
\usepackage{fancyhdr}
\usepackage{tikz}
\usepackage{davidcorzo}

\title{Tarea \#8 - David Corzo}
\date{2020 March 04, 09:43PM}
\author{David Gabriel Corzo Mcmath}

\begin{document}
\maketitle
%%%%%%%%%%%%%%%%%%%%%%%%%%%%%%%%%%%%%%%%%%%%%%%%%%%%%%%%%%%%%%%%%%%%%%%%%%%%%%%%%%%%%%%%%%%%%%%%%%%%%%%%%%%%%%%%%%%%%%%%%%%%%%%%%%%%%%%%%%%%%%

\section{Encuentre $\displaystyle \dervpar{y}{x}$}
%----------------------------------------------------------------------------------------
\subsection{$\displaystyle y\tan^{-1}(x)=x\sin^{-1}(y)+x^2y^2$}  

%----------------------------------------------------------------------------------------
\subsection{$\displaystyle yx+x^3\ln(y)=\left(x^2+y^2\right)^2$}

%%%%%%%%%%%%%%%%%%%%%%%%%%%%%%%%%%%%%%%%%%%%%%%%%%%%%%%%%%%%%%%%%%%%%%%%%%%%%%%%%%%%%%%%%%%%%%%%%%%%%%%%%%%%%%%%%%%%%%%%%%%%%%%%%%%%%%%%%%%%%%

\section{Encuentre las derivadas parciales para las sigs. funciones implícitas}

%----------------------------------------------------------------------------------------
\subsection{$\displaystyle \sin(xy)+\cos(yz)=\cot(zx)$}

%----------------------------------------------------------------------------------------
\subsection{$\displaystyle \sqrt{x^2y^2+y^2z^2}=\frac{1}{x-2y-3z} $}




%%%%%%%%%%%%%%%%%%%%%%%%%%%%%%%%%%%%%%%%%%%%%%%%%%%%%%%%%%%%%%%%%%%%%%%%%%%%%%%%%%%%%%%%%%
\section{Encuentre la ecuación del plano tangente a la superficie dada en el punto específicado.}


%----------------------------------------------------------------------------------------
\subsection{$\displaystyle z = \frac{2x+3}{4y+1} $, $(0,0,0)$}

%----------------------------------------------------------------------------------------
\subsection{$\displaystyle z=\sec(xy^2)$, $\left(\frac{\pi }{3},1,2\right)$}



%%%%%%%%%%%%%%%%%%%%%%%%%%%%%%%%%%%%%%%%%%%%%%%%%%%%%%%%%%%%%%%%%%%%%%%%%%%%%%%%%%%%%%%%%%
\section{Encuentre la aproximación lineal $L(x,y)$ de la función en el punto indicado.}


%----------------------------------------------------------------------------------------
\subsection{$\displaystyle z=\frac{x}{x+y} $, $(4,-2)$}

%----------------------------------------------------------------------------------------
\subsection{$\displaystyle z=e^{-xy}\sin(y)$, $\left(\frac{\pi }{2},0\right)$}



%%%%%%%%%%%%%%%%%%%%%%%%%%%%%%%%%%%%%%%%%%%%%%%%%%%%%%%%%%%%%%%%%%%%%%%%%%%%%%%%%%%%%%%%%%
\section{Encuentre las ecuaciones paramétricas de las rectas tangente a superficie $z=f(x,y)$ en el punto indicado. $L_1$ es la tangente en la dirección de $x$ y $L_2$ es la tangente en la dirección de $y$.}


%----------------------------------------------------------------------------------------
\subsection{$\displaystyle z=\sqrt{x^2+y^2}$, $(3,4 )$}

%----------------------------------------------------------------------------------------
\subsection{$\displaystyle z=2\sin^2(3x-2y)+4\cos^2(x+y)$, $\left(\frac{\pi }{4}, \frac{\pi }{4}\right)$}
Recuerde encontrar la función vectorial para encontrar la recta tangente a la superficie $z=f(x,y)$.


\end{document}

