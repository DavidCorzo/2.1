\documentclass{book}
\title{Cost Analysis}
\author{David Gabriel Corzo Mcmath}
\date{2020-Jan-08 10:00:30}
%%%%%%%%%%%%%%%%%%%%%%%%%%%%%%%%%%%%%%%%%%%%%%%%%%%%%%%%%%%%%%%%%%%%%%%%%%%%%%%%%%%%%%%%%%%%%%%%%%%%%%%%%%%%%%%%%%%%%%%%%%%%%%%%%%%%%%%%%%%%%%%
\usepackage[margin = 1in]{geometry}
\usepackage{graphicx}
\usepackage{fontenc}
\usepackage{pdfpages}
\usepackage[spanish]{babel}
\usepackage{amsmath}
\usepackage{amsthm}
\usepackage[utf8]{inputenc}
\usepackage{enumitem}
\usepackage{mathtools}
\usepackage{import}
\usepackage{xifthen}
\usepackage{pdfpages}
\usepackage{transparent}
\usepackage{color}
\usepackage{fancyhdr}
\usepackage{lipsum}
\usepackage{sectsty}
\usepackage{titlesec}
\usepackage{calc}
\usepackage{lmodern}
\usepackage{xpatch}
\usepackage{blindtext}
\usepackage{bookmark}
\usepackage{fancyhdr}
\usepackage{xcolor}
\usepackage{tikz}
\usepackage{blindtext}
\usepackage{hyperref}
\usepackage{listing}
\usepackage{spverbatim}
\usepackage{fancyvrb}
\usepackage{fvextra}
\usepackage{amssymb}
\usepackage{pifont}
\usepackage{longtable}
\usepackage{davidcorzo}
\usetikzlibrary{arrows,shapes}

% \newcommand{\estadoderesultados}[3]{
%     \begin{center}
%        \begin{tabular}{ | p{8cm} | p{5cm} | }
%            \hline
%                 Ventas &  #1   \\
%                 Costo de ventas & #2 \\ 
%            \hline
%                 Utilidad marginal &  \\
%                 Costos fijos & #3 \\ 
%            \hline
%                 Utilidad de operación &  \\ 
%                 ISR & \\ 
%            \hline
%                 Utilidad neta & \\ 
%             \hline
%        \end{tabular}
%     \end{center}
% }
%%%%%%%%%%%%%%%%%%%%%%%%%%%%%%%%%%%%%%%%%%%%%%%%%%%%%%%%%%%%%%%%%%%%%%%%%%%%%%%%%%%%%%%%%%%%%%%%%%%%%%%%%%%%%%%%%%%%%%%%%%%%%%%%%%%%%%%%%%%%%%%
% Begin block styles
\tikzblockdefinitions
% End block styles 
%%%%%%%%%%%%%%%%%%%%%%%%%%%%%%%%%%%%%%%%%%%%%%%%%%%%%%%%%%%%%%%%%%%%%%%%%%%%%%%%%%%%%%%%%%


\begin{document}
\maketitle
\tableofcontents

\chapter{Segunda Clase - 2020-01-08}
\section{Clase introductoria}
\begin{itemize}
    \item Hay dos tipos de datos en estadística; 
        \begin{enumerate}
            \item Cualitativo: el cualitativo es por 
            \item Cuantitativo: 
        \end{enumerate}
    
    \item Distribución de frecuencias: nos dice qué tan frecuente es la distribución de los datos en un set.
\end{itemize}


\chapter{Tercera clase - 2020-01-13}
\section{¿Qué es el marketing?}
\begin{itemize}
    \item Todo aquello que una empresa u organización realiza con el objeto de identificar, conocer, cultivar y satisfacer el mercado que sirve y ser retribuido por él de manera sostenida.
    \item \emph{\textbf{Ejemplo: }Problema acerca de tarjeta de crédito. Las marcas problemáticas afectan su posicionamiento.}
    \item Se necesita hacer énfasis que el target tiene que ser investigado, conocido, entendido a profundidad.
    \item \emph{\textbf{Ejemplo: }Krispy Cream: la noción de ``tal empresa va a hacer que otra quibre''}.
    \item Analizar: \textbf{Nos preguntamos:} ¿por qué compramos donas? Es fácil saber por qué compran, \textbf{lo difícil es saber por NO compran.}
    \item Historia de McDonald's: 
        \begin{itemize}
            \item Cuando se fundó la franquicia, se inició la idea de la cajita feliz con juguete; se pudieron deducir estas conclusiones por que la Sr. Cofiño estaba en el restaurante dando la comida; ``el peor lugar para tomar decisiones es el escritorio, uno tiene que estar en la jugada''.
            \item Ahora Mc está teniendo muchos detractores ya que la gente valora más los saludable por ejemplo. 
        \end{itemize}    
    \item Ejemplo de madre e hija:
        \begin{itemize}
            \item Aprender a ver la escena.
            \item Aprender a absorber todo el contexto.
        \end{itemize}
\end{itemize}

\section{Ejemplo de señor leyendo:}
\begin{itemize}
    \item El ambiente se adecua al target, sería mal si se propone poner música a alto volúmen.
    \item Todos nosotros vamos a necesitar alguna cosa, estas son oportunidades de negocios.
    \item Se analiza que el target necesita un ambiente para poder juntarse a leer, \textbf{identificar las necesidades del target}; resumir el target a una necesidad, conocer los intereses del target.  
\end{itemize}

\section{Ejemplo de la moto:}
\begin{itemize}
    \item \textbf{Nos preguntamos:} ¿Qué pensamos?
    \item \textbf{Nos preguntamos:} ¿Qué necesidades tendrá esta familia? 
    \item \textbf{Nos preguntamos:} ¿Será la cultura?
    \item Hay que entender los targets, por qué quieren usar una moto con 3 pasajeros y el piloto.
    \item Aspecto relevante: tener en cuenta que una de las cosas más importantes es investigar el target.
    \item \textbf{Nos preguntamos:} ¿Será porque se vuelve económicamente imposible tener carro por la distancia entre la casa y el trabajo? \emph{\textbf{La respuesta a esta problemática es: }tenemos que entender cuál es el problema.}
\end{itemize}

%%%%%%%%%%%%%%%%%%%%%%%%%%%%%%%%%%%%%%%%%%%%%%%%%%%%%%%%%%%%%%%%%%%%%%%%%%%%%%%%%%%%%%%%%%%%%%%%
\section{Ejemplos del marketing (buenos / malos)}
\subsection{Anécdota colgate:}
\begin{itemize}
    \item Insight: Darnos cuenta que había un niño que llevaba cuatro veces al día por cositas, esto no se hubiera podido analizar en el escritorio.
    \item Marketing es un poco como trabajo de detective, como trabajo de investigador.
    \item Hay que conocer el mercado; a partir del ejemplo de la moto se puede intentar coordinar el mercado para ejercer funciones empresariales para resolver esta problemática.
\end{itemize}

\subsection{Ejemplo de maratón:}
\begin{itemize}
    \item Especulamos que es por una buena causa, les interesa su salud, les interesa el reto, les interesa para ver que podemos hacer.
    \item Ejemplo en Chile: el bicicross;  así van y vienen las tendencias.
\end{itemize}

\subsection{Ejemplo de Donald Trump:}
\begin{itemize}
    \item Él ganó su candidatura en EEUU por que supo identificar su target.
    \item En política se utiliza mucho esto de identificar su target.
\end{itemize}

\subsection{Ejemplo de Madonna:}
\begin{itemize}
    \item Madonna se ha mantenido por décadas, ella como artista genera más dinero que las grandes marcas en GT.
    \item Otros ejemplos como Shakira, generan mucho dinero.
    \item Walkman: es el abuelito del iPod, identificaron que los adolecentes llevaban su boom box e innovó a crear el walkman, donde se guardaba música, ahora no tenían que cargar la grabadora con sus baterías, el walkman ue por mucho tiempo el aparato para escuchar música. Esto un ejemplo clásico de identificar las necesidades del target.
    \item Los targets van cambiando sus intereses a través del tiempo.
\end{itemize}

\subsection{Ejemplo DudeWipes:}
\begin{itemize}
    \item Muchas veces con que se identifique la necesidad del target si no que diseñar un producto con el target en mente. Agarraron el mismo producto que era para bebés solo que lo orientaron para hombres.
    \item Las toallas humedas las modificaron un poco, como no tienen olor, son más gruesas, son más grandes, etcétera; este es un producto \textbf{muy bien enfocado} desde el color, hasta el logos, hasta el nombre de la marca.
    \item Todo el modelo de negocios está orientado al target.
    \item Se pueden introducir a nuevos sectores del mercado; posiblemente no podía usar toallitas de bebé se introduce estas nuevas toallitas, lugares como barbería.
    \item Otro ejemplo es el shampoo Ego; Se enfocó en el target de los hombres; se posiciona de tal manera que resuenan mucho mejor con el target.
    \item Video promoción de DudeWipes - \url{https://www.youtube.com/watch?v=4jMgM0pKEUw}
    \item Dude wipes se separó de los otros productos casi identicos.
    \item Otro ejemplo puntual es - \url{https://www.youtube.com/watch?v=vdWHucg900U}
    \item Carreer builder es un sitio para conseguir trabajo, en el comercial sale el mismo actor que en marketing es el target que trata de conseguir que los targets de la empresa visiten el sitio; Este video tiene ese problema, implica que todos los demas ajenos al target es un mono.
    \item Regla básica en marketing es \textbf{No hablar mal de nadie}.
    \item Ejemplo de Coca~Cola y pepsi; su estrategia de marketing es riesgosa por sacar anuncios en contra de pepsi.
    \item Ejemplo de Beneton: El ejemplo de un Sueco con un negro de África.
\end{itemize}

\subsection{\textbf{Nos preguntamos:} ¿Cómo evadir que las personas no encuentren cosas ajenas al marketing a la intención?}
\begin{itemize}
    \item Una solución es leer de todo. 
    \item Entender si ya se había intentado y qué tal funcionó.
    \item Que en tu mesa de trabajo hayan expertos; experimentar.
\end{itemize}

\subsection{Analizar: El debate presidencial de Donald Trump}
\begin{itemize}
    \item El audio que salió acerca de la charla en el bus.
\end{itemize}


\chapter{Clase - 2020-01-20}
\section{Punto de equilibrio}
\begin{itemize}
    \item Sirve para relacionar el costo del volumen de lo que puedo producir.
    \item Para eso tengo que saber los costos variables, fijos.
    \item El punto de equilibrio son donde yo no gano ni pierdo. Para calcular esto tengo que sumar todos los costos y dividirlo dentro de lo que le gano. A este precio es el precio de equilibrio.
        \[
          \frac{\sum_{}^{}\text{Todos costos fijos}}{\text{Precio de venta - Costos varibles unitarios (utilidad marginal)}} = \text{Unidades de equilibrio }
        \]
        
    \item Para calcular las unidades que debo vender contando la utilidad que quiero sumo la cantidad de utilidad deseada a los costo y divido por el margen de contribución. Utilidad bruta: 
        \[
        \frac{\sum_{}^{}(\text{costos}) + \text{ Utilidad deseada} }{\text{Margen de contribución}} = \text{Cantidad de unidades que debo vender}
        \]
    
    \item Margen de contribución: \[
      \text{  Margen de contribución  } = \frac{\text{  Utilidad marginal } \times 100 }{\text{  Precio de venta  } } 
    \]
    \item 
        \begin{align*}
            \text{  Utilidad marginal  } = \text{  Precio  } - \text{  Costos variables  } \\ 
        \end{align*}
    \item Entonces se propone el objetivo de vender tal cantidad de unidades para sacar x cantidad de unidades, eso por supuesto no depende de mi si no del mercado y demanda.
    \item \emph{\textbf{Interesante:} Al hacer un estudio de mercado se deduce cuántas unidades me dejará vender el mercado como máximo en un periodo de tiempo.}
    \item \emph{\textbf{Recordar lo siguiente: }}
        \begin{center}
           \begin{tabular}{ | p{5cm} | p{5cm} | }
               \hline
                    Ventas & x    \\
                    (-) Costos de venta & x \\ 
               \hline
                    Utilidad bruta & 76,000 \\ 
                    (-) Costos fijos & 56,000 \\ 
                \hline
                    Utilidad en operación & 20,000 \\ 
                    (-) ISR & 5,000 \\ 
                    Utilidad neta & 15,000 \\ 
                \hline
           \end{tabular}
        \end{center}
    
    \item \emph{\textbf{Ejemplo: }Galletas de desnutrición, el mercado está en el interior, \textbf{Nos preguntamos:} ¿puedo hacer una galleta a Q1.00?}
    \item Siempre hay que hacer un estudio de mercado para evaluar la entrada al negocio y al mercado.
\end{itemize}

%%%%%%%%%%%%%%%%%%%%%%%%%%%%%%%%%%%%%%%%%%%%%%%%%%%%%%%%%%%%%%%%%%%%%%%%%%%%%%%%%%%%%%%%%%%%%%%%

\section{Costo volumen por unidad}
\begin{itemize}
    \item Utilidad:  
    \[
      \text{Utilidad } = \text{Ingreso total} - \text{Costo variable total} - \text{Costo fijo total}
    \]
    
    \item Unidades para lograr objetivo: 
        \[
         \text{Unidades para lograr objetivo} = \frac{\text{Utilidad objetivo} + \text{Costo fijo total}}{\text{Margen de contribución por x unidad}} 
        \]
\end{itemize}

%%%%%%%%%%%%%%%%%%%%%%%%%%%%%%%%%%%%%%%%%%%%%%%%%%%%%%%%%%%%%%%%%%%%%%%%%%%%%%%%%%%%%%%%%%%%%%%%
\section{ISR - impuesto sobre la renta}
\begin{itemize}
    \item Para calcular cuanto tengo que ganar para tener x cantidad de utilidad:
        \[
          \frac{\text{Utilidad esperada}}{(1-\%)} 
        \]
    
    \item 
        \[
          \text{Unidades para lograr objetivo} = \frac{\frac{\text{Utilidad esperada}}{(1-0.25)} + \text{Costo fijos totales}}{\text{Margen de contribución por x unidad}} 
        \]
\end{itemize}

\section{Análisos de riesgo}
\begin{itemize}
    \item Margen de seguridad: es el porcentaje máximo en que las ventas esperadas pueden disminuir y aún generar una utilidad.
        \[
          \text{Margen de seguridad } = \frac{\text{Ventas esperadas} - \text{Precio de equilibrio}}{\text{Ventas esperadas}} 
        \]
    
    \item Se considera alto a partir de 25\%. 
\end{itemize}

%%%%%%%%%%%%%%%%%%%%%%%%%%%%%%%%%%%%%%%%%%%%%%%%%%%%%%%%%%%%%%%%%%%%%%%%%%%%%%%%%%%%%%%%%%%%%%%%

\section{Comparación de procesos de producción}
\begin{enumerate}
    \item Determinar el punto de equilibrio para cada proceso.
    \item Determinar cuales son nuestras demás variables.
    \item Evaluar el margen de seguridad.
\end{enumerate}
Clave: el mejor es el que permita encontrar el precio de equilibrio más rápido.


\chapter{Clase - 2020-01-29}
\section{Dudas}
\begin{itemize}
    \item \textbf{Nos preguntamos:} ¿cuál es la diferencia entre población y muestra?
        \begin{itemize}
            \item Las muestras son parciales, la población es el total.
            \item Población es el concepto de todo lo que existe, existe y va a existir en algún predeterminado lugar.
        \end{itemize}
\end{itemize}

%%%%%%%%%%%%%%%%%%%%%%%%%%%%%%%%%%%%%%%%%%%%%%%%%%%%%%%%%%%%%%%%%%%%%%%%%%%%%%%%%%%%%%%%%%%%%%%%

\section{Medidas de localización}
\begin{itemize}
    \item Las medidas de localización dan una idea de lo que está pasando en un set de observaciones.
\end{itemize}

%%%%%%%%%%%%%%%%%%%%%%%%%%%%%%%%%%%%%%%%%%%%%%%%%%%%%%%%%%%%%%%%%%%%%%%%%%%%%%%%%%%%%%%%%%%%%%%%

\section{Medidas de variabilidad}

\subsection{Rango}
Problemas de la media \& solución es es la introducción del \textbf{rango}.
\begin{figure}[htbp]
    \centering
    \includegraphics[width=6cm]{./../__Imagenes__/2020-01-16-EstadisticaI.jpeg}
    \caption{Misma media, diferente rango de datos}
    \label{}
\end{figure} 
\begin{itemize}
    \item Los datos están confusos ya que a pesar de tener la misma media los datos varían.
    \item Entonces se introduce el rango que se calcula como:
        \[
            \text{Rango} = \text{Máximo} + \text{Mín}
        \]
\end{itemize}

%%%%%%%%%%%%%%%%%%%%%%%%%%%%%%%%%%%%%%%%%%%%%%%%%%%%%%%%%%%%%%%%%%%%%%%%%%%%%%%%%%%%%%%%%%%%%%%%

\subsubsection{Rango intercuartílico}
La diferencia entre el cuartil tres y el cuartil uno.
\[
   R_{\text{Intercuartílico}} = Q_{3} - Q_{1}
\]
Entre $Q_{3}$ y el $Q_{1}$, \textbf{Nos preguntamos:} ¿qué diferencias hay? es el 50\% de todos los que se parecen entre sí.

%%%%%%%%%%%%%%%%%%%%%%%%%%%%%%%%%%%%%%%%%%%%%%%%%%%%%%%%%%%%%%%%%%%%%%%%%%%%%%%%%%%%%%%%%%%%%%%%
\subsection{Varianza muestral:}
\begin{itemize}
    \item \emph{\textbf{Definición de ``Varianza Muestral":} cuánto varian en promedio los datos respecto a la media.}
    \item Se denota por una $S^2$
    \item \[
        S^2 = \frac{\sum_{i=1}^{n}(X_{i}-\bar{x})}{n-1} 
      \]     
    
    \item Este en el ejemplo esta expresado en centímetros$^2$. 
\end{itemize}

%%%%%%%%%%%%%%%%%%%%%%%%%%%%%%%%%%%%%%%%%%%%%%%%%%%%%%%%%%%%%%%%%%%%%%%%%%%%%%%%%%%%%%%%%%%%%%%%

\subsection{Desviación estándar}
\begin{itemize}
    \item \emph{\textbf{Definición de ``Desviación estándar":} cuánto varían los datos respecto a la media.}
    \item Es la raíz cuadrada de la varianza, se denota por solo $S$.
    \item \[
      \sqrt[]{S^2}  = S = \sqrt[]{ \frac{\sum_{i=1}^{n}(X_{i}-\bar{x})}{n-1} }
    \]
    
    \item En el ejemplo que tenemos está expresado en centímetros.
    \item Desviación estándar significa que en promedio los datos difieren respecto a la media.
    \item Cuando hay una desviación estándar está alta nos dice qué tanto se parecen los datos, qué tan variable es el grupo de datos.
    \item DE: 15 respecto a otra de DV:7, dice que hay más variación en la desviación estándar de 15.
\end{itemize}

%%%%%%%%%%%%%%%%%%%%%%%%%%%%%%%%%%%%%%%%%%%%%%%%%%%%%%%%%%%%%%%%%%%%%%%%%%%%%%%%%%%%%%%%%%%%%%%%
\section{Excel}
\begin{itemize}
    \item Para fijar una celda usar la letra de la columna encerrada por signos de dólar, así: \$E\$56
    \item Para desviación estándar usar fórmula: =DESVEST.M(<TodosLosDatosOriginales>) .
\end{itemize}

%%%%%%%%%%%%%%%%%%%%%%%%%%%%%%%%%%%%%%%%%%%%%%%%%%%%%%%%%%%%%%%%%%%%%%%%%%%%%%%%%%%%%%%%%%%%%%%%

\section{Ejemplo}
\begin{figure}[htbp]
    \centering
    \includegraphics[width=8cm]{../__Imagenes__/2020-01-16-01-ESTADISTICA.jpg}
    \caption{Niveles de IQ entre hombres y mujeres}
    \label{}
\end{figure} 

\begin{itemize}
    \item Las deducciones son que los hombres tienen una desviación estándar mayor.
\end{itemize}


\chapter{Clase - 2020-02-12}
\section{\textbf{Nos preguntamos:} ¿Hay en el hombre algo que se le pueda llamar naturaleza?}
\begin{itemize}
    \item \begin{center}
       \begin{tabular}{ | p{8cm} | p{8cm} | }
           \hline
               Si & No     \\
           \hline
                \underline{Instinto} = Naturaleza & \\ 
                \begin{itemize}
                    \item Reacción
                    \item Algo con lo que nacemos 
                    \item Impulso 
                    \item Supervivencia 
                \end{itemize} & 
                \begin{itemize}
                    \item Conjunto de incentivos 
                    \item Ser vivo diferentes formas de actuar o sentir
                \end{itemize} \\ 
            \hline
                \multicolumn{2}{c}{$\overbrace{\text{Intinto = Naturaleza}}^{\text{Ser vivo}}$} \\
            \hline
                \multicolumn{2}{c}{Para los griegos es aquello que conozco que puedo conocer pero no puedo modificar, la naturaleza es aquello que tiene orden sin que el humano lo haga. }
       \end{tabular}
    \end{center}
    
    \item \textbf{Nos preguntamos:} ¿Hay cosas en el hombre que puedan influenciarlo inconscientemente?
\end{itemize}

%%%%%%%%%%%%%%%%%%%%%%%%%%%%%%%%%%%%%%%%%%%%%%%%%%%%%%%%%%%%%%%%%%%%%%%%%%%%%%%%%%%%%%%%%%%%%%%%

\subsection{Video \& Análisis}
\begin{itemize}
    \item \url{https://www.youtube.com/watch?v=LjCzPp-MK48} Flores
    \item \url{https://www.youtube.com/watch?v=PyMBI0ee1qs} Caballo
    \item \url{https://www.youtube.com/watch?v=zpARvJoO4aY} Nacimiento de orca en SeaWorld
    \item \url{https://www.youtube.com/watch?v=h82ltr84_Yg} Nacimiento de bebé humano
    \item Franco de Vita: Si tu no estas ft. Amaia Montero.
    \item Franco de Vita: Será
\end{itemize}

%%%%%%%%%%%%%%%%%%%%%%%%%%%%%%%%%%%%%%%%%%%%%%%%%%%%%%%%%%%%%%%%%%%%%%%%%%%%%%%%%%%%%%%%%%%%%%%%

\subsection{La felicidad}
\begin{itemize}
    \item $\underbrace{\text{Alma}}_{\text{De par en par}}$ $\rightarrow$ Nada $\rightarrow$ Esperanza 
    \item \emph{\textbf{Interesante:} La friendzone ``La felicidad depende de la decisión de otra persona''.}
    \item Abrir el alma es lo más difícil del ser humano. 
\end{itemize}

\section{Continuación de clase}
\begin{itemize}
    \item Naturaleza:
        \begin{enumerate}
            \item Ciclo de vida 
            \item Crecimiento 
            \item Reproducción 
            \item Vida  
            \item Involuntario 
        \end{enumerate}
    
    \item La naturaleza, y análisis: 
        \begin{itemize}
            \item Esto lo estudia la biología
            \item Los perros son cazadores por \emph{Naturaleza}.
        \end{itemize}
    
    \item \textbf{Nos preguntamos:} ¿Entonces hay naturaleza en el humano?
        \begin{itemize}
            \item Hay un orden dado, que se recibe, no se crea.
            \item Hay comportamiento humano que es natural.
            \item Tenemos un temperamento diferente.
            \item Lo dado de la psyche es la psicología.
            \item Si algo tiene alma $\rightarrow$ Está vivo.
            \item \textbf{Nos preguntamos:} ¿Cualquier comportamiento raro es por que está loco? 
                \begin{itemize}
                    \item \emph{Citación:``no se puede hacer nada si así es <tal persona>."} Acudiendo a la locura para explicar la conducta rara.
                    \item Nos afecta la \emph{natura}, hay altas tazas de depresión en países donde no sale el sol.
                    \item \emph{\textbf{Ejemplo: }En Holanda cuando sale el sol todos dejan lo que están haciendo y salen a recibir el sol, alegra el alma. }
                \end{itemize}
        \end{itemize}
    
    \item \emph{\textbf{Respuesta:} \textbf{Nos preguntamos:} ¿Hay algo en el hombre que se cataloga como naturaleza?} Sí hay una parte natural del hombre.
    \item \textbf{Nos preguntamos:} ¿Cómo explicamos la depresión?  
        \begin{itemize}
            \item Es enfermedad de la psyche, cómo estamos dominados por fuerzas inconscientes.
        \end{itemize}
\end{itemize}



\chapter{Clase de repaso - 2020-02-24}
\section{Otras medidas de localización}
\begin{itemize}
    \item Media ponderada: 
        \begin{itemize}
            \item \[
              \bar{x} = \frac{\sum_{i=1}^{n}W_i\cdotX_i} {\sum_{i=1}^{n}W_i} 
            \]
            \begin{itemize}
                \item Donde $W$ es ``weight'' o el peso que le vamos a delegar a cada clase.
                \item Toma en cuenta del peso de lo que estamos midiendo y a base de eso resulta la media ponderada. La media aritmética no toma en cuenta el peso.
                \item $X_i$ es la observación.
            \end{itemize}            
            \item \emph{\textbf{Definición de ``media ponderada":} }
        \end{itemize}
    \item Media a partir de datos agrupados: 
        \begin{itemize}
            \item \[
              \bar{x} = \frac{\sum_{i=1}^{n}f_iM_i}{n} 
            \]
            
            \item La media a partir de datos agrupados se tiene un resultado con un nivel moderado de incertidumbre. 
            \item \emph{Citación:``Cuando agrupamos datos perdemos información."}         
            \item La media aritmética no va a ser exactamente igual que la media a partir de datos agrupados.   
        \end{itemize}
    
    \item Varianza a partir de datos agrupados: 
        \begin{itemize}
            \item \[
              S^2= \frac{\sum_{}^{}f_i(M_i-\bar{x})^2}{n-1} 
            \]
                \begin{itemize}
                    \item $M_i$ es la media de las clases, si tengo un clase de 80-90 $M_i = \frac{(80+90)}{2}$.
                    \item  
                \end{itemize}
            \item 
        \end{itemize}
\end{itemize}

\section{Excel steps}
\begin{itemize}
    \item To show empty data fields:
        \begin{itemize}[label=$\downarrow$]
           \item right click en row labels (random) 
           \item Configuración de campo ...  
           \item Diseño e impresión 
           \item Mostrar elementos sin datos (cheque)
        \end{itemize}
\end{itemize}


\chapter{Clase - 2020-03-04 Post Parcial}
\section{Array}
\begin{itemize}
    \item Collección de datos del mismo tipo.
    \item En memoria se hace de manera consecutiva.
    \item Con arraylists no se almacena de manera consecutiva.
    \item Elementos:
        \begin{itemize}
            \item Data type: Any valid data type.
            \item Name: How we'll identify.
            \item Size: Maximum numbers of values to hold.
        \end{itemize}
    
    \item En el procesador:
        \begin{itemize}
            \item Tiene una eficiencia mayor.
            \item Como se guarda consecutivamente solo se tiene que buscar una vez en memoria por ser consecutiva.
            \item Por eso se necesita declarar con cuántos elementos inicializar el array.
        \end{itemize}
    
    \item Tener presente cuánto ocupa en memoria cada tipo de elementos.
    \item La función .length recorre todo el array y retorna el número de elementos que tiene.
\end{itemize}


%%%%%%%%%%%%%%%%%%%%%%%%%%%%%%%%%%%%%%%%%%%%%%%%%%%%%%%%%%%%%%%%%%%%%%%%%%%%%%%%%%%%%%%%%%%%%%%%

\subsection{Storing values in the array}
\begin{itemize}
    \item Cuando ingresamos información a un array lo hacemos por orden de índices.
    \item Accesar: por el índice [i].
        \begin{itemize}
            \item System.Out.println(A[i]); // Prints the element stored in that index in memory.
        \end{itemize}
    \item Operations on arrays:
        \begin{itemize}
            \item Deleting elements: cuando se elimina un elemento, se tiene que hacer un corrimiento de las cantidades correspondientes; si por ejemplo tiene que un array de n elementos, se desplaza indices $n$ por los $n + 1$.
            \item A[i] = A[i + 1];
        \end{itemize}
    
    \item Merging arrays:
        \begin{itemize}
            \item array1;
            \item array2; 
            \item merged = array1 + array2; 
        \end{itemize}
    
    \item Passing arrays to functions:
        \begin{enumerate}
            \item Passing entire arrays:
                \begin{itemize}
                    \item Le pasa todo el arreglo, con todo y la posición de memoria.
                \end{itemize}

            \item Passing individual elements: esto permite trabajar el contenido del array por aparte, creo una copia y trabajo con la información, entonces no modifico el array original, si paso la dirección de memoria del índice o de la posición de memoria si voy a alterar el arreglo original.
                \begin{itemize}
                    \item Passing data values 
                    \item Passing addresses 
                \end{itemize}
        \end{enumerate}
\end{itemize}


%%%%%%%%%%%%%%%%%%%%%%%%%%%%%%%%%%%%%%%%%%%%%%%%%%%%%%%%%%%%%%%%%%%%%%%%%%%%%%%%%%%%%%%%%%%%%%%%
\subsection{Caché}
\begin{enumerate}
    \item Almacenamiento 
    \item Temporal 
    \item Fácil Acceso
\end{enumerate}

\subsubsection{Beneficios}
\begin{itemize}
    \item Tener más efectividad.
    \item Reduce el impacto cuando tenemos que interactuar  .
    \item Está enfocado en \textbf{leer}.
\end{itemize}



%%%%%%%%%%%%%%%%%%%%%%%%%%%%%%%%%%%%%%%%%%%%%%%%%%%%%%%%%%%%%%%%%%%%%%%%%%%%%%%%%%%%%%%%%%%%%%%%

\section{Two-Dimensional Array}
\begin{itemize}
    \item Almacena matrices en memoria, tiene varias implicaciones:
        \begin{itemize}
            \item Tener contabilidad de dos indices.
            \item Agregar una línea implica agregar n cantidad de el array anidado.
        \end{itemize}
    
    \item Se almacena de la misma manera que un array de una dimensión.
    \item Cuando se almacena un array bidimensional, se hacen operaciones matemáticas para ubicar en memoria los indices.
    \item Inicialización:
        \begin{itemize}
            \item Se declaran valores al momento de declarar.
            \item Se declaran valores parametrados.
        \end{itemize}
\end{itemize}


%%%%%%%%%%%%%%%%%%%%%%%%%%%%%%%%%%%%%%%%%%%%%%%%%%%%%%%%%%%%%%%%%%%%%%%%%%%%%%%%%%%%%%%%%%%%%%%%

\section{Multi-Dimensional array}
\begin{itemize}
    \item Multidimensional array:
    \begin{itemize}
        \item \emph{\textbf{Ejemplo: }Graficador de imágenes}
            \begin{align*}
                grafo \overbrace{[t_1]}^{\text{  Temporalidad  }}\underbrace{[x_1][y_1]}_{\text{  Cuadrícula  }} & = \underbrace{\text{  Color  }}_{255}\\ 
            \end{align*}
        
        \item Entonces el tiempo puede tener manifestación en un gif, las imágenes que resultan se pintan en la pantalla. Mientras más pixeles tenemos mejor la calidad.
            \begin{figure}[htbp]
                \centering
                \includegraphics[width=12cm]{./../__Imagenes__/2020-01-27-ED-01.jpeg}
                \includegraphics[width=12cm]{./../__Imagenes__/2020-01-27-ED-02.jpeg}
                \caption{Ejemplo de multidimensional arrays \& el black box exercise}
                \label{}
            \end{figure}
    \end{itemize}
    
    \item Los arrays multidimensionales podemos combinarlos con objetos : 
        \begin{itemize}
            \item Nos va permitir más que un solo valor.
            \item Se puede almacenar objetos en el array, por lo general se va a tener que tener el mismo tamaño.
            \item En Java se puede declarar un array de tipos de datos objetos.
        \end{itemize}
\end{itemize}




\end{document}
