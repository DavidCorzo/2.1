\documentclass{article}

\usepackage{generalsnips}
\usepackage{calculussnips}
\usepackage[margin = 1in]{geometry}
\usepackage{pdfpages}
\usepackage[spanish]{babel}
\usepackage{amsmath}
\usepackage{amsthm}
\usepackage[utf8]{inputenc}
\usepackage{titlesec}
\usepackage{xpatch}
\usepackage{fancyhdr}
\usepackage{tikz}
\usepackage{hyperref}
\usepackage{enumitem}
\usepackage{longtable}
\title{Repaso Teoria}
\date{2020 May 12}
% \author{Cost Analysis}

\begin{document}
\maketitle
%%%%%%%%%%%%%%%%%%%%%%%%%%%%%%%%%%%%%%%%%%%%%%%%%%%%%%%%%%%%%%%%%%%%%%%%%%%%%%%%%%%%%%%%%%%%%%%%%%%%%%%%%%%%%%%%%%%%%%%%%%%%%%%%%%%%%%%%%%%%%%

\section{Tipos de contabilidad}
\begin{itemize}
    \item \textbf{Contabilidad financiera: } se muestra una fecha determinada, todo lo que ha pasado en una empresa. Es más como presentar. Contabilidad orientada a externos. 
    \item \textbf{Contabilidad administrativa: } se usan internamente para tomar en cuenta en el futuro y se utilizan los costos, es para uso interno. Esta es la contabilidad de costos. 
\end{itemize}

%----------------------------------------------------------------------------------------
\section{Fórmulas}        
\begin{center}
    \begin{longtable}[c]{ | p{17cm} | }
            \hline 
            Mano de obra directa: MO - MO indirecta 
            \\
            \hline 
            Costo de producción: MO directa + Materia directa + gastos indirectos 
            \\
            \hline 
            Costos unitarios: $\displaystyle \frac{\text{ Costos de producción  }}{\text{ Total unidades producidas }} $ 
            \\
            \hline 
            Costos de conversión: MO directa + costos indirectos de fabricación
            \\
            \hline 
            Costos primos: materiales directos + MO directa 
            \\
            \hline 
                 Para calcular el punto de equilibrio en unidades:
                    \[
                      \text{  Unidades en equilibrio  } = \frac{\sum \text{  (Todos los costos fijos)  }}{\text{  Precio de venta - Costos variables unitario (ó utilidad marginal)  }} 
                    \]
                    \begin{itemize}[label=\#]
                        \item Recordar que hay que siempre redondear para arriba.
                    \end{itemize}
            \\
            \hline 
                 Calcular el equilibrio a partir de utilidad deseada:
                    \[
                      \text{  Cantidad de unidades que debo vender  } = \frac{\sum (\text{  costos fijos  }) + \text{  Utilidad deseada  }}{\text{  Margen de contribución  }} 
                    \]
                    \begin{itemize}[label=\#]
                        \item El margen de contribución es ella utilidad marginal solo que en porcentaje.
                    \end{itemize}
            \\
            \hline 
                 Margen de contribución:
                    \[
                      \text{  Margen de contribución  } = \frac{\text{  Utilidad marginal  }\times 100}{\text{  Precio de venta  }} 
                    \]
            \\  
            \hline  
                 Para calcular la utilidad marginal:
                    \[
                      \text{  Utilidad marginal  } = \text{  Precio de venta } - \text{  Costos variables  }
                    \]
            \\
            \hline 
                 Para calcular la utilidad de operación:
                    \[
                      \text{  Utilidad en operación } = \text{  Ingreso total  } - \text{  Costo variable total  } - \text{  Costo fijo total  }
                    \]
            \\
            \hline 
                 Para calcular las unidades que debo vender para llegar al equilibrio cuando quiero una utilidad:
                    \[
                      \text{  Unidades para lograr el objetivo  } = \frac{\text{  Utilidad objetivo  } + \text{  Costo fijo total  }}{\text{  Margen de contribución por ``x'' unidad  }} 
                    \]
            \\
            \hline 
                 Para calcular las unidades con utilidad esperada e ISR:
                    \[
                      \text{  Unidades para lograr objetivo  } = \frac{\frac{\text{  Utilidad esperada  } } {(1-0.25)}  + \text{  Costos fijos totales  }}{ \text{  Margen de contribución por ``x'' unidad  }} 
                    \]
            \\
            \hline 
                 Para calcular el margen de seguridad:
                    \[
                      \text{  Margen de seguridad  } = \frac{\text{  Ventas esperadas  }-\text{  Precios de equilibrio  }}{\text{  Ventas esperadas  }} 
                    \]
                    \begin{itemize}[label=\#]
                        \item Se considera alto a partir del 25\%
                        \item Análisis de riego 
                    \end{itemize}
            \\
            \hline 
                 Para calcular el equilibrio tomando en cuenta varios productos:
                    \[
                      P_e \; \text{  en unidades  }= \frac{\sum \text{Costos fijos}}{\sum (\text{  media de la utilidad marginal de los productos  })} 
                    \] 
            \\
            \hline
    \end{longtable}
\end{center}


\section{Ejemplo de precio de equilibrio }
\begin{center}
    \begin{tabular}{ |p{3cm}|p{3cm}|p{3cm}|p{3cm}| }
        \hline
            &  Grandes & Medianas & Pequeñas \\
        \hline
            Costo variable unitario & Q8.00 & Q6.24 & Q4.8 \\ 
            \hline
            Mercado & 40\% & 20\% & 40\% \\ 
            \hline
            Precio de venta & Q12.00 & Q11.24 & Q9.50 \\ 
            Utilidad marginal & 
            Q12.00 - Q8.00 = Q4 & Q11.24 - Q6.24 = 5 & Q9.50 - Q 4.8 = Q4.7 \\ 
            \hline
            Utilidad marginal promedio & \multicolumn{3}{|c|}{4*40\% + 5*20\% * 4.7*40\% = 4.48} \\ 
        \hline
    \end{tabular}
\end{center}

\subsection{Punto de equilibrio en unidades y quetzales}
\[
  P_e = \frac{90,000}{4.44} = 20,271 \;\text{ unidades }
\]
Para cada producto: 
\begin{center}
    \begin{tabular}{ c }
        \hline
            Grandes: 20,271 * 40\% = 8,109 \\
            Medianas: 20,271 * 20\% = 4,055 \\
            Pequeñas: 20,271 * 40\% = 8,109 \\
        \hline
    \end{tabular}
\end{center}

\subsection{Número de unidades para una utilidad después del ISR de Q20,000}
\[
  P_e = \frac{90,000 + \p{\cfrac[]{20,000}{1-0.25}} }{4.48} = 26277
\]
\begin{center}
    \begin{tabular}{ c }
        \hline
            26277 * 40\% = 10510 unds $\rightarrow$ 10510 * 12 = Q26276.27 \\
            26277 * 20\% = 5255 unds $\rightarrow$ 5255 * 11.24 = Q59070 \\
            26277 * 40\% = 10510 unds $\rightarrow$ 10510 * 9.50 = Q99850 \\
        \hline
    \end{tabular}
\end{center}


\section{Estrategias de pricing}
\begin{itemize}
    \item Neutralización: entrar con el precio de l competencia.
    \item Penetración: entrar con un precio artificialmente bajo temporalmente.
    \item Skimming: diferenciación al principio y después baja. 
    \item Psicología: 0.99
    \item Productos económicos: empaques grandes a precios más bajos al por mayor.
\end{itemize}


%----------------------------------------------------------------------------------------
\section{Interpretación de las razones}
\begin{itemize}
    \item \large\textbf{Razones de liquidez:}\normalsize
    \begin{itemize}
        \item Razón circulante: activo corriente / pasivo corriente 1.97
            \begin{itemize}
                \item Por cada quetzal de pasivo hay 1.97 quetzales de activo.  
            \end{itemize}
        \item Prueba de ácido: (activo corriente - inventario)/ pasivo corriente:  ej. 1.51
            \begin{itemize}
                \item Por cada quetzal que tengo de pasivo, tengo 1.51 para pagar con activos.
            \end{itemize}
    \end{itemize}

    \item \large\textbf{Razones de solvencia:}\normalsize
        \begin{itemize}
            \item Razón de la deuda total: Pasivo total / activo total
                \begin{itemize}
                    \item Si tiene 1, TODO lo debe, mientras más bajo mejor. 
                \end{itemize}

            \item Razón de la deuda capital: Pasivo total / capital: 
                \begin{itemize}
                    \item Mientras más cerca de uno es peor, (depende de la industria) mientras más lejos de uno es mejor. A veces las empresas se pueden endeudar más que otras. 
                \end{itemize}
            \item Multiplicador: Activo total / capital total: 54\%, 1.84, al dividir 1/1.84 nos da un porcentaje.
                \begin{itemize}
                    \item El el 54\% de los activos están cubiertos por los accionistas o el capital.
                    \item Mientras más alto es mejor. 
                \end{itemize}
        \end{itemize}

    \item Razones de actividad o rotación de activos:
        \begin{itemize}
            \item Rotación de inventarios: Costo / promedio inventario: 7.09
                \begin{itemize}
                    \item Número de veces en el que el inventario es acabado y remplazado. Usualmente en un año.
                    \item Mientras más alto mejor. 
                \end{itemize}

            \item Días de ventas en inventario: 365 / Rotación de inventario 
                \begin{itemize}
                    \item Mientras menos, se está vendiendo más rápido, es la cantidad de días que toma vender todo un inventario. 
                \end{itemize}
                
            \item Rotación de cuentas por cobrar: ventas / promedio de cuentas por cobrar 
                \begin{itemize}
                    \item Cantidad de veces en un año que se recogen las cuentas por cobrar. 
                \end{itemize}
                
            \item Días de ventas en cuentas por cobrar: 365/rotación de cuentas por cobrar 
                \begin{itemize}
                    \item Cantidad de días que transcurre hasta recolectar las cuentas por cobrar en un año, es bueno si aumenta, malo si baja. 
                \end{itemize}
                
            \item Rotación de activos fijos: Ventas / Activos fijos netos (es decir sin depreciación): 
                \begin{itemize}
                    \item De todos mis ingresos, qué tan eficientes somos para invertirlos en activos no corrientes. 
                    \item Mientras más alta de uno, mejor por que quiere decir que cada inversión sobre los activos no corrientes me están ayudando a generar más ventas. 
                \end{itemize}
                
            \item Rotación de activos totales: Ventas / activos totales 
                \begin{itemize}
                    \item Del total de ventas, cuánto nos ayudan los activos para mantener ese total de activos totales. Tuvo un rendimiento de 0.85 quetzales por cada quetzal de activos totales. 
                \end{itemize}
        \end{itemize}
        
    \item \large\textbf{Medidas de rentabilidad:}\normalsize
        \begin{itemize}
            \item Margen de utilidad: Utilidad neta / ventas
                \begin{itemize}
                    \item Qué porcentaje de las ventas es utilidad neta. 
                \end{itemize}
                
            \item Rendimiento sobre los activos: utilidad neta / activos totales 
                \begin{itemize}
                    \item Mide el rendimiento de su activo a partir de la utilidad neta. De la cantidad de utilidad que se genera, cuál es la proporción de los activos totales que representa en proporción a la utilidad neta. (Depende de la industria)
                \end{itemize}
                
            \item Rendimiento sobre el capital: Utilidad neta / capital total
                \begin{itemize}
                    \item Representa cada quetzal que invirtieron los accionistas, representa el rendimiento del capital en proporción a las utilidades netas. 
                \end{itemize}
        \end{itemize}

    \item \large\textbf{Medidas de valor de mercado:}\normalsize
        \begin{itemize}
            \item Utilidades por acción: Utilidad neta / acciones en circulación
                \begin{itemize}
                    \item No lo vimos. 
                \end{itemize}
                
            \item Razón de precio: precio por acción / utilidades por acción: 
                \begin{itemize}
                    \item No lo vimos. 
                \end{itemize}
                
            \item Razón de valor de mercado a valor en libros: valor de mercado por acción / calor en libros por acción: 
                \begin{itemize}
                    \item No lo vimos. 
                \end{itemize}
        \end{itemize}
    \item \large\textbf{Análisis vertical: }\normalsize
        \begin{itemize}
            \item Es la evaluación del funcionamiento de la empresa en un periodo ya especificado.
        \end{itemize}
    
    \item \large\textbf{Análisis horizontal: }\normalsize
        \begin{itemize}
            \item Se realiza con Estados Financieros de diferentes períodos y se examina la tendencia que tienen las cuentas en el transcurso del tiempo ya establecido para su análisis.
        \end{itemize}
\end{itemize}


\section{Costos}
\begin{itemize}
    \item Clasificación de costos: 
        \begin{itemize}
            \item Costo directo: costos que son rastreables (que se pueden medir por unidad).
            \item Costo indirecto: costos que no son fácilmente rastreables. 
            \item Costo variable: se paga proporcional a la cantidad, depende de la cantidad. 
            \item Costo fijo: se paga una magnitud fija por servicio, no depende de cantidad. 
        \end{itemize}
    
    \item Costos mixtos: 
        \begin{itemize}
            \item Costo semivarible: parte fija y parte variable. 
            \item Costo escalonado: cada cierto nivel de producción se incrementan los costos fijos. 
        \end{itemize}

    \item Elementos del costo: 
        \begin{itemize}
            \item Material: materia prima, directa o indirecta.
            \item Mano de obra: Directo o indirecto.
            \item Gastos fabricación: gastos adicionales que no tienen que ver ni con el material no con MO. \# Solo incluir de fabricación. 
        \end{itemize}

    \item Relación con la producción:
        \begin{itemize}
            \item Costo primo: materiales directos + mano de obra directa.
            \item Costos de conversión: mano de obra directa + costos indirectos de fabricación (incluye materiales indirectos + mano de obra indirecta, todo lo que diga indirecto)
            \item Costo de producción: mano de obra directa +  materiales directos + gastos indirectos de fabricación. (TODO)
            \item Costo unitario: elementos del costo / número de unidades a producir.
        \end{itemize}
\end{itemize}




%%%%%%%%%%%%%%%%%%%%%%%%%%%%%%%%%%%%%%%%%%%%%%%%%%%%%%%%%%%%%%%%%%%%%%%%%%%%%%%%%%%%%%%%%%%%%%%%%%%%%%%%%%%%%%%%%%%%%%%%%%%%%%%%%%%%%%%%%%%%%%
\end{document}

