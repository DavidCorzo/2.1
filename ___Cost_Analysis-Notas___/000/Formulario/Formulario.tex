\documentclass{article}
\title{Formulario}
\author{David Corzo}
\date{2020-Feb-21}
%%%%%%%%%%%%%%%%%%%%%%%%%%%%%%%%%%%%%%%%%%%%%%%%%%%%%%%%%%%%%%%%%%%%%%%%%%%%%%%%%%%%%%%%%%%%%%%%%%%%%%%%%%%%%%%%%%%%%%%%%%%%%%%%%%%%%%%%%%%%%%%
\setlength{\voffset}{-0.50in}
\setlength{\headsep}{5pt}

\usepackage[margin=0.75in]{geometry}
\usepackage{graphicx}
\usepackage{fontenc}
\usepackage{pdfpages}
\usepackage[spanish]{babel}
\usepackage{amsmath}
\usepackage{amsthm}
\usepackage[utf8]{inputenc}
\usepackage{enumitem}
\usepackage{mathtools}
\usepackage{import}
\usepackage{xifthen}
\usepackage{pdfpages}
\usepackage{transparent}
\usepackage{color}
\usepackage{fancyhdr}
\usepackage{lipsum}
\usepackage{sectsty}
\usepackage{titlesec}
\usepackage{calc}
\usepackage{lmodern}
\usepackage{xpatch}
\usepackage{blindtext}
\usepackage{bookmark}
\usepackage{fancyhdr}
\usepackage{xcolor}
\usepackage{tikz}
\usepackage{blindtext}
\usepackage{hyperref}
\usepackage{listing}
\usepackage{spverbatim}
\usepackage{fancyvrb}
\usepackage{fvextra}
\usepackage{amssymb}
\usepackage{pifont}
\usepackage{longtable}
\usepackage{tkz-euclide}
%\usetkzobj{all}
%%%%%%%%%%%%%%%%%%%%%%%%%%%%%%%%%%%%%%%%%%%%%%%%%%%%%%%%%%%%%%%%%%%%%%%%%%%%%%%%%%%%%%%%%%%%%%%%%%%%%%%%%%%%%%%%%%%%%%%%%%%%%%%%%%%%%%%%%%%%%%%
\begin{document}
\maketitle

\begin{center}
\begin{longtable}[c]{ | p{17cm} | }
    
\hline 
     Para calcular el punto de equilibrio en unidades:
        \[
          \text{  Unidades en equilibrio  } = \frac{\sum \text{  (Todos los costos fijos)  }}{\text{  Precio de venta - Costos variables unitario (ó utilidad marginal)  }} 
        \]
        \begin{itemize}[label=\#]
            \item Recordar que hay que siempre redondear para arriba.
        \end{itemize}
\\
\hline 
     Calcular el equilibrio a partir de utilidad deseada:
        \[
          \text{  Cantidad de unidades que debo vender  } = \frac{\sum (\text{  costos fijos  }) + \text{  Utilidad deseada  }}{\text{  Margen de contribución  }} 
        \]
        \begin{itemize}[label=\#]
            \item El margen de contribución es ella utilidad marginal solo que en porcentaje.
        \end{itemize}
\\
\hline 
     Margen de contribución:
        \[
          \text{  Margen de contribución  } = \frac{\text{  Utilidad marginal  }\times 100}{\text{  Precio de venta  }} 
        \]
\\  
\hline  
     Para calcular la utilidad marginal:
        \[
          \text{  Utilidad marginal  } = \text{  Precio de venta } - \text{  Costos variables  }
        \]
\\
\hline 
     Para calcular la utilidad de operación:
        \[
          \text{  Utilidad en operación } = \text{  Ingreso total  } - \text{  Costo variable total  } - \text{  Costo fijo total  }
        \]
\\
\hline 
     Para calcular las unidades que debo vender para llegar al equilibrio cuando quiero una utilidad:
        \[
          \text{  Unidades para lograr el objetivo  } = \frac{\text{  Utilidad objetivo  } + \text{  Costo fijo total  }}{\text{  Margen de contribución por ``x'' unidad  }} 
        \]
\\
\hline 
     Para calcular las unidades con utilidad esperada e ISR:
        \[
          \text{  Unidades para lograr objetivo  } = \frac{\frac{\text{  Utilidad esperada  } } {(1-0.25)}  + \text{  Costos fijos totales  }}{ \text{  Margen de contribución por ``x'' unidad  }} 
        \]
\\
\hline 
     Para calcular el margen de seguridad:
        \[
          \text{  Margen de seguridad  } = \frac{\text{  Ventas esperadas  }-\text{  Precios de equilibrio  }}{\text{  Ventas esperadas  }} 
        \]
        \begin{itemize}[label=\#]
            \item Se considera alto a partir del 25\%
        \end{itemize}
\\
\hline 
     Para calcular el equilibrio tomando en cuenta varios productos:
        \[
          P_e \; \text{  en unidades  }= \frac{\sum \text{Costos fijos}}{\sum (\text{  media de la utilidad marginal de los productos  })} 
        \] 
\\
\hline
    \end{longtable}
\end{center}


\end{document}


