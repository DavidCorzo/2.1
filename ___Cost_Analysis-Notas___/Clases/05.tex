\section{Pricing}
\begin{itemize}
    \item \emph{\textbf{Definición de ``pricing":} son las técnicas que me permiten determinar a qué precio puedo poner mi producto, es el arte de saber comprender como cuánto un cliente estaría dispuesto a pagar por un producto o servicio, intensando obtener el máximo margen de utilidad posible de éste.}
    \item Hay una gama de dificultades que no controlamos:
        \begin{itemize}
            \item Precios de materia prima 
            \item Situación político-económica del país.
        \end{itemize}
    
    \item Hay diferentes tipos de productos:
        \begin{itemize}
            \item Los productos superfluos: son productos de lujo.
            \item Los productos necesarios: son productos que necesito.
        \end{itemize}
    
    \item Objetivos: 
        \begin{enumerate}
            \item Maximizar las ganancias: uno vende no por benevolencia, es por que quiero derivar beneficios.
            \item Aumentar los volúmenes de venta: a veces si no balanceo las unidades vendidas con el precio puedo perder.
            \item Consolidar un prestigio: tiene que ver con lealtad, si tengo un buen producto puedo elevar más el precio.
            \item Neutralizar la guerra de precios: todos pierden en una guerra de precios. 
        \end{enumerate}
    
    \item Deteminación de la estrategia de pricing:
        \begin{itemize}
            \item Lo más importante es la estrategia con el cliente.
            \item Depende del tipo de mercado que quiero llegar. 
            \item Establecer una estrategia de precios que sea compatible para el tipo de mercado al que llegaré.
            \item Hacer un estudio de mercado antes de lanzar un producto al mercado.
            \item Evaluar si se tiene capacidad de vender las unidades necesarias para sobrevivr.
            \item Controlar todos los factores.
        \end{itemize}
    
\end{itemize}



%%%%%%%%%%%%%%%%%%%%%%%%%%%%%%%%%%%%%%%%%%%%%%%%%%%%%%%%%%%%%%%%%%%%%%%%%%%%%%%%%%%%%%%%%%%%%%%%%%%
\section{Estrategias de pricing}

% --------------------------------------------------------------------------------------------
\subsection{Neutralizar}
\begin{itemize}
    \item Entrar y poner el mismo precio que la competencia.
    \item Usualmente no puedo mover el precio por ser nuevo.
\end{itemize}


% --------------------------------------------------------------------------------------------
\subsection{Penetración}
\begin{itemize}
    \item Quiero entrar a un mercado y bajo el precio de una manera simbólica, ofrezco el producto a precios bajos y después lo subo.
\end{itemize}

% --------------------------------------------------------------------------------------------
\subsection{Skimming}
\begin{itemize}
    \item Se establece un valor de venta por arriba de la competencia.
    \item Esto se hace por prestigio, por ejemplo Apple.
\end{itemize}


% --------------------------------------------------------------------------------------------
\subsection{Psicológica}
\begin{itemize}
    \item Cuando algo le da un pricing de 9.99, 999.99, la diferencia de un centavo el cliente se siente más atraído psicológicamente.
\end{itemize}

% --------------------------------------------------------------------------------------------
\subsection{Productos económicos}
\begin{itemize}
    \item Vender paquetes al por mayor, un ejemplo es PriceSmart.
    \item Productos necesarios, le bajan el precio bajo la condición de comprar más.
\end{itemize}
