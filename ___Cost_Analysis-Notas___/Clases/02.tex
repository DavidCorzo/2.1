\section{Clasificación de costos}
\begin{itemize}
    \item Contbilidad financiera \& Contabilidad administrativa.
        \begin{itemize}
            \item La contabilidad administrativa es la contabilidad de costos.
        \end{itemize}
\end{itemize}

\subsection{Modelo de planeación y de control}
Seguir estos pasos :
\begin{itemize}
    \item Planeación: Objetivos y programas.
    \item Control: Efectividad y eficiencia.
        \begin{itemize}
            \item Efectivo es poder alcanzar la meta.
            \item Efectividad es poder alcanzar las metas con la menor cantidad de recursos.
            \item \emph{\textbf{Ejemplo: } se hace un pedido y se cumple a un costo de 109,000. Fue efectivo por que alcanzó la meta; no fue eficiente por que uso más recursos de lo presupuestado.}
        \end{itemize}
    \item Costo presupuestado vs. costo real: 
        \begin{itemize}
            \item Presupuesto es una estimación de los costos, esto se utiliza para poder medir la eficiencia; \emph{\textbf{Ejemplo: }quiero en enero producir 100,000 unidades, tengo que tener un presupuesto para poder comparar si se puede, y comparar el costo del presupuesto con el costo real}.
            \item Lo que se busca es optimizar los recursos y que tengan su mejor rendimiento.
        \end{itemize}
    \item Congruencia de metas:
        \begin{itemize}
            \item Congruencia en las metas, saber las capacidades que se tiene y los costos que incurren, tener congruencia en la meta.
        \end{itemize}
\end{itemize}

%%%%%%%%%%%%%%%%%%%%%%%%%%%%%%%%%%%%%%%%%%%%%%%%%%%%%%%%%%%%%%%%%%%%%%%%%%%%%%%%%%%%%%%%%%%%%%%%
\subsection{Conceptos y clasificación de costos}
\begin{itemize}
    \item Costo:
        \begin{itemize}
            \item Es lo que a mi me cuesta hacer un producto.
            \item \emph{\textbf{Ejemplo: }Si me costo Q8, al momento de venderlo a Q8.01 ya estoy ganando.}
        \end{itemize}
    
    \item Gastos:
        \begin{itemize}
            \item 
        \end{itemize}
    
    \item Ingresos:
        \begin{itemize}
            \item Todo lo que entra a la empresa.
            \item \emph{\textbf{Ejemplo: }intereses.}
        \end{itemize}
    
    \item Pérdidas:
        \begin{itemize}
            \item Cuando el costo es mayor al ingreso, todo lo que me produce un costo.
        \end{itemize}
\end{itemize}

%%%%%%%%%%%%%%%%%%%%%%%%%%%%%%%%%%%%%%%%%%%%%%%%%%%%%%%%%%%%%%%%%%%%%%%%%%%%%%%%%%%%%%%%%%%%%%%%
\subsection{Elementos de costo}
\begin{enumerate}
    \item Materiales: materia prima, etcétera, puede ser directa o indirecta.
    \item Mano de obra: toda la gente que trabaja para proveer el producto. Puede ser indirecto o directo, directo o indirecto recae con qué exactitud se puede rastrear el costo.
    \item Gastos de fabricación: todos aquellos gastos adicionales que no tienen que ver con la materia prima o de la mano de obra. \emph{\textbf{Ejemplo: }electricidad, internet, etcétera.}
\end{enumerate}

%%%%%%%%%%%%%%%%%%%%%%%%%%%%%%%%%%%%%%%%%%%%%%%%%%%%%%%%%%%%%%%%%%%%%%%%%%%%%%%%%%%%%%%%%%%%%%%%
\subsubsection{Ejemplo}
\begin{itemize}
    \item Los cortadores de madera son costos directos.
    \item Los supervisores son un costo indirecto por que no tiene nada que ver con la materia prima ni la mano de obra.
    \item Servicios de cámaras son seguridad, no tienen nada que ver con la fabricación; son seguridad, es indirecto.
\end{itemize}

%%%%%%%%%%%%%%%%%%%%%%%%%%%%%%%%%%%%%%%%%%%%%%%%%%%%%%%%%%%%%%%%%%%%%%%%%%%%%%%%%%%%%%%%%%%%%%%%
\subsection{Relación con la producción}
Los costos primos y costos de conversión nunca se juntan por que se estaría contabilizando dos veces la mano de obra. 
\begin{itemize}
    \item Costos primos:
        \begin{itemize}
            \item Materiales directos y la mano de obra directa, sin estos costos no puedo generar materia prima.
        \end{itemize}
    
    \item Costos de conversión:
        \begin{itemize}
            \item Lo que me cuesta a mi convertir la materia prima; 
            \item Gastos indirectos de fábrica.
        \end{itemize}
\end{itemize}

%%%%%%%%%%%%%%%%%%%%%%%%%%%%%%%%%%%%%%%%%%%%%%%%%%%%%%%%%%%%%%%%%%%%%%%%%%%%%%%%%%%%%%%%%%%%%%%%
\subsection{Relación con el volumen}
\begin{itemize}
    \item Costos variables:
        \begin{itemize}
            \item Son costos que  se mantienen constante, es el costo que debo incrementar proporcional a la cantidad de productos que vaya a producir.
        \end{itemize}
    
    \item Costos fijos:
        \begin{itemize}
            \item Son costos que si o si siempre se van a pagar, costos de arrendamiento por ejemplo.
        \end{itemize}
\end{itemize}

%%%%%%%%%%%%%%%%%%%%%%%%%%%%%%%%%%%%%%%%%%%%%%%%%%%%%%%%%%%%%%%%%%%%%%%%%%%%%%%%%%%%%%%%%%%%%%%%
\subsubsection{Maximizar costos}
\begin{itemize}
    \item Se intenta producir lo más que se pueda para disminuir los costos fijos.
\end{itemize}

%%%%%%%%%%%%%%%%%%%%%%%%%%%%%%%%%%%%%%%%%%%%%%%%%%%%%%%%%%%%%%%%%%%%%%%%%%%%%%%%%%%%%%%%%%%%%%%%
\section{Costos mixtos}
\begin{itemize}
    \item Costos semivariables: hay una parte fija y hay otra parte variable.
        \begin{itemize}
            \item Costo fijo: se paga un costo fijo por cierto servicio.
            \item Costo variable: pero aparte del costo fijo se agrega una cantidad proporcional a la cantidad del servicio que adquiero.
            \item \emph{\textbf{Ejemplo: }La cuota del celular.}
        \end{itemize}
    
    \item Costo escalonado: cada cierto nivel de producción se incrementan los costos fijos.
        \begin{itemize}
            \item Proporcional a el numero de unidades en producción, se necesita a veces incrementar costos fijos por el volumen de producción, \emph{\textbf{Ejemplo: }Si un supervisor supervisa a 15 personas máximo, y contrato a más trabajadores tengo que también contratar a otro supervisor.}
        \end{itemize}
\end{itemize}

%%%%%%%%%%%%%%%%%%%%%%%%%%%%%%%%%%%%%%%%%%%%%%%%%%%%%%%%%%%%%%%%%%%%%%%%%%%%%%%%%%%%%%%%%%%%%%%%
\section{Áreas funcionales}
\begin{itemize}
    \item Costos de manufactura: 
        \begin{itemize}
            \item Todos los costos relacionados con la manufactura del producto.
        \end{itemize}
    
    \item Costos de mercadeo:
        \begin{itemize}
            \item Todos los costos de marketing que se hacen como publicidad.
        \end{itemize}
    
    \item Costos administrativos:
        \begin{itemize}
            \item Tener las oficinas y departamentos.
        \end{itemize}
    
    \item Costos dinancieros:
        \begin{itemize}
            \item Los préstamos y lo que se paga por interés.
            \item No es el giro normal de la empresa.
        \end{itemize}
\end{itemize}
