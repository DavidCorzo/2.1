\section{Punto de equilibrio}
\begin{itemize}
    \item Sirve para relacionar el costo del volumen de lo que puedo producir.
    \item Para eso tengo que saber los costos variables, fijos.
    \item El punto de equilibrio son donde yo no gano ni pierdo. Para calcular esto tengo que sumar todos los costos y dividirlo dentro de lo que le gano. A este precio es el precio de equilibrio.
        \[
          \frac{\sum_{}^{}\text{Todos costos fijos}}{\text{Precio de venta - Costos varibles unitarios (utilidad marginal)}} = \text{Unidades de equilibrio }
        \]
        
    \item Para calcular las unidades que debo vender contando la utilidad que quiero sumo la cantidad de utilidad deseada a los costo y divido por el margen de contribución. Utilidad bruta: 
        \[
        \frac{\sum_{}^{}(\text{costos}) + \text{ Utilidad deseada} }{\text{Margen de contribución}} = \text{Cantidad de unidades que debo vender}
        \]
    
    \item Margen de contribución: \[
      \text{  Margen de contribución  } = \frac{\text{  Utilidad marginal } \times 100 }{\text{  Precio de venta  } } 
    \]
    \item 
        \begin{align*}
            \text{  Utilidad marginal  } = \text{  Precio  } - \text{  Costos variables  } \\ 
        \end{align*}
    \item Entonces se propone el objetivo de vender tal cantidad de unidades para sacar x cantidad de unidades, eso por supuesto no depende de mi si no del mercado y demanda.
    \item \emph{\textbf{Interesante:} Al hacer un estudio de mercado se deduce cuántas unidades me dejará vender el mercado como máximo en un periodo de tiempo.}
    \item \emph{\textbf{Recordar lo siguiente: }}
        \begin{center}
           \begin{tabular}{ | p{5cm} | p{5cm} | }
               \hline
                    Ventas & x    \\
                    (-) Costos de venta & x \\ 
               \hline
                    Utilidad bruta & 76,000 \\ 
                    (-) Costos fijos & 56,000 \\ 
                \hline
                    Utilidad en operación & 20,000 \\ 
                    (-) ISR & 5,000 \\ 
                    Utilidad neta & 15,000 \\ 
                \hline
           \end{tabular}
        \end{center}
    
    \item \emph{\textbf{Ejemplo: }Galletas de desnutrición, el mercado está en el interior, \textbf{Nos preguntamos:} ¿puedo hacer una galleta a Q1.00?}
    \item Siempre hay que hacer un estudio de mercado para evaluar la entrada al negocio y al mercado.
\end{itemize}

%%%%%%%%%%%%%%%%%%%%%%%%%%%%%%%%%%%%%%%%%%%%%%%%%%%%%%%%%%%%%%%%%%%%%%%%%%%%%%%%%%%%%%%%%%%%%%%%

\section{Costo volumen por unidad}
\begin{itemize}
    \item Utilidad:  
    \[
      \text{Utilidad } = \text{Ingreso total} - \text{Costo variable total} - \text{Costo fijo total}
    \]
    
    \item Unidades para lograr objetivo: 
        \[
         \text{Unidades para lograr objetivo} = \frac{\text{Utilidad objetivo} + \text{Costo fijo total}}{\text{Margen de contribución por x unidad}} 
        \]
\end{itemize}

%%%%%%%%%%%%%%%%%%%%%%%%%%%%%%%%%%%%%%%%%%%%%%%%%%%%%%%%%%%%%%%%%%%%%%%%%%%%%%%%%%%%%%%%%%%%%%%%
\section{ISR - impuesto sobre la renta}
\begin{itemize}
    \item Para calcular cuanto tengo que ganar para tener x cantidad de utilidad:
        \[
          \frac{\text{Utilidad esperada}}{(1-\%)} 
        \]
    
    \item 
        \[
          \text{Unidades para lograr objetivo} = \frac{\frac{\text{Utilidad esperada}}{(1-0.25)} + \text{Costo fijos totales}}{\text{Margen de contribución por x unidad}} 
        \]
\end{itemize}

\section{Análisos de riesgo}
\begin{itemize}
    \item Margen de seguridad: es el porcentaje máximo en que las ventas esperadas pueden disminuir y aún generar una utilidad.
        \[
          \text{Margen de seguridad } = \frac{\text{Ventas esperadas} - \text{Precio de equilibrio}}{\text{Ventas esperadas}} 
        \]
    
    \item Se considera alto a partir de 25\%. 
\end{itemize}

%%%%%%%%%%%%%%%%%%%%%%%%%%%%%%%%%%%%%%%%%%%%%%%%%%%%%%%%%%%%%%%%%%%%%%%%%%%%%%%%%%%%%%%%%%%%%%%%

\section{Comparación de procesos de producción}
\begin{enumerate}
    \item Determinar el punto de equilibrio para cada proceso.
    \item Determinar cuales son nuestras demás variables.
    \item Evaluar el margen de seguridad.
\end{enumerate}
Clave: el mejor es el que permita encontrar el precio de equilibrio más rápido.
