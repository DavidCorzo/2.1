\section{Análisis financiero}
\begin{itemize}
    \item Análisis vertical: solo miro un año. 
    \item Análisis horizontal: comparo múltiples años. 
\end{itemize}

%----------------------------------------------------------------------------------------
\subsection{Razones financieras}
\begin{itemize}
    \item Medidas de liquidez: mide la capacidad de pago a corto plazo. 
    \item Medidas de solvencia: mide la capacidad de pago a largo plazo. 
    \item Medidas de actividad o rotación de activos: mide la eficiencia con que se utilizan los activos. 
    \item Medidas de rentabilidad: mide la eficiencia de la utilización de los activos para generar sus operaciones. 
    \item Medidas de valor de mercado: mide el precio del valor de mercado por acción del capital accionario. 
\end{itemize}

\subsubsection{Razones de liquidez}
\begin{itemize}
    \item Activo corriente / Pasivo corriente
    \item Prueba de ácido: (Activo corriente - Inventario) / Pasivo corriente
\end{itemize}

\subsubsection{Razones de solvencia}
Si me da dos estoy demasiado endeudado: 
\begin{itemize}
    \item Razón de la deuda total: Pasivo Total / Activo Total
    \item Razón de la deuda capital: deuda total / capital total 
    \item Multiplicador: activo total / capital total 
\end{itemize}

\subsubsection{Razón de actividad o rotación de activos}
\begin{itemize}
    \item Rotación de inventarios: costo de ventas / inventario 
    \item Días de ventas en inventario: 365/Rotación de inventario 
    \item Rotación de cuentas por cobrar: ventas  / cuentas por cobrar; va dar un número de veces, las veces que roto mi inventario. 
    \item Días de ventas en cuentas por cobrar: 365 / rotación de cuentas por cobrar 
    \item Rotación de activos fijos: ventas / activos no corrientes netos 
    \item Rotación de activos totales: ventas / activos totales 
\end{itemize}

\subsubsection{Medidas de rentabilidad}
\begin{itemize}
    \item Margen de utilidad: utilidad neta / ventas, si sale más de 10 estoy siendo muy rentable. 
    \item Rendimiento sobre los activos: utilidad neta / activos totales, de todo lo que obtengo cuál es el rendimiento. 
    \item Rendimiento sobre capital: utilidad neta / capital contable total, de todo lo que obtengo cuál es el rendimiento.  
\end{itemize}
El rendimiento sobre capital sólo puede ser igual al rendimiento sobre capital, nunca puede ser mayor. 


\subsubsection{Medidas de valor de mercado}
\begin{itemize}
    \item Utilidades por acción (UPA): utilidad neta / acciones en circulación 
    \item Razón precio - utilidades: precio por acción / utilidades por acción, para saber cómo está el precio por acción. 
    \item Razón de valor de mercado a valor en libros: valor de mercadio por acción / valor en libros por acción, cuál es el valor de mercado comparado con el de libros. 
\end{itemize}


