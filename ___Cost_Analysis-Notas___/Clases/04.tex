\section{Punto de equilibrio}
\begin{itemize}
    \item Ejemplo: 
        \begin{center}
           \begin{tabular}{ | p{5cm} | p{5cm} | p{5cm} | p{5cm} | }
               \hline
                    Producto & A & B & C    \\ 
                \hline
                    Precio de venta & Q. 15 & Q. 10 & Q. 8 \\  
                    Costo variable por unidad & 10 & Q. 6 & Q. 5\\ 
                    Utilidad marginal & Q. 5 & Q. 4 & Q. 3 \\ 
               \hline
               Proporción mercado & 13\% & 50\% & 20\% \\ 
               $\bar{x}$ Utilidad marginal & Q. 1.50 & Q. 2.00 & Q. 0.60 \\ 
           \end{tabular}
            \begin{align*}
                P_e = \frac{1,000,000}{\sum_{}^{}\bar{x}\text{  utilidad marginal  }} = 243,903 \text{  Unidades totales  } \\ 
                \therefore \quad \\ 
                \text{  Producto A  } = 73,171 (243,903 \times 30\%) \times Q15 =& Q. 1,097,565 \\ 
                \text{  Producto B  } = 121,952 (243,903 \times 50\%) \times Q10 =& Q. 1,219,520 \\ 
                \text{  Producto C  } = 48,781 (243,903 \times 20\%) \times Q8 =& Q. 390,249 \\ 
                \sum_{}^{}\text{  Producto A,B,C  } = Q. 2,707,333 \implies \text{  Punto de equlibrio  } = Q. 2,707,334 \\ 
            \end{align*}
        \end{center}
\end{itemize}

\section{Material de apoyo}
\begin{itemize}
    \item \begin{figure}[htbp] 
        \centering
        % \includegraphics[width=cm]{}
        % \includegraphics[width=cm]{}
        % \includegraphics[width=cm]{}
        \caption{}
        \label{}
    \end{figure}
\end{itemize}
