\section{Planeación estratégica}
%----------------------------------------------------------------------------------------
\subsection{Introducción}


%----------------------------------------------------------------------------------------
\subsection{Preguntas a realizarse}
\begin{itemize}
    \item \pregunta{Dónde está la empresa} 
    \item \pregunta{Cómo logra el cambio deseado} 
    \item \pregunta{Hacia dónde quiere is la empresa} 
\end{itemize}
\begin{itemize}[label=\#]
    \item Tengo que establecer una misión \& visión.
    \item Tengo que ir acorde a los cambios que yo pueda realizar.
    \item \pregunta{Cómo voy a llegar} 
\end{itemize}

\subsection{Acitivades a realizar}
\begin{itemize}
    \item Definición de estrategia: tengo que definir a dónde voy y cómo lo haré.
    \item Definición de los planes de acción: parte de la definición de estrategia es cómo lo haré, el como lo haré es el plan de acción y el presupuesto es un plan de acción.
\end{itemize}


%----------------------------------------------------------------------------------------
\subsection{Presupuestos}
\begin{itemize}
    \item Son valiosos a la medida que sirven de guía que ayudan a monitorear si lo implantado está bien y se están logrando los resultados esperados. Los presupuestos son excelentes herramientas que facilitan la:
        \begin{itemize}
            \item Administración de objetivo 
            \item Administración por excepción
        \end{itemize}
\end{itemize}


%----------------------------------------------------------------------------------------
\subsection{Elementos del presupuesto}
\begin{itemize}
    \item Un plan: tengo objetivos que quiero alcanzar .
    \item Integrador: involucra a todos los departamentos de la empresa.
    \item Coordinador: todos los departamentos trabajan en conjunto, todos tenemos la misma meta.
    \item En términos financieros: la dimensional que se puede usar.
    \item Operaciones y recursos: tengo que ver los gastos de la empresa , de dónde viene el financiamiento por ejemplo.
    \item Periodo determinado: corto plazo. 
\end{itemize}


%----------------------------------------------------------------------------------------
\subsection{Ventajas de hacer un presupuesto}
\begin{itemize}
    \item Permite definir los objetivos.
    \item Define la estructura: delega responsabilidad y autoridad a ciertas personas.
    \item Si hay incentivos se verá un incremento en la participación y se incrementa la cercanía a los objetivos esperados.
    \item Obliga a llevar archivos contables.
    \item Facilita la administración óptima de diferentes insumos.
    \item Obliga realizar autoanálisis periódicamente.
    \item Facilita el control administrativo.
    \item Mejorar la eficiencia(alcanzar las metas) y eficacia(alcanzar la meta con bajos recursos)
\end{itemize}


%----------------------------------------------------------------------------------------
\subsection{Desventajas/limitaciones de hacer un presupuesto}
\begin{itemize}
    \item Esta basado en estimaciones 
    \item Debe ser adaptado a los cambios según surjan.
    \item No es de ejecución automática.
    \item No toma el lugar de la administración.
    \item Toma tiempo y cuesta prepararlos.
    \item No se debe esperar resultados inmediatos.
\end{itemize}


%----------------------------------------------------------------------------------------
\subsection{Características}
\begin{itemize}
    \item Flexibilidad: pueden hacerse cambio cuando se necesite.
    \item Facilitadores de control administrativo.
    \item Cuantifica los objetivos y metas.
    \item Facilita la autoevaluación.
    \item El presupuesto a corto plazo debe ser parte del presupuesto a largo plazo 
    \item Debe presentar indicadores en forma condensada que reflejen los objetivos logrados y por lograr (\%)
\end{itemize}


%----------------------------------------------------------------------------------------
\subsection{Presupuesto maestro}
\begin{itemize}
    \item Sistema convencional del presupuesto maestro: tomo en cuenta el pasado, la historia, tomar en cuenta la historia y el peso, me baso en presupuestos anteriores.
    \item Presupuesto base cero: empiezo de cero, no veo el pasado, no pongo nada de lo que tengo de experiencia.
    \item Planeación del programa y sistema presupuestario: 
    \item Elaboración del presupuesto maestro: 
        \begin{itemize}
            \item Inicio estableciendo las metas y objetivos.
        \end{itemize}
\end{itemize}



%----------------------------------------------------------------------------------------
\subsection{Presupuesto de ventas: Puntos a considerar}
\begin{itemize}
    \item Pronósticos económicos generales
    \item Ventas y utilidades de la industria 
    \item Condiciones del inventario 
    \item Condiciones competitivas 
    \item Se puede ver la información histórica por producto y por canal de distribución.
\end{itemize}

\subsubsection{Para hacer el presupuesto de ventas}
\begin{itemize}
    \item Sabes cuánto voy a vender y multiplicarlo por el precio.
\end{itemize}



%----------------------------------------------------------------------------------------
\subsection{Presupuesto de producción: puntos a considerar}
\begin{itemize}
    \item Terminar de copiar.
\end{itemize}
