\section{Conceptos fundamentales}
\begin{itemize}
    \item Activos: Es lo que yo tengo derecho de.
    \item Pasivos: Son aquellas obligaciones que tengo que incurrir.
    \item Patrimonio: Todo lo que le pertenece a la empresa, es el capital; el pasivo + capital cuadra con el activo. El capital no solo es dinero. 
    \item Ingreso o ventas: todos los beneficios que se derivan por la actividad de vender, los ingresos pueden ser 
    \item Costos y gastos: Costos es lo que me cuesta hacer un producto, los gastos es lo que me cuesta venderlo, es todo el gasto adicional que realiza para realizar sus operaciones.
\end{itemize}

%%%%%%%%%%%%%%%%%%%%%%%%%%%%%%%%%%%%%%%%%%%%%%%%%%%%%%%%%%%%%%%%%%%%%%%%%%%%%%%%%%%%%%%%%%%%%%%%
\subsection{Ejemplo}
\[
  \text{Activo} = \text{Pasivo} (+) \text{Capital} 
\]

\subsection{Activos}
\begin{itemize}
    \item Corrientes: son los que se pueden liquidar a \textbf{corto} plazo a efectivo.
        \begin{itemize}
            \item Disponibles 
            \item Exigibles
        \end{itemize}
    \item No corrientes: son los que se liquidan a \textbf{largo} plazo a efectivo. 
        \begin{itemize}
            \item Realizables
            \item fijos 
        \end{itemize}
\end{itemize}

%%%%%%%%%%%%%%%%%%%%%%%%%%%%%%%%%%%%%%%%%%%%%%%%%%%%%%%%%%%%%%%%%%%%%%%%%%%%%%%%%%%%%%%%%%%%%%%%
\subsection{Pasivos}
\begin{itemize}
    \item Corrientes:
        \begin{itemize}
            \item Bancos: Un prestamo, una hipotéca, carta de crédito.
            \item Proveedores: las obligaciones incurridas por el inventario que yo debo.
            \item Gastos acumulados por pagar: son pagos que se acumulan a través del tiempo.
            \item Prestaciones laborales: pagos adicionales que establece la ley a sus empleados, el bono 14 o el aguinaldo.
            \item Provisión para indemnización: La reserva para cuando los empleados son despedidos y tienen que ser pagados por indemnización.
        \end{itemize}
\end{itemize}


%%%%%%%%%%%%%%%%%%%%%%%%%%%%%%%%%%%%%%%%%%%%%%%%%%%%%%%%%%%%%%%%%%%%%%%%%%%%%%%%%%%%%%%%%%%%%%%%
\subsection{Patrimonio}
\begin{itemize}
    \item Capital - Aportaciones: en aportaciones lo que los accionistas aportan a la sociedad.
    \item Utilidades retenidas: se retiene el 5\% del dinero de los años anteriores.
        \begin{itemize}
            \item Reserva legal: aquellas reservas de dinero que debo tener por ley, es obligatorio no es voluntario como las reservas estatuarias.
            \item Reservas estatutarias: hacer reservas a favor de los empleados; ejemplo que el 10\% lo voy a separar para futuros proyectos.
            \item Resultados acumulados: las ganancias de los aos anteriores o pérdidas, la suma de todo.  
        \end{itemize}
\end{itemize}

%%%%%%%%%%%%%%%%%%%%%%%%%%%%%%%%%%%%%%%%%%%%%%%%%%%%%%%%%%%%%%%%%%%%%%%%%%%%%%%%%%%%%%%%%%%%%%%%
\subsection{Ingresos}
\begin{itemize}
    \item Ventas: por ventas.
    \item Intereses ganados: como los intereses por bancos.
    \item Otros ingresos: se venden activos por arriba de su precio de depreciación.
\end{itemize}

%%%%%%%%%%%%%%%%%%%%%%%%%%%%%%%%%%%%%%%%%%%%%%%%%%%%%%%%%%%%%%%%%%%%%%%%%%%%%%%%%%%%%%%%%%%%%%%%
\subsection{Gastos}
\begin{itemize}
    \item Costos: El costo al que yo adquirí en el inventario. 
    \item Gastos: todos los gasto que se tengan, seguridad, luz, teléfono.
    \item Otros gastos: No son usuales pero son gastos, una multa, intereses bancarios, etc. 
\end{itemize}

%%%%%%%%%%%%%%%%%%%%%%%%%%%%%%%%%%%%%%%%%%%%%%%%%%%%%%%%%%%%%%%%%%%%%%%%%%%%%%%%%%%%%%%%%%%%%%%%
\subsection{Estados financieros básicos}
\begin{itemize}
    \item Balance general: enseñar la situación de una empresa en una fecha específica. ``Balance general al <fecha>'', es como una foto de ese día.
        \begin{itemize}
            \item Ejemplo: la billetera, el balance general, tiene Q3; el balance es de 303 por 300 en efectivo y 3 en bancos.
                \begin{center}
                   \begin{tabular}{ | p{5cm} | p{5cm} | }
                       \hline
                        Activo & Pasivo     \\
                       \hline
                       Caja y bancos Q303 & -- \\ 
                       -- & Capital Q303 \\ 
                       \hline
                   \end{tabular}
                \end{center}
        \end{itemize}

    \item Estado de resultados: si la empresa está ganando o perdiendo, es un resumen de sus ingresos y sus gastos.
    \item Estado de patrimonio: mostrar las variaciones que hubieron en el capital durante un año, puede aumentar o disinuir.
    \item Estado de flujos de efectivo: todo el efectivo que se recudé en un año y dónde están invertidos. 
    \item Notas a los estados financieros: la información general que ayuda a entender sus estados financieros; como cual método de depreciación uso, clientes morosos. 
\end{itemize}

%%%%%%%%%%%%%%%%%%%%%%%%%%%%%%%%%%%%%%%%%%%%%%%%%%%%%%%%%%%%%%%%%%%%%%%%%%%%%%%%%%%%%%%%%%%%%%%%
\subsection{Auditoría externa}
\begin{itemize}
    \item Es importante para comprobar que la información financiera esté registrada correctamente.
\end{itemize}

%%%%%%%%%%%%%%%%%%%%%%%%%%%%%%%%%%%%%%%%%%%%%%%%%%%%%%%%%%%%%%%%%%%%%%%%%%%%%%%%%%%%%%%%%%%%%%%%
\section{Ejercicio: ¿invertimos en la empresa o no?}
\begin{itemize}
    \item La empresa es Disney, si tenía que invertir.
\end{itemize}
