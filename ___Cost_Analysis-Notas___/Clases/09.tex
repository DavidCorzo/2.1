\section{Presupuesto de compra de materiales directos}
Puntos a considerar:
\begin{itemize}
    \item Cantidades a comprar 
    \item Programa de entrega
    \item Cuanto se necesita para producir una unidad 
\end{itemize}

Tener en cuenta que el presupuesto de materiales directos es vital dado a que si no se tienen materiales directos no se puede producir.


%----------------------------------------------------------------------------------------
\subsection{\pregunta{Qué necesito para hacer el presupuesto} }
\begin{itemize}
    \item presupuesto de producción en unidades 
    \item Inventario inicial en unidades 
    \item inventario final en unidades 
    \item Precio 
\end{itemize}

\textbf{Importante: La base es el de ventas y el de producción.}


%----------------------------------------------------------------------------------------
\subsection{Presupuesto de consumo de materiales directos}

\subsection{Presupuesto de Mano de obra}
Puntos a considerar:
\begin{itemize}
    \item Recursos humanos proporciona la información relacionada con este tema.
\end{itemize}
Necesito:
\begin{itemize}
    \item Presupuesto de producción en unidades 
    \item Hora de mano de obra directa en unidades 
    \item Tasa por hora de mano de obra directa 
\end{itemize}
Fórmula: Unidades de producción requeridas* tarifa por hora de mano de obra.
