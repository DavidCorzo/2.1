\section{Repaso de parcial}
\begin{itemize}
    \item Contabilidad:
        \begin{itemize}
            \item Administrativa: es la contabilidad de los costos. 
            \item Financiera: contabilidad orientada a externos.
        \end{itemize}
    
    \item Clasificación de costos:
        \begin{itemize}
            \item Costos Fijos: gastos que no me los puedo quitar produzca o no produzca.
            \item Costos Variables: van relacionados directamente con la producción, me los puedo quitar si no produzco.
            \item Costos Directos: son costos que son rastreables. Se puede decir con exactitud cuánto se va. Por lo general son los fijos pero sólo los fijos.
            \item Costos Indirectos: costos que no son rastrables. No se puede determinar con exactitud cuánto se va. Por lo general son los variables pero no sólo los variables.
        \end{itemize}
    
    \item Elementos del costo: 
        \begin{itemize}
            \item Materiales: puede ser costo \textbf{directo o indirecto}.
                \begin{itemize}
                    \item Es todo aquello que yo tengo que convertir.
                    \item \emph{\textbf{Ejemplo: }Caña de azúcar, conlleva un procedimiento para producir azúcar.}
                    \item Material directo: material cuyo costo es fácilmente rastreable.
                    \item Material indirecto: material cuyo costo no es fácilmente rastreable.
                \end{itemize}
            \item Mano de obra: puede ser costo \textbf{directo o indirecto}.
                \begin{itemize}
                    \item  
                \end{itemize}
            \item Costos de fabricación: puede ser únicamente costo \textbf{indirecto}.
                \begin{itemize}
                    \item 
                \end{itemize}
        \end{itemize}
        \begin{itemize}[label=\#]
            \item La suma de todos los elementos del costo ma va a dar el costo del producto.
        \end{itemize}
    
    \item Los elementos del costo se pueden dividir en dos categorías:
        \begin{itemize}
            \item Costos primos: todo lo que se necesite para primeramente producir.
                \begin{itemize}
                    \item son la materia prima directa.
                    \item con la mano de obra directa.
                \end{itemize}
            
            \item Costos de conversión: todo lo que gasto para poder convertir la materia prima.
                \begin{itemize}
                    \item Materia prima indirecta.
                    \item Mano de obra indirecta.
                    \item Gastos indirectos 
                \end{itemize}
            
            \item ojo:
                \begin{itemize}[label=\#]
                    \item Si quiero saber el costo unitario:
                        \[
                          \text{  Costo unitario  } = \frac{\sum(\text{Elementos del costo})}{\text{Número de unidades producida}} 
                        \]
                    \item Los costos primos y costos de conversión se manejan por separado para evitar la doble contabilidad de costos.
                \end{itemize}
        \end{itemize}
    
    \item Estudio de mercado:
        \begin{itemize}
            \item \textbf{¿}Cuántas unidades puedo vender\textbf{?} el estudio de mercado responde a esto.
            \item Pasos:
                \begin{enumerate}
                    \item Hacer un estudio de mercado 
                    \item Determinar el punto de equilibrio para determinar si voy a ganar o no.
                    \item 
                \end{enumerate}
            
            \item Punto de equilibrio:
                \[
                  P_e = \frac{\sum(\text{  Costo fijo   })}{\text{  Utilidad marginal  }} 
                \]
                \emph{\textbf{Ejemplo: }Tengo costos fijos de 1,200,000 y los el costo es de 8 y lo planeo vender a 15, el estudio de mercado determinó que se venderán 15,000 unidades, \textbf{¿}entro o no\textbf{?}}
                \[
                  P_e \frac{\text{  1,200,000  }}{\text{  7  }} = 171,429 \,\text{  Unidades  } 
                \]
                No debo entrar por que no voy a vender mas de 15,000 unidades.
        \end{itemize}
    
    \item Punto de equilibrio a partir de utilidad esperada de operación:
        \begin{itemize}
            \item Supongamos que el estudio de mercado determina que puedo vender 200,000 unidades.
            
            \item \[
                P_e = \frac{\text{  Costos fijos totales  } + \text{  Utilidad esperada  }}{\text{  Utilidad marginal  }} = \frac{1,200,000 + 100,000}{7} = \frac{1,300,000}{7} = 185,715 \, \text{  unidades  } 
              \]
            
            \item Conclusión: si mi meteré al mercado por que llego al equilibrio mucho antes de lo que el estudio de mercado dice.
        \end{itemize}

    
    \item Calcular punto de equilibrio para saber cuántas unidades tiene que vender después de llegar al impuesto sobre la renta:
        \begin{itemize}
            \item \[
                P_e = \frac{\text{  Costos fijos totales  } + \frac{\text{Utilidad esperada}}{0.75} }{\text{  Utilidad marginal  }} = \frac{1,200,000 + \frac{100,000}{0.75}}{7} = 190,474\, \text{  Unidades  } 
            \] 
            
            \item Para sacar el punto de equilibrio en dinero multiplicar $190,474\times15$ el 15 es el precio de venta, el resultado de este cálculo es la venta total.
        \end{itemize}
    
    
    \item Estado de resultados para sacar la utilidad neta:
        \begin{center}
           \begin{tabular}{ | p{3cm} | p{2cm} | p{2cm} | p{2cm} | }
               \hline
                    & Producto A & Producto B & Producto C    \\
               \hline
                    Precio de venta & 10 & 20 & 25 \\ 
                    Costo de ventas & 8 & 15 & 20 \\ 
                \hline
                    Utilidad marginal & 2 & 5 & 5 \\ 
                    Mercado & 35\% & 25\% & 40 \% \\ 
                \hline
                    $\bar{x}$ Marginal & 0.70 & 1.25 & 2 \\ \hline 
                    \multicolumn{4}{|c|}{$\sum (\bar{x} \text{  Utilidad marginal  }) = 3.95$} \\ 
                \hline
           \end{tabular}
        \end{center}
        \begin{itemize}
            \item Punto de quilibrio en unidades globales:
                \begin{center}
                    \begin{tikzpicture}[node distance = 5cm, auto]
                        \node [large_block] (1) {$P_e = \frac{520,000}{3.95} = 131,646 \, \text{  Unidades  }$};
                        \node [block,right of=1] (2) {25\% =  B}; 
                        \node [block,above of=2] (3) {35\% = A};
                        \node [block,below of=2] (4) {40\% = C};   
                        \path [line] (1) -- (2);
                        \path [line] (1) -- (3);
                        \path [line] (1) -- (4);
                    \end{tikzpicture}
                \end{center}
        \end{itemize}
    
    \item Media vez yo decida entrar al mercado debo escoger una estrategia de pricing:
        \begin{itemize}
            \item Skimming: por tener un prestigio notable pongo un precio mucho más alto sabiendo que la gente lo comprará.
            \item Psicológica: poner las cosas a 9.99, 999.99, etc. 
            \item Penetración: introduzco mi producto a un precio menor al de la competencia.
            \item Neutralización: poner mi precio al precio de la competencia.
            \item Productos económicos: venta al por mayor. Típicamente productos de la canasta básica.
        \end{itemize}
\end{itemize}
