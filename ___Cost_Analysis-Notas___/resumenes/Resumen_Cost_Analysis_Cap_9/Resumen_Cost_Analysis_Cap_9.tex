\documentclass{article}

\usepackage{generalsnips}
\usepackage{calculussnips}
\usepackage[margin = 1in]{geometry}
\usepackage{pdfpages}
\usepackage[spanish]{babel}
\usepackage{amsmath}
\usepackage{amsthm}
\usepackage[utf8]{inputenc}
\usepackage{titlesec}
\usepackage{xpatch}
\usepackage{fancyhdr}
\usepackage{tikz}
\usepackage{hyperref}
\usepackage{autobreak}
\title{Resumen Cost Analysis Cap 9}
\date{2020 March 08, 06:20PM}
\author{David Gabriel Corzo Mcmath}

\begin{document}
\maketitle
%%%%%%%%%%%%%%%%%%%%%%%%%%%%%%%%%%%%%%%%%%%%%%%%%%%%%%%%%%%%%%%%%%%%%%%%%%%%%%%%%%%%%%%%%%%%%%%%%%%%%%%%%%%%%%%%%%%%%%%%%%%%%%%%%%%%%%%%%%%%%%
\allowdisplaybreaks

\section{Naturaleza del presupuesto}
\begin{itemize}
    \item \termdefinition{Presupuesto}{es una expresión cuantitativa de los objetivos gerenciales  un medio para controlar el progreso hacia el logro de tales objetivos. Hay diferentes tipos y que cubren diferentes periodos. Usualmente se quiere un presupuesto que corresponden a 12 meses.} 
    \item \termdefinition{Costos estándares}{Costos por unidad que se espera lograr en determinado proceso de producción.} 
    \item \termdefinition{Costos presupuestados}{Costos totales que se espera incurrir.}
    \item \termdefinition{Director de presupuesto}{El responsable de cordinar todas las partes del presupuesto. Este presenta informes directamente al comité de presupuesto.}  
    \item \termdefinition{Gerencia de línea}{Los que preparan los diferentes componentes como ventas por producto, territorio, produccíon por producto.} 
    \item \termdefinition{Manual de presupuesto}{proporciona los datos respectivos para cada segmento del presupuesto maestro, donde se asigne la responsabilidad de su preparación, y se describan las formas, las políticas y los procedimientos presupuéstales afines a la presupuestación} 
\end{itemize}


%----------------------------------------------------------------------------------------
\subsection{Presupuesto maestro}
\begin{itemize}
    \item \termdefinition{Presupuesto maestro}{todos los costos que se esperan incurrir a nivel macro de la empresa} 
\end{itemize}


%----------------------------------------------------------------------------------------
\subsection{Sistema convencional del presupuesto maestro}
\begin{itemize}
    \item \termdefinition{Sistema convencional del presupuesto meastro}{Un sistema de presupuesto que se basa del presupuesto maestro anterior, Los resultados reales que se lograron en el periodo anterior más las expectativas del periodo próximo determinan si se incrementará o disminuirá cada ítem en el presupuesto maestro.} 
\end{itemize}


%----------------------------------------------------------------------------------------
\subsection{Presupuesto base cero (PBC)}
\begin{itemize}
    \item \termdefinition{PBD}{el gerente de un centro de responsabilidad debe justificar cada actividad planeada y su costo total estimado como si fuera la primera vez que va a realizarse, se catalogan las actividades de cada centro de acuerdo a qué tan importantes son, el presupuesto tras haber sido enviado por el gerente se somete a revisión por el comité de presupuesto para hacer las eliminaciones o adiciones necesarias, esto se hace cada perioro presupuestal.}
\end{itemize}




%----------------------------------------------------------------------------------------
\subsection{Planeación del programa y sistema presupuestario (PPSP)}
\begin{itemize}
    \item \termdefinition{PPSP}{se centra en el resultado (programas o actividades) de la organización;  La fuerza del PPSP está en la asignación de los recursos (efectivo) limitados de la organización a aquellas actividades o programas que prometen el mayor rendimiento}; 
    \item Cuatro pasos para el PPSP:
        \begin{enumerate}
            \item Alta gerencia planea estratégicamente metas a corto y largo plazo.
            \item Se identifican todos los programas o actividades alternativos para lograr las metas y objetivos de la organización.
            \item Se pronostica los costos y beneficios cuantitativos y cualitativos de cada actividad o programa alternativo.
            \item Se hace un presupuesto de lo que se seleccionó junto con un plan detallado por ítem.
        \end{enumerate}
\end{itemize}


%----------------------------------------------------------------------------------------
\section{Elaboración del presupuesto maestro}
\begin{itemize}
    \item Se empieza planeando estratégicamente (fijar metas a largo plazo).
    \item El presupuesto es una herramienta para mobilizar a la compañía con recursos limitados a donde quiere llegar.
    \item Pasos:
        \begin{itemize}
            \item Pronostico de venta 
            \item Elaboración de estado de ingresos presupuestado, presupuesto de vaja y balance general presupuestado. \termdefinition{estado financiero presipuestado}{lo mismo que estado de resultados solo que en versión futura tomando en cuenta el presupuesto.} 
        \end{itemize}
    
    \item Dos extremos en el desarrollo del presupuesto maestro:
        \begin{enumerate}
            \item El enfoque de la alta gerencia 
            \item Enfoque ``la base'' de la organización
        \end{enumerate}
\end{itemize}



%----------------------------------------------------------------------------------------
\subsection{Presupuesto de ventas}
\begin{itemize}
    \item \termdefinition{Presupuesto de ventas}{el pronóstico de ventas suministra los datos para elaborar presupuestos de producción, de compra, de gastos de venta y administrativos; se debe elaborar con cuidado.} 
    \item El pronostico de ventas empieza con a preparación de estimaciones de ventas realizadas por cada uno de los vendedores, estas se consolidan y son presentadas al gerente general de mercadeo que las revisa y aprueba para elaborar el presupuesto de ventas.
    \item Estimar las ventas es un proceso que se puede dar de varias formas, y hay libros dedicados a la elaboración de estas estimaciones.
    \item Requerimientos para la elaboración, programa 1a:
        \begin{enumerate}
            \item Presupuesto de ventas, en unidades.
            \item Precio de venta por unidad.
        \end{enumerate}
        
        \begin{align*}\begin{autobreak}
          \text{ Presupuesto de ventas }= \text{ Presupuesto de ventas (unidades) } \times \text{ precio de venta por unidad }
        \end{autobreak}\end{align*}
        
\end{itemize}

%----------------------------------------------------------------------------------------
\subsection{Presupuesto por producción}
\begin{itemize}
    \item \termdefinition{Presupuesto por producción}{ es el presupuesto de ventas ajustado por los cambios en el inventario} 
    \item Función: quiero que para x fecha estén listas y cantidad de unidades, necesito un presupuesto para saber qué necesito para que estén listas y unidades en x fecha.
    \item Está relacionado con el presupuesto de ventas.
    \item Se debe determinar si la fábrica es capaz de suplir lo que se estableció en la estimación de ventas.
    \item Requerimientos para la elaboración, programa 1b:
        \begin{enumerate}
            \item Presupuesto de ventas, en unidades.
            \item Inventario final, en unidades.
            \item Inventario inicial, en unidades.
        \end{enumerate}
        
        \begin{align*}\begin{autobreak}
          \text{ Presupuesto de producción (unidades) } = 
          \text{ Presupuesto de ventas (unidades) } 
        + \text{ Inventario final deseado (unidades) } 
        - \text{ Inventario inicial (unidades) }
        \end{autobreak}\end{align*}
\end{itemize}



%----------------------------------------------------------------------------------------
\subsection{Presupuesto de compras de materiales directos}
\begin{itemize}
    \item \termdefinition{Presupeusto de ompras de materiales directos}{es un presupuesto que sirve para mantener el inventario de de los materiales directos.}
    \item Función: para tener x cantidades directos para fecha y.
    \item Por lo general, se dispone de una hoja de especificación o fórmula para cada producto que muestra el tipo y la cantidad de cada material directo por unidad de producción.
    \item  El presupuesto de suministros y de materiales indirectos se incluye, por lo general, en el presupuesto de costos indirectos de fabricación.
    \item Requerimientos para elaboración, programa 1c:
        \begin{enumerate}
            \item Presupuesto de producción, en unidades.
            \item Inventario final, en unidades.
            \item Inventario inicial, en unidades.
            \item Precio de compra, por unidad.
        \end{enumerate}
        
        \noindent\begin{align*}\begin{autobreak}
          \text{ Compra de materiales directos requeridos (unidades) } = 
          (\text{Presupuesto de producción (unidades)} 
          \times \text{ Materiales directos requeridos para producir una unidad })
          + \text{ Inventario final deseado (unidades)} 
          - \text{ Inventario inicial (unidades) } 
        \end{autobreak}\end{align*}
        
        
        \begin{align*}\begin{autobreak}
          \text{ Costo de compra de materiales directos } = 
          \text{ Compras de materiales directos requeridos (unidades) } \times
           \text{ Precio de compra por unidad }
        \end{autobreak}\end{align*}
        
\end{itemize}


%----------------------------------------------------------------------------------------
\subsection{Presupuesto de consumo de materiales directos}
\begin{itemize}
    \item \termdefinition{Presupuesto de consumo de materiales directos}{presupuesto utilizado para planear las actividades operacionales, este etima el consumo de materiales directos que se incurrirá en las actividades.}
    \item Requerimientos para elaboración, programa 1d:
        \begin{enumerate}
            \item Presupuesto de materiales directos de producción, en unidades 
            \item Precio de compra, en unidades 
        \end{enumerate}
        
        \begin{align*}\begin{autobreak}
          \text{ Presupuesto de consumo de materiales directos } = 
          \text{ Materiales directos requeridos(unidades) } 
          \times \text{ Costo unitario de materiales directos }
        \end{autobreak}\end{align*}
        
\end{itemize}


%----------------------------------------------------------------------------------------
\subsection{Presupuesto de mano de obra directa}
\begin{itemize}
    \item \termdefinition{Presupuesto de mano de obra}{El departamento de personal deberá expresar en los presupuestos de mano de obra directa e indirecta los tipos y la cantidad de empleados requeridos y cuándo se necesitan en este presupuesto.} 
    \item  La mano de obra indirecta se incluye en el presupuesto de costos indirectos de fabricación. 
    \item  El presupuesto de mano de obra directa debe estar coordinado con los de producción, de compras y con las demás partes del presupuesto maestro.
    \item Requerimientos para la elaboración, programa 1e:
        \begin{enumerate}
            \item Presupuesto de producción, en unidades.
            \item Horas de mano de obra directa, unidades.
            \item Tasa por hora de mano de obra directa.
        \end{enumerate}
        
        \begin{align*}\begin{autobreak}
          \text{ Presupuesto de mano de obra directa } = 
          \text{ Unidades de producción requeridas } 
          \times \text{ Horas de mano de obra por unidad  } 
          \times \text{ Tarifa por hora de mano de obra directa }
        \end{autobreak}\end{align*}
        
\end{itemize}


%----------------------------------------------------------------------------------------
\subsection{Presupuesto de costos indirectos de fabricación}
\begin{itemize}
    \item \termdefinition{Presupuesto de costos indirectos de fabricación}{Los jefes de departamento deben ser responsables de los costos incurridos por sus respectivos departamentos, estos se hacen en el presupuesto de costos indirectos de fabricación.} 
    \item Los costos fijos y variables se separan como sigue: los costos fijos tienen valores totales asignados en dólares en tanto que a los costos variables se les asignan tasas.
    \item Requerimientos para la elaboración, programa 1f:
        \begin{enumerate}
            \item Presupuesto de horas de mano de obra directa.
            \item Costos fijos 
            \item Costos variables
        \end{enumerate}
        
        \begin{align*}\begin{autobreak}
          \text{ Presupuesto de horas de mano de obra directa } = 
          \text{ Costos indirectos fijos totales } 
          + \text{ (Total de horas presupuestadas de mano de obra indirecta) }
           \times \text{ Tasa de costos variables por hora de mano de obra directa }
        \end{autobreak}\end{align*}
        
\end{itemize}      



%----------------------------------------------------------------------------------------
\subsection{Presupuesto de inventarios finales}
\begin{itemize}
    \item \termdefinition{Presupuesto de inventarios finales}{Con las cantidades del inventario presupuestado al final del mes (que se necesitan para el inventario de materiales directos y de artículos terminados) en el presupuesto del costo de los artículos vendidos y el balance general presupuestado se calcula el presupuesto de inventarios finales.}
    \item Requisitos para la elaboración, programa 1g: 
        
        \begin{align*}\begin{autobreak}
          \text{ Costo de inventario final presupuestado } = 
          \text{ Inventario final (unidades) }
          \times  \text{ Costo estándar por unidad }
        \end{autobreak}\end{align*}
        
\end{itemize}



%----------------------------------------------------------------------------------------
\subsection{Presupuesto del costo de los artículos vendidos}
\begin{itemize}
    \item \termdefinition{Presupuesto del costo de los artículos vendidos}{Presupuesto que describe el costo de los artículos vendidos, pueden tomarse de presupuestos individuales previamente descritos y ajustados por los cambios en inventario.} 
    \item Requisitos para la elaboración, programa 1h:
        \begin{enumerate}
            \item Presupuesto de consumo de materiales directos 
            \item Presupuesto de mano de obra directa 
            \item Presupuesto de costos indirectos de fabricación 
            \item Inventario inicial de artículos de fabricación 
            \item Inventario inicial de artículos terminados
            \item Inventario final de artículos terminados
        \end{enumerate}
        
        \begin{align*}\begin{autobreak}
          \text{ Presupuesto de costo de los artículos vendidos } = 
          \text{ Presupuesto de consumo de materiales directos } + 
          \text{ Presupuesto de mano de obra directa }
           + \text{ Presupuesto de costos indirectos de fabricación }
            + \text{ Inventario inicial de artículos terminados }
             - \text{ Inventario final de artículos terminados }
        \end{autobreak}\end{align*}
        
\end{itemize}




%----------------------------------------------------------------------------------------
\subsection{Presupuesto de gastos de venta}
\begin{itemize}
    \item \termdefinition{Presupuesto de gastos de venta}{Presupuesto conpuesto por: Los principales gastos fijos son salarios y depreciación; los principales gastos variables, como comisiones y publicidad, se basan en las cifras en dólares por concepto de ventas y, por tanto, varían directamente con las ventas} 
    \item Requisitos para la elaboración, programa 1i:
        \begin{enumerate}
            \item Ventas en dólares 
            \item Gastos fijos
            \item Gastos variables
        \end{enumerate}
        
        \begin{align*}\begin{autobreak}
          \text{ Presupuesto de gastos de venta } =

           \text{ Gastos fijos totales por ítem }

            + [\text{ Ventas en dólares  }
            \times \text{ Tasa de gastos variables (\%) } por item] 

        \end{autobreak}\end{align*}
        
\end{itemize}

%----------------------------------------------------------------------------------------
\subsection{Presupuesto de gastos administrativos}
\begin{itemize}
    \item \termdefinition{Presupuesto de gastos administrativos}{estos gastos puede asignarse a operaciones como compra o investigación, pero en este caso consideraremos todas las partidas como gastos fijos no asignables} 
    \item Requisitos para la elaboración, programa 1j:
        \begin{enumerate}
            \item Gastos fijos
        \end{enumerate}
        
        \begin{align*}\begin{autobreak}
        \text{ Presupuesto de gastos administrativos } = \sum \p{\text{Gastos fijos} } 
        \end{autobreak}\end{align*}
        
\end{itemize}


%----------------------------------------------------------------------------------------
\section{Estado de ingresos presupuestado}
\begin{itemize}
    \item \termdefinition{Estado de ingresos presupuestado}{El resultado final de todos los presupuestos operativos, como ventas, costo de los artículos vendidos, gastos de ventas y administrativos, se resume en el estado de ingresos presupuestado. Allí se presenta el resultado neto de las operaciones del periodo presupuestado.}
\end{itemize}


%----------------------------------------------------------------------------------------
\subsection{Presupuesto de caja}
\begin{itemize}
    \item \termdefinition{Presupuesto de caja}{se reconoce como una herramienta gerencial básica, y la cuidadosa planeación del efectivo se considera un elemento de rutina en una gerencia eficiente} 
    \item contribuyen en forma significativa a la estabilización de los saldos de caja y a mantener estos saldos razonablemente cercanos a las continuas necesidades de efectivo.
    \item En la mayor parte de las empresas, las entradas de caja provienen principalmente de la recaudación de cuentas por cobrar y de las ventas de contado.
    \item Requerimientos para su elaboración, programa 2c: 
        \begin{itemize}
            \item Saldo en cada, inicial 
            \item Entradas presupuestadas de caja para el periodo 
            \item Salidas presupuestadas de cada para el periodo.
        \end{itemize}
        
        \begin{align*}\begin{autobreak}
          \text{ Saldo final en caja } = 
          \text{ Saldo inicial en caja } 
          +  \text{ Entradas presupuestadas de caja para el periodo } 
          - \text{ Salidas presupuestadas de caja para el periodo }
        \end{autobreak}\end{align*}
        
\end{itemize}




%----------------------------------------------------------------------------------------
\subsection{Estado presupuestado de los flujos de caja}
\begin{itemize}
    \item \termdefinition{Estado presupuestdo de los flujos de caja}{Un estado de los flujos de caja presenta el cambio durante el periodo en caja y en los equivalentes de caja (inversiones a corto plazo de alta liquidez) se clasifican en un estado de flujos de caja según las actividades operacionales, de inversión y financieras}
    \item \termdefinition{Actividades operacionales}{Las actividades operacionales usualmente involucran la producción y el despacho de artículos y el suministro de servicios} 
    \item \termdefinition{Actividades de inversión}{ activos que la empresa posee o utiliza en la producción de artículos o servicios (distintos de los materiales que hacen parte del inventario de la empresa). } 
    \item \termdefinition{Acitividades financieras}{incluyen la obtención de recursos por parte de los propietarios y el suministro a éstos de un rendimiento sobre su inversión y un rendimiento de ella.} 
    \item La reconciliación de la utilidad neta para los flujos de caja exige ajustar la utilidad neta para eliminar: 1) los efectos de todos los aplazamientos de entradas y desembolsos de caja operacionales previos, como cambios durante el periodo en inventario, ingreso diferido y similares, y todas las acumulaciones de las salidas y pagos de caja operacionales futuros esperados, como cambios durante el periodo en cuentas por cobrar y cuentas por pagar, y 2) los efectos de todos los ítemes cuyos efectos de caja están financiando los flujos de caja, como depreciación, amortización de goodwill, ganancias o pérdidas sobre las ventas de propiedad, planta y equipo. Esto se debe representar en un programa separado.
\end{itemize}



%----------------------------------------------------------------------------------------
\section{Aspectos de comportamiento de la presupuestación}
\begin{itemize}
    \item En realidad el presupuesto aprobado representa un consenso y un compromiso pactado entre muchas personas dentro de la organización.
\end{itemize}























%%%%%%%%%%%%%%%%%%%%%%%%%%%%%%%%%%%%%%%%%%%%%%%%%%%%%%%%%%%%%%%%%%%%%%%%%%%%%%%%%%%%%%%%%%%%%%%%%%%%%%%%%%%%%%%%%%%%%%%%%%%%%%%%%%%%%%%%%%%%%%
\end{document}

