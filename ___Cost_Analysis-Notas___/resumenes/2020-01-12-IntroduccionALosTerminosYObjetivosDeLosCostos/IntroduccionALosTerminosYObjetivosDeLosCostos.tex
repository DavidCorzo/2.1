\documentclass{article}
\title{Introduccion a los terminos y objetivos de los costos}
\author{David Gabriel Corzo Mcmath}
\date{2020-Jan-12 20:38:00}
%%%%%%%%%%%%%%%%%%%%%%%%%%%%%%%%%%%%%%%%%%%%%%%%%%%%%%%%%%%%%%%%%%%%%%%%%%%%%%%%%%%%%%%%%%%%%%%%%%%%%%%%%%%%%%%%%%%%%%%%%%%%%%%%%%%%%%%%%%%%%%%
\usepackage[margin = 1in]{geometry}
\usepackage{graphicx}
\usepackage{fontenc}
\usepackage{pdfpages}
\usepackage[spanish]{babel}
\usepackage{amsmath}
\usepackage{amsthm}
\usepackage[utf8]{inputenc}
\usepackage{enumitem}
\usepackage{mathtools}
\usepackage{import}
\usepackage{xifthen}
\usepackage{pdfpages}
\usepackage{transparent}
\usepackage{color}
\usepackage{fancyhdr}
\usepackage{lipsum}
\usepackage{sectsty}
\usepackage{titlesec}
\usepackage{calc}
\usepackage{lmodern}
\usepackage{xpatch}
\usepackage{blindtext}
\usepackage{bookmark}
\usepackage{fancyhdr}
\usepackage{xcolor}
\usepackage{tikz}
\usepackage{blindtext}
\usepackage{hyperref}
\usepackage{listing}
\usepackage{spverbatim}
\usepackage{fancyvrb}
\usepackage{fvextra}
\usepackage{amssymb}
\usepackage{pifont}
\usepackage{longtable}
%%%%%%%%%%%%%%%%%%%%%%%%%%%%%%%%%%%%%%%%%%%%%%%%%%%%%%%%%%%%%%%%%%%%%%%%%%%%%%%%%%%%%%%%%%%%%%%%%%%%%%%%%%%%%%%%%%%%%%%%%%%%%%%%%%%%%%%%%%%%%%%
\begin{document}
\maketitle

\section{Costos \& historia}
\begin{itemize}
    \item Las organizaciones tienden a fluctuar en su énfasis en los costos, usualmente si la empresa está bien no le ponen mucho énfasis; sin embargo cuando las cosas no van del todo bien el énfasis en este concepto se vuelve vital.
    \item \emph{\textbf{Ejemplo: } Tenesse Products, está teniendo una baja en ventas; Se reúnen con sus directivos para afinar estrategia. Se dan cuenta que en los costos está la clave para darle un giro a la situación.}
\end{itemize}

%%%%%%%%%%%%%%%%%%%%%%%%%%%%%%%%%%%%%%%%%%%%%%%%%%%%%%%%%%%%%%%%%%%%%%%%%%%%%%%%%%%%%%%%%%%%%%%%
\section{Costos y terminología de costos}
\begin{itemize}
    \item \emph{\textbf{Definición de ``costo":} un recurso sacrificado o perdido para alcanzar un objetivo en específico.}
    \item La dimensional de medición de costo generalmente se mide por las magnitudes monetarias que se incurrieron.
    \item \emph{\textbf{Definición de ``Costo real":} el costo histórico, o el que fue incurrido para cierto objetivo.}
    \item \emph{\textbf{Definición de ``Costo presupuestado":} el costo predicho o pronosticado (costo futuro).}
    \item \emph{\textbf{Definición de ``Objeto del costo":} es todo aquello para lo que sea necesaria una medida de costos.}
    \item \emph{\textbf{Definición de ``Acumulación del costo":} es la recopilación de información de costos en forma organizada a través de un sistema contable.}
    \item \emph{\textbf{Definición de ``Asignación del costo":}esun término general que abaraca: }
        \begin{enumerate}
            \item El rastreo de costos acumulados que tienen una relación directa con el objeto de costo.
            \item El prorateo de costos acumulados que tienen una relación \textbf{in}directa con el objeto de costo.
        \end{enumerate}
\end{itemize}

%%%%%%%%%%%%%%%%%%%%%%%%%%%%%%%%%%%%%%%%%%%%%%%%%%%%%%%%%%%%%%%%%%%%%%%%%%%%%%%%%%%%%%%%%%%%%%%%
\section{Costos directos e indirectos}
%%%%%%%%%%%%%%%%%%%%%%%%%%%%%%%%%%%%%%%%%%%%%%%%%%%%%%%%%%%%%%%%%%%%%%%%%%%%%%%%%%%%%%%%%%%%%%%%
\subsection{Rastreo del costo y prorrateo del costo}
\begin{itemize}
    \item Los costos directos de un objeto de costo son aquellos que se pueden rastrear de una manera económicamente factible. \emph{\textbf{Ejemplo: }El costo de las botellas.}
    \item Los costos indirectos de un objeto de costo son aquellos que no se pueden rastrar de una manera económicamente factible. \emph{\textbf{Ejemplo: }Sueldos de los supervisores.}
\end{itemize}
%%%%%%%%%%%%%%%%%%%%%%%%%%%%%%%%%%%%%%%%%%%%%%%%%%%%%%%%%%%%%%%%%%%%%%%%%%%%%%%%%%%%%%%%%%%%%%%%
\subsection{Cobro excesivo al gobierno de EEUU}
\begin{itemize}
    \item Los gerentes y contadores administrativos deben colaborarse por que pueden darse casos en los que se asignan mal recursos para los costos indirectos.
    \item Empresas como Boeing cobraron de más por gastos indirectos a EEUU, esto trajo con sigo varias consecuencias negativas ya que se iniciaron procesos penales y se cobraron altas multas.
\end{itemize}
%%%%%%%%%%%%%%%%%%%%%%%%%%%%%%%%%%%%%%%%%%%%%%%%%%%%%%%%%%%%%%%%%%%%%%%%%%%%%%%%%%%%%%%%%%%%%%%%
\subsection{Factores que influyen en la clasificación de los costos directos e indirectos}
\begin{itemize}
    \item La importancia del costo de que se trata, ``Mientras menor el costo, menos factible se vuelve rastrearlo''; \emph{\textbf{Ejemplo: }si uno quisera rastrear el costo del papel que se uso para imprimir la factura por cliente sería casi imposible.}
    \item La tecnología disponible para recopilar información, los avances en la tecnología hacen cada vz más posible rastrear costos, es factible poder volver estos en costos directos y ya no indirectos.
    \item El diseño de las operaciones, resulta más fácil clasificar un costo como directo si las instalaciones se utilizan exclusivamente para un objeto de costo, es decir se usan para hacer un solo producto y sólo funciona para eso. \emph{\textbf{Ejemplo: }si el salarario de el supervisor de un departamente está incluido si se considera dicho departamento el objeto de costo y no solo el producto que produce.}
\end{itemize}

%%%%%%%%%%%%%%%%%%%%%%%%%%%%%%%%%%%%%%%%%%%%%%%%%%%%%%%%%%%%%%%%%%%%%%%%%%%%%%%%%%%%%%%%%%%%%%%%
\section{Patrones de comportamiento del costo: costos variables y costos fijos}
\begin{itemize}
    \item 
\end{itemize}














\end{document}
