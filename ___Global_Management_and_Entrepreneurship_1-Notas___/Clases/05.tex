\section{Ciclo de vida de las organizaciones}
\begin{itemize}
    \item El ciclo de vida de las organización sirve como autodiagnóstico.
    \item Uso de factores internos y externos.
    \item Solo cuando se tiene una ventaja competitiva se puede a partir de ahí crear alianzas estratégicas. 
\end{itemize}

%%%%%%%%%%%%%%%%%%%%%%%%%%%%%%%%%%%%%%%%%%%%%%%%%%%%%%%%%%%%%%%%%%%%%%%%%%%%%%%%%%%%%%%%%%%%%%%%

\subsection{Pregunta de quiz}
\begin{itemize}
    \item \textbf{Nos preguntamos:} ¿que estrategias de negocio pueden emplear las empresas para mantenerse con ventaja competitiva? respondes a tendencias, poder detectar tendencias y traducirlas a innovación. Se puede innovar en tecnología para que no se vuelva obsoleta.
    \item La tecnología se capitaliza por el ámbito de procesos, servicios.
\end{itemize}

%%%%%%%%%%%%%%%%%%%%%%%%%%%%%%%%%%%%%%%%%%%%%%%%%%%%%%%%%%%%%%%%%%%%%%%%%%%%%%%%%%%%%%%%%%%%%%%%

\section{Globalización aldea global}
\begin{itemize}
    \item Nuevos productos:
        \begin{itemize}
            \item Existentes 
            \item Tendencias 
        \end{itemize}
    
    \item Nuevos servicios: 
        \begin{itemize}
            \item Cosas como entrega a domicilio 
            \item Valor agregado 
        \end{itemize}
        
    \item Nuevos mercados:
        \begin{itemize}
            \item Territorios
            \item Segmentos 
        \end{itemize}

    \item Estrategia mercado
\end{itemize}

%%%%%%%%%%%%%%%%%%%%%%%%%%%%%%%%%%%%%%%%%%%%%%%%%%%%%%%%%%%%%%%%%%%%%%%%%%%%%%%%%%%%%%%%%%%%%%%%

\section{Alianzas estratégicas}
\begin{itemize}
    \item \emph{\textbf{Definición de ``franquisisa":} es una empresa que te permite entrar en tu región con un nombre de una empresa conocida. }
    \item 
\end{itemize}

%%%%%%%%%%%%%%%%%%%%%%%%%%%%%%%%%%%%%%%%%%%%%%%%%%%%%%%%%%%%%%%%%%%%%%%%%%%%%%%%%%%%%%%%%%%%%%%%

\section{Michael Porter}
\begin{itemize}
    \item Sique vivo 2020-01-20, es profesor en la escuela de negocios en Harvard. 
    \item Dedujo que los países tienen ventajas comparativas, en C.A. hay ventajas comparativas:
        \begin{itemize}
            \item La posición geográfica.
            \item Clima, un clima muy estable, los ``microclimas'', los climas varían de municipio a municipio.
        \end{itemize}
    
    \item Las ventajas comparativas son dados, las ventajas competitivas son creadas.
    \item Los territorios competitivos invirtieron en tecnología, invensión, etcétera; ahora naciones como Israel que hicieron esto se pudo desarrollar económicamente.
    \item El diamante de Michael Porter, 
    \item Mercados de baja competitividad es un mercado de pocas personas, pocas oportunidades, podas empresas que no son muy rivales.
    \item \emph{\textbf{Definición de ``ventaja competitiva ":} capacidad de mejorar e innovar continuamente}.
\end{itemize}
