\subsection{Organizaciones grandiosas}
\begin{itemize}
    \item Desempeño superior sostenible 
    \item Clientes profundamente leales 
    \item Cultura ganadora: Steak holders
        \begin{itemize}
            \item Proveedores 
            \item Comunidad 
            \item Empleados / colaboradores 
            \item Accionistas 
        \end{itemize}

    \item Contribución distintivas
\end{itemize}



%----------------------------------------------------------------------------------------
\subsection{Oportunidades de negocio}
\begin{itemize}
    \item Tendencias $\rightarrow $ Oportunidades de negocio. 
        \begin{itemize}
            \item En el FODA se detectan oportunidades de negocio.
            \item No todas las oportunidades las puedo aprovechar.
            \item Las oportunidades de negocio se redacta muy específicamente.
        \end{itemize}
\end{itemize}



%----------------------------------------------------------------------------------------
\subsection{Funciones de administrador}
\begin{itemize}
    \item Planear
    \item Organizar
    \item Dirigir
    \item Controlar 
\end{itemize}


%----------------------------------------------------------------------------------------
\subsection{Planeación}
\begin{itemize}
    \item Análisis del entorno:
        \begin{itemize}
            \item FODA 
            \item 5 fuerzas competitivas de Michael Porter 
        \end{itemize}
        
    \item Valores     
    \item Misión 
    \item Visión     
    \item EGC (Estrategia Global de Competencia): define la manera en que competirá en el mercado la organización con el propósito de desarrollar su ventaja competitiva y consolidar su misión y visión.
        \begin{itemize}
            \item 3 Michael Protestarla 
            \item 3 Líderes del mercado
        \end{itemize}
    
    \item Plazos
    \item OG
    \item Áreas calves 
\end{itemize}


%----------------------------------------------------------------------------------------
\subsection{La estrategia global de conpetencia}
\begin{itemize}
    \item \termdefinition{Estrategia global de competencia}{La estrategia global de competencia define la manera en que copetirá en el mercado la organización con el propósito de desarrollar su ventaja comparativa} 
    \item Elementos medulares:
    \begin{itemize}
        \item Producto / servicio
        \item Costos / precios 
        \item Mercadeo 
        \item Ventas: canales de venta / distribución.
    \end{itemize} 
    
    \item Determina 4 elementos medulares:
        \begin{center}
           \begin{supertabular}{ | p{6cm} | p{6cm} | p{6cm} |}
               \hline
                    Liderazgo en costos & 
                    Diferenciación &
                    Enfoque
                       \\
               \hline
                \begin{enumerate}
                    \item Estrategia de producto / servicio: básico, básico + valor agregado, especializado.
                        \begin{itemize}
                            \item El opuesto es diferenciació.
                        \end{itemize}
                    \item Estrategia de precios y costos: precio alto, baja oferta, costos bajos y precios bajos, alto volúmen de oferta.
                    \item Estrategia de mercadeo: la marca comunica la calidad, estatus, prestigio funcionalidad, versatilidad, propuesta de valor del producto, valores de la organización.
                    \item Estrategia de ventas: canales de venta / distribución: tiendas propias, distribuidores autorizados / exclusivos, puntos de ventas masivos, mayoristas, minoristas, en línea.
                \end{enumerate}
                Ejemplos:
                    \begin{itemize}
                        \item Lapiceros bic 
                        \item Relojes citizen
                    \end{itemize}
                & 
                \begin{itemize}
                    \item Estrategia producto o servicio: básicos +  valor agregado
                    \item Estrategia de precio costo: bajos volúmenes / precios altos / excedente del consumidor.
                    \item Estrategia de mercadeo: marca, estatus / prestigio / calidad; sacan ediciones especiales / personalizar.
                    \item Estrategia de ventas: Tiendas propias / distribuidoreas autorizadas. 
                \end{itemize} 
                Ejemplos: 
                    \begin{itemize}
                        \item Lapiceros Mont Blanc 
                        \item Relojes rolex 
                        \item Ferrari 
                    \end{itemize}
                & 
                \begin{itemize}
                    \item \pregunta{Qué es}: una estrategia global de competencia, así como liderazgo en costos y diferenciación hay una tercera (enfoque), en el caso de enfoque primero es el segmento; Segmento: necesidad específica de un grupo de consumidores, producto / servicios especializados.
                    \item Estrategia de producto: el producto es especializado.
                    \item Estrategia de precios y costos: puede ser de volúmen / exclusividad.
                    \item Estrategia de mercadeo: enfocada para el segmento de mercado
                    \item Estrategia de ventas: puntos de venta
                \end{itemize}
                Ejemplos: 
                    \begin{itemize}
                        \item GNC: segmento alto valor nutricional y salud.
                        \item Pañales Huggies: mamás primerizas.
                        \item Restaurante de comida vegana 
                        \item Gatorade - deportistas  
                        \item Endulcurante especialiadas: diabéticos. 
                        \item Revistas especializatas: runnerworld the economist.
                        \item FUNDAL - Fundación para niños sordo-ciegos.
                    \end{itemize}
                \\ 
               \hline
               Liderazgo en costos: Excelencia operativa & Diferenciación: Innovación & Enfoque: Intimidad con el consumidor \\ 
               \hline
                    \begin{itemize}
                        \item Sistematizacipon de procesos de operaciones y ventas, franquisias.
                        \item 
                    \end{itemize}
                    Ejemplos:
                        \begin{itemize}
                            \item Copa airlines 
                            \item Mcdonald's 
                            \item Starbucks 
                            \item Netflix 
                            \item Uber 
                            \item Toyota 
                        \end{itemize}
                    & 
                    \begin{itemize}
                        \item Vuelven obsoletos sus productos, clientes, lover fans.
                    \end{itemize}
                    Ejemplo:
                        \begin{itemize}
                            \item Apple 
                            \item Samsung 
                            \item Nike 
                            \item Pixar 
                            \item Disney 
                        \end{itemize}
                    & 
                    \begin{itemize}
                        \item Estrategias de liderazgo en costos fueron superadas por la estrategia de excelencia operativa:
                            \begin{itemize}
                                \item Se intentó a ahorrar en la operación. Instituir la producción en masa. Entonces empiezan a revisar, optimizar, sistematizar los procesos de producción. Generaron ahorros en todos los sistemas que tenían.
                                \item Surge McDonald's con esta estrategia, surge el concepto de franquicia. McDonald's tienen productos estandarizados, se expanden por medio de franquisias.
                                \item Copa es otro ejemplo, pasó de la estrategia de liderazgo en costos a la excelencia operacional.
                                \item Ojo: las empresas que fomentan excelencia operativa tiene que necesariamente tener liderazgo en costos, se implican mútuamente, el liderazgo en costos es la base y excelencia operativa es una construcción que se empieza a construir a partir de la base.
                            \end{itemize}
                    \end{itemize}
                    Ejemplos:  
                        \begin{itemize}
                            \item Abogados 
                            \item Consultores 
                            \item Médicos 
                            \item Diseño de software 
                            \item Plataforma CRM 
                            \item Maqinaria industrial
                            \item Equipo farmacéutico 
                        \end{itemize}
                    \\
               \hline
           \end{supertabular}
        \end{center}
\end{itemize}

\begin{center}
    \begin{tikzpicture}[node distance = 4cm, auto]
        \node [block,, text width=10em] (1) {Liderazgo en costos};
        \node [block, right of=1, right of=1, text width=10em] (2) {Diferenciación};
        \node [block, below of=1, left of=2,, text width=10em] (3) {Enfoque};
        \path [line] (1) -- (2);
        \path [line] (2) -- (1);
        \path [line] (3) -- (1);
        \path [line] (3) -- (2);   
    \end{tikzpicture}
\end{center}
%%%%%%%%%%%%%%%%%%%%%%%%%%%%%%%%%%%%%%%%%%%%%%%%%%%%%%%%%%%%%%%%%%%%%%%%%%%%%%%%%%%%%%%%%%

