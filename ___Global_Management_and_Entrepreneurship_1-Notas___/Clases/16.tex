\section{Planeación estratégica - Fuerzas competitivas de Michael Porter}
\subsection{Presentación}
\begin{enumerate}
    \item Proveedores: 
        \begin{itemize}
            \item A partir de la materia prima emana todo en la industria. 
        \end{itemize}
    
    \item Competidores directos: 
        \begin{itemize}
            \item Evaluar las economías de escala.
        \end{itemize}
    
    \item Clientes
    \item Competidores potenciales 
    \item Productos sustitutos 
\end{enumerate}

\subsection{Fuerzas competitivas}
\begin{itemize}
    \item \termdefinition{Fuerzas competitivas}{Es un modelo estratégico que sirve para definir el nivel de competitividad dentro de la industria. Se debe de tomar en cuenta la misión, visión y valores de una empresa para poder definir las estrategias con base a lo que se desarrolle con esta herramienta.} 
    \item Conclusiones: 
        \begin{itemize}
            \item \pregunta{Altamente competitiva} Sí, pero la verdad es que no por que cuando se tiene mucha competencia a largo plazo las utilidades de la oferta es casi cero, se mantiene en equilibrio. 
            \item \pregunta{Invertiría}, no. 
        \end{itemize}
\end{itemize}

\begin{tikzpicture}
	\begin{pgfonlayer}{nodelayer}
		\node [style=white] (0) at (-1, 0.75) {Competencia de mercado};
		\node [style=white] (1) at (-0.75, 4.5) {Proveedores};
		\node [style=white] (2) at (4, 0.75) {Sustitutos};
		\node [style=white] (3) at (-1, -3) {Clientes};
		\node [style=white] (4) at (-6.5, 0.75) {Nuevos entrantes};
		\node [style=none] (5) at (2.75, 3) {4. Poder de negociación de los proveedores};
		\node [style=none] (6) at (-5.75, 1.75) {3. Amenaza de los nuevos entrantes};
		\node [style=none] (7) at (3.5, 0) {5. Amenaza de productos sustitutos};
		\node [style=none] (8) at (-4.25, -2.25) {1. Poder de negociación de los clientes};
		\node [style=none] (9) at (-0.75, 0.25) {2. Rivalidad entre las empresas};
	\end{pgfonlayer}
	\begin{pgfonlayer}{edgelayer}
		\draw (0) to (1);
		\draw (0) to (4);
		\draw (0) to (3);
		\draw (0) to (2);
	\end{pgfonlayer}
\end{tikzpicture}


\subsection{Análisis de la fuerza competitiva, Matriz de posicionamiento del mercado}
5 Fuerzas de competitivas de Michael Porter:
\begin{enumerate}
    \item Proveedores: precio y calidad (alta- baja oferta)
    \item Productos Sustitutos Actuales: precio y calidad / otras variables relevantes 
    \item Clientes: no, actual y no. Potencial (tamaño del mercado)
    \item Nuevos entrantes: Competencia potencial precio y calidad.
    \item Competencia en el mercado: alta o baja.
\end{enumerate}
Ejemplo complementario: 
\begin{enumerate}
    \item \pregunta{Qué fuerza competitiva} Sustitutos 
    \item \pregunta{Cuántas variables} 4- (Exclusividad, AccesibilidadClásico, Deportivo)
    \item \pregunta{Cuadrante más competitivo} Clásico y accesible 
    \item Cuadrante de mayor oportunidades de participación en el mercado: El que tenga menos participantes y mayor potencial de mercado. Barreras de entrada o salida al mercado. 
    \item Economías de escala, números de competidores, diferenciación del producto, requerimiento de capital, política gubernamental, ventaja en costos, acceso a canales de distribución. Puntos previos o alianzas estratégicas.
\end{enumerate}

\begin{tikzpicture}
	\begin{pgfonlayer}{nodelayer}
		\node [style=none] (0) at (-1.5, 6) {};
		\node [style=none] (1) at (-1.5, -3) {};
		\node [style=none] (2) at (-7.5, 1.5) {};
		\node [style=none] (3) at (4.75, 1.5) {};
		\node [style=none] (4) at (-1.5, 6.5) {Precio (+)};
		\node [style=none] (5) at (-8.5, 1.5) {Calidad (-) };
		\node [style=none] (6) at (5.75, 1.5) {Calidad (+)};
		\node [style=none] (7) at (-1.5, -3.5) {Precio (-)};
	\end{pgfonlayer}
	\begin{pgfonlayer}{edgelayer}
		\draw (2.center) to (3.center);
		\draw (0.center) to (1.center);
	\end{pgfonlayer}
\end{tikzpicture}

\subsection{Plazos}
\begin{itemize}
    \item Los plazos dentro de la planeación estratégica, se establecen los criterios de pensamiento estratégico que deben tener presentes a lo largo del tiempo para asegurar el logro de la Misión y Visión de la organización. 
    \item Estos criterios permitirán a la organización proveer y anticiparse con recursos financieros, materia prima, integrar equipos de trabajo, capacitar equipos, gestionar permisos y cumplir con regulaciones legales, fiscales, etc. Crear las condiciones adecuadas, para así concretar las mejores negociaciones y acuerdos con los stake holders. 
    \item 6 Criterios para establecer los plazos estratégicos:
        \begin{enumerate}
            \item Periodo fiscal: este es el criterio más conservador y menos estratégico (por que básicamente en mantener estable la ejecución pero innovar muy poco), Fundamentalmente consiste en monitorear y dar seguimiento a la ejecución de planes y cumplimientos de estándares dentro de los mismos.
            \item Estacionalidad de la demanda: este criterio está orientado al entendimiento del comportamiento de la demanda durante un período de tiempo. Una vez organización entiendo el comportamiento de la demanda de los productos podrá generar estrategias de precios, mercader, ventas que le permita aumentar las ventas durante dichos perídos.
        \end{enumerate}
\end{itemize}
