\section{La planeación}
\begin{itemize}
    \item Se divide en dos:
        \begin{center}
           \begin{tabular}{ | p{7cm} | p{7cm} }
               \hline
                   Planeación estratégica & Planeación operativa     \\
               \hline
                    \begin{itemize}
                        \item Largo plazo 
                        \item Se crea en la Gerencia General y línea de fuego.
                        \item Es de dominio público (steake holder), recordar qué es un steak holder:
                            \begin{itemize}[label=\#]
                                \item Accionistas 
                                \item Clientes 
                                \item Sociedad 
                                \item Proveedores 
                                \item Colaboradores
                            \end{itemize}
                    \end{itemize}
                    & 
                    \begin{itemize}
                        \item Corto plazo (día / día) 
                        \item Se crea el gerente de cada área con su equipo
                        \item Es de conocimiento interno y específico.
                    \end{itemize}
                    \\ 
                \hline
           \end{tabular}
        \end{center}
\end{itemize}


%----------------------------------------------------------------------------------------
\subsection{Planeación estratégica}

Pasos para la planeación estratégica: 
\begin{enumerate}
    \item Análisis del entorno:
        \begin{itemize}
            \item FODA 
            \item Diamante de Michael Porter
        \end{itemize}

    \item Valores: 
        \begin{itemize}
            \item Diferencias entre principios y valores.
            \item Principios son: 
                \begin{itemize}
                    \item Universales: aplican a todos.
                    \item Atemporal: no se corroe a través del tiempo.
                    \item Objetivos: actúan independientemente de las personas, independiente a nuestra opinión.
                \end{itemize}
            
            \item Son los principios que nosotros valoramos:
                \begin{itemize}
                    \item Son dependientes a las creencias y opiniones de los individuos.
                    \item Hay principios que valoramos gerencialmente. Por ejemplo la confianza.
                    \item Los valores son de dominio público.
                \end{itemize}
            
            \item Preguntas:
                \begin{itemize}
                    \item \textbf{¿}Cuáles son las guías para la toma de decisiones\textbf{?}: son los valores, qué es lo que valoramos..
                    \item \textbf{¿}En dónde observaremos que vivimos nuestros valores\textbf{?}: pruebas que esto pasa.
                    \item \textbf{¿}Por cuáles valores somos reconocidos como organización\textbf{?}: 
                \end{itemize}
            
            \item Ejemplos de preguntas:
                \begin{itemize}
                    \item La limpieza se puede observar en $\langlerangle{\text{donde}}$ 
                \end{itemize}
            
            \item Somos los principios que caloramos como organización a la luz de los cuales tomamos las decisiones ejecutamos nuestra acciones y en base a los cuales construimos las relaciones con stakeholders. Responden a las preguntas filtro.
        \end{itemize}
\end{enumerate}

