\section{Continuación de clase pasada}
\begin{itemize}
    \item \textbf{¿}Porqué invertir (tiempo y dinero, etc.) en desarrollo de las herramientas, procesos y sistemas de administración\textbf{?}
        \begin{itemize}
            \item Para maximizar utilidad 
            \item Porductividad 
            \item Aumentar la eficacia de la operación de los negocios.
        \end{itemize}
\end{itemize}



%%%%%%%%%%%%%%%%%%%%%%%%%%%%%%%%%%%%%%%%%%%%%%%%%%%%%%%%%%%%%%%%%%%%%%%%%%%%%%%%%%%%%%%%%%
\section{Administración científica}
%----------------------------------------------------------------------------------------
\subsection{Frederick Taylor} 
\begin{itemize}
    \item El padre de la administración científica: Frederick Taylor.
    \item F. Taylor aplicó el \textbf{método científico} a la gestión organizacional. 
    \item Propuso la hipótesis que la productividad dependía de la remuneración que se les daba a los empleados productivos. La producción era proporcional a la remuneración.
    \item Con forme el tiempo transcurrió, la remuneración era un factor influyente pero no la principal razón.
    \item Productividad es maximizada por: 
        \begin{enumerate}
            \item Aumento salarial 
            \item Especialización de los colaboradores
        \end{enumerate}
    
    \item A partir de los aportes a la administración es una ciencia también.
\end{itemize}


%----------------------------------------------------------------------------------------
\subsection{Max Weber}
\begin{itemize}
    \item Era seguidor de Taylor y era economista; Se le conoce como el gran economista y administrador.
    \item Propuso sistemas organizados que potencien la especialización.
    \item El ya no creía en la especialización individual si no que en la especialización colectiva, a este término se le delegaba la burocracia.
\end{itemize}


%----------------------------------------------------------------------------------------
\subsection{Lilian Frank Grilbeth}
\begin{itemize}
    \item Aportes en psicología a las prácticas administrativas.
\end{itemize}


%----------------------------------------------------------------------------------------
\subsection{Gantt}
\begin{itemize}
    \item Propone el modelo de planeación de las gráficas de Gantt, que permite gestionar mucho mejor la operatividad de la compañía.
\end{itemize}

%%%%%%%%%%%%%%%%%%%%%%%%%%%%%%%%%%%%%%%%%%%%%%%%%%%%%%%%%%%%%%%%%%%%%%%%%%%%%%%%%%%%%%%%%%
\section{Administración moderna}
%----------------------------------------------------------------------------------------
\subsection{Henry Fayol}
\begin{itemize}
    \item El padre de la administración moderna: Henry Fayol.
    \item El hace dos grandes aportes:
        \begin{enumerate}
            \item  Identificó las cuatro funciones administrativas medulares:
                \begin{enumerate}
                    \item Planear
                    \item Organizar
                    \item Dirigir 
                    \item Controlar 
                \end{enumerate}
                \begin{itemize}[label=\#]
                    \item Es un proceso, primero planeo, después me organizo, etc.
                    \item Originalmente eran cinco funciones administrativas medulares:
                        \begin{enumerate}
                            \item Planeación 
                            \item Organización 
                            \item Integración del personal 
                            \item Dirección 
                            \item Control 
                        \end{enumerate}
                        \begin{itemize}[label=\#]
                            \item En la clase trabajaremos con cuatro, la integración del personal y la dirección se unen. Dentro de dirección esta integración personal y el liderazgo.
                        \end{itemize}
                \end{itemize}
            
            \item Su segundo gran aporte fueron los 14 principio de Fayol:
                \begin{itemize}
                    \item División del trabajo: permite la especialización.
                    \item Espírito de cuerpo: visión compartida.
                    \item Estabilidad loboral: trabajo a futuro, reduce la incertidumbre del empleado. Si hay mucha rotación de personal se baja l productividad.
                    \item Equidad: tratar por igual a los colaboradores.
                    \item Autoridad y responsabilidad: propone los primeros organigramas de los negocios. Si la (Autoridad > Responsabilidad) hay abuso de la autoridad, si la ( Autoridad < Responsabilidad ) se sobre carga de culpa al gerente y no le delegan el poder de tomar decisiones. Lo ideal es (Responsabilidad == Autoridad).
                    \item Iniciativa: reconoce la capacidad del empleado proponer ideas y ser escuchado.
                    \item Unidad de mando: un jefe por equipo.
                \end{itemize}
        \end{enumerate}
\end{itemize}


%----------------------------------------------------------------------------------------
\subsection{Wilfredo Pareto}
\begin{itemize}
    \item Si yo hago 200 actividades en el negocio, identifica el 20\% de esto que te permitan llegar al 80\% o más de los resultados. 
    \item Es decir especialízate en ese \textbf{20\%} de actividades, hoy en día se le delega el nombre ``menos come más''.
\end{itemize}


%%%%%%%%%%%%%%%%%%%%%%%%%%%%%%%%%%%%%%%%%%%%%%%%%%%%%%%%%%%%%%%%%%%%%%%%%%%%%%%%%%%%%%%%%%
\section{Administracón de conductas o psicología}
%----------------------------------------------------------------------------------------
\subsubsection{Hugo Münstemberg}    
\begin{itemize}
    \item El padre de la administración de conductas o psicología: Hugo Münstemberg.
    \item Propuso el considerar el factor de estrés en el empleado.
    \item Propuso la consideración de el la salud emocional del empleado, propone cuidar la estabilidad emocional, por ejemplo al pirámide de Maslow que es un sistema de motivación y conductas, la estabilidad emocional es proporcional a la productividad.
    \item Münstemberg propone la integración de la consideración:
        \begin{enumerate}
            \item Mente
            \item Corazón 
            \item Cuerpo
        \end{enumerate}
\end{itemize}



%%%%%%%%%%%%%%%%%%%%%%%%%%%%%%%%%%%%%%%%%%%%%%%%%%%%%%%%%%%%%%%%%%%%%%%%%%%%%%%%%%%%%%%%%%
\section{Administración contemporánea}
%----------------------------------------------------------------------------------------
\subsection{Ejemplos de la administración contemporánea}
\begin{itemize}
    \item J. Collins: Prof. Standford:
        \begin{itemize}
            \item From good to great 
            \item Great by choice 
        \end{itemize}
    
    \item G. Kawasaki: Emprendimiento y liderazgo
        \begin{itemize}
            \item El arte de empezar 
        \end{itemize}
    
    \item Mary Parker: El buen modelaje de lo jefes impacta posivo.
    \item Fred Kofman: Economista, integra filosofía a la gestión administrativa.
    \item Jim Senegal: Cosco CEO.
    \item Clayton Christensen: Profesor de Harvard.
    \item Edward Demin: Padre de la calidad total, fue un norteamericano experto en negocio pero sobre todo especializado en calidad, su gran aporte son los círculos de calidad total; él sostenía que se debía verificar la calidad no sólo al final si no en cada etapa de la producción. 
    \item Philip Kotler: Padre del marketing, aporta las cinco ``p's'' del marketing.
    \item Peter Drucker: Padre de la planeación, aporta que antes de todo se debe planear.
    \item Michael Porter: Padre de la estratégia.
\end{itemize}
\begin{itemize}[label=\#]
    \item \textbf{NO} hay padre de la administración contemporánea, sólo padres de especialidades de la administración contemporánea.
\end{itemize}
