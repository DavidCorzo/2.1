\section{Recordatorios}
\begin{itemize}
    \item Parcial 24 de febrero.
    \item Quiz de capítulo \# 1.
    \item Avanzaremos en contenido.
\end{itemize}


%%%%%%%%%%%%%%%%%%%%%%%%%%%%%%%%%%%%%%%%%%%%%%%%%%%%%%%%%%%%%%%%%%%%%%%%%%%%%%%%%%%%%%%%%%%%%%%%%%%
\section{Ciclo de vida de las empresas}
\begin{itemize}
    \item Etapas:
        \begin{enumerate}
            \item Start ups: rol de emprendedor. 
            \item Crecimiento: Rol de administración.
            \item Maturez Rol de administración.
        \end{enumerate}
    
    \item Las funciones del administrador: \textbf{Proceso vivo:}
        \begin{center}
            \begin{tikzpicture}[node distance = 3cm, auto]
                \node [block] (1) {$\underbrace{\text{  Planeación  }}_{\text{  Global management I }}$}; 
                \node [block,below of=1,right of=1] (2) {$\underbrace{\text{Organización}}_{Global management II}$}; 
                \node [block,below of=2, left of=2] (3) {Dirección};
                \node [block,below of=1, left of=1] (4) {Control}; 
                \path [line] (1) -- (2);
                \path [line] (2) -- (3);
                \path [line] (3) -- (4);
                \path [line] (4) -- (1); 
            \end{tikzpicture}   
        \end{center}
        \begin{enumerate}
            \item Planeación: se divide en dos, la planeación estratégica \& planeación operativa. \textbf{¿Qué?}
            \item Organización: Diseñar las estructuras operativas y equipos del negocio. \textbf{Crear el organigrama de la compañía}. \textbf{¿Cómo?}. \textbf{¿Quién ejerce qué función, qué puestos?}
            \item Dirección: En esta función se contrata a las personas.
                \begin{enumerate}
                    \item Integración de personal (contratación)
                    \item Liderazgo
                \end{enumerate}
            \item Control: ¿Cómo lo estamos haciendo?, Sistema de verificación/ejecución y control de la compañía.
        \end{enumerate}
    
    \item Las dos principales funciones:
        \begin{center}
           \begin{tabular}{| p{5cm} | p{5cm} | }
               \hline
                    Planeación Estratégica & Planeación Operativa    \\
               \hline
                    \begin{itemize}
                        \item Visión de $\underbrace{\text{  Largo plazo  }}_{\text{Tendencias, oportunidades de negocio y innovación}}$: Poder visualizar y especular con más precisión usando herramientas como proyecciones. Va a tomar tiempo poder desarrollar destrezas administrativas. Dónde quiero ver a mi empresa en 15 años.
                        \item Amplio impacto en toda la organización: afecta a todos los miembros de la compañía.
                        \item Se diseña en la línea de fuego gerencial de la compañía.
                    \end{itemize} & 

                    \begin{itemize}
                        \item Visión de Corto Plazo: donde quiero ver a mi empresa en 1 año.
                        \item Enfoque en la ejecución del día a día: 
                        \item La diseñan los gerentes del área alineados con la planeación estratégica.
                        \item Dominio público
                    \end{itemize} 
                    \\ 
                \hline
           \end{tabular}
        \end{center}
        \begin{itemize}[label=\#]
            \item Tom Petters \& Peter Bucker, son los padres de la administración contemporánea.
            \item Ante un contexto cambiante era fundamental un plan de negocios.
            \item Es importante saber qué puesto está a \textbf{cargo} de las decisiones.
        \end{itemize}
        \begin{center}
            \begin{tikzpicture}[node distance = 2cm, auto]
                \node [block] (1) {$\underbrace{\text{Accionistas}}_{\text{  Propietarios  }}$};
                \node [block,below of=1] (2) {El consejo de administración o junta directiva};  
                \node [block,below of=2] (3) {CEO o Gerente general}; 
                \path [line] (1) -- (2);
                \path [line] (2) -- (3);
                % \rule{16cm}{1pt}
                \node [block,below of=3] (4) {Línea de fuego}; 
                % \rule{16cm}{1pt}
                \path [line] (3) -- (4);
                
                % --------------------------------------------------------------------------------------------
                
                
                % --------------------------------------------------------------------------------------------
                
                \node [block, below of=4] (7) {Gerente de calidad};  
                \node [block, right of=7] (8) {Gerente financiero};  
                \node [block, right of=8] (9) {Gerente de talento}; 
                
                
                % --------------------------------------------------------------------------------------------
                
                \node [block, left of=7] (6) {Gerente de ventas};
                \node [block,below of=6] (6.1) {Gerente de tiendas varios};   
                \node [block,left of=6] (5) {Gerente de producción};
                % --------------------------------------------------------------------------------------------
                
                \path [line] (4) -- (5);
                \path [line] (4) -- (6);
                \path [line] (4) -- (7);
                \path [line] (4) -- (8);
                \path [line] (4) -- (9);
                 \path [line] (6) -- (6.1);
                 % --------------------------------------------------------------------------------------------
            \end{tikzpicture}
        \end{center}
        \begin{itemize}
            \item Los propietarios son los accionistas.
            \item El consejo de administración o junta directiva la hacen lso accionistas.
            \item El CEO es el orquestador y de él se delegan cargos gerenciales para ayudarlo a orquestrar mejor.
                \begin{itemize}
                    \item Gerente de producción: el que se encarga de gestiones de producción.
                    \item Gerente de ventas: Permite un insider look a la situación de las tiendas.
                \end{itemize}
            
            \item Gerente de calidad: tiene que ser peer de el gerente de ventas, no puede ser su subordinado.
            \item Gerente de distribución: se encarga de las distribuciones.
            \item Gerente financiero: el que toma las decisiones estratégicas en base de la capacidad de recursos de la compañía.
            \item Gerente de talento: la persona que define o diseña la estratégia para gestionar a la gente.
        \end{itemize}
        \begin{itemize}[label=\#]
            \item La primera línea de fuego junto con el CEO, son los que diseñan la planeación estratégica del negocio.
            \item El gerente de negocio tiene que saber perfectamente la función de cada gerente.
        \end{itemize}
    
    \item Por ejemplo:
        \begin{center}
           \begin{tabular}{ | p{5cm} | p{5cm} | }
               \hline
                    \multicolumn{2}{|c|}{San Martín} \\
               \hline
                    Core (corazón) & Soporte \\ 
               \hline
                    \begin{itemize}
                        \item Proveedores $\rightarrow$ producción 
                        \item $\underbrace{\text{  Comercial  } }_{\text{  Estrategia comercial \& Estrategia de precios  }}$ $\rightarrow$ Puntos de venta (tiendas)
                        \item Logística $\rightarrow$ Distribución
                    \end{itemize} &
                    \begin{itemize}
                        \item Financiera 
                        \item Talento 
                    \end{itemize} \\ 
                \hline
           \end{tabular}
        \end{center}
    
    \item Destreza:
        \begin{itemize}[label=\#]
            \item Se tiene que tener la capacidad de venderse a uno mismo.
            \item Hay que desarrollar la capacidad numérica.
        \end{itemize}
\end{itemize}




%%%%%%%%%%%%%%%%%%%%%%%%%%%%%%%%%%%%%%%%%%%%%%%%%%%%%%%%%%%%%%%%%%%%%%%%%%%%%%%%%%%%%%%%%%%%%%%%%%%
\section{Video de copa airlines}
\begin{itemize}
    \item Comenzamos por averiguar qué quieren los clientes. \textbf{Llegar a tiempo}.
    \item \emph{Citación:``vimos cuánto tiempo tomaba limpiar la nave con respecto a otras aerolíneas"}.
    \item Cascadear: filtrar los objetivos y evaluar qué estamos haciendo para llegar a esas metas.
    \item Aseguramos que cada persona esté trabajando para avanzar el cumplimiento de esos objetivos.
    \item \emph{Citación:``Lo podemos resolver aquí mismo"}
    \item \emph{Citación:``Antes todas las decisiones las tomaba una personas y esto retrasaba todo"}
    \item \emph{Citación:``Vamos en búsqueda de los pasajeros no lo esperamos"}
    \item Llegadas a tiempo, esa es nuestra meta corporativa.
\end{itemize}


%%%%%%%%%%%%%%%%%%%%%%%%%%%%%%%%%%%%%%%%%%%%%%%%%%%%%%%%%%%%%%%%%%%%%%%%%%%%%%%%%%%%%%%%%%%%%%%%%%%
\section{Análisis del video}
\begin{itemize}
    \item La meta corporativa:
        \begin{center}
            \begin{tikzpicture}[node distance = 3cm, auto]
                \node [block] (1) {Meta corporativa}; 
                \node [decision,right of=1] (2) {Meta Pública}; 
                \node [decision,below of=2, right of=1] (3) {Meta interna}; 
                
                % --------------------------------------------------------------------------------------------
                \path [line] (1) -- (3);
                % --------------------------------------------------------------------------------------------
                \node [block,below of=1] (4) {Planeación estratégica que tiene 8 pasos}; 
                \node [block,left of=4] (5) {Visión};
                \node [block,below of=5] (6) {\small Análisis de la situación actual}; 
                \node [block,below of=4] (7) {Misión}; 
                \node [block,right of=7] (8) {Valores};
                % \node [block,below of=4] (9) {Estratégia de negocio};  
                
                % --------------------------------------------------------------------------------------------
                \path [line] (1) -- (2);
                \path [line] (1) -- (3);
                \path [line] (1) -- (4);
                \path [line] (4) -- (5);
                \path [line] (4) -- (6);
                \path [line] (4) -- (7);
                
            \end{tikzpicture}
        \end{center}
        
        \begin{itemize}[label=\#]
            \item Tenemos que entender cómo medir el cumplimiento del objetivo.
            \item Análisis de la situación actual, se contactan a los steakholders.
        \end{itemize}
    
    \item En el caso de Copa Airlines:
        \begin{itemize}
            \item La meta pública es ``llegar a tiempo''. La meta estratégica, amarrado a esto venían 3 metas operativas:
                \begin{enumerate}
                    \item Con las maletas, agilizar.
                    \item Con los papeles, agilizar. \emph{\textbf{Ejemplo: }Sao Paulo Brasil, cuando se le olvidaron los papeles y no pudieron desbordar a los pasajeros.}
                    \item Con la comida, agilizar. 
                \end{enumerate}
            \item La meta interna es \textbf{medible}, aquí hay KPI(Key Performance Indicators); 50\% $\rightarrow$ 98\% --- en 10 años.
            \item La visión: Ser la línea aéra \#1; conexiones L.A.
            \item Misión: Transporte aéreo de pasajeros y cargo del aeropuerto de Panamá- a LA.
            \item Valores:
                \begin{itemize}
                    \item Clientes 
                    \item Puntualidad 
                    \item Colaboradores 
                    \item Excelencia 
                    \item Logro de resultados
                \end{itemize}
        \end{itemize}
\end{itemize}
