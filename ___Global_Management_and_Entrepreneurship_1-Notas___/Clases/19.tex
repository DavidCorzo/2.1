\section{Video de Autor de 7 hábitos altamente efectivos}

%----------------------------------------------------------------------------------------
\subsection{\url{https://www.youtube.com/watch?v=ytjeyqlTjyY}}
\begin{itemize}
    \item Las metas son inalcanzables:
        \begin{itemize}
            \item $r^2$ realista y realizable
        \end{itemize}
    
    \item Objetivos no claros:
        \begin{itemize}
            \item Medible 
            \item Verificable 
        \end{itemize}
    
    \item Metas no priorizadas:
        \begin{itemize}
            \item No determinaban bien el orden de metas 
        \end{itemize}
    
    
    \item Los empleados no se sienten involucrados:
        \begin{itemize}
            \item \pregunta{por qué alguien no se siente involucrados} \pregunta{Cómo mi trabajo contribuye al logro de las metas de la organización} 
        \end{itemize}
        
    \item Los empleados no entienden la forma de redacción de la junta directiva: 
        \begin{itemize}
            \item Tener claro qué voy a comunicar, asegurarme que lo entiendan y tener cuidado cuándo lo comunico.
            \item 1. Qué, 2. Cuándo, 3. Cómo 
            \item La comunicación es clave.
            \item No has hecho un plan operativo con tu equipo. 
        \end{itemize}
    
    \item Los cuatro salarios:
        \begin{itemize}
            \item Intelectual 
            \item Sentimental 
            \item Espiritual
        \end{itemize}
\end{itemize}


%%%%%%%%%%%%%%%%%%%%%%%%%%%%%%%%%%%%%%%%%%%%%%%%%%%%%%%%%%%%%%%%%%%%%%%%%%%%%%%%%%%%%%%%%%
\section{Planeación operativa}
Es el segundo paso de la planeación, es mucho más específica, más específica en las áreas clave. 
\begin{enumerate}
    \item Identificar las áreas clave.
        \begin{itemize}
            \item Objetivos globales 
            \item Estrategia menor 
            \item Táctica 
        \end{itemize}
    
    \item Política 
    \item Reglas 
    \item Procedimiento 
    \item Plan de contingencia 
    \item Presupuesto 
\end{enumerate}

En la planeación estratégica , vimos que los objetivos globales responden a la pregunta \pregunta{qué} deseamos lograr estratégicamente en cada área clave. Ahora en la planeación operativa buscaremos las estrategias menores y tácticas que nos permitan asegurar el lograr los objetivos 
\newline \newline 
Las estrategias menores responden a la pregunta: \pregunta{Cómo lograremos el objetivo} Las estrategias menores, son sumamente específicas y orientan la identificación de las acciones tácticas. Son muy concretas y altamente alineadas con el objetivo global. \newline \newline 
Las tácticas responden a la pregunta \pregunta{Cómo lograremos la estrategia menor} Las tácticas son acciones sumamente específicos que al ejecutarse aseguran el logro de estrategia menor para lograr el objetivo. Son acciones altamente predictivas (80/20)
\newline \newline 

% \subsection{Ejemplo de COPA}
% \begin{center}
%     \begin{tabular}{ |c|l|l|l| }
%         \hline
%             1. Áreas clave & 2. Objetivos globales & 3. Estrategia menor & 4. Táctica \\
%         \hline
%             Ventas  y reservaciones &
%             \begin{enumerate}
%                 \item 
%             \end{enumerate} & 
%             \begin{enumerate}
%                 \item 
%             \end{enumerate}
%     \end{tabular}
% \end{center}


%%%%%%%%%%%%%%%%%%%%%%%%%%%%%%%%%%%%%%%%%%%%%%%%%%%%%%%%%%%%%%%%%%%%%%%%%%%%%%%%%%%%%%%%%%
\section{Políticas y reglas}
Las políticas y reglas de un Área clave, son guías para la toma de decisiones que facultan a los equipos operativos para tomar decisiones en la ejecución de sus rateas diarias. \newline \newline 
Las políticas y reglas son elaboradas por lóderes/jefes de los equipos de acuerdo a necesidades identificadas en sus equipos de trabajo para agilizar la toma de decisiones. \newline \newline 
Puede haber tantas como sean necesarias para poder delegar la toma de decisiones en los supordinados, asegurando un mínimo de errores cometidos en las mismas. \newline \newline 
Son súmamente específicas, empoderan y descentralizan la toma de decisiones (velocidad en la toma de decisiones para resolver situaciones eliminando los procesos burocráticos que podrían hacer lenta la ejecución).\newline \newline 
Diferencia entre política y regla:
\begin{itemize}
    \item \termdefinition{Las políticas}{es flexible al ofrcer rangos para la toma de decisiones.} 
    \item \termdefinition{Las reglas}{Son rígidas. No hay flexibiliad de ningun tipo. } 
\end{itemize}

Cada área clave tiene su plan operativo que consiste en: políticas, reglas, plan de contingencia, presupuesto.
