\section{Globalización y tendencias}
\begin{itemize}
    \item ``Donde hay una empresa de éxito, alguien tomó alguna vez una decisión valiente''
    \item Técnicas de management empresarial, ``basta de management empresarial''. 
        \begin{itemize}
            \item Retos empresariales
            \item Globalización 
        \end{itemize}
        \begin{figure}[htbp]
            \centering
            %\includegraphics[width=6cm]{2020_01_08_retos_empresariales.jpg}
            %\includegraphics[width=6cm]{2020_01_08_globalizacion.jpg}
            \caption{}
            \label{}
        \end{figure} 
\end{itemize}

%%%%%%%%%%%%%%%%%%%%%%%%%%%%%%%%%%%%%%%%%%%%%%%%%%%%%%%%%%%%%%%%%%%%%%%%%%%%%%%%%%%%%%%%%%%%%%%%
\section{Términos nuevos}
\begin{itemize}
    \item Recilencia: capacidad de afrontar retos.
    \item Posicionamiento: la posición de la marca en el mercado.
    \item Steakholders: todos los interesados o forman parte del proyecto.
\end{itemize}


\subsection{Grupos significativos}
\begin{itemize}
    \item Innovación y las buenas ideas.
    \item Competencia: es una realidad que no estas solos en el mercado, el compentidor puede ser una plataforma para generar alianzas (no siempre).
    \item Liderazgo: \emph{\textbf{Ejemplo: }El CEO lo cambian, cambian los valores, tienden a bajar el valor de las acciones.}; dime con quién andas y te diré quien eres. Esto va ligado con la reputación, la primera marca que construimos es la de nuestro nombre. 
    \item Clientes y mercado: el reto más grande no está en adquirir clientes si no en retenerlos.
    \item Gente y organización \& habilidades gerenciales: 
    \item Rentabilidad y crecimiento: 
\end{itemize}

%%%%%%%%%%%%%%%%%%%%%%%%%%%%%%%%%%%%%%%%%%%%%%%%%%%%%%%%%%%%%%%%%%%%%%%%%%%%%%%%%%%%%%%%%%%%%%%%

\section{Tarea}
\begin{itemize}
    \item Entrevistar a dos empresarios, preguntar: cuales son los dos retos más importantes que estás enfrentando en la empresa.
\end{itemize}

%%%%%%%%%%%%%%%%%%%%%%%%%%%%%%%%%%%%%%%%%%%%%%%%%%%%%%%%%%%%%%%%%%%%%%%%%%%%%%%%%%%%%%%%%%%%%%%%

\section{Objetivos estratégicos en un negocio}
\begin{itemize}
    \item Rentabilidad y sostenibilidad (crecimiento / nuevos mercados)(valor) 
    \item Impacto (satisfacción al cliente / social)
    \item Reconocimiento / éxito
    \item Posicionamiento 
    \item Autorealización 
    \item Mejora continua 
    \item Crear conocimiento 
    \item Proveer oportunidades laborales
\end{itemize}

%%%%%%%%%%%%%%%%%%%%%%%%%%%%%%%%%%%%%%%%%%%%%%%%%%%%%%%%%%%%%%%%%%%%%%%%%%%%%%%%%%%%%%%%%%%%%%%%

\section{Los resultados que buscan las organizaciones}
\begin{enumerate}
    \item Desempeño superior sostenible, cultura ganadora, clientes profundamente leales, contribución distintiva.
    \item Una cosa que no se puede copiar es la cultura organizacional, la cultura empresarial ganadora.
    \item Las empresas ganadoras tienen clientes muy leales.
    \item Clientes con buenas características son los que vienen acompañados o referidos.
    \item Contribución distintiva: capacidad de transcender y generar impacto en situaciones políticas, económicas.
\end{enumerate}
