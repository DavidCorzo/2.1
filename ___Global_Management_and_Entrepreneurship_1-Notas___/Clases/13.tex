\section{Planeación estratégica}
\begin{enumerate}
    \item Análisis del entorno 
    \item Valores
    \item Misión 
    \item Visión  
    \item Estrategia global de competencia 
    \item Plazos 
    \item Objetivos globales 
    \item Áreas clave 
\end{enumerate}


\subsection{Misión}
\begin{itemize}
    \item \termdefinition{Misión}{La misión es parte de la planeacion estratégica que explica el propósito de existir de las organizaciones, qué ofrecen a sus cloentees, qué los hace únicos en el mercado y pa promesa para sus clientes: \pregunta{qué pueden esperar de trabajar y establecer una relacion con la organización} Respinde a las preguntas de filtro. Es el momento presente de la organización} 
    \item Misión:
        \begin{itemize}
            \item ¿Quiénes somos?
            \item ¿Cuál es la promesa de servicio?
            \item ¿Qué ofrecemos que nos hace únicos?
            \item ¿Quiénes son nuestros clientes?
        \end{itemize}
\end{itemize}


%----------------------------------------------------------------------------------------
\subsection{Visión}
\begin{itemize}
    \item \termdefinition{Visión}{Es la planeación estratégica que expresa el legado de la organización, su proyección a futuro y en qué desean consolidarse como referente o fuente de inspiración para la industria o sociedad}
    \item Preguntas de la visión:
        \begin{itemize}
            \item \pregunta{Hacia dónde vamos} 
            \item \pregunta{En qué queremos ser un referente} 
            \item \pregunta{Cómo nos cvemos en el futuro} 
            \item \pregunta{Cuál será nuestro legado} 
        \end{itemize}

\end{itemize}


%----------------------------------------------------------------------------------------
\subsection{Estrategia global competitiva}
\begin{itemize}
    \item Michael Porter: padre de la competitividad, define la estratégia.
    \item \termdefinition{Estratégia}{Factor diferenciador en el mercado que te otorga la ventaja competitiva} 
    \item Estrategias de MP:
        \begin{itemize}
            \item Liderazgo en costos: 
                \begin{itemize}
                    \item El liderazgo en costos es opuesto a diferenciación. 
                \end{itemize}
                \begin{enumerate}
                    \item Producto: Básico estandarizado. Hacen un producto básico estandarizado, como ejemplo de la estrategia de liderazgo en costos ilustra como se compra en altos volúmenes a precios bajos. Producción al por mayor.
                    \item Estrategia de ingreso: Altos volúmenes a previos bajos.Altos volúmenes a precios bajos maximiza la utilidad:
                        \[
                        \text{ Utilidad }= \text{ Ingreso total } - \text{ Costo total }
                        \]
                        \[
                        \pi = (P\cdot Q) + (CV+CF)
                        \]
                    
                    \item Puntos de venta son masivos y tercerizados.
                    \item Estrategia de mercado: baja inversión intensa en alianzas promocionadas.
                \end{enumerate}
                \begin{itemize}[label=\#]
                    \item Ejemplos como lapiceros BIC, sodas, farmacias, fármacos de cierta categoría.
                \end{itemize}

            \item Enfoque: 
                \begin{enumerate}
                    \item Producto: básico + valor agregado, es un bien de status.
                    \item Estrategia de mercadeo: vende estatus / prestigio / selecto.
                    \item Estrategia de ingresos: precio altos y bajas cantidades.
                    \item Ountos de venta: tiendas propias o distribuidores autoizados.
                \end{enumerate}
            \item Diferenciación: 
        \end{itemize}
        \begin{itemize}[label=\#]
            \item 
        \end{itemize}
\end{itemize}
