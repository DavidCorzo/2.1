\section{Productividad}
\begin{itemize}
    \item Eficacia: lograr los resultados \textbf{esperados}. 
    \item Eficiencia: los que demuestran resultados \textbf{esperados} con la mínima cantidad de recursos, \textbf{exceden expectativas}.
    \item Efectividad: es el logro de la \textbf{eficiencia} sostenida a través del tiempo.
\end{itemize}

\subsection{Las estrategias mencionadas anteriormente}
\begin{itemize}
    \item Está bien cualquiera de las estrategias de productividad
\end{itemize}

%%%%%%%%%%%%%%%%%%%%%%%%%%%%%%%%%%%%%%%%%%%%%%%%%%%%%%%%%%%%%%%%%%%%%%%%%%%%%%%%%%%%%%%%%%%%%%%%
\section{Factores}
\begin{itemize}
    \item Hay factores que no puedo controlar:
        \begin{itemize}
            \item Tenemos que entender como empresarios de \textbf{disernir} qué factores controlamos y cuáles no:
                \begin{enumerate}
                    \item Económicos 
                    \item Ecológicos 
                    \item Políticos y legales 
                    \item Tecnología
                    \item Éticos 
                    \item Sociales y culturales 
                \end{enumerate}
            
            \item \emph{\textbf{Ejemplo: }Los factores externos como aquellos de un cambio de precio en la industria de transporte.}
            \item \textbf{Nos preguntamos:} ¿Se dispara el precio de la leche? las industrias que emplean la leche como derivada de su producto o como complemento de su producto.
            \item \emph{\textbf{Ejemplo: }Industria de cemento deja de producir cemento} 
        \end{itemize}
    
    \item Factores internos que puedo controlar: 
        \begin{itemize}
            \item Los puedo controlar.
            \item Estrategia: se debe saber qué tan estable es el territorio en el que se está haciendo negocios.
            \item Los países que son inestables no atraen inversión extrangera.
        \end{itemize}
\end{itemize}

%%%%%%%%%%%%%%%%%%%%%%%%%%%%%%%%%%%%%%%%%%%%%%%%%%%%%%%%%%%%%%%%%%%%%%%%%%%%%%%%%%%%%%%%%%%%%%%%
\section{Estrategia}
\begin{itemize}
    \item Steakholders:
        \begin{itemize}
            \item Son clientes, accionistas, inversionistas, proveedores, etcétera. Cualquier persona interesadas en la empresa. Es importante tener una buena relación con dichos steakholders.
        \end{itemize}
    
    \item Tendencias: 
        \begin{itemize}
            \item Pueden ser conductas, actitudes, preferencias que puedan indicar hacia donde se mueve el consumidor.
            \item Las tendencias de la demanda de los consumedores se abre una oportunidad de negocios que se deben aprovechar.
        \end{itemize}
    
    \item Conosza el entorno en el que opera su empresa: se necesita saber el entorno.
        \begin{itemize}
            \item \textbf{Nos preguntamos:} ¿qué servicios o productos vende?
            \item \textbf{Nos preguntamos:} ¿a que industria permanece?
        \end{itemize}
        
    \item \emph{\textbf{Definición de ``Industria:":} El conjunto de empresas que se dedican a un mismo grupo de productos o servicios.}
\end{itemize}

%%%%%%%%%%%%%%%%%%%%%%%%%%%%%%%%%%%%%%%%%%%%%%%%%%%%%%%%%%%%%%%%%%%%%%%%%%%%%%%%%%%%%%%%%%%%%%%%

\section{Manejan información correcta de la fuente correcta}
\begin{itemize}
    \item Las tendencias nor permiten detectar oportunidades de negocios.
    \item \emph{\textbf{Ejemplo: }China crecerá aun más, esta es una tendencia mal redactada por que no tiene cifras.}
    \item Una tendencia se manifiesta en cifras, no en declaraciones.
\end{itemize}
