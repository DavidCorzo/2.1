\documentclass{article}

\usepackage{generalsnips}
\usepackage{calculussnips}
\usepackage[margin = 1in]{geometry}
\usepackage{pdfpages}
\usepackage[spanish]{babel}
\usepackage{amsmath}
\usepackage{amsthm}
\usepackage[utf8]{inputenc}
\usepackage{titlesec}
\usepackage{xpatch}
\usepackage{fancyhdr}
\usepackage{tikz}
\usepackage{hyperref}
\title{Resumen Cap 6 UPG}
\date{2020 April 18, 12:47PM}
\author{David Gabriel Corzo Mcmath}

\begin{document}
\maketitle
%%%%%%%%%%%%%%%%%%%%%%%%%%%%%%%%%%%%%%%%%%%%%%%%%%%%%%%%%%%%%%%%%%%%%%%%%%%%%%%%%%%%%%%%%%%%%%%%%%%%%%%%%%%%%%%%%%%%%%%%%%%%%%%%%%%%%%%%%%%%%%

\section*{La toma de decisiones}
\begin{itemize}
    \item \termdefinition{Toma de decisiones}{La selección de un curso de acción entre varias alternativas, es el nucleo de la planeación.} 
    \item Se considera la principal tarea de los gerentes. 
    \item Cada decisión debe orientarse a otros planes. 
\end{itemize}

%%%%%%%%%%%%%%%%%%%%%%%%%%%%%%%%%%%%%%%%%%%%%%%%%%%%%%%%%%%%%%%%%%%%%%%%%%%%%%%%%%%%%%%%%%
\section{Importancia y limitaciones de la toma de decisiones racional}
\begin{itemize}
    \item \textbf{La toma de decisiones es el núcleo de la planeación.}
    \item Proceso que conduce a decidir.
        \begin{enumerate}
            \item Establecer las premisas.
            \item Identificar las alternativas. 
            \item Evaluar las alternativas en términos de la meta que se busca.
            \item Elegir una alternativa, es decir, tomar una decisión. 
        \end{enumerate}
    
    \item \textbf{La toma de decisiones es uno de los pasos de planeación}.
\end{itemize}



%----------------------------------------------------------------------------------------
\subsection{Racionalidad en la toma de decisiones}
\begin{itemize}
    \item Decidir con \textbf{racionalidad}: personas que intentan alcanzar metas que \textbf{no pueden lograrse sin acción}, entienden los cursos de acción que emplearan para llegar a las metas con \textbf{limitaciones existentes}. Pueden \textbf{evaluar alternativas}. 
    \item Personas con racionalidad total piensan:
        \begin{enumerate}
            \item No se puede cambiar el pasado entonces todas las decisiones deben aplicarse hacia el futuro, esto siempre supone incertidumbres.
            \item Es difícil identificar todas las alternativas posibles para llegar a una meta, en especial cuando la decisión supone hacer algo completamente nuevo. 
            \item Casi nunca pueden analizarse todas las alternativas, incluso con una computadora. 
        \end{enumerate}
\end{itemize} 


%----------------------------------------------------------------------------------------
\subsection{Toma de decisiones 10-10-10}
\begin{itemize}
    \item \termdefinition{Regla 10-10-10}{Promueve la idea de Suzy Welch, que sostiene analizar las consecuencias de las decisiones que tomamos, por ejemplo: la decisión que tomó le traerá consecuencias los próximos ¿10 minutos, 10, meses o 10 años?}
    \item Es una herramienta para los gerentes, para hacerlos pensar en las consecuencias a corto, mediano y largo plazo.
\end{itemize}


%----------------------------------------------------------------------------------------
\subsection{Racionalidad limitada o ligada}
\begin{itemize}
    \item Un administrador debe estar consciente de la racionalidad limitada o ligada, y además, aceptarla; es decir, de las limitaciones de información, tiempo y certidumbre que restringen la racionalidad, aun si el gerente se esfuerza por ser completamente racional. 
    \item \termdefinition{Sufisfacer}{Elegir, dadas las circunstancias, un curso de acción lo suficientemente bueno, aunque no del todo satisfactorio} 
    \item Equilibrio entre la racionalidad y la garantía. 
\end{itemize}



%%%%%%%%%%%%%%%%%%%%%%%%%%%%%%%%%%%%%%%%%%%%%%%%%%%%%%%%%%%%%%%%%%%%%%%%%%%%%%%%%%%%%%%%%%
\section{Desarrollo de alternativas y el factor limitante}
\begin{itemize}
    \item Después de tener claras las premisas, se buscan alternativas. Entre estas alternativas se evalúan los factores limitantes. 
    \item \termdefinition{Factor limitante}{Algo que obstruye el logro de un objetivo deseado} 
    \item \termdefinition{El principio del factor limitante}{Buscar qué alternativas tienen menos factores limietantes para determinar cuál de todas tiene el mejor curso de acción.} 
\end{itemize} 


%----------------------------------------------------------------------------------------
\subsection{Heurística en la toma de decisiones}
\begin{itemize}
    \item Cuando parece haber demasiadas alternativas entre las cuales elegir, los directivos confían en sus propias reglas de decisión. Estas reglas se conocen como heurísticas y permiten hacer juicios complejos de manera más simple. 
    \item Las decisiones tomadas con heurística pueden variar según quién las tome. 
    \item Los gerentes deben ser conscientes de su propia heurística, de cómo puede sesgar sus decisiones e intentar compensarlas mediante un proceso de decisión exhaustivo.
    \item Los valores y las inclinaciones cognitivas de los altos directivos de las compañías se observan en las estrategias y la efectividad de las organizaciones. 
\end{itemize} 

%%%%%%%%%%%%%%%%%%%%%%%%%%%%%%%%%%%%%%%%%%%%%%%%%%%%%%%%%%%%%%%%%%%%%%%%%%%%%%%%%%%%%%%%%%
\section{Evaluación de alternativas}
\begin{itemize}
    \item Cuando ya se han determinado las alternativas se evalúan.
    \item Dos maneras de evaluar, la cuantitativa y la cualitativa. 
\end{itemize}

%----------------------------------------------------------------------------------------
\subsection{Factores cuantitativos y cualitativos}
\begin{itemize}
    \item  \termdefinition{Factores cuantitativos}{ aquéllos que pueden medirse en términos numéricos, como el tiempo o los varios costos fijos y operativos} 
    \item \termdefinition{Factores cualitativos}{son aquellos difíciles de medir numéricamente, como la calidad de las relaciones laborales, el riesgo de cambios tecnológicos o el clima político internacional, para evaluar estos factores se hace una escala de prioridades (calificarlos de acuerdo a su importancia, influencia, etc).} 
    \item \textbf{Los dos factores sin muy importantes.}
\end{itemize}

%----------------------------------------------------------------------------------------
\subsection{Análisis marginal}
\begin{itemize}
    \item \termdefinition{Análisis marginal}{Técnica que permite comparar el ingreso y el costo adicionales que surgen al aumentar la producción, en la evaluación de alternativas.} 
    \item Para maximizar: cuando ingreso y costo adicionales sean iguales (si el ingreso adicional de una mayor cantidad es superior a su costo adicional)
    \item  si el ingreso adicional de una mayor cantidad es menor a su costo adicional, puede obtenerse mayor utilidad al producir menos. 
\end{itemize}


%----------------------------------------------------------------------------------------
\subsection{Análisis de efectividad de costos}
\begin{itemize}
    \item Una mejoría del análisis marginal tradicional.
    \item \termdefinition{Análisis de costo-efectividad}{ análisis costo-beneficio, que busca el mejor índice entre el costo y el beneficio.} 
\end{itemize}


%----------------------------------------------------------------------------------------
\subsection{Seleccionar una alternativa: tres enfoques}
\begin{enumerate}
    \item Experiencia 
    \item Experimentación 
    \item Investigación y análisis
\end{enumerate}

\subsubsection{Experiencia}
\begin{itemize}
    \item La experiencia es beneficiosa pero no siempre garantiza desarrollar la intuición para saber qué hacer.
    \item Muchas personas no aprenden de sus errores, ello es claro en gerentes que nunca parecen adquirir el sano juicio requerido por la empresa moderna.
    \item Es una espada de dos filos.
    \item Depender de la experiencia pasada como guía para una acción futura puede ser peligroso porque:
        \begin{enumerate}
            \item  Muy pocas personas reconocen las razones que subyacen a sus errores o fracasos. 
            \item Las lecciones de la experiencia pueden ser completamente inaplicables a nuevos problemas: las buenas decisiones deben evaluarse frente a los sucesos futuros, en tanto que la experiencia pertenece al pasado.
        \end{enumerate}
    
    \item En cambio, si una persona analiza su experiencia con cuidado, más que seguirla a ciegas, y extrae de ella los motivos fundamentales de éxitos o fracasos, puede serle útil como base para el análisis de decisiones
\end{itemize}

\subsubsection{Experimentación}
\begin{itemize}
    \item Una forma evidente de decidir entre alternativas es probar una de ellas y ver qué ocurre.
    \item Probar las diversas alternativas y ver cuál es la mejor.
    \item  Probablemente la técnica experimental sea la más cara de todas.
    \item casi siempre se requiere cierta experimentación en el proceso de selección del curso de acción correcto.
\end{itemize}

\subsubsection{Invetigación y análisis}
\begin{itemize}
    \item enfoque que conlleva la resolución de un problema mediante su previa comprensión, lo que supone la búsqueda de las relaciones entre las variables, restricciones y premisas más decisivas que afectan la meta perseguida, es el enfoque de lápiz y papel (o mejor, de computadora e impresora) en la toma de decisiones. 
    \item Resolver un problema de planeación requiere desglosarlo en sus partes componentes y estudiar los diversos factores cuantitativos y cualitativos.
    \item Es más barato que la experimentación generalmente. 
    \item Usar modelos matemáticos.
\end{itemize}


%%%%%%%%%%%%%%%%%%%%%%%%%%%%%%%%%%%%%%%%%%%%%%%%%%%%%%%%%%%%%%%%%%%%%%%%%%%%%%%%%%%%%%%%%%
\section{Decisiones programadas y no programadas}
\begin{itemize}
    \item \termdefinition{Decisión programada}{Se utilizan para trabajos estructurados o rutinarios.} 
    \item \termdefinition{Decisión no programada}{Se emplean en situaciones no estructuradas, nuevas y mal definidas de naturaleza  no recurrente.} 
    \item En general, las decisiones estratégicas son de  tipo no programadas, ya que devienen de juicios subjetivos. 
    \item La mayoría de las decisiones no son del tipo completamente programadas ni del tipo no programadas: son una combinación de ambas.
    \item Mientras más alto el nivel las decisiones tienden a ser no programadas, mientras más bajo el nivel las decisiones tienden a ser programadas. 
\end{itemize}




%%%%%%%%%%%%%%%%%%%%%%%%%%%%%%%%%%%%%%%%%%%%%%%%%%%%%%%%%%%%%%%%%%%%%%%%%%%%%%%%%%%%%%%%%%
\section{Toma de decisiones en condiciones de certidumbre, incertidumbre y riesgo}
\begin{itemize}
    \item En una situación con certidumbre, las personas están razonablemente seguras de lo que ocurrirá cuando tomen una decisión; asimismo, la información está disponible y se considera confiable, además de que se conocen las relaciones de causa y efecto que le subyacen.
    \item En cambio, en una situación de incertidumbre las personas sólo tienen una escasa base de datos, no saben si los datos son confiables y están inseguros sobre si la situación puede cambiar o no, además de que no pueden evaluar las interacciones de sus diferentes variables.
    \item Para decidir con incertidumbre acudir a análisis de probabilidad, modelos matemáticos, intuición, buscar la \emph{mejor estimación}.
\end{itemize}


%%%%%%%%%%%%%%%%%%%%%%%%%%%%%%%%%%%%%%%%%%%%%%%%%%%%%%%%%%%%%%%%%%%%%%%%%%%%%%%%%%%%%%%%%%
\section{Creatividad e incertidumbre}
\begin{itemize}
    \item Creatividad $\displaystyle \neq $ Innovación.
    \item \termdefinition{Creatividad}{Habilidad y poder de desarrollar nuevas ideas.} 
    \item \termdefinition{Innovación}{ El uso de nuevas ideas.} 
\end{itemize}


%----------------------------------------------------------------------------------------
\subsection{Proceso creativo}
Consta de: 
\begin{enumerate}
    \item Escaneo inconsciente 
    \item Intuición 
    \item Percepción 
    \item Formulación lógica
\end{enumerate}

%
\subsubsection{Escaneo inconsiente}
Usa la inconsciencia para decidir o crear nuevas cosas, es de donde se origina la intuición.

%
\subsubsection{Intuición}
Lo que se intuye a partir de conocer el problema, La intuición requiere tanto tiempo para funcionar como que las personas encuentren nuevas combinaciones e integren diversos conceptos e ideas. Así, debe pensarse a fondo en el problema. Diversas técnicas promueven el razonamiento intuitivo, como la lluvia de idea

%
\subsubsection{Percepción}
Lo interesante es que la percepción se puede presentar cuando el pensamiento no está directamente enfocado en el problema que nos ocupa. Más aún, las nuevas percepciones pueden durar sólo unos minutos y los gerentes efectivos se benefician de tener papel y lápiz a la mano para tomar nota de sus ideas creativas.

(Primera formulación o redacción de lo que se ha creado en la intuición.)

%
\subsubsection{Formulación lógica o verificación}
La percepción necesita probarse mediante la lógica o el experimento


%----------------------------------------------------------------------------------------
\subsection{Lluvia de ideas (brain storming)}
\begin{itemize}
    \item Algunas técnicas se enfocan en interacciones de grupo y otras en acciones individuales; una de las más conocidas para facilitar la creatividad fue desarrollada por Alex F. Osborn, a quien se le ha llamado el Padre de la Lluvia de Ideas.
    \item El propósito de este enfoque es mejorar la solución de problemas al encontrar soluciones nuevas o atípicas.
    \item Reglas:
        \begin{itemize}
            \item No se critica ninguna idea 
            \item Cuanto más radical sean las ideas, mejor.
            \item Se insiste en la cantidad de la producción de ideas.
            \item Se alienta a que los demás mejoren las ideas.
        \end{itemize}
    
    \item Algunas personas no les funciona esto. Investigaciones demostraron que los individuos podían desarrollar mejores ideas trabajando solos que en grupo; sin embargo, investigación adicional mostró que en algunas situaciones el enfoque de grupo puede funcionar mejor.
\end{itemize}


%----------------------------------------------------------------------------------------
\subsection{Limitaciones del análisis de grupo tradicional}
\begin{itemize}
    \item Sería incorrecto asumir que la  creatividad florece sólo en grupos; de hecho, la acostumbrada reunión en grupo puede inhibir  la creatividad.
    \item  los miembros de un grupo pueden seguir una idea con la exclusión de otras alternativas; los expertos en un tema pueden no estar dispuestos a expresar sus ideas en un grupo por temor a ser ridiculizados; los gerentes de menor nivel jerárquico pueden sentirse inhibidos para expresar sus puntos de vista ante un grupo con gerentes de mayor jerarquía; las presiones a conformarse pueden desalentar la expresión de opiniones divergentes; la necesidad de llevarse bien con otros puede ser más fuerte que la de explorar alternativas creativas.
\end{itemize}


%----------------------------------------------------------------------------------------
\subsection{El gerente creativo}
\begin{itemize}
    \item A menudo se asume que la mayoría de las personas no son creativas y que tienen poca habilidad para desarrollar nuevas ideas. Este supuesto, por desgracia, puede ser perjudicial para la organización, ya que en el ambiente apropiado prácticamente todas las personas son capaces de ser creativas —aunque el grado de creatividad varía de manera considerable entre los individuos.
    \item En términos generales, las personas creativas son inquisitivas y presentan muchas ideas nuevas y originales: pocas veces están satisfechas con el statu quo.
    \item Por último, los individuos creativos pueden causar problemas al ignorar las políticas, las reglas y reglamentos establecidos. John Kao, que enseñó en la Harvard Business School, sugiere que las personas creativas deben tener la libertad suficiente para seguir sus ideas, pero no tanta que pierdan el tiempo o no tengan el necesario para colaborar con otros en la búsqueda de metas comunes.
    \item Como resultado, la creatividad de la mayoría de los individuos es quizá en muchos casos subutilizada, a pesar del hecho de que las innovaciones originales pueden ser de gran beneficio para la empresa. Para cultivar la creatividad, en especial en el área de planeación, pueden utilizarse técnicas individuales y de grupo efectivas. Aun así, recuérdese que la creatividad no es un sustituto del juicio gerencial: es el gerente quien debe determinar y sopesar los riesgos involucrados al seguir ideas originales y traducirlas en prácticas innovadoras.
\end{itemize}





%%%%%%%%%%%%%%%%%%%%%%%%%%%%%%%%%%%%%%%%%%%%%%%%%%%%%%%%%%%%%%%%%%%%%%%%%%%%%%%%%%%%%%%%%%%%%%%%%%%%%%%%%%%%%%%%%%%%%%%%%%%%%%%%%%%%%%%%%%%%%%
\end{document}

