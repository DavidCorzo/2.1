\documentclass{article}
\title{Global Capitulo 3}
\author{David Gabriel Corzo Mcmath}
\date{2020-Jan-20 00:22:04}
%%%%%%%%%%%%%%%%%%%%%%%%%%%%%%%%%%%%%%%%%%%%%%%%%%%%%%%%%%%%%%%%%%%%%%%%%%%%%%%%%%%%%%%%%%%%%%%%%%%%%%%%%%%%%%%%%%%%%%%%%%%%%%%%%%%%%%%%%%%%%%%
\usepackage[margin = 1in]{geometry}
\usepackage{graphicx}
\usepackage{fontenc}
\usepackage{pdfpages}
\usepackage[spanish]{babel}
\usepackage{amsmath}
\usepackage{amsthm}
\usepackage[utf8]{inputenc}
\usepackage{enumitem}
\usepackage{mathtools}
\usepackage{import}
\usepackage{xifthen}
\usepackage{pdfpages}
\usepackage{transparent}
\usepackage{color}
\usepackage{fancyhdr}
\usepackage{lipsum}
\usepackage{sectsty}
\usepackage{titlesec}
\usepackage{calc}
\usepackage{lmodern}
\usepackage{xpatch}
\usepackage{blindtext}
\usepackage{bookmark}
\usepackage{fancyhdr}
\usepackage{xcolor}
\usepackage{tikz}
\usepackage{blindtext}
\usepackage{hyperref}
\usepackage{listing}
\usepackage{spverbatim}
\usepackage{fancyvrb}
\usepackage{fvextra}
\usepackage{amssymb}
\usepackage{pifont}
\usepackage{longtable}
%%%%%%%%%%%%%%%%%%%%%%%%%%%%%%%%%%%%%%%%%%%%%%%%%%%%%%%%%%%%%%%%%%%%%%%%%%%%%%%%%%%%%%%%%%%%%%%%%%%%%%%%%%%%%%%%%%%%%%%%%%%%%%%%%%%%%%%%%%%%%%%
\begin{document}
\maketitle


\section{Introducción}
\begin{itemize}
    \item Las empresas internacionales enfrentan retos más restrictivos de la administración.
    \item Cosas como la cultura y costumbre pueden anteponer un conflicto a la administración.
\end{itemize}

%%%%%%%%%%%%%%%%%%%%%%%%%%%%%%%%%%%%%%%%%%%%%%%%%%%%%%%%%%%%%%%%%%%%%%%%%%%%%%%%%%%%%%%%%%%%%%%%
\section{Administración internacional y corporaciones}
\begin{itemize}
    \item \emph{\textbf{Definición de ``Administración internacional":} se enfoca de la manera de operar de las empresas internacionales en países anfitriones, concentra en problemas gerenciales con el flujo de personas, bienes y dinero para mejorar la administración en situaciones que incluyen el cruce de fronteras nacionales.}
    \item Por lo regular están más conscientes de factores ambientales.
    \item Enfrentan muchos factores distintos además de las empresas nacionales, cosas como:
        \begin{itemize}
            \item Condiciones legales 
            \item Condiciones políticas 
            \item Condiciones económicas 
            \item Diferente nivel educativo
        \end{itemize}
    
    \item Las empresas internacionales buscan el mercado \textbf{mundial}.
\end{itemize}
%%%%%%%%%%%%%%%%%%%%%%%%%%%%%%%%%%%%%%%%%%%%%%%%%%%%%%%%%%%%%%%%%%%%%%%%%%%%%%%%%%%%%%%%%%%%%%%%

\subsection{Naturaleza y propósito de las empresas internacionales: aspectos relevantes de las empresas internacionales}
\begin{itemize}
    \item Exportación: 
        \begin{itemize}
            \item País \emph{matriz} $\underbrace{\rightarrow}_{\text{Bienes y servicios}}$ país \emph{anfitrión}.
        \end{itemize}
    
    \item Acuerdo de licencia:
        \begin{itemize}
            \item País sede $\underbrace{\rightarrow}_{\text{Experiencia técnica \& administrativa}}$ país anfitrión.
        \end{itemize}
    
    \item Contratos administrativos:
        \begin{itemize}
            \item País sede $\underbrace{\rightarrow}_{\text{Materias primas y personal}}$ país anfitrión  
        \end{itemize}
    
    \item Subsidiarias:
        \begin{itemize}
            \item País sede $\underbrace{\rightarrow}_{\text{Capital y experiencia}}$ país anfitrión 
        \end{itemize}
\end{itemize}
%%%%%%%%%%%%%%%%%%%%%%%%%%%%%%%%%%%%%%%%%%%%%%%%%%%%%%%%%%%%%%%%%%%%%%%%%%%%%%%%%%%%%%%%%%%%%%%%

\subsubsection{Los automoviles baratos son lo de hoy}
\begin{itemize}
    \item Responden a las necesidades del mercado de tener un carro barato.
\end{itemize}
%%%%%%%%%%%%%%%%%%%%%%%%%%%%%%%%%%%%%%%%%%%%%%%%%%%%%%%%%%%%%%%%%%%%%%%%%%%%%%%%%%%%%%%%%%%%%%%%

\subsection{Efectos unificadores}
\begin{itemize}
    \item \emph{\textbf{Definición de ``Efectos unificadores":} Cuando la casa matríz comparte conocimientos y técnicas con otras casas anfitrionas para mejorar el desarrollo en dicho país anfitrión.}
    \item Esto puede ser un arma de dos filos por que al ellos aprender pueden sentirse capaces de subsistir en el mercado sin la casa matríz y cortar comunicación con ella.
\end{itemize}
%%%%%%%%%%%%%%%%%%%%%%%%%%%%%%%%%%%%%%%%%%%%%%%%%%%%%%%%%%%%%%%%%%%%%%%%%%%%%%%%%%%%%%%%%%%%%%%%

\subsubsection{Conflictos potenciales}
\begin{itemize}
    \item Diferencias socioculturales
\end{itemize}

Para combatir estos problemas es importante aprender a tratar con estos problemas de una manera diplomática y con habilidad social, y prevenir estos conflictos a toda costa.



%%%%%%%%%%%%%%%%%%%%%%%%%%%%%%%%%%%%%%%%%%%%%%%%%%%%%%%%%%%%%%%%%%%%%%%%%%%%%%%%%%%%%%%%%%%%%%%%
\subsection{Corporaciones multinacionales}
\begin{itemize}
    \item \emph{\textbf{Definición de ``Corporaciones multinacionales":} ien su sede en un país determinado pero operan en muchos otros.}
\end{itemize}

%%%%%%%%%%%%%%%%%%%%%%%%%%%%%%%%%%%%%%%%%%%%%%%%%%%%%%%%%%%%%%%%%%%%%%%%%%%%%%%%%%%%%%%%%%%%%%%%
\subsection{De la orientación etnocéntrica a la geocéntrica}
\begin{itemize}
    \item \emph{\textbf{Definición de ``Etnocéntrica":} una empresa con orientación a operar en el extrangero pero desde una casa matríz. Clave: centralización.}
    \item \emph{\textbf{Definición de ``Policéntrica":} Orientada a la noción que es mejor dar a las subsidiarias extrangeras con personal local, libertad administrativa por que ellos sabrán mejor qué hacer por ser de ahí. Clave: descentralizado.}
    \item \emph{\textbf{Definición de ``Regiocéntrica":} recluta personal experto con base en una región. Clave: tercerización.}
    \item \emph{\textbf{Definición de ``Geocéntrica":} considera un sistema interdependiente que opera en muchos países, este es el enfoque que toman las multinacionales. Clave: lo normal.}
\end{itemize}
%%%%%%%%%%%%%%%%%%%%%%%%%%%%%%%%%%%%%%%%%%%%%%%%%%%%%%%%%%%%%%%%%%%%%%%%%%%%%%%%%%%%%%%%%%%%%%%%
\subsection{Ventajas de las multinacionales}
\begin{itemize}
    \item Aprovechar más \textbf{oportunidades de negocio}.
    \item Conseguir \textbf{dinero para operaciones de todo el mundo}.
    \item Puede decidir poner producción en países donde les sea más \textbf{rentable}.
    \item Mejor \textbf{acceso a recursos} y materiales naturales.
    \item Pueden reclutar a gerentes y personal de la reserva mundial de \textbf{mano de obra}.
\end{itemize}

%%%%%%%%%%%%%%%%%%%%%%%%%%%%%%%%%%%%%%%%%%%%%%%%%%%%%%%%%%%%%%%%%%%%%%%%%%%%%%%%%%%%%%%%%%%%%%%%

\subsection{Desafíos multinacionales}
\begin{itemize}
    \item Nacionalismo 
    \item La gente aprende de las técnicas y surge la posibilidad que se vayan de la empresa y compitan en contra nuestra.
    \item Tener un buena relación con el gobierno anfitrión.
\end{itemize}

%%%%%%%%%%%%%%%%%%%%%%%%%%%%%%%%%%%%%%%%%%%%%%%%%%%%%%%%%%%%%%%%%%%%%%%%%%%%%%%%%%%%%%%%%%%%%%%%

\subsection{De corporaciones multinacionales a mundiales o transnacionales}
\begin{itemize}
    \item \emph{\textbf{Definición de ``Corporación mundial o transnacional":} Compañías que contemplan al mundo entero como un solo mercado.}
    \item Típicamente para estas los mercados locales se han vuelo muy pequeños.
    \item Desarrollan productos que consideran al mundo entero.
    \item Estas compañías deben siempre estar afinando sus tácticas en los ambientes nacionales y locales.
\end{itemize}

%%%%%%%%%%%%%%%%%%%%%%%%%%%%%%%%%%%%%%%%%%%%%%%%%%%%%%%%%%%%%%%%%%%%%%%%%%%%%%%%%%%%%%%%%%%%%%%%
%%%%%%%%%%%%%%%%%%%%%%%%%%%%%%%%%%%%%%%%%%%%%%%%%%%%%%%%%%%%%%%%%%%%%%%%%%%%%%%%%%%%%%%%%%%%%%%%
\section{Alianzas entre países y bloques económicos}
\begin{itemize}
    \item Los países a veces tienden a competir entre sí, a veces hasta grupos de países compiten entre sí.
\end{itemize}

%%%%%%%%%%%%%%%%%%%%%%%%%%%%%%%%%%%%%%%%%%%%%%%%%%%%%%%%%%%%%%%%%%%%%%%%%%%%%%%%%%%%%%%%%%%%%%%%
\subsection{Unión europea}
\begin{itemize}
    \item Hizo reformas legislativas para facilitar el comercio entre los países europeos con el objetivo de competir con otras regiones como Asia, y EEUU.
\end{itemize}

%%%%%%%%%%%%%%%%%%%%%%%%%%%%%%%%%%%%%%%%%%%%%%%%%%%%%%%%%%%%%%%%%%%%%%%%%%%%%%%%%%%%%%%%%%%%%%%%

\subsection{Tratado de libre comercio de América del norte y otros bloques de libre comercio en Ámera Latina}+
\begin{itemize}
    \item TLCAN, medidas legislativas para facilitar el comercio de los países americanos.
    \item Esto brindaba más acceso al mercado, procedimientos de aduanas, etcétera.
    \item Tenía un poco de críticas como la del planteamiento que sólo beneficiaban a las naciones desarrolladas.
\end{itemize}

%%%%%%%%%%%%%%%%%%%%%%%%%%%%%%%%%%%%%%%%%%%%%%%%%%%%%%%%%%%%%%%%%%%%%%%%%%%%%%%%%%%%%%%%%%%%%%%%
\subsection{Asociación de naciones de Sudeste Asiático}
\begin{itemize}
    \item Formaron un bloque comrecial para hacerle frente al TLCAN, Union Europea, etcétera; de manera económica y política.
\end{itemize}

%%%%%%%%%%%%%%%%%%%%%%%%%%%%%%%%%%%%%%%%%%%%%%%%%%%%%%%%%%%%%%%%%%%%%%%%%%%%%%%%%%%%%%%%%%%%%%%%
\subsection{Función de India en la economía mundial}
\begin{itemize}
    \item India es un mercado muy grande por ende es relevante.
\end{itemize}

%%%%%%%%%%%%%%%%%%%%%%%%%%%%%%%%%%%%%%%%%%%%%%%%%%%%%%%%%%%%%%%%%%%%%%%%%%%%%%%%%%%%%%%%%%%%%%%%
\section{Administración internacional: diferencias culturales y entre países}
\begin{itemize}
    \item Hay diferentes prácticas gerenciales.
\end{itemize}

%%%%%%%%%%%%%%%%%%%%%%%%%%%%%%%%%%%%%%%%%%%%%%%%%%%%%%%%%%%%%%%%%%%%%%%%%%%%%%%%%%%%%%%%%%%%%%%%
\subsection{Comportamientos en distintas culturas}
\begin{itemize}
    \item Geert Hofstede, investigador que condujo un estudio para indagar en la cultura de un país y cómo eso influye en el comportamiento de los empleados; llegó a filtrarlo en cinco categorías:
        \begin{enumerate}
            \item Individualismo frente al colectivismo:
                \begin{itemize}
                    \item \emph{\textbf{Definición de ``Individalismo":} Las personas concentran en sus propios intereses y en quienes las rodean, las tareas son más importantes que las relaciones.}
                    \item \emph{\textbf{Definición de ``Colectivismo":} Se concentran en el grupo y se espera contar con su apoyo, las relaciones son más importantes que la orientación a la tarea.}
                \end{itemize}
            \item Distanciamiento del poder contra el acercamiento a éste:
                \begin{itemize}
                    \item \emph{\textbf{Definición de ``Distanciamiento del poder":} La sociedad acepta la distribución desigual del poder y lo respeta.}
                    \item \emph{\textbf{Definición de ``Acercamiento al poder":} La sociedad acepta menos al poder.}
                \end{itemize}
            \item Tolerancia a la incertidumbre frente a su evasión:
                \begin{itemize}
                    \item \emph{\textbf{Definición de ``Tolerancia a la incertidumbre":} la gente acepta la incertidumbre, toma riesgos.}
                    \item \emph{\textbf{Definición de ``Evasión de la incertidumbre":} Temor a la incertidumbre.}
                \end{itemize}
            \item Masculinidad frente a la feminidad:
                \begin{itemize}
                    \item \emph{\textbf{Definición de ``Masculinidad":} conducta energética, cofianza en sí mismo, enfoque en lo material, el dinero y el éxito.}
                    \item \emph{\textbf{Definición de ``Feminidad":} Orientación a la relación, favorece la calidad de vida, importancia de la modestia.}
                \end{itemize}
            \item Orientación al corto frente al largo plazo:
                \begin{itemize}
                    \item \emph{\textbf{Definición de ``Orientación a largo plazo":} Se caracteriza por el trabajo arduo, tendencia a ahorrar.}
                    \item \emph{\textbf{Definición de ``Orientación a corto plazo":} Menor enfoque en el futuro, impulso al consumo.}
                \end{itemize}
        \end{enumerate}
\end{itemize}

%%%%%%%%%%%%%%%%%%%%%%%%%%%%%%%%%%%%%%%%%%%%%%%%%%%%%%%%%%%%%%%%%%%%%%%%%%%%%%%%%%%%%%%%%%%%%%%%
\subsection{Francia: Le Plan y le Candre}
\begin{itemize}
    \item Ejemplo de diferencias en prácticas administrativas.
    \item La gente tiende a aspirar ser funcionario público.
    \item El gobierno se propone metas como mejorar la efectividad del uso de recursos, evitar la expansión de áreas no económicas.
\end{itemize}

%%%%%%%%%%%%%%%%%%%%%%%%%%%%%%%%%%%%%%%%%%%%%%%%%%%%%%%%%%%%%%%%%%%%%%%%%%%%%%%%%%%%%%%%%%%%%%%%
\subsection{Alemania: autoridad y codeterminación}
\begin{itemize}
    \item Tienden a manejar prácticas denominadas ``autoritarismo benevolente''.
    \item \emph{\textbf{Definición de ``cordeterminación":} requiere la filiación de la fuerza laboral en el consejo de supervisión y en el comité ejecutivo de la corporación.}
\end{itemize}

%%%%%%%%%%%%%%%%%%%%%%%%%%%%%%%%%%%%%%%%%%%%%%%%%%%%%%%%%%%%%%%%%%%%%%%%%%%%%%%%%%%%%%%%%%%%%%%%
\subsection{Factores que influyen en la administración de países occidentales}
\begin{itemize}
    \item Tienden a existir prácticas como las de ser más competitivo, promover el liderazgo o las decisiones en consenso.
\end{itemize}

%%%%%%%%%%%%%%%%%%%%%%%%%%%%%%%%%%%%%%%%%%%%%%%%%%%%%%%%%%%%%%%%%%%%%%%%%%%%%%%%%%%%%%%%%%%%%%%%
\subsection{Administración Coreana}
\begin{itemize}
    \item Promueven principios como ``Chaebol'' o ``Inhwa'', 
    \begin{itemize}
        \item \emph{\textbf{Definición de ``Chaebol":} modelo gerencial coreano que se caracteriza por una estrecha convivencia entre gobierno y conglomerados industriales}; 
        \item \emph{\textbf{Definición de ``Inhwa":} concepto organizacional que promueve la armonía.}
    \end{itemize}
\end{itemize}


%%%%%%%%%%%%%%%%%%%%%%%%%%%%%%%%%%%%%%%%%%%%%%%%%%%%%%%%%%%%%%%%%%%%%%%%%%%%%%%%%%%%%%%%%%%%%%%%
%%%%%%%%%%%%%%%%%%%%%%%%%%%%%%%%%%%%%%%%%%%%%%%%%%%%%%%%%%%%%%%%%%%%%%%%%%%%%%%%%%%%%%%%%%%%%%%%
\section{La administración japonesa y la teoría Z}
Adopción de prácticas como empleo vitanlicio, wa, teoría Z, toma de decisiones en consenso: 
\begin{itemize}
    \item \emph{\textbf{Definición de ``Empleo vitalicio":} concepto de ``empleados permanentes'' hace énfasis en la antigüedad del tiempo que la persona ha estado trabajando ahí y se empiezan a dar beneficios a los empleados antigüos. Puede aumentar los costos pero es cultura japonesa y tienden a ocasionar choques con nuevos y más jóvenes empleados.}
    \item \emph{\textbf{Definición de ``wa":} armonía}
    \item \emph{\textbf{Definición de ``Toma de decisiones en consenso":} concepto que las nuevas ideas deben de venir desde abajo, siempre respetando la jerarquía organizacional pero se toma la decisión de acuerdo a lo que digan los que están hasta abajo.}
    \item \emph{\textbf{Definición de ``Teoría Z":} Sostiene que la responsabilidad la absorbe el individuo, toman muy en cuenta las relaciones informales y democráticas basadas en confianza, las metas, autoridad y reglas guían el comportamiento corporativo.}
\end{itemize}

%%%%%%%%%%%%%%%%%%%%%%%%%%%%%%%%%%%%%%%%%%%%%%%%%%%%%%%%%%%%%%%%%%%%%%%%%%%%%%%%%%%%%%%%%%%%%%%%

\subsection{El auge de China: Deng Xiaoping cambió a China de una economía planificada a otra de mercado}
\begin{itemize}
    \item Gracias a esto China ha podido desarrollarse con una impresionante velocidad.
\end{itemize}

%%%%%%%%%%%%%%%%%%%%%%%%%%%%%%%%%%%%%%%%%%%%%%%%%%%%%%%%%%%%%%%%%%%%%%%%%%%%%%%%%%%%%%%%%%%%%%%%

\subsection{El auge de India}
\begin{itemize}
    \item Se han vuelto expertos en calidad y la alta tecnología.
\end{itemize}

%%%%%%%%%%%%%%%%%%%%%%%%%%%%%%%%%%%%%%%%%%%%%%%%%%%%%%%%%%%%%%%%%%%%%%%%%%%%%%%%%%%%%%%%%%%%%%%%

\section{La ventaja competitiva de las naciones de Porter}
Michael Porter, propone factores que contribuyen a mejorar una nación en términos de ventaja competitiva.
\begin{itemize}
    \item Condiciones del factor: recursos de una nación, costos laborales, etcétera.
    \item Condiciones de demanda: tamaño del mercado.
    \item Proveedores: una compañía prospera cuando las empresas que la apoyan están en la misma área.
    \item Estrategia y estructura de la empresa: rivalidad entre competidores.
\end{itemize}

%%%%%%%%%%%%%%%%%%%%%%%%%%%%%%%%%%%%%%%%%%%%%%%%%%%%%%%%%%%%%%%%%%%%%%%%%%%%%%%%%%%%%%%%%%%%%%%%

\section{Lograr una ventaja competitiva mundial mediante la administración de calidad}
Defensores de la calidad proponen calidad en EEUU pero son rechazados por haber propuesto un cambio en las prácticas de administración, dos de ellos Demin, Juran se van a Japón y ahí si los escuchan.

%%%%%%%%%%%%%%%%%%%%%%%%%%%%%%%%%%%%%%%%%%%%%%%%%%%%%%%%%%%%%%%%%%%%%%%%%%%%%%%%%%%%%%%%%%%%%%%%
\subsection{Premio nacional a la calidad Malcom Baldrige 1996}
Compiten diversas categorías:
\begin{itemize}
    \item Empresas manufactureras 
    \item Compañías de servicios 
    \item Pequeñas empresas 
\end{itemize}
Criterios para ganar:
\begin{enumerate}
    \item Liderazgo: liderazgo y lograr alto desempeño.
    \item Información y análisis: examinar la efectividad de la compañía.
    \item Planeación estratégica: traducción de planes a operaciones.
    \item Desarrollo y administración: recursos humanos.
    \item Administración: procesos de administración.
    \item Organizaciones orientadas a resultados: examina resultados.
    \item Enfoque en el cliente: qué tanto importa el cliente. 
\end{enumerate}

\subsection{ISO 9000}
\begin{itemize}
    \item Se encarga de certificar a las empresas y asegurar que cumplen con los requisitos de calidad necesarios.
\end{itemize}

\subsection{Un modelo europeo de la administración}
\begin{itemize}
    \item Evalúa las cinco variables (liderazgo, administración de las personas, política y estrategia,, recursos y procesos). 
\end{itemize}











\end{document}
