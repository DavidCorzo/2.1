\documentclass{article}
\title{Resumen De Resumen De Guy Kiyosaky}
\author{David Gabriel Corzo Mcmath}
\date{2020-Feb-09 11:09:11}
%%%%%%%%%%%%%%%%%%%%%%%%%%%%%%%%%%%%%%%%%%%%%%%%%%%%%%%%%%%%%%%%%%%%%%%%%%%%%%%%%%%%%%%%%%%%%%%%%%%%%%%%%%%%%%%%%%%%%%%%%%%%%%%%%%%%%%%%%%%%%%%
\usepackage[margin = 1in]{geometry}
\usepackage{graphicx}
\usepackage{fontenc}
\usepackage{pdfpages}
\usepackage[spanish]{babel}
\usepackage{amsmath}
\usepackage{amsthm}
\usepackage[utf8]{inputenc}
\usepackage{enumitem}
\usepackage{mathtools}
\usepackage{import}
\usepackage{xifthen}
\usepackage{pdfpages}
\usepackage{transparent}
\usepackage{color}
\usepackage{fancyhdr}
\usepackage{lipsum}
\usepackage{sectsty}
\usepackage{titlesec}
\usepackage{calc}
\usepackage{lmodern}
\usepackage{xpatch}
\usepackage{blindtext}
\usepackage{bookmark}
\usepackage{fancyhdr}
\usepackage{xcolor}
\usepackage{tikz}
\usepackage{blindtext}
\usepackage{hyperref}
\usepackage{listing}
\usepackage{spverbatim}
\usepackage{fancyvrb}
\usepackage{fvextra}
\usepackage{amssymb}
\usepackage{pifont}
\usepackage{longtable}
%%%%%%%%%%%%%%%%%%%%%%%%%%%%%%%%%%%%%%%%%%%%%%%%%%%%%%%%%%%%%%%%%%%%%%%%%%%%%%%%%%%%%%%%%%%%%%%%%%%%%%%%%%%%%%%%%%%%%%%%%%%%%%%%%%%%%%%%%%%%%%%
\begin{document}
\maketitle

\section{El arte de empezar}
\begin{itemize}
    \item Responde preguntas sencillas, \textbf{no} empieza con grandes ambiciones.
    \item Para alcanzar el éxito se empieza formulando preguntas \textbf{muy sencillas}.
    \item Preguntas sencillas ejemplos:
        \begin{itemize}
            \item Y ¿entonces qué?
            \item ¿Existe una mejor manera? 
            \item ¿Es factible? 
            \item ``Si lo fabricamos, vendrán a por ello''
            \item ¿Dónde está la debilidad del líder del mercado? (analiza enfermedades que puedan debilitar al líder del mercado y ve por ello).
        \end{itemize}
    
    \item \textbf{No} preguntarse ¿cómo ganar dinero?, sólo por el momento da respuestas a estas preguntas sencillas.
\end{itemize}


%%%%%%%%%%%%%%%%%%%%%%%%%%%%%%%%%%%%%%%%%%%%%%%%%%%%%%%%%%%%%%%%%%%%%%%%%%%%%%%%%%%%%%%%%%%%%%%%
\subsection{Encuentra compañeros de alma ideales}
\begin{itemize}
    \item Encontrar buenos socios, sin miedo a pensar en grande.
    \item Encontrar por lo menos dos ``amigos del alma'', la nueva iniciativa necesita funcionar con un equipo.
    \item Características deseables de socios:
        \begin{itemize}
            \item Compartir visión y compromiso.
            \item Fundadores con intuición similar en cuanto a cómo evolucionará el startup como el mercado.
            \item Compartir un nivel similar de compromiso. 
        \end{itemize}
    
    \item Diferencias deseables:
        \begin{itemize}
            \item Experiencia, experiencia en marketing y un inventor. 
            \item Orientación, quien se enfoque en los detalles y quien se enfoque en lo general, los dos son necesarios.
            \item Puntos de vista, abundantes puntos de vista, diversidad del equipo.
        \end{itemize}
    
    \item Planifica para el peor escenario.
\end{itemize}


%%%%%%%%%%%%%%%%%%%%%%%%%%%%%%%%%%%%%%%%%%%%%%%%%%%%%%%%%%%%%%%%%%%%%%%%%%%%%%%%%%%%%%%%%%%%%%%%
\subsection{Adopta un modelo de negocio}
\begin{itemize}
    \item Estate listo para cambiar tu modelo de negocio varias veces.
    \item Que tu modelo de negocio responda :  
        \begin{itemize}
            \item ¿En los bolsillos de quién está el dinero que necesitas? Identificar al cliente.
            \item ¿Cómo lo harás para conseguir que ese dinero vaya a parar a tus bolsillos? Crear un mecanismo de ventas que garantice ingresos.
        \end{itemize}
    
    \item ``Si la primera versión de tu producto no te avergüenza lo has lanzado demasiado tarde al mercado''
    \item Olvidate del perfeccionismo, lanza el producto sin que sea perfecto.
\end{itemize}


%%%%%%%%%%%%%%%%%%%%%%%%%%%%%%%%%%%%%%%%%%%%%%%%%%%%%%%%%%%%%%%%%%%%%%%%%%%%%%%%%%%%%%%%%%%%%%%%
\section{El arte del lanzamiento}
\begin{itemize}
    \item Tener identificadas las necesidades del mercado para evadir la destrucción creativa. Ejemplo de las refris en el siglo 19 \& 20.
    \item Construir prototipos antes de lanzar al mercado.
    \item No producir en masa al principio, solo preocuparse que el mercado adopte tu producto.
    \item Posicionamiento: establecer distinciones con la competencia y tener un lugar dominante. La publicidad $\neq $ posicionamiento. El posicionamiento habita en las opiniones de los consumidores.
    \item El mérito es el nuevo marketing.
    \item Sal de la oficina, ve qué se necesita y que se puede hacer mejor, y descubrir qué quiere el cliente, sal a la calle personalmente y comprueba cómo utilizan tu producto.
    \item Llevar a cabo una preautopsia, analizar por que un producto puede morir en lugar de analizar pore que murió.
    \item El liderazgo importa: lo más fácil es la contabilidad y la gestión de procesos por que se puede ``aprender o contratar a alguien que lo haya aprendido'', el liderazgo es lo difícil.
    \item Ficha a gente mejor que tu: 
        \begin{itemize}
            \item Kamikazes dispuestos a trabajar 80 horas a la semana.
            \item Implementadores que creen infraestructura.
            \item Operadores que se sientan felices gestionando un proyecto en marcha.
        \end{itemize}
    
    \item Haz mejores a tus empleados, no buscar un empleado perfecto, buscar un empleado e irlo mejorando.
    \item No pidas a los empleados algo que tú nunca harías, ten empatía.
\end{itemize}


%%%%%%%%%%%%%%%%%%%%%%%%%%%%%%%%%%%%%%%%%%%%%%%%%%%%%%%%%%%%%%%%%%%%%%%%%%%%%%%%%%%%%%%%%%%%%%%%
\section{El arte de conseguir financiación}
\begin{itemize}
    \item Es un mal necesario.
    \item Hay varias formas de conseguir financiación.
    \item El arte del bootstraping:
        \begin{itemize}
            \item Sacrificar comidas y lujos de hoy para invertir en tu negocio como se pueda, es y tiene que ser necesariamente a corto plazo.
            \item Emplea herramientas como open-source-code, servicios gratis en la nube, trabajadores virtuales, usar la redes sociales para el marketing.
            \item Este método es para \textbf{liquidez} no rentabilidad.
        \end{itemize}
    
    \item El arte de recurrir a incubadoras y aceleradoras:
        \begin{itemize}
            \item Proveen mentorización, proveen ayuda como: capital inicial, compañerismo y fertilización cruzada, mentorización y formación, desarrollo del negocio, un camino hacía más financiación, tareas administrativas, espacio de oficina.
            \item Pediran un porcentaje de la empresa típicamente un 1\% como compensación.
        \end{itemize}
    
    \item El arte del crowd funding:
        \begin{itemize}
            \item La forma de financiación más democrática, transparente y abierta que existe.
            \item Consiste en crear un proyecto y mandarlo a personas interesadas y en lugar de pedir participaciones de la compañía se dan recompensas llamadas  ``kickers'', estos kickers son como descuentos, premios, etcétera.
            \item Se difunde a través de redes sociales.
            \item Esta gente financia el proyecto, se utiliza para productos destinados al consumo.
        \end{itemize}
    
    \item El arte de cortejar a los ``ángeles'':
        \begin{itemize}
            \item Son los business angels, estos quieren aportar a la sociedad y talvés ganar dinero.
            \item Algunas recomendaciones son: no los infravalores, toma en cuenta que disfrutan como lo hacen los empresarios jóvenes, suele estar integrado por su pareja y no por socios, sé agradable, tienen un enfoque paternalista. 
        \end{itemize}
    
    \item El (Duro) arte de tratar con capitalistas de riesgo:
        \begin{itemize}
            \item Ellos no son tus amigos, no saben más que tú, ellos invierten en muchos startups al mismo tiempo teniendo en cuenta que la mayoría no serán rentables, al finalizar un año tienes que generarles ganancias si no te cortarán financiamiento.
            \item Tener contactos, consigue una buena tarjeta de presentación.
            \item Demuestra tracción: buscar la garantía demostrada, solo esto permitirá que la gente saque de su bolsillo dinero y lo ponga en el tuyo.
            \item Si no tienes suficiente financiación para demostrar tracción, demuéstrala por medio de:
                \begin{itemize}
                    \item Ventas reales.
                    \item Pruebas sobre terreno e instalaciones piloto.
                    \item Acuerdo para realizar pruebas.
                    \item Tener un contacto que permita realizar pruebas.
                \end{itemize}
            
            \item Tener mínimo un contacto para realizar las pruebas.
            \item Cataliza la fantasía: en lugar de decir que el producto vaya a generar una inimaginable suma de dinero expresar a los inversores una necesidad en el mercado y que ellos hágan sus cálculos mentales.
            \item Ejemplos de catalizar la fantasía:
                \begin{itemize}
                    \item Todo el que ha practicado dicha actividad sabe ... 
                    \item Pero se topan con tal problema ... 
                    \item Entonces para eso presento tal solución ... 
                    \item Las personas apreciarán esta solución y obtener dinero a partir de ahí será fácil.
                \end{itemize}
        \end{itemize}
    
    \item Reconoce y crea un enemigo:
        \begin{itemize}
            \item Reconocer la competencia, si se cree que no hay competencia se asume que es por que no hay mercado.
            \item Reconocer competencia indirecta, osea de productos sustitutos.
        \end{itemize}
    
    \item Busca un abogado experto en derecho financiero corporativo, necesitaras ayuda legal.
    \item Anticípate el futuro:
        \begin{itemize}
            \item Capital semilla: el producto se empieza a desarrollar con el capital financiero de inversores.
            \item Serie A: El producto empieza a generar.
            \item Serie B: Se necesita más financiamiento para hacer crecer más el negocio.
            \item Serie C: Los inversores compran , el emprendedor no vende, le adulan a un equipo ganador.
        \end{itemize}
    
    \item Gestionar la junta directiva:
        \begin{itemize}
            \item Necesitarás gestionar con ellos por que son los inversores.
            \item Necesitas dos tipos de personas: conocedores de la creación de empresas, conocedores profundos del mercado.
            \item Haz reuniones frecuentes con la junta, eso transmite disciplina y responsabilidad.
        \end{itemize}
    
    \item *Ver preguntas y respuestas a los capitalistas en lectura
\end{itemize}















\end{document}
