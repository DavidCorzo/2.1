\documentclass{article}
\title{Guia Discusion Socratica - Amazon \& Whole Foods}
\author{David Gabriel Corzo Mcmath}
\date{2020-Feb-03 09:30:24}
%%%%%%%%%%%%%%%%%%%%%%%%%%%%%%%%%%%%%%%%%%%%%%%%%%%%%%%%%%%%%%%%%%%%%%%%%%%%%%%%%%%%%%%%%%%%%%%%%%%%%%%%%%%%%%%%%%%%%%%%%%%%%%%%%%%%%%%%%%%%%%%
\usepackage[margin = 1in]{geometry}
\usepackage{graphicx}
\usepackage{fontenc}
\usepackage{pdfpages}
\usepackage[spanish]{babel}
\usepackage{amsmath}
\usepackage{amsthm}
\usepackage[utf8]{inputenc}
\usepackage{enumitem}
\usepackage{mathtools}
\usepackage{import}
\usepackage{xifthen}
\usepackage{pdfpages}
\usepackage{transparent}
\usepackage{color}
\usepackage{fancyhdr}
\usepackage{lipsum}
\usepackage{sectsty}
\usepackage{titlesec}
\usepackage{calc}
\usepackage{lmodern}
\usepackage{xpatch}
\usepackage{blindtext}
\usepackage{bookmark}
\usepackage{fancyhdr}
\usepackage{xcolor}
\usepackage{tikz}
\usepackage{blindtext}
\usepackage{hyperref}
\usepackage{listing}
\usepackage{spverbatim}
\usepackage{fancyvrb}
\usepackage{fvextra}
\usepackage{amssymb}
\usepackage{pifont}
\usepackage{longtable}
%%%%%%%%%%%%%%%%%%%%%%%%%%%%%%%%%%%%%%%%%%%%%%%%%%%%%%%%%%%%%%%%%%%%%%%%%%%%%%%%%%%%%%%%%%%%%%%%%%%%%%%%%%%%%%%%%%%%%%%%%%%%%%%%%%%%%%%%%%%%%%%
\begin{document}
\maketitle

\section{Síntesis}
Cuando una empresa o grupo de empresa se mete a un nuevo sector de mercado, típicamente no es muy aceptable por el mismo cliente, el asunto es que los clientes tienen un modelo mental muy cerrado en el que cosas como este cambio aunque sea de la misma empresa se ven de manera muy sospechosa. Cuando Amazon se metió a este nuevo sector de mercado agrego un nivel de complejidad a su modelo de negocio ya que no solamente tuvo que heredar el modelo de negocio existente que llevaba Whole foods si no que muchas veces incompatible con el modelo de Amazon. Pero sorprendentemente a pesar de este reto pudo resonar con sus clientes casi ya que con la misma efectividad con la que Whole foods lo hacía. Este proceso se llevó a cabo de una manera muy peculiar, Amazon empleó ciertas estratégias innovadoras para poder ponerlo a funcionar, eventualmente lo hizo. Amazon ha sido el pionero más acertado en mercados virtuales y han a través del tiempo afinado bastante su estratégia, el hecho que se estaban metiendo a un sector de mercado en el que no sabían mucho era una movida algo riesgosa pero como un empresario tiene que incurrir en incertidumbre para poder derivar ganacias de sus actividades, en el caso de Amazon contaban con un capital bastante completo y abundante de tal manera que en cierto sentido no tenían mucho que perder.

\section{Referencias bibliográficas}
\subsection{Servicios de Drone}
En el mercado se han observado tendencias que los consumidores tienden a levantar protesta al hecho que tienen que esperar por un paquete mucho tiempo, esto se debe a que el transporte aéreo y terrestre tiene muchas cosas que lo hacen costoso que no sea así. Amazon tiene una costumbre empresarial a querer innovar usando tecnología y automatización, en este caso se observa a primera instancia esto con el innovador servicio de entregas por drones. Esto permitirá agilizar el proceso de entrega a los consumidores y brindar al consumidor bienes más rápidos y que por ende aportarán valor al cliente.
\url{}

\subsection{Amazon prime}
Amazon está comprometido a no perder el tiempo del consumidor


\section{Identificar términos desconocidos}
\subsection{Término ``Retail''}
El término retail es especialmente importante, es un tipo de actividad empresarial que sostiene la idea de querer proveer productos a los consumidores y/o clientes bajo la suposición que son para consumo propio y no para revender. En cierta forma llevan un modelo de negocio esencialmente minorista en el que no venden grandes cantidades de producto (por mayor) si no que venden solo el volúmen que el consumidor tienda a desear.
\subsection{Término ``CEO''}
Se refiere a uno de los puestos más importantes en una empresa, este término hace referencia a ``\emph{Chief Excecutive Officer}'' esta persona tiene la responsabilidad de hacer las gestiones gerenciales de toda la empresa, es más toma la mayoría de decisiones referentes a el rumbo de la empresa.

\section{Reflecionar en el caso Amazon}
\subsection{Semejanza con otras industrias}

\subsection{¿Por qué comprar Whole foods?}

\subsection{Describa el papel de liderazgo, innovación \& tecnología}

\subsection{Formas de creación de valor}

\subsection{Cuáles son los retos más importantes de Amazon y su fusión con Whole foods}






\end{document}
