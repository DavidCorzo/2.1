\documentclass{article}
\title{ Global Management And Entrepreneurship \\ Oportunidades De Negocio \& cooperación entre Industrias \\ 
\Huge Cooperación entre la industria tecnológica y la industria de telecomunicaciones }
\author{David Gabriel Corzo Mcmath}
\date{2020-Feb-01 09:40:17}
%%%%%%%%%%%%%%%%%%%%%%%%%%%%%%%%%%%%%%%%%%%%%%%%%%%%%%%%%%%%%%%%%%%%%%%%%%%%%%%%%%%%%%%%%%%%%%%%%%%%%%%%%%%%%%%%%%%%%%%%%%%%%%%%%%%%%%%%%%%%%%%
\usepackage[margin = 1in]{geometry}
\usepackage{graphicx}
\usepackage{fontenc}
\usepackage{pdfpages}
\usepackage[spanish]{babel}
\usepackage{amsmath}
\usepackage{amsthm}
\usepackage[utf8]{inputenc}
\usepackage{enumitem}
\usepackage{mathtools}
\usepackage{import}
\usepackage{xifthen}
\usepackage{pdfpages}
\usepackage{transparent}
\usepackage{color}
\usepackage{fancyhdr}
\usepackage{lipsum}
\usepackage{sectsty}
\usepackage{titlesec}
\usepackage{calc}
\usepackage{lmodern}
\usepackage{xpatch}
\usepackage{blindtext}
\usepackage{bookmark}
\usepackage{fancyhdr}
\usepackage{xcolor}
\usepackage{tikz}
\usepackage{blindtext}
\usepackage{hyperref}
\usepackage{listing}
\usepackage{spverbatim}
\usepackage{fancyvrb}
\usepackage{fvextra}
\usepackage{amssymb}
\usepackage{pifont}
\usepackage{longtable}
%%%%%%%%%%%%%%%%%%%%%%%%%%%%%%%%%%%%%%%%%%%%%%%%%%%%%%%%%%%%%%%%%%%%%%%%%%%%%%%%%%%%%%%%%%%%%%%%%%%%%%%%%%%%%%%%%%%%%%%%%%%%%%%%%%%%%%%%%%%%%%%
\begin{document}
\maketitle

\section{Beneficios de anexar las industrias}
Desde el el gran denominado renacimiento tecnológico el mundo se ha vuelto cada vez más interconectado, las empresas tienen más capacidad de lograr sus objetivos pero también más oportunidad de fracasar y aún con más repercusiones que nunca antes. Las industrias, son conjuntos de empresas que conforman un denominado sector del mercado por vender productos o brindar servicios similares. Este fenómeno que se ha estallado en las últimas décadas deja en la mente que sin duda alguna se tiene que trabajar con varias industrias para derivar cualquier ganancia significativa. Dado a que estas industrias son sectores de mercado \emph{especializados}, es decir que tienen más capacidad de brindar bienes de mejor calidad, más barato y con más influencia. Dejar que yo me especialice en lo que yo soy bueno y que las demás personas se especialicen en lo que ellos son buenos, crea efectividad ya que puedo dar un mejor servicio por que yo tener el denominado \emph{know how} y como tengo el \emph{know how} tengo más posibilidades de equivocarme menos y adquirir más experiencia tomando mejores decisiones con más probabilidades de acertar a mis objetivos. Esta cooperación especializada es precisamente lo que se gana anexando industrias, se gana que cada quien trabaje en lo que es bueno haciendo y después consolidar esa creación en algo que nos fue más barato producir de esa forma a que yo produzca todo.

\section{Industrias de tecnología y de telecomunicaciones}
Entre las industrias más exitosas hoy en día están las de tecnología y telecomunicaciones, la de tecnología que constituye miles de miles de millones de dólares en EEUU es la que de momento está teniendo su auge, en muy cercana proximidad de esta industria se encuentra la de telecomunicaciones. Dichas industrias muchas veces producen lo que se denomina \emph{bienes complementarios} es decir que muchas de las cosas de las que producen se complementan en un producto final consolidado. Por ejemplo: un teléfono celular \emph{Smart Phone} utiliza una parte tecnológica en términos de circuitos eléctricos e innovadores funcionalidades electrónicas, pero consolidado en el producto está la capacidad de poder hacer llamadas telefónicas y enviar mensajes por medio de una red mundial que permite comunicar al usuario con cualquier otro celular conectado en dicha red, esta es la parte de telecomunicaciones; telecomunicaciones y tecnología son productos que muchas veces son complementarios porque como explique anteriormente un celular viene consolidado con las funcionalidades que pertenecen a las dos industrias. Para ejemplificar las posibles alianzas que se pueden tener en las empresas que pertencen a las diversas industrias elegiré el producto \textbf{celular} \emph{Smart Phone}.

%%%%%%%%%%%%%%%%%%%%%%%%%%%%%%%%%%%%%%%%%%%%%%%%%%%%%%%%%%%%%%%%%%%%%%%%%%%%%%%%%%%%%%%%%%%%%%%%

\section{Alianzas estratégicas y oportunidad de negocio \# 1}
Crear una alianza estratégica con estas industrias sería lo mejor, mejorar la coordinación entre estas industrias es una labor que podría terminar en una muy fructífera alianza con prácticamente cero trabajo adicional. La primera alianza que yo propondría es que las empresa tecnológicas formen alianzas estratégicas en el ámbito de brindar al cliente un producto complementario, con sigo lo que implica es que dicho producto sería más fácil de producir, más rápido de producir \& también sería menos laborioso por que cada quien puede hacer su parte especializada y nadie está intentando hacer ejercer en ninguna práctica en la que no estén ya  especializados ya. Las empresas de celulares \emph{Smart Phones} todos tienen así como mencioné anteriormente la característica que demandan una parte tecnológica y una parte de telecomunicación, la alianza estratégica que propondré será precisamente que se pacte un acuerdo entre todas las empresas involucradas para poder surtirse mutuamente estos componentes especializados y poder manufacturar el producto de la forma más efectiva posible.  


%%%%%%%%%%%%%%%%%%%%%%%%%%%%%%%%%%%%%%%%%%%%%%%%%%%%%%%%%%%%%%%%%%%%%%%%%%%%%%%%%%%%%%%%%%%%%%%%

\section{Alianzas estratégicas y oportunidad de negocio \# 2}
Cuando se decide en una empresa o industria empezar a manufacturar un celular \emph{Smart phone} se tiene que crear en masa diversos componentes electrónicos que llevan cada uno su abstracto nivel de complejidad, es por esto que la alianza estratégica aquí sería que las empresas que se encargan mútuamente a surtirse estos componentes electrónicos tengan ua mejor cooperación y potencialmente llegar a crear componentes cada vez más compatibles con las redes de telecomunicaciones existentes. La nueva red 5G que será lanzada en muchos países prontamente tendrá una mayor capacidad de manejo de datos móviles, tanto así que es muy probable que la gente empiece a demandar celulares con un mayor poder de procesamiento para poder enviar y recibir datos. Un objeto computacional como un video resulta ser muy complicado y costoso enviar usando datos móviles, es por eso que las empresas tecnológicas tienen que mantenerse muy al tanto de qué nuevas necesidades surge el cliente, actualmente es tener un celular más poderoso capaz de procesar más datos para trabajar mejor con la transmisión y recepción de datos por medio de las redes de telecomunicaciones actuales.


%%%%%%%%%%%%%%%%%%%%%%%%%%%%%%%%%%%%%%%%%%%%%%%%%%%%%%%%%%%%%%%%%%%%%%%%%%%%%%%%%%%%%%%%%%%%%%%%

\section{Alianzas estratégicas y oportunidad de negocio \# 3}
Dada a la complementariedad de los \emph{Smart Phones} se puede intentar que algunas empresas tengan diversos pactos de precios respecto a sus productos, algunas empresas de telefonía ya lo hacen, la práctica de promocionar a menores precios cierto producto complementario de manera que el cliente pueda comprar de una vez un paquete con el teléfono que quiere y con el plan de telefonía de su preferencia. Con estos bienes complementarios existe tal cosa como las ofertas a bajísimo precio al momento del cliente comprar el bien con el servicio ya que ambas empresas ganan un porcentaje por cada venta que efectúan. Alianzas estratégicas derivadas de esta coordinación es que a precios menores mayor demanda y tendrán más compras por los bienes que venden a comparación a que si los estuviesen vendiendo a independientemente.


%%%%%%%%%%%%%%%%%%%%%%%%%%%%%%%%%%%%%%%%%%%%%%%%%%%%%%%%%%%%%%%%%%%%%%%%%%%%%%%%%%%%%%%%%%%%%%%%

\section{Blibiografía}
\begin{enumerate}
    \item \url{https://www.electroschematics.com/mobile-phone-how-it-works/}
    \item \url{https://en.wikipedia.org/wiki/Mobile_telephony}
    \item \url{https://en.wikipedia.org/wiki/Mobile_telephony}
\end{enumerate}








\end{document}
