\documentclass{article}
\title{14 Principios De Fayol \& Mas}
\author{David Gabriel Corzo Mcmath}
\date{2020-Feb-17 19:20:39}
%%%%%%%%%%%%%%%%%%%%%%%%%%%%%%%%%%%%%%%%%%%%%%%%%%%%%%%%%%%%%%%%%%%%%%%%%%%%%%%%%%%%%%%%%%%%%%%%%%%%%%%%%%%%%%%%%%%%%%%%%%%%%%%%%%%%%%%%%%%%%%%
\usepackage[margin = 1in]{geometry}
\usepackage{graphicx}
\usepackage{fontenc}
\usepackage{pdfpages}
\usepackage[spanish]{babel}
\usepackage{amsmath}
\usepackage{amsthm}
\usepackage[utf8]{inputenc}
\usepackage{enumitem}
\usepackage{mathtools}
\usepackage{import}
\usepackage{xifthen}
\usepackage{pdfpages}
\usepackage{transparent}
\usepackage{color}
\usepackage{fancyhdr}
\usepackage{lipsum}
\usepackage{sectsty}
\usepackage{titlesec}
\usepackage{calc}
\usepackage{lmodern}
\usepackage{xpatch}
\usepackage{blindtext}
\usepackage{bookmark}
\usepackage{fancyhdr}
\usepackage{xcolor}
\usepackage{tikz}
\usepackage{blindtext}
\usepackage{hyperref}
\usepackage{listing}
\usepackage{spverbatim}
\usepackage{fancyvrb}
\usepackage{fvextra}
\usepackage{amssymb}
\usepackage{pifont}
\usepackage{longtable}
\usepackage{url}
\makeatletter
\g@addto@macro{\UrlBreaks}{\UrlOrds}
\makeatother
%%%%%%%%%%%%%%%%%%%%%%%%%%%%%%%%%%%%%%%%%%%%%%%%%%%%%%%%%%%%%%%%%%%%%%%%%%%%%%%%%%%%%%%%%%%%%%%%%%%%%%%%%%%%%%%%%%%%%%%%%%%%%%%%%%%%%%%%%%%%%%%
\begin{document}
\maketitle

\section{Los catorce principios de Henry Fayol}
Según Henry Fayol que es considrado el padre de la gerencia moderna, todos los gerentes deben seguir los 14 principios de la administración que detalló él a lo largo de su vida. A continuación los 14 principios de Fayol.


%----------------------------------------------------------------------------------------
\subsection{División del trabajo}
Según Fayol, la organización debe de estar dividida entre los individuos y departamentos, esta división plantea él deja que los individuos y grupos de individuos se especialicen. Entonces es de vital importancia que los puestos delegados a cada persona sea congruente a su especialidad, de esta manera se logrará una organización más productiva, eficaz, eficiente y con eficacia.


%----------------------------------------------------------------------------------------
\subsection{Autoridad y responsabilidad}
Para Fayol era de suma importancia llegar al equilibrio entre la autoridad y la responsabilidad, si se delega mucha autoridad pero poca responsabilidad lo que termina ocurriendo es que los gerentes van a tender a abusar de su autoridad, por otro lado si se tiene poca autoridad y mucha responsabilidad es probable que termine ocurriendo que se culpe al gerente por todas las cosas y hasta de las que no tiene nada que ver. 


%----------------------------------------------------------------------------------------
\subsection{Disciplina}
La disciplina como afirmó Fayol es respetar las normas y reglamentos de la organización, hay personas diferentes en la organización, muchas de ellas no tienen auto-disciplina por lo que se tiene que agregar al sistema disciplinario sanciones para darles incentivos a dichas personas de ser disciplinadas.


%----------------------------------------------------------------------------------------
\subsection{Unidad de mando}
La unidad de mando es precisamente lo que el nombre indica, es que entre un subordinado o grupo de subordinados haya sólamente un jefe o gerente, también implica que dicho(s) subordinado(s) deben responder sólamente a una persona. Este concepto es importante ya que se necesita no solamente tener claro el organigrama si no que también al tener más de una unidad de mando los subordinados pueden recibir ordenes que se contradigan entre sí.


%----------------------------------------------------------------------------------------
\subsection{Unidad de dirección}
Todas las estrategias de dirección de la organización deben de ser dirigidas por un solo gerente, este debe emplear los procedimientos y planes que permitan que las organizaciones lleguen a sus objetivos, los gerentes deben poder dirigir hacia la \textbf{dirección} que la organización establezca.


%----------------------------------------------------------------------------------------
\subsection{Subordinación de interés individual al interés general}
Este principio sostiene básicamente sostiene la idea que el interés colectivo o general de la organización se va a favorecer más que el interés individual de cada miembro de la organización.


%----------------------------------------------------------------------------------------
\subsection{Remuneración}
Sostiene la idea que para maximizar el rendimiento de los empleados se tiene que compensar su trabajo de una manera justa, es decir tener una buena política de remuneración.
Toma en cuenta que esto no implica exclusivamente remuneración monetaria, también incluye remuneración no monetaria como premios, viajes, vacaciones, etc. 


%----------------------------------------------------------------------------------------
\subsection{Centralización}
Este principio se adapta al hecho que ninguna organización jerárquica puede ser plenamente centralizada o no centralizada, pues si es centralizada totalmente lo que ocurrirá es que los subordinados no tendrán autoridad y todo lo tendrá que decidir una sola persona con el arduo labor de conocer lo suficiente para considerar sus decisiones de una manera que funcione a un nivel local, y si se tiene una descentralización total lo que ocurre es que no hay autoridad y ningún gerente puede orientar ni dirigir a la organización a los objetivos establecidos. El truco es encontrar el \textbf{equilibrio} entre la centralización y la descentralización en una organización.


%----------------------------------------------------------------------------------------
\subsection{Cadena escalar}
En las organizaciones resulta de tremenda utilidad poder una jerarquía, pero para que el sistema jerárquico funcione debe tenerse claro que los superiores deben saber quienes son sus subordinados y los subordinados deben saber quién es su superior, deben implementar líneas de comunicación efectiva entre los mandos de tal manera que la información fluya para así tomar mejores decisiones y definir mejores estratégias. Sin embargo hay instancias en las que resulta más útil que se rompa la cadena, típicamente en instancias en la que se necesita acción rápida.


%----------------------------------------------------------------------------------------
\subsection{Orden}
Persigue la idea que para que una organización funcione óptimamente se debe de la misma manera que es más útil ordenar las cosas materiales en su lugar ordenar socialmente a los miembros de la organización, es decir que cada persona esté en el lugar adecuado, solo así se puede derivar los conceptos de especialización y de ahorro de recursos ya que si todo está en su lugar es más fácil todo.


%----------------------------------------------------------------------------------------
\subsection{Equidad}
Los superiores y los gerentes deben de dar un trato de equidad a su personal, la equidad en este sentido es una combinación de bondad y justicia. Esto permite que a lo largo del tiempo el gerente y sus subordinados puedan tener lealtad y devoción un respecto del otro, este concepto también ayuda formar las cadenas de comunicación efectiva mencionados anteriormente.


%----------------------------------------------------------------------------------------
\subsection{Estabilidad del personal}
Dado a que para cualquier humano la curva de aprendizaje para cualquier habilidad es a veces muy vasta, este principio sostiene la idea que al empleado se le debe de dar tiempo para aprender su trabajo y llegar a ser eficiente, eventualmente cuando llega a ser eficaz debe ser permanente, esto significa que el empleado debe tener seguridad laboral, no s conveniente andar cambiando a este empleado a otros puestos que exigen otras habilidades que tendrá que aprender, lo mejor es que se especialice en cierta habilidad y que después pueda hacer eso por bastante tiempo. Esto es congruente con la propuesta hecha por Adam Smith de la división del trabajo, él también decía que los cambios de tarea tienden a producir resultados ineficientes en una organización.

%----------------------------------------------------------------------------------------
\subsection{Iniciativa}
Cuando un empleado desea proponer una iniciativa que cree que funcionará y que hará que la organización cumpla sus objetivos de manera efectiva es conveniente escucharlos y permitir (si es en efecto el caso) la iniciativa propuesta por el empleado, esto produce resultados satisfactorios de dos maneras: 1. el empleado deriva satisfacción de poner sus planes en práctica y 2. la organización es más eficaz para cumplir sus objetivos. 

%----------------------------------------------------------------------------------------
\subsection{Espíritu de cuerpo}
Los miembros de una organización deben de ser unidos, es útil que el gerente cree un ambiente el cual se promueva la unidad y se desfavorezca la división. Esto permitirá la mejor y eficaz cooperación entre empleados, se dbe evitar la división y la política dado a que son temas altamente divididos y no aportarán a los objetivos propios de la organización.


%%%%%%%%%%%%%%%%%%%%%%%%%%%%%%%%%%%%%%%%%%%%%%%%%%%%%%%%%%%%%%%%%%%%%%%%%%%%%%%%%%%%%%%%%%
\section{Libros literarios de autores de la administración}
El management es un arte y una ciencia a la vez en el sentido que el arte por el talento que se requiere y ciencia por la forma de emplear las herramientas del management en la práctica. A continuación dos libros que me intrigaron por su contenido innovador y destacado en los temas de management.

\subsection{Conscious Business Fred Kofman}
Este libro es especialmente interesante por que adapta muchos excelentes ideas filosóficas y las orienta a la administración de una organización, este libro describe la relación que hay entre personas consientes de sus alrededores y la eficiencia y productividad. Es un libro muy profundo en el sentido que te hace reflexionar en qué es lo que los humanos hacen que se ha vuelto habitual y normal pero que son prácticas que son altamente desconsideradas y eficientes; también resalta la necesidad de que los empleados se traten con respeto y consideración, también favorece altamente conceptos como la empatía y eficiencia para lograr tener una organización consciente. 

\subsection{Scrum: The Art of Doing Twice the Work in Half the Time, Jeff Sutherland}
Es un libro que describe la implementación de un innovador e interesante modelo de organización llamado Scrum, precisamente uno de los creadores de la metodología Scrum es el autor de este libro, Jeff Sutherland. Scrum es difícil de resumir en un solo párrafo pero en breve es una metodología de agilidad empresarial que emplea conceptos como los de considerar la capacidad de los trabajadores, darles libertad a los trabajadores y emplear autoridad sólo en casos necesarios, también en diversos puestos se emplean nuevas funciones, por ejemplo el dueño de la empresa se vuelve en el ``product owner'', los trabajadores en el ``team'', adicional se agrega al modelo un puesto casi gerencial llamado ``scrum master'' este se encarga de cerciorarse que Scrum se esté aplicando correctamente y esté dando resultados.



%%%%%%%%%%%%%%%%%%%%%%%%%%%%%%%%%%%%%%%%%%%%%%%%%%%%%%%%%%%%%%%%%%%%%%%%%%%%%%%%%%%%%%%%%%
\section{Fuentes}
\begin{enumerate}[label={[\arabic*]}]
    \item \url{https://www.webyempresas.com/diferencia-entre-autoridad-y-poder-en-administracion/}
    \item \url{https://www.webyempresas.com/los-14-principios-de-henry-fayol/}
\end{enumerate}




















\end{document}
