\documentclass{article}

\usepackage{generalsnips}
\usepackage{calculussnips}
\usepackage[margin = 0.8in]{geometry}
\usepackage{pdfpages}
\usepackage[spanish]{babel}
\usepackage{amsmath}
\usepackage{amsthm}
\usepackage[utf8]{inputenc}
\usepackage{titlesec}
\usepackage{xpatch}
\usepackage{fancyhdr}
\usepackage{tikz}
\usepackage{hyperref}
\usepackage{enumitem}
\usepackage{url}
\title{Discusion Socratica \#2 - Caso Ferrari}
\date{2020 March 22}
\author{David Gabriel Corzo Mcmath}
\thispagestyle{empty}

\begin{document}
\maketitle
%%%%%%%%%%%%%%%%%%%%%%%%%%%%%%%%%%%%%%%%%%%%%%%%%%%%%%%%%%%%%%%%%%%%%%%%%%%%%%%%%%%%%%%%%%%%%%%%%%%%%%%%%%%%%%%%%%%%%%%%%%%%%%%%%%%%%%%%%%%%%%


%%%%%%%%%%%%%%%%%%%%%%%%%%%%%%%%%%%%%%%%%%%%%%%%%%%%%%%%%%%%%%%%%%%%%%%%%%%%%%%%%%%%%%%%%%
\section{Referencias bibliográficas}
% Referencias bibliográficas: Investigar dos referencias e indicar como éstas se relacionan con el argumento principal de la lectura.
\subsection{``The overarching goal is to create an experience—a sensual experience,'' - Stefan Thomke}
El modo Ferrari como se le suele decir en inglés ``the Ferrari way'' es básicamente una marca, tiene un posicionamiento interesante en el mercado. Stefan Thomke es un profesor de Harvard que hizo un estudio de caso a Ferrari, él pudo deducir que el producto implícito de Ferrari es en cierto aspecto la experiencia, como se menciono en la lectura los procesos de ingenería y diseño no sólo se enfocan qué tan rápido es su carro si no también en la comodidad y como en muchos casos la rapidez y la comodidad son caracteristicas mútuamente excluyentes que hacen que la combinación que ha logrado ofrecer Ferrri a sus clientes sea la combinación óptima. La capacidad de Ferrari a adaptarse a las necesidades de sus clientes es impresionante, es una de las empresas que mejor se ha adaptado a la creciente y cambiante demanda en el mercado. 



%----------------------------------------------------------------------------------------
\subsection{``I don’t know what technologies we will use in the future,” the driver said. “But I do know that any new technology will be deployed in our car the Ferrari way.'' - Ferrari's Chief Test Driver}
La alta capacidad que tiene Ferrari para poder no sólo manufacturar sus productos si no que adaptarse a las diferentes tendencias en el mercado es muy buena, Ferrari cuenta con uno de los equipos más sofisticados de ingenieros y técnicos, esto les permite adaptarse a la innovación que surge a través del tiempo, dado a esta capacidad que tiene Ferrari a adaptarse con tanta agilidad al fenómeno de la innovación en el mercado dí esta cita que me pareció que ilustra con exactitud esta capacidad, las tecnologías del futuro no las sabemos, pero lo que sabemos es que seguramente las vamos a estar utilizando. 


%%%%%%%%%%%%%%%%%%%%%%%%%%%%%%%%%%%%%%%%%%%%%%%%%%%%%%%%%%%%%%%%%%%%%%%%%%%%%%%%%%%%%%%%%%
\section{Dos conceptos o términos desconocidos}
% Identificar por lo menos dos conceptos o términos desconocidos por el estudiante que se encuentren en la lectura.  
\subsection{Las nueve dimensiones}
Una forma que adoptó Ferrari para media a los gerentes, ``competencia técnica, capacidad de gestión, resultados obtenidosen función del número de autos entregados, EBITDA y pasión'' este concepto es una derivación de un estándar para medir la efectividad y eficacia de un gerente. Durante el periodo de formalización de Ferrari adopto este framework para maximizar sus prácticas de gestión y posiciones.


%----------------------------------------------------------------------------------------
\subsection{La manera Ferrari}
Este es un concepto que se refiere a la manera de operar de Ferrari, el modelo de negocio de Ferrari con todo lo que eso implica, un modelo que ha funcionado a través de la vasta cantidad de tiempo que lleva operando. 


%%%%%%%%%%%%%%%%%%%%%%%%%%%%%%%%%%%%%%%%%%%%%%%%%%%%%%%%%%%%%%%%%%%%%%%%%%%%%%%%%%%%%%%%%%
\section{Análisis de la industria}
% Realice un análisis de la industria, utilizando las 5 Fuerzas de Michael Porter y el FODA.  Explique detalladamente cada uno.
\subsection*{Requisitos del analisis}
\begin{enumerate}
    \item Misión: ``We build cars, symbols of Italian excellence the world over, and we do so to win on both road and track. Unique creations that fuel the Prancing Horse legend and generate a “World of Dreams and Emotions”.''
    \item Visión: ``Ferrari, Italian Excellence that makes the world dream''
    \item Valores: 
        \begin{enumerate}
            \item Individual and team 
            \item Emotion 
            \item Integrity 
            \item Tradition and innovation 
            \item Passion and excelence 
        \end{enumerate}
\end{enumerate}

\subsection{Fuerzas de Michael Porter}
\subsubsection{Proveedores}%
Ferrari tiene un vasto número de proveedores de los componentes a sus productos que necesitan, pero es importante considerar lo siguiente``La empresa evitó la externalización de componentes que consideraba vitales para su proceso de producción'', como podemos ver Ferrari tiene un modelo de economía de escala y economía de alcance en la que ellos quieren hacer todos los componentes que sean necesarios y consideran que ellos son los mejores en esa labor. Entonces Ferrari tiene mucho poder de negociación en este aspecto porque la cantidad de proveedores necesaria es poca y fácilmente cambiable.


\subsubsection{Productos sustitutos}%
Realmente sustitutos a Ferrari hay una vasta cantidad, pero Ferrari es importante pensarlo como un producto que brinda una diversa cantidad de satisfacciones, por un lado está el hecho que es un medio de transformación y por otro lado la experiencia Ferrari y entre otras cosas, es por eso que Ferrari tiene muchos sustitutos en términos de vender un carro, pero tiene casi ninguno si la satisfacción que se busca es la experiencia Ferrari. Hay pocos productos sustitutos que cubren la exclusividad que ellos proveen. 


\subsubsection{Clientes}%
Los clientes de Ferrari son en su mayoría muy leales, algunos dispuestos hasta a esperar meses para poder comprar un carro Ferrari. Entonces el poder de negociación que Ferrari tiene con sus clientes es bastante alto, este bien de lujo para los clientes de Ferrari se vuelve un bien casi inelástico que es algo curioso por que los bienes de lujo son regularmente elásticos.

\subsubsection{Nuevos entrantes}%
Realmente nuevos entrantes tendrían que tener una fuente de financiación muy fuerte y duradera por que intentar implementar todos los sistemas y una marca que compita con Ferrari sería exorbitantemente difícil.

\subsubsection{Competencia en el mercado}%
Ferrari no tiene barreras de entrada legales, sin embargo tiene barreras de entrada, sería muy difícil implementar toda la tecnología y capital que tiene Ferrari, esto no sería nada fácil de hacer aún si se tuviese el capital para hacerlo. La inversión en maquinaria y procesos que requeriría entrar a este mercado es vasta, además se tiene que considerar toda la experiencia y aprendizaje que Ferrari ha adquirido a lo largo del tiempo. Sin embargo competidores existen que hacen que mútuamente estén incentivados a innovar y a ganarle uno al otro.

%----------------------------------------------------------------------------------------
\subsection{Analisis FODA}
\begin{center}
    \begin{tabular}{ |p{8cm}|p{8cm}| }
        \hline
            Fortalezas: & Oportunidades \\
            \begin{itemize}
                \item Tiene mucha capacidad de adaptarse a los cambios en el mercado.
                \item Su modelo de negocio es eficiente, cosas como la remuneración generosa y los descansos a los empleados.
                \item Sabe en qué determinarse para crear su producto. 
                \item Tiene clientes profundamente leales.
            \end{itemize} & 
            \begin{itemize}
                \item Las tendencias tecnológicas.
                \item Las tendencias de automatización.
                \item Regulaciones del medio ambiente permiten que Ferrari demuestre su capacidad de incorporar motores eléctricos de la misma eficiencia que los de combustión interna. (Anexo 6)
                \item La oferta de Ferrari podría incrementar por cambios en precios de las materias primas.
            \end{itemize} \\ 
        \hline
            Debilidades: & Amenazas \\ 
            \begin{itemize}
                \item El marketing no es óptimo. 
                \item La estructura de la organización no se presta a la flexibilidad. 
                \item Su modelo de negocio no es del todo compatible con otros por lo que le cuesta encontrar colaboradores.
            \end{itemize} & 
            \begin{itemize}
                \item Regulaciones de los motores de combustión interna, si Ferrari no se muestra lo suficientemente eficiente para sacar el motor eléctrico, empresas como Tesla podrían tomar el mercado. (Anexo 5)
                \item La competencia es feroz.
                \item Las tendencias que se han popularizado de subir el salario mínimo podrían afectar a la compañía. 
                \item Productos de imitación pueden tomar el mercado, otros productos de menor calidad que brinden una pseudo-sensación Ferrari podrían afectar los ingresos de Ferrari.
                \item El marketing privativo puede tornar a dar un resultado no esperado si no se hace con cuidado.
            \end{itemize} \\ 
        \hline
    \end{tabular}
\end{center}







%%%%%%%%%%%%%%%%%%%%%%%%%%%%%%%%%%%%%%%%%%%%%%%%%%%%%%%%%%%%%%%%%%%%%%%%%%%%%%%%%%%%%%%%%%
\section{Tres principales tendencias de esta industria}
% ¿Cuáles son las tres principales tendencias que hay en esta industria y cómo se relacionan con lo que Ferrari está haciendo actualmente? 
En la industria de automóviles de lujo habían surgido demandas de tres principales características: 
\begin{enumerate}
    \item Electrificación de tren motriz 
    \item Conectividad y  
    \item automatización en la conducción.
\end{enumerate}
Dado a el surgimiento de estas tendencias Ferrari y sus competidores empezaron a perseguir las metas que dictaba el mercado. El modelo de negocio de Ferrari está basado en marketing por diferenciación por ende producen menos carros de lo que realmente se demanda, eso para mantener la exclusividad de auto de lujo. 

%%%%%%%%%%%%%%%%%%%%%%%%%%%%%%%%%%%%%%%%%%%%%%%%%%%%%%%%%%%%%%%%%%%%%%%%%%%%%%%%%%%%%%%%%%
\section{Mayores retos o desafíos de Ferrari}
% ¿Cuáles son los mayores retos o desafíos que ha tenido Ferrari en los últimos 5 años? Compárelo con 2 marcas que se encuentren en la misma industria.
Ferrari es una compañía que ya lleva aproximadamente 80 años de existir a pesar de esto no ha demostrado incapacidad de adaptarse a los avances en la tecnología, sin embargo esto no significa que no sea reto. El reto de innovar con carros de gran capacidad ha sido un gran reto de superar, sin embargo Ferrari ha logrado superarlo. Recientemente se ha intentado responder a la tendencia de la automatización y los motores eléctricos. En cierto aspecto Ferrari siendo una empresa que se ha dedicado tantos años a la manufactura de motores de combustión interna les costará adaptarse a esta nueva demanda que ha surgido en el mercado, pero desde ya Ferrari está intentando competir en el mercado con esta nueva tendencia, planea sacar carros eléctricos en un futuro cercano. Una de las maneras que podemos observar es viendo las acciones de la empresa Ferrari respecto a algunas de sus competidores. 
\subsection*{Acciones Ferrari}
\begin{itemize}
    \item Closing price: 179.21 
    \item Ferrari 52 week low stock price is 127.73 or 1.4\% below current share price.
    \item Ferrrari 52 week high stock price is 180.95 or 39.7\% 
    \item Average Ferrari Stock price 156.46.
\end{itemize}

\subsection*{Acciones Ford}
\begin{itemize}
    \item Ford Motor stock closing price was 42.45.
    \item 52-week high stock price is 10.56, which is 163.3\%
    \item 52-week low stock price is 3.96, which is 1.2\%
\end{itemize}

\subsection*{Acciones Tesla}
\begin{itemize}
    \item Tesla stock closing price was 917.42
    \item 52-week high stock price is 968.99, which is 123.1\% 
    \item 52-week low stock price is 176.99, which is 59.2\%
\end{itemize}

Como se puede observar Tesla ahora ha sido el líder del mercado de carros eléctricos y este nuevo surgimiento ha sido un reto para Ferrari y otros por ya haber implementado un proceso para manufactura de motores de combustión interna.

\subsection*{Otro reto Ferrari}
Se mencionó en este texto que uno de los más grandes retos es llegar al equilibrio de brindar un producto con características de comodidad y rapidez, la competencia ``esta dispuesta a agregar cuero de sillón con tal de aumentar la comodidad'' esto a coste de la rapidez, es por esto que Ferrari dice ser una mezcla perfecta de las dos, en la que se aprecian las dos características, esto es un reto de ingeniería y diseño, un reto que Ferrari ha logrado solucionar muy bien. Aun que Tesla tiene la ventaja en motores eléctricos lo más probable es que los mismos clientes que compren Ferraris en el futuro compren los carros eléctricos que producirá Ferrari.


\subsection*{Otro reto Ferrari}
Al principio cuando se vendían pocos volúmenes a los consumidores lo que ocurría era que Ferrari disminuía sus ingresos, por que a pocos volúmenes no se maximizan beneficios siempre, este problema Ferrari logró resolverlo con su estratégia de marketing privativo, que sostiene la idea de querer escasear el producto a propósito para maximizar su exclusividad, el resultado un carro lujoso que pertenece a clientes fanáticos e ideales para empresas como Ferrari, logró superar este reto con mucha astucia. Sus competidores como los mencionados anteriormente no han ejercido prácticas como tales y por eso es que Ferrari con su estrategia de marketing privativo tiene una gran cantidad de mercado reservado para ellos.






%%%%%%%%%%%%%%%%%%%%%%%%%%%%%%%%%%%%%%%%%%%%%%%%%%%%%%%%%%%%%%%%%%%%%%%%%%%%%%%%%%%%%%%%%%
\section{Estrategia global de competencia de Ferrari}
% ¿Cuál es la Estrategia Global de Competencia de Ferrari? Justifíquela y respalde con datos.
La estrategia global de competencia de Ferrari es de diferenciación, ya que se observa que es evidente que Ferrari vende carros de lujo, son bienes con mucho valor agregado, además los vende a bajos volúmenes y a precios altos, son también principalmente un bien de status, también tienen la característica de que sólo distribuidores autorizados pueden vender. \newline 
Respaldo mis argumentos con la lectura: ``Lo que hace a la exclusividad es el volumen que se entrega vs. la demanda'', explicó. ``Si tengo una demanda de 20 000 autos y entrego 16 000, estoy duplicando el negocio y sigo siendo exclusivo''; esto lo dijo Enrico Galliera el director de marketing y comercial de Ferrari. Otro ejemplo en el que se puede observar evidencias que la estrategia de competencia global de Ferrari es la diferenciación es el hecho que escasean su producto para hacerlo más exclusivo, esto es clave dado a que ellos saben que tienen una amplia gama de clientes profundamente leales que están dispuestos hasta a esperar y ``sufrir'' para convertirse en  \emph{Ferrarista} es decir dispuestos a se parte del marketing privativo de Ferrari con tal de comprarles su producto, son casi clientes fanáticos.



%%%%%%%%%%%%%%%%%%%%%%%%%%%%%%%%%%%%%%%%%%%%%%%%%%%%%%%%%%%%%%%%%%%%%%%%%%%%%%%%%%%%%%%%%%
\section{Principales áreas clave}
% De acuerdo a la Estrategia Global de Competencia, ¿cuáles son las principales áreas clave para Ferrari?
\begin{center}
    \begin{tabular}{ |p{3cm}|p{1cm}|p{6cm}|p{6cm}| }
        \hline
            Áreas Clave & Tipo & Descripción y justificación del área clave & Alineación con objetivos globales \\ 
        \hline
            Comodidad y rapidez & Core & Dado a que Ferrari es una compañía que se como parte de lo que ofrece a comparación de sus competidores es la combinación mágica de comodidad y rapidez que pueden en un momento muy bien llegar a ser mútuamente excluyentes. & Genera la experiencia única de Ferrari mencionada en su misión. 
            \\
        \hline
            Ingenería y diseño & Core & El equipo de ingenieros y diseñadores de Ferrari es uno de los más sofisticados en el mundo, ellos permiten que el carro sea lo mejor de lo mejor. & Ellos coperan a construir las ``símbolos de excelencia italiana en el mundo'' mencionado en la misión \\ 
        \hline
            Experiencia Ferrari & Soporte & La experiencia Ferrari es una de las cosas que Ferrari ha logrado que sus clientes valoren. & Esto va directamente relacionado con la misión donde dice ``generate a world of dreams and emotions'' y también con la visión donde dice ``Ferrari, italian excelence that makes the world dream''. \\ 
        \hline

        \hline
    \end{tabular}
\end{center}






%%%%%%%%%%%%%%%%%%%%%%%%%%%%%%%%%%%%%%%%%%%%%%%%%%%%%%%%%%%%%%%%%%%%%%%%%%%%%%%%%%%%%%%%%%
\section{Entrevista hipotética a CEO Ferrari}
% Si usted pudiera entrevistar al CEO de Ferrari, ¿qué preguntas le haría? 
\begin{enumerate}
    \item \pregunta{Consideraría que si Ferrari subiera las unidades producidas bajaría la exclusividad del producto hoy en día} 
    \item \pregunta{Cuál diría usted que es una de las tendencias más recientes aparte de las que ya mencionó} 
    \item \pregunta{Cuánto tiempo se tardan en recuperar la inversión} 
    \item Según fuentes el costo de producción y el precio no deriva una ganancia grandísima, \pregunta{Apoyaría esta afirmación} 
    \item \pregunta{Dónde cree que estará Ferrari en el futuro} 
    \item Mucha gente dice que su modelo de negocios tiene el nombre ``The Ferrari Way'' \pregunta{De dónde opina usted que se derivaron estas prácticas gerenciales} 
    \item \pregunta{Consideraría beneficioso que ferrari se metiera a otros mercados, por ejemplo los aviones} 
\end{enumerate}


%%%%%%%%%%%%%%%%%%%%%%%%%%%%%%%%%%%%%%%%%%%%%%%%%%%%%%%%%%%%%%%%%%%%%%%%%%%%%%%%%%%%%%%%%%
\section{Fuentes de investigación}
\begin{enumerate} 
    \item \url{https://www.ferrari.com/es-GT/news/3}
    \item \url{https://www.mtvehicles.com/blog/how-much-does-it-actually-cost-manufacturers-to-make-a-car/} 
    \item \url{https://hbswk.hbs.edu/item/the-ferrari-way}
    \item \url{https://www.inc.com/don-reisinger/ferraris-electric-tesla-killer-is-coming-but-you-still-have-time-to-save-up.html}
    \item \url{https://www.epa.gov/ghgemissions/sources-greenhouse-gas-emissions}
    \item \url{http://fernfortuniversity.com/term-papers/swot/1433/1285-ferrari.php}
    \item \url{https://corporate.ferrari.com/en/about-us/ferrari-dna}
    \item \url{https://www.procurementbulletin.com/ferraris-supply-chain-is-facing-more-demand/}
    \item \url{https://www.macrotrends.net/stocks/charts/TSLA/tesla/stock-price-history}
    \item \url{https://www.macrotrends.net/stocks/charts/TSLA/tesla/stock-price-history}
    \item \url{https://www.macrotrends.net/stocks/charts/RACE/ferrari/stock-price-history}
    \item \url{https://www.macrotrends.net/stocks/charts/F/ford-motor/stock-price-history}
    \item \url{https://www.nytimes.com/2014/04/19/sports/autoracing/ferraris-big-challenge.html}
    \item \url{http://fernfortuniversity.com/term-papers/swot/1433/1285-ferrari.php}
    \item \url{https://www.cnet.com/roadshow/news/ferrari-electric-car-tesla-roadster/}
\end{enumerate}















%%%%%%%%%%%%%%%%%%%%%%%%%%%%%%%%%%%%%%%%%%%%%%%%%%%%%%%%%%%%%%%%%%%%%%%%%%%%%%%%%%%%%%%%%%%%%%%%%%%%%%%%%%%%%%%%%%%%%%%%%%%%%%%%%%%%%%%%%%%%%%
\end{document}

