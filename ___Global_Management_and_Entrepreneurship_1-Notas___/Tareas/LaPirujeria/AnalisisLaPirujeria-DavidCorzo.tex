\documentclass{article}
%%%%%%%%%%%%%%%%%%%%%%%%%%%%%%%%%%%%%%%%%%%%%%%%%%%%%%%%%%%%%%%%%%%%%%%%%%%%%%%%%%%%%%%%%%%%%%%%%%%%%%%%%%%%%%%%%%%%%%%%%%%%%%%%%%%%%%%%%%%%%%%
\usepackage{generalsnips}
\usepackage{enumitem}
\usepackage{supertabular}
\usepackage{url}
\usepackage{float}
\packagesneeded
\usepackage[top=0.6in, bottom=1in, left=0.70in, right=0.70in]{geometry}
\title{La pirujería - Planeación estratégica y operativa}
\date{2020/04/14}
\author{David Gabriel Corzo Mcmath}
%%%%%%%%%%%%%%%%%%%%%%%%%%%%%%%%%%%%%%%%%%%%%%%%%%%%%%%%%%%%%%%%%%%%%%%%%%%%%%%%%%%%%%%%%%%%%%%%%%%%%%%%%%%%%%%%%%%%%%%%%%%%%%%%%%%%%%%%%%%%%%%
\begin{document}
\maketitle
%%%%%%%%%%%%%%%%%%%%%%%%%%%%%%%%%%%%%%%%%%%%%%%%%%%%%%%%%%%%%%%%%%%%%%%%%%%%%%%%%%%%%%%%%%
\section{La pirujería}
\subsection{FODA}
\begin{center}
   \begin{tabular}{ | p{8.25cm} | p{8.25cm} | }
        \hline
            \textbf{Fortalezas: } 
            \begin{itemize}
                \item Capacidad sobresaliente de crecer y expandirse.
                \item Parametrizan muy bien el riesgo e implementan medidas correctas para sustraerlo.
                \item El mantenimiento de ciertos procesos es rutinario y preventivo.
                \item Entregan con precisión y eso caracteriza su estrategia.
            \end{itemize}
            &
            \textbf{Oportunidades: } 
            \begin{itemize}
                \item Nuevos mercados de expansión.
                \item Tecnificar la producción de pirujos.
                \item Apertura de nuevas franquicias.
                \item Tendrán una licitación.
            \end{itemize}
            \\ 
        \hline
            \textbf{Debilidades: } 
            \begin{itemize}
                \item No tienen una forma de proceder si el precio de la harina y el aceite suben.
                \item No cuentan aún con una planeación por escrito.
            \end{itemize}
            &
            \textbf{Amenazas: } 
            \begin{itemize}
                \item Precio de la harina en aumento.
                \item Precio del aceite en aumento.
                \item Ajuste salarial que que entró en vigencia en enero 2020. 
            \end{itemize}
            \\ 
        \hline
   \end{tabular}
\end{center}



%----------------------------------------------------------------------------------------
\subsection{Valores}
Valores de la pirujería:
\begin{itemize}
    \item Preservar la receta: Promulgar la receta auténtica guatemalteca de pirujos.
    \item Precisión: Entregar producto con precisión en la entrega.
    \item Expansión: abrir nuevas franquicias.
    \item Calidad: producir pirujos de calidad con recetas auténticas guatemaltecas.
\end{itemize}

\subsubsection{Misión y visión}
\begin{itemize}
    \item \textbf{Misión: ``Llegar a todos los rincones de Guatemala, para que se conozcan sus deliciosos pirujos como un producto autentico guatemalteco y único en su categoría.''}
        \begin{itemize}
            \item ¿Quiénes somos?
                \begin{itemize}
                    \item Una empresa que se compromete con sus clientes a brindar priujos de alta calidad con servicios altamente precisos en su operación.
                \end{itemize}
                
            \item ¿Cuál es la promesa de servicio?
                \begin{itemize}
                    \item Precisión en la entrega y calidad en el producto. 
                \end{itemize}
                
            \item ¿Qué ofrecemos que nos hace únicos?
                \begin{itemize}
                    \item Un producto autentico guatemalteco y único en su categoría.
                \end{itemize}
                
            \item ¿Quiénes son nuestros clientes?
                \begin{itemize}
                    \item Personas interesadas en el consumo de pirujos de calidad.
                \end{itemize}
        \end{itemize}
    
    \item \textbf{ Visión: ``Lograr expandirnos por medio de franquisias y proveyendo productos de calidad a través de ellas''.}
        \begin{itemize}
            \item \pregunta{Hacia dónde vamos} 
                \begin{itemize}
                    \item Vamos a expandir nuestras tiendas por medio de franquicias para que más personas puedan degustar pirujo de calidad.
                \end{itemize}
                
            \item \pregunta{En qué queremos ser un referente} 
                \begin{itemize}
                    \item Calidad, crecimiento, servicio.
                \end{itemize}
                
            \item \pregunta{Cómo nos vemos en el futuro} 
                \begin{itemize}
                    \item Expandidos por muchos rincones de Guatemala. 
                \end{itemize}
                
            \item \pregunta{Cuál será nuestro legado} 
                \begin{itemize}
                    \item Nuestra receta y producto de calidad que la gente haya crecido a amar con forme el tiempo transcurra.
                \end{itemize}
                
        \end{itemize}
\end{itemize}



%----------------------------------------------------------------------------------------
\subsection{Estrategia global de competencia}
\begin{center}
    \begin{tabular}{ |p{3cm}|p{8cm}| }
        \hline
            Estrategia global de competencia & Justificación \\
        \hline
            Liderazgo en costos: Excelencia operativa & Dado al hecho que se mantiene muy bien la maquinaria, y se intenta implementar nuevos sistemas y equipos capaces de producir más eficientemente la estrategia global de competencia es la de esta empresa. \\ 
        \hline
    \end{tabular}
\end{center}



%----------------------------------------------------------------------------------------
\section{Plazos de la planeación estratégica}
\begin{center}
    \begin{tabular}{ |p{2cm}|p{2cm}|p{5cm}|p{5cm}| }
        \hline
            Plazo & Criterios & Justificación & Objetivos globales\\
        \hline 
            Corto plazo (1 año): 
            & 
            Velocidad de cambio en la tecnología
            & 
            Dado a que por mecanismos de seguridad y de sustracción de riesgo la empresa debe abrir una nueva franquicia cada por año, debe poder recuperar las inversiones para poder realizar eso. 
            & 
            Objetivo: minimizar el riesgo y aumentar la velocidad de expansión para el 31 de diciembre de 2021.
            \\ 
        \hline
            Mediano plazo (5 año): 
            & 
            Barreras de entrada a nuevos mercados
            & 
            Se intenta abrir una tienda cada año, las barreras de entrada al mercado pueden ser un gran problema si se llegasen a complicar, es por eso que se tiene que considerar en la planeación. 
            & 
            Objetivo: tener un sistema de expansión que por lo menos le permita abrir una tienda más anualmente para el 31 de diciembre de 2021. 
            \\ 
        \hline
            Largo plazo (10 año): 
            & 
            Periodo fiscal 
            & 
            Eventualmente conforme nuevos gobiernos vayan tomando posesión del poder se tendrán problemas fiscales, es correcto planear para tal escenario. 
            & 
            Objetivo: Expandirse a tener una nueva planta de producción en Costa Rica para febrero 2021 y tener suficiente presencia en la región. 
            \\ 
        \hline
    \end{tabular}
\end{center}




%%%%%%%%%%%%%%%%%%%%%%%%%%%%%%%%%%%%%%%%%%%%%%%%%%%%%%%%%%%%%%%%%%%%%%%%%%%%%%%%%%%%%%%%%%
\section{Fuerzas competitivas de Michael Porter}
\begin{enumerate}
    \item Proveedores: Muchos proveedores en el mercado además la mayoría brinda muy buena calidad de materia prima. 
    \item Productos sustitutos actuales: Sustitutos de pan claro que hay, pero la pirujería tiene recetas diferentes y por ende no es del todo igual a sus competidores en términos de calidad. 
    \item Clientes: La demanda de pirujos diarios es de 15,000 es una vasta cantidad de producto que los clientes demandan y el mercado es vasto y grande. 
    \item Nuevos entrantes: No hay muchos nuevos entrantes al mercado por que es muy competitivo. 
    \item Competencia en el mercado: Es un mercado competitivo por que es competencia entre bienes esenciales o básicos, por consiguiente tiende a haber más competencia. 
\end{enumerate}




%%%%%%%%%%%%%%%%%%%%%%%%%%%%%%%%%%%%%%%%%%%%%%%%%%%%%%%%%%%%%%%%%%%%%%%%%%%%%%%%%%%%%%%%%%
\section{Áreas clave de la empresa}
\begin{center}
    \begin{supertabular}{ |p{2cm}|p{2cm}|p{5.5cm}|p{5.5cm}| }
        \hline
            Áreas clave 
            & Core / Soporte 
            & Descripción y justificación 
            & Alineación con objetivos globales \\ 
        \hline
            Costos y finanzas 
            & Core 
            & El mercado en el que están es sumamente competitivo es por eso que se tiene que tener un liderazgo en costos y recuperación de inversiones para poder mantener la empresa a flote. 
            & Objetivo global: abrir una nueva planta de producción en Costa Rica para Febrero 2021 \& tener suficiente presencia en la región.
            \\ 
        \hline
            Maquinaria y mantenimiento 
            & Core 
            & El mantenimiento de sus máquinas y la modernización de las mismas ha sido un área medular en toda esta operación, sin este tipo de reinversiones no se pudiese tener la constante actualización de máquinas en la empresa y como consecuencia se tendría que tener menos producción. 
            & Objetivo global: renovación del equipo para febrero del 2021. \\ 
        \hline
            Calidad en el producto y entrega precisa 
            & Core 
            & Tener congruencia entre lo que se vende y lo que se ofrece es fundamental para una organización 
            & Objetivo global: Expandir operación de una franquicia más anual para el 31 de diciembre de 2020. \\ 
        \hline
            Gestión de talento 
            & Soporte 
            & Tener a gente con cultura organizacional y cultura ganadora es importante para poder tener una organización en crecimiento. Es por eso que se tiene que tener una buena gestión de talento.
            & Objetivo: minimizar el riesgo y aumentar la velocidad de expansión para el 31 de diciembre de 2020 \\ 
        \hline
    \end{supertabular}
\end{center}


\section{Áreas clave, estrategias menores y tácticas}
\begin{center}
    \begin{supertabular}{ |p{1.5cm}|p{4cm}|p{4cm}|p{4cm}| }
        \hline
            Áreas clave 
            & Objetivos Globales 
            & Estrategia menor 
            & Estrategia táctica \\
        \hline
            Maquinaria y mantenimiento 
            & La renovación del equipo también esperan abrir una nueva planta de producción en Costa Rica para Febrero 2021 
            % estrategia menor
            & Por medio de capital ahorrado se intentará garantizar el logro de los objetivos de renovación de equipo, comprando nuevos equipos y restaurando los preexistentes cuando sea necesario utilizando re-inversiones de los accionistas.  
            % estrategia táctica
            & Por medio de considerar nuevas participaciones a los accionistas tener incentivos a conseguir el capital para poder invertir en nuevas tecnologías que puedan maximizar la producción de la empresa de la manera más eficiente posible. \\ 
        \hline
            Costos y finanzas 
            & Este sistema de expansión les ha permitido minimizar el riesgo y aumentar la velocidad de expansión, pues han ido abriendo 1 franquicia por año durante los últimos 5 años y desean seguir haciéndolo de ésta forma para no correr riesgos, aumentar la expansión por una franquicia más para el 31 de diciembre de 2020. 
            % estrategia menor
            & Aumentar las ventas ofreciendo un incentivo de remuneración por buen trabajo, para aumentar la productividad se debe remunerar correctamente el trabajo de los empleados. 
            % estrategia táctica
            & Tener un sistema de premiación individual y colectivo que consistan individualmente dar incentivos a trabajar en equipo al igual que individualmente, según califique el trabajo por hacer, estos incentivos serán comisiones pagadas al fin de mes. Tener instituido esto para el 31 de diciembre del 2020.   \\ 
        \hline  
            Calidad en el producto y entrega precisa 
            & Poder llegar a todos los rincones de Guatemala, para el 31 de diciembre de 2021 poder expandirse no sólo a Guatemala si no que a Costa Rica también. 
            % estrategia menor
            & Poder gestionar con proveedores y franquicias en el extranjero para gestionar exportaciones y movimiento fuera del país.
            % estrategia táctica 
            & Por medio de estas nuevas posibles alianzas estratégicas establecer una amplia red de distribución pertinente a cada país para así poder expandirse a lo largo de Centro América a lo largo de los próximos 5 años \\ 
        \hline
            Gestión de talento 
            & Tener suficiente presencia en la región para febrero de 2021. 
            % estrategia menor
            & Lograr por medio de alianzas estratégicas tener mayor presencia en la región con capital invertido de los inversionistas. 
            % estrategia táctica
            & Por medio de capital de los inversionistas lograr implementar una red de tiendas en la región de Costa Rica promocionandonos con el uso de la publicidad y de promoción.  \\ 
        \hline
    \end{supertabular}
\end{center}


\section{Fuentes de investigación}  
\begin{enumerate}
    \item \url{https://elperiodico.com.gt/opinion/2018/09/27/pan-y-tortilla/}
    \item \url{http://biblioteca.usac.edu.gt/tesis/06/06_2243.pdf}
    \item \url{https://www.cmimolinosmodernos.com/blog-de-molinos-modernos/nuevas-tendencias-de-panificaci%C3%B3n}
\end{enumerate}



\end{document}
