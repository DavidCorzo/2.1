\documentclass{article}
\title{\Huge Global Management And Entreprenourship \\ Startups \\ Análisis del startup \emph{Brioagro Tecnologies}}
\author{David Gabriel Corzo Mcmath}
\date{2020-Feb-0 09:59:02}
%%%%%%%%%%%%%%%%%%%%%%%%%%%%%%%%%%%%%%%%%%%%%%%%%%%%%%%%%%%%%%%%%%%%%%%%%%%%%%%%%%%%%%%%%%%%%%%%%%%%%%%%%%%%%%%%%%%%%%%%%%%%%%%%%%%%%%%%%%%%%%%
\usepackage[margin = 1in]{geometry}
\usepackage{graphicx}
\usepackage{fontenc}
\usepackage{pdfpages}
\usepackage[spanish]{babel}
\usepackage{amsmath}
\usepackage{amsthm}
\usepackage[utf8]{inputenc}
\usepackage{enumitem}
\usepackage{mathtools}
\usepackage{import}
\usepackage{xifthen}
\usepackage{pdfpages}
\usepackage{transparent}
\usepackage{color}
\usepackage{fancyhdr}
\usepackage{lipsum}
\usepackage{sectsty}
\usepackage{titlesec}
\usepackage{calc}
\usepackage{lmodern}
\usepackage{xpatch}
\usepackage{blindtext}
\usepackage{bookmark}
\usepackage{fancyhdr}
\usepackage{xcolor}
\usepackage{tikz}
\usepackage{blindtext}
\usepackage{hyperref}
\usepackage{listing}
\usepackage{spverbatim}
\usepackage{fancyvrb}
\usepackage{fvextra}
\usepackage{amssymb}
\usepackage{pifont}
\usepackage{longtable}
\usepackage{url}
\makeatletter
\g@addto@macro{\UrlBreaks}{\UrlOrds}
\makeatother
%%%%%%%%%%%%%%%%%%%%%%%%%%%%%%%%%%%%%%%%%%%%%%%%%%%%%%%%%%%%%%%%%%%%%%%%%%%%%%%%%%%%%%%%%%%%%%%%%%%%%%%%%%%%%%%%%%%%%%%%%%%%%%%%%%%%%%%%%%%%%%%
\begin{document}
\maketitle

\section{Startups - Descripción e información preliminar}
Los startups son uno de los medios más populares para poder emprender una gran idea, usualmente se lleva a cabo en regiones donde son bienvenidos los startups, es decir, regiones donde el gobierno al menos en las primeras etapas te ayuda con cobrarte menos impuestos y asesoramiento tributario, al igual que tienen programas para startups, es decir reconocen el objeto startups y estas empresas pueden tener menos incertidumbre por esto. Muchos países y regiones apoyan este ambiente fértil para los startups, saben que las empresas primitivas como los startups pueden llegar a aportar gran valor a la sociedad y por ende sacan programas como el \emph{\textbf{mipimes}} que es básicamente un programa tributario en términos de contabilidad para las empresas micro, pequeñas \& medianas.

\subsection{B2B \& B2C}
La abreviación \emph{B2B} es Business to Business \& la abreviación \emph{B2C} es Business to consumer. Una de las más importantes cosas de tener claro es quién es nuestro mercado, en este sentido es muy relevante entender si le estas vendiendo tu producto a consumidores finales o a consumidores intermedios, es importante saber si tu producto es recibido por el mercado como materia prima o como bienes de consumo final.


%%%%%%%%%%%%%%%%%%%%%%%%%%%%%%%%%%%%%%%%%%%%%%%%%%%%%%%%%%%%%%%%%%%%%%%%%%%%%%%%%%%%%%%%%%%%%%%%
\section{\emph{BrioAgro Tecnologies} ¿Qué es?}
La empresa \emph{BrioAgro Tecnologies} es una empresa que es B2B ya que provee productos a proveedores esencialmente, proveedores de productos agrícolas. La empresa desarrolló un dispositivo con altas capacidades tecnológicas. Se trata de un dispositivo que es capaz de brindar más control al agricultor de sus siembras, este dispositivo recopila toda la información relevante que detecte y la sube a la nube de tal manera que permite el control más efectivo de sus cosechas y si algo llegase a estar mal poder hacer algo al respecto de una manera más rápida que ir personalmente a ver qué pasa con la siembra.
\newline 
El sistema provee monitoreo las 24h del día y el agricultor recibe información en tiempo real. De cierta manera es como que si le estuviesen reduciendo grandemente la incertidumbre al agricultor.


%%%%%%%%%%%%%%%%%%%%%%%%%%%%%%%%%%%%%%%%%%%%%%%%%%%%%%%%%%%%%%%%%%%%%%%%%%%%%%%%%%%%%%%%%%%%%%%%
\section{¿Clientes?}
Los clientes de esta empresa son diversos, pero entre su público objetivo están las personas con grandes siembras agrícolas, el producto está enfocado de tal manera que se instale a lo largo de grandes planicies de siembra, y se vende al por mayor. Cuando se toman en cuenta todos estos aspectos es razonable concluir que esta empresa quiere proveerle a personas con grandes plantaciones y no mucho a una persona con un jardín de 15m$^2$.


%%%%%%%%%%%%%%%%%%%%%%%%%%%%%%%%%%%%%%%%%%%%%%%%%%%%%%%%%%%%%%%%%%%%%%%%%%%%%%%%%%%%%%%%%%%%%%%%
\section{¿B2B? o ¿B2C?}
Dado a que la empresa está claramente orientada a personas que tienen una vasta cantidad de terreno que cuidar es una empresa B2B, estos agricultores son proveedores de supermercados y de otras empresas, es una empresa Business to Business. No es un producto de consumo final, se usa para agilizar a otra empresa que es proveedora de otras.


%%%%%%%%%%%%%%%%%%%%%%%%%%%%%%%%%%%%%%%%%%%%%%%%%%%%%%%%%%%%%%%%%%%%%%%%%%%%%%%%%%%%%%%%%%%%%%%%
\section{¿Retos?}
Uno de los retos que ha enfrentado \emph{BrioAgro Tecnologies} ha sido resolver el \emph{reto de digitalización} agrícola en el sentido que hay mucha competencia de startups y empresas que todas quieren aportar soluciones a los retos plantedos por grandes empresas en el sector agro alimentario. En un caso en concreto fueron contratados por la empresa ``Grupo An'' que querían una solución al reto de agricultura que he descrito anteriormente. Ellos querían hacer que sus operaciones fueran mejores ``mejorar de la gestión, la productividad, medioambiente y trazabilidad-seguridad'', querían que ``a través de tecnologías que ayuden a mejorar procesos en la detección de malas hierbas, sensórica, automatización de riesgo y datos''. Al haber recibido 380 solicitudes tenían mucha competencia, pero aún con todas las empresas competidoras pudieron presentar una solución más atractiva que las demás.

%%%%%%%%%%%%%%%%%%%%%%%%%%%%%%%%%%%%%%%%%%%%%%%%%%%%%%%%%%%%%%%%%%%%%%%%%%%%%%%%%%%%%%%%%%%%%%%%
\section{¿Logros?}
El startup a logrado mantenerse en el mercado desde que empezó, esto indica que su producto si está resonando con los consumidores y que sí está aportando valor a sus ellos. ``Esta empresa ha recibido una ayuda cofinanciada al 50\% por el Fondo Europeo de Desarrollo Regional a través del Programa Operativo FEDER 2014-2020 de Navarra'', la empresa ha tenido ayuda financiera y ha logrado mantenerse en operación.



%%%%%%%%%%%%%%%%%%%%%%%%%%%%%%%%%%%%%%%%%%%%%%%%%%%%%%%%%%%%%%%%%%%%%%%%%%%%%%%%%%%%%%%%%%%%%%%%
\section{Síntesis}
Los startups y su sobrevivencia depende plenamente de sus capacidades de coordinar el mercado y responder a tendencias que el mercado pueda tener. Poder evaluar que mi producto pueda en realidad aportar valor al consumidor, en cierto sentido las ganancias que una empresa pueda derivar de sus operaciones es una señal del mercado que se están haciendo las cosas bien, es importante estar muy pendiente de estas señales principalmente en las empresas de nivel de crecimiento tan primitivo como la de los startups.


%%%%%%%%%%%%%%%%%%%%%%%%%%%%%%%%%%%%%%%%%%%%%%%%%%%%%%%%%%%%%%%%%%%%%%%%%%%%%%%%%%%%%%%%%%%%%%%%
\section{Bibliografía}
\begin{enumerate}[label={[\arabic*]}]
    \item \url{https://brioagro.es/author/jlbustos/}
    \item \url{https://brioagro.es/esta-empresa-ha-recibido-una-ayuda-cofinanciada-al-50-por-el-fondo-europeo-de-desarrollo-regional-a-traves-del-programa-operativo-feder-2014-2020-de-navarra/}
    \item \url{https://brioagro.es/}
    \item \url{https://brioagro.es/riego-inteligente-smartcity/}
    \item \url{https://www.prensalibre.com/economia/emprendedores-startups-y-mipymes/}
\end{enumerate}

























\end{document}
