\documentclass{article}
\title{Why uber is an economist's dream}
\author{David Gabriel Corzo Mcmath}
\date{2020-Jan-08 08:38:20}
%%%%%%%%%%%%%%%%%%%%%%%%%%%%%%%%%%%%%%%%%%%%%%%%%%%%%%%%%%%%%%%%%%%%%%%%%%%%%%%%%%%%%%%%%%%%%%%%%%%%%%%%%%%%%%%%%%%%%%%%%%%%%%%%%%%%%%%%%%%%%%%
\usepackage[margin = 1in]{geometry}
\usepackage{graphicx}
\usepackage{fontenc}
\usepackage{pdfpages}
\usepackage[spanish]{babel}
\usepackage{amsmath}
\usepackage{amsthm}
\usepackage[utf8]{inputenc}
\usepackage{enumitem}
\usepackage{mathtools}
\usepackage{import}
\usepackage{xifthen}
\usepackage{pdfpages}
\usepackage{transparent}
\usepackage{color}
\usepackage{fancyhdr}
\usepackage{lipsum}
\usepackage{sectsty}
\usepackage{titlesec}
\usepackage{calc}
\usepackage{lmodern}
\usepackage{xpatch}
\usepackage{blindtext}
\usepackage{bookmark}
\usepackage{fancyhdr}
\usepackage{xcolor}
\usepackage{tikz}
\usepackage{blindtext}
\usepackage{hyperref}
\usepackage{listing}
\usepackage{spverbatim}
\usepackage{fancyvrb}
\usepackage{fvextra}
\usepackage{amssymb}
\usepackage{pifont}
\usepackage{longtable}
%%%%%%%%%%%%%%%%%%%%%%%%%%%%%%%%%%%%%%%%%%%%%%%%%%%%%%%%%%%%%%%%%%%%%%%%%%%%%%%%%%%%%%%%%%%%%%%%%%%%%%%%%%%%%%%%%%%%%%%%%%%%%%%%%%%%%%%%%%%%%%%
\begin{document}
\maketitle

\section{Uber}
\begin{itemize}
    \item Uber is privately owned, it's threatning to destroy the car and taxi industry.
    \item The details of the transactions are captured in a data base, this is useful.
    \item The demand curve: 
        \begin{itemize}
            \item What is a demand curve? 
            \item No definitive specific answer.
            \item It's an artifitial construct to analyse and organise the world around us.
            \item Real world examples.
            \item Not faked, we understand what it is but we havent' figured out how to actually see it.
            \item Uber allows us to see a demand curve come to life.
        \end{itemize}
    
    \item Data:
        \begin{itemize}
            \item Uber refused for a year to hand it's data to economist's to analyse.
            \item This is a instance of a real life demand curve.
            \item This data tells people what price is right.
            \item Consumer surplus, the benefits derived from a transactions.
            \item For so much of what we buy we would be willing to buy it for so much more; example: water.
            \item Measurement of the willingness to pay; there is no way to surely measure willingness to pay.
            \item Example:
                \begin{itemize}
                    \item universe 1: 1\$
                    \item universe 2: 2\$
                    \item ... 
                    \item Some universe the consumer will not be willing to pay for that apple.
                    \item Take in to account \textbf{marginal theory of value.}
                \end{itemize}
            
            \item Measurement:
                \begin{itemize}
                    \item Search prices: the demand is visualized when the app is opened to tell the price according to the demand of the area and the time.
                    \item When people open the app and see the prices and don't make the purchase thus visualizing the price searcher's dilema.
                    \item How often does someone take the trip with diferent prices? now we can visualize this.
                    \item This helps the price searching at the market with very efective measurement.
                    \item Two people can be shown diferent prices for the same ride.
                    \item Uber always rounds the price.
                    \item At these discontinueties we can predict the perfect search price.
                    \item Regression descontinuety analysis, these are ways to discretely measure and make the best represention of a demand curve.
                \end{itemize}
        \end{itemize}
\end{itemize}

%%%%%%%%%%%%%%%%%%%%%%%%%%%%%%%%%%%%%%%%%%%%%%%%%%%%%%%%%%%%%%%%%%%%%%%%%%%%%%%%%%%%%%%%%%%%%%%%

\section{Price searching}
\begin{itemize}
    \item The price spikes are justified, example: hurricane. This is a break down of market phenomenon, the market should be able to provide to all consumers. 
    \item Arbitrage, like people to go across state lines to sell goods that are on low demand.
    \item The total Uber consumer surplus added up to 7 billion dollars.
\end{itemize}

%%%%%%%%%%%%%%%%%%%%%%%%%%%%%%%%%%%%%%%%%%%%%%%%%%%%%%%%%%%%%%%%%%%%%%%%%%%%%%%%%%%%%%%%%%%%%%%%

\section{Consumer surplus}
\begin{itemize}
    \item How much utility you actually get from something minus what you payed for it.
    \item Expected big numbers of consumer surplus for Uber.
    \item UberX, it's 80\% of the rides you use, the over all consumer surplus 7 billion and spent 4 billion, they would be able to pay 11 billion, win for everyone, the consumers got twice the benefit, and Uber kept a fraction profit, the consumers got to keep almost six times the profits. This only un the US alone.
    \item The consumer surplus influences policy, there is some enfasis that Uber could have monopolic powers.
    \item Monopolies are very powerful, when they are broken the loss is great, this is what happened with taxies charing the market for Uber.
    \item The people who are loosing are the people that have taxy medalions; this is natural destructive creation phenomenon of the market.
\end{itemize}

%%%%%%%%%%%%%%%%%%%%%%%%%%%%%%%%%%%%%%%%%%%%%%%%%%%%%%%%%%%%%%%%%%%%%%%%%%%%%%%%%%%%%%%%%%%%%%%%

\section{Uber perks}
\begin{itemize}
    \item The median driver drives an average of 10 hours per week. Other costs such as ensurance.
    \item Problems with jobs as traditionally worked, is that they don't have an option to do in the modern labor market, this gives flexibility to the worker.
\end{itemize}

%%%%%%%%%%%%%%%%%%%%%%%%%%%%%%%%%%%%%%%%%%%%%%%%%%%%%%%%%%%%%%%%%%%%%%%%%%%%%%%%%%%%%%%%%%%%%%%%

\section{Self driving cars}
\begin{itemize}
    \item Now implemented in Pitsburg self-driving cars, these services lower even more the transaciton costs.
    \item Replacing human drivers with self driving cars, is another topic.
\end{itemize}

%%%%%%%%%%%%%%%%%%%%%%%%%%%%%%%%%%%%%%%%%%%%%%%%%%%%%%%%%%%%%%%%%%%%%%%%%%%%%%%%%%%%%%%%%%%%%%%%

\section{Have we found the perfect demand curve?}
\begin{itemize}
    \item It's a very close aproximation.
    \item Usefull for long term aproximation, for better strategy, for better price searching.
    \item How does price searching effect uber:
        \begin{itemize}
            \item The elasticity of demand is estimated in a -0.6, this is very unelastic, people are not very reactive to price.
        \end{itemize}
    
    \item Is this to create an equilibrioum?, it is easier for uber to have less responsive people so that this inelastic demand.
    \item We can measure the surplus with pokemon go data.
\end{itemize}



\end{document}


