\documentclass{article}

\usepackage{generalsnips}
\usepackage{calculussnips}
\usepackage[top=1in, bottom=0.5in, left=0.5in, right=0.5in]{geometry}
\usepackage{pdfpages}
\usepackage[spanish]{babel}
\usepackage{amsmath}
\usepackage{amsthm}
\usepackage[utf8]{inputenc}
\usepackage{titlesec}
\usepackage{xpatch}
\usepackage{fancyhdr}
\usepackage{tikz}
\usepackage{hyperref}
\title{Mankiw - Cap 14 \& Videos \& Artículo}
\date{2020 March 17, 11:19PM}
\author{David Gabriel Corzo Mcmath}

\begin{document}
\maketitle
%%%%%%%%%%%%%%%%%%%%%%%%%%%%%%%%%%%%%%%%%%%%%%%%%%%%%%%%%%%%%%%%%%%%%%%%%%%%%%%%%%%%%%%%%%%%%%%%%%%%%%%%%%%%%%%%%%%%%%%%%%%%%%%%%%%%%%%%%%%%%%

\section{Las empresas en mercados competitivos}
\begin{itemize}
    \item Recordará que un mercado es competitivo si cada comprador y vendedor es pequeño en comparación con el tamaño del mercado y, por consiguiente, tiene poca capacidad de influir en los precios del mercado.
    \item si una empresa puede influir en el precio de mercado del bien que vende, se dice que tiene \emph{poder de mercado}.
\end{itemize}
\section{Competencia}
\begin{itemize}
    \item \termdefinition{Mercado competitivo}{Mercado con muchos compradores y vendedores que intercambian productos idénticos, de tal forma que cada comprador y vendedor son tomadores de precios.} 
        \begin{itemize}
            \item Existen muchos compradores y vendedores en el mercado.
            \item Los bienes ofrecidos por los diversos vendedores son básicamente los mismos. 
            \item Las empresas pueden entrar y salir libremente del mercado. 
        \end{itemize}
    
    \item las acciones de un solo comprador o vendedor en el mercado tienen un efecto insignificante en el precio de mercado.
    \item Compradores y vendedores en mercados competitivos deben aceptar el precio que el mercado determina y, por tanto, se dice que son tomadores de precios. 
\end{itemize}

\subsection{Los ingresos de una empresa competitiva}
\begin{itemize}
    \item Para todas las empresas, el ingreso promedio es igual al precio del bien.
    \item Ingresos promedios: 
        \[
          \text{ Ingresos promedios } = \frac{\text{ Ingresos totales }}{\text{ Cantidad vendida }} 
        \]
    
    \item Ingreso marginal: 
        \[
            \text{ Ingresos marginales } = \text{ Cambio en los ingresos totales cuando se agrega una unidad }
        \]
\end{itemize}


\subsection{Maximización de beneficios y curva de oferta de una empresa competitiva}
\begin{itemize}
    \item Para maximizar el ingreso marginal tiene que se igual que costo marginal:
        \[
          \text{ Max. }: \text{ Ingreso marginal  } \; == \; \text{ Costo marginal } 
        \]
\end{itemize}


\subsection{La curva de costo marginal y decisión de la empresa respecto a la oferta}
\begin{itemize}
    \item Recordar:
        \begin{itemize}
            \item  La curva de costo total promedio (CTP) tiene forma de U
            \item La curva de costo marginal interseca la curva de costo total promedio en el mínimo del costo total promedio.
            \item La figura también muestra una línea horizontal en el precio de mercado (P). 
            \item  La línea del precio es horizontal porque la empresa es tomadora de precios
        \end{itemize}
    
    \item Enconces:
        \begin{itemize}
            \item Si el ingreso marginal es mayor que el costo marginal, la empresa debe aumentar la producción. 
            \item Si el costo marginal es mayor que el ingreso marginal, la empresa debe disminuir la producción. 
            \item En el nivel de producción que maximiza los beneficios, el ingreso marginal y el costo marginal son exactamente iguales.
        \end{itemize}
    
    \item Entonces:
        \begin{itemize}
            \item  Para cualquier precio dado, la cantidad que maximiza los beneficios de una empresa competitiva se encuentra en la intersección del precio con la curva de costo marginal
                \[
                  \text{ Max. }: \; \text{ CM } == \text{ Precio }
                \]
        \end{itemize}
\end{itemize}


\subsection{La decisión de la empresa de cerrar a corto plazo}
\begin{itemize}
    \item Recordar:
        \begin{itemize}
            \item Las decisiones a corto y largo plazo difieren porque la mayoría de las empresas no puede evitar los costos fijos a corto plazo, pero sí a largo plazo
        \end{itemize}
    
    \item Cierre: 
        \begin{itemize}
            \item Cierre se refiere a una decisión a corto plazo de no producir nada durante un periodo específico, debido a las condiciones actuales del mercado.
        \end{itemize}
    
    \item Salida:
        \begin{itemize}
            \item  Salida se refiere a la decisión a largo plazo de abandonar el mercado.
        \end{itemize}
    
    \item Ejemplo:
        \begin{itemize}
            \item Al tomar la decisión a corto plazo de cerrar por una temporada, se dice que el costo fijo de la tierra es un costo hundido
        \end{itemize}
    
    \item Decisiones: (IT: Ingresos Totales, CV: Costos Variables)
        \begin{itemize}
            \item Cerrar si: 
                \[
                  P < CVP
                \]
            
            \item  Cuando la empresa opta por producir, compara el precio que recibe por una unidad típica con el costo variable promedio en el que debe incurrir para producirla.
            \item Si la empresa produce algo, producirá la cantidad en la que el costo marginal es igual al precio del bien. Sin embargo, si el precio es menor que el costo variable promedio en esa cantidad, la empresa estará mejor si cierra y no produce nada. 
        \end{itemize}
    
    \item \termdefinition{Costos hunididos}{Costo en el que se ha incurrido y que no se puede recuperar.} 
\end{itemize}


\subsection{Lo pasado, pasado está y otros costos hundidos}
\begin{itemize}
    \item \termdefinition{Costo hundidos}{costo hundido cuando ya se incurrió en él y no es posible recuperarlo.} 
    \item Los costos hundidos deben de ser considerados irrelevantes.
    \item  Por consiguiente, la empresa sale del mercado si el ingreso que obtendría de producir es menor que sus costos totales. 
    \item La empresa saldrá del mercado si: (P: Precio, CTP: Costo Total Promedio)
        \[
          P < CTP
        \]
    
    \item El criterio de entrada es exactamente lo contrario del criterio de salida. 
        \[
          P > CTP
        \]
    
    \item Beneficios de las empresas:
        \[
          \text{ Beneficios } = \p{P-CTP} \times Q
        \]
    
    \item Si el precio disminuye por debajo del costo total promedio, será mejor que la empresa salga del mercado.
\end{itemize}

\section{La curva de la oferta en un mercado competitivo}
\begin{itemize}
    \item El corto plazo:
        \begin{itemize}
            \item  Para obtener la curva de la oferta del mercado, se suma la cantidad ofrecida por cada empresa en el mercado
        \end{itemize}
    
    \item El largo plazo: 
        \begin{itemize}
            \item Al final de este proceso de entrada y salida, las empresas que sigan operando en el mercado tendrán cero beneficios económicos.
        \end{itemize}
    
    \item Entonces: 
        \begin{itemize}
            \item Si el precio está por encima del costo total promedio, los beneficios son positivos y esto estimula la entrada de nuevas empresas. 
            \item Si el precio es menor que el costo total promedio, los beneficios son negativos y esto ocasiona la salida del mercado de algunas empresas. 
            \item El proceso de entrada y salida concluye cuando el precio y el costo total promedio son iguales.
            \item Acabamos de señalar que la libre entrada y salida de las empresas obliga a que el precio sea igual al costo total promedio.
            \item Como resultado, la curva de oferta a largo plazo del mercado debe ser horizontal en este precio.
            \item Finalmente, el número de empresas en el mercado se ajusta para que el precio sea igual al mínimo del costo total promedio y existan suficientes empresas para satisfacer toda la demanda a este precio. 
        \end{itemize}
    
    \item ¿Por qué las empresas competitivas siguen operando si obtienen cero beneficios?
        \begin{itemize}
            \item  Como resultado, en el equilibrio de cero beneficios, el beneficio económico es cero, pero la utilidad contable es positiva.
            \item \pregunta{Los costos explicitos son positivos y los costos implícitos son 0} 
        \end{itemize}
    
    \item Un desplazamiento de la demanda a corto y largo plazo:
        \begin{itemize}
            \item Debido a que las empresas pueden entrar y salir a largo plazo, pero no a corto plazo, la respuesta del mercado a un cambio en la demanda depende del horizonte de tiempo.
            \item  Las empresas obtienen beneficios cero, por lo que el precio es igual al mínimo del costo total promedio.
            \item A largo plazo se tiene cero beneficios, después de un proceso.
        \end{itemize}
    
    \item ¿Por qué la curva de la oferta a largo plazo tiene pendiente positiva?
        \begin{itemize}
            \item Hasta el momento hemos visto que la entrada y salida pueden hacer que la curva de la oferta a largo plazo del mercado sea perfectamente elástica.
            \item Hay, sin embargo, dos razones por las que la curva de la oferta a corto plazo del mercado tiene pendiente positiva. La primera es que algunos de los recursos utilizados en la producción pueden estar disponibles sólo en cantidades limitadas.
            \item Otra razón que explica una curva de la oferta con pendiente positiva es que las empresas tienen diferentes costo. Tenga en cuenta que si las empresas tienen diferentes costos, algunas obtienen beneficios incluso a largo plazo. En este caso, el precio de mercado refleja el costo total promedio de la empresa marginal (la empresa que saldría del mercado si el precio fuera menor).
            \item Debido a que las empresas pueden entrar y salir más fácilmente a largo que a corto plazo, la curva de la oferta a largo plazo es típicamente más elástica que la curva de la oferta a corto plazo.
        \end{itemize}
\end{itemize}

\section{Conclusión}
\begin{itemize}
    \item Recordar:  las personas racionales piensan en términos marginales.
    \item  El análisis marginal ha proporcionado una teoría de la curva de la oferta en un mercado competitivo y, como resultado, un mayor entendimiento de los resultados del mercado.
    \item cuando compramos un bien de una empresa en un mercado competitivo, podemos estar seguros de que el precio que pagamos es cercano al costo de producir ese bien.
    \item  si las empresas son competitivas y maximizan sus beneficios, el precio de un bien será igual al costo marginal de producirlo. 
    \item Además, si las empresas pueden entrar y salir libremente del mercado, el precio es también igual al menor costo total promedio posible de la producción. 
\end{itemize}


\section{Resumen}
\begin{itemize}
    \item Debido a que una empresa competitiva es tomadora de precios, sus ingresos son proporcionales a la cantidad que produce. El precio del bien será igual al ingreso promedio de la empresa y a su ingreso marginal. 
    \item  Para maximizar sus beneficios, la empresa determina una cantidad de producción tal que el ingreso marginal sea igual al costo marginal. Debido a que el ingreso marginal de una empresa competitiva es igual al precio de mercado, la empresa selecciona la cantidad con la que el precio es igual al costo marginal. Por tanto, la curva de costo marginal de la empresa es su curva de la oferta. 
    \item  A corto plazo, cuando una empresa no puede recuperar sus costos fijos, la empresa decidirá cerrar temporalmente si el precio del bien es menor que el costo variable promedio. A largo plazo, cuando la empresa puede recuperar sus 
    costos fijos y variables, optará por salir del mercado si el precio es menor que el costo total promedio. 
    \item  En un mercado con libre entrada y salida, los beneficios a largo plazo son cero. En este equilibrio a largo plazo, todas las empresas producen a su escala eficiente, el precio es igual al mínimo del costo total promedio y el número de empresas se ajusta para satisfacer la cantidad demandada a este precio. 
    \item  Los cambios en la demanda tienen diferentes efectos dependiendo del horizonte de tiempo. A corto plazo, un incremento en la demanda incrementa el precio y produce beneficios, y una disminución de la demanda reduce los precios y provoca pérdidas. Pero si las empresas pueden entrar y salir libremente del mercado, el número de empresas a largo plazo se ajusta para restablecer el equilibrio de cero beneficios en el mercado.
\end{itemize}


\section{The future of electric scooter sharing companies – 4 scenarios 4 years from now}
\begin{itemize}
    \item While Bird might have seemed original at the time, there have since been over a dozen other companies that have attempted to join the nascent electric scooter sharing industry. (suply curve attracting new supliers).
    \item Many experts claim the businesses aren’t even sustainable. (long term benefits are null)
    \item All of the companies charge the same prices in the US: \$1 to rent an electric scooter and \$0.15 per minute afterwards. International rates vary. (perfectly competitive)
    \item But after a dozen or so companies joined the competition, now there’s a stalemate where no single company can raise their initial price without causing a mass exodus of users to its competitors. (perfectly competitive)
    \item those scooters have a lifespan of just 30-90 days before being broken beyond repair from nearly constant commercial misuse. (interesting)
    \item Scenario 1: All of the electric scooter companies run out of money and fail:
        \begin{itemize}
            \item helping of venture capital funding 
            \item losing as much as US \$6 million – \$23 million per month 
        \end{itemize}
    
    \item Scenario 2: The little guys fail while the big guys survive by raising prices: 
        \begin{itemize}
            \item Price war (collusion of suppliers, this is illegal pricing, but it happens as implicit collusion, non competitive market) 
        \end{itemize}
    
    \item Scenario 3: Companies merge and avoid direct competition to reach profitability:
        \begin{itemize}
            \item Larger companies aquire the small ones. 
            \item Create mini-monopolies.
        \end{itemize}
    
    \item Scenario 4: Electric scooters die out in favor of alternatives:
        \begin{itemize}
            \item Substitutes (demand determinant)
            \item Regulations could make it dificult. 
        \end{itemize}
    
    \item Either there will be a decline in scooters or an increase, only time will tell.
\end{itemize}

\section{MRU: Incentives}
\begin{itemize}
    \item ``Economy beat sentiments and benevolence.''
\end{itemize}


\section{MRU: Maximizing profit under competition}
\begin{itemize}
    \item Profit: 
        \[
          \text{ Profit } = \pi(Q) = \text{ Total Revenue } - \text{ Total Cost }
        \]
        \begin{itemize}
            \item Total Revenue: Price $\times $ Quantity 
            \item Total Costs:  Fixed Costs + Variable Costs(Q) 
        \end{itemize}
    
    \item Maximize profit: 
        \begin{itemize}
            \item Take the derivative of that function and set it equal to zero:
                \begin{center}
                   \begin{align*}
                       \dervpar{\pi }{Q} = \dervpar{TR}{Q} - \dervpar{TC}{Q} = MR-MC = 0 \qimplies \text{ MR } = \text{ MC } \\ 
                   \end{align*}
                \end{center}
            
            \item If Marginal Revenue > Marginal Cost $\rightarrow$ Producing more will add to your profit. 
            \item If Marginal Revenue < Marginal Cost $\rightarrow$ Producing less will add to your profit.
        \end{itemize}
    
    \item Marginal Revenue un a competitve market is equal to price. Marginal cost is going to be an upward's sloping curve, when Marginal Revenue is equal to Marginal Cost that is the maximize. 
    \item Maximizing profit explains firm's behaviour: 
    
    \item Average cost let's us visualize profit in the graph. 
\end{itemize}


\section{Maximizing profit and the average cost curve}
\begin{itemize}
    \item Average cost is: 
        \[
          \text{ AC } = \frac{\text{ TC }}{\text{ Q }} = \frac{\text{ FC }+\text{ VC }}{\text{ Q }} 
        \]
        As Q gets bigger the fraction aproaches 0.
    
    \item Profit: 
        \[
          \pi = Q \times (P-AC)
        \]
    
    \item When the price (marginal revenue) intersects the marginal cost curve draw a vertical line, then when that line intersects both marginal cost and average cost draw a rectangle to the y-axis.  If the price is below the average cost the rectangle will represent loss, on the contrary if the price is above then the resulting rectangle will be profit. 
    \item Firm enter or exit: 
        \begin{itemize}
            \item Long run:
                \begin{itemize}
                    \item Firms enter: P > AC 
                    \item Firms exit: P < AC 
                    \item When P = Min AC 
                        \begin{itemize}
                            \item Profits are 0. No incentive to enter or exit.
                        \end{itemize}
                \end{itemize}
        \end{itemize}
    
    \item Why remain if zero profits? 
        \begin{itemize}
            \item They are at equilibrioum. 
            \item It's normal profit. 
        \end{itemize}
    
    \item Consider entry and exit costs. 
\end{itemize}











%%%%%%%%%%%%%%%%%%%%%%%%%%%%%%%%%%%%%%%%%%%%%%%%%%%%%%%%%%%%%%%%%%%%%%%%%%%%%%%%%%%%%%%%%%%%%%%%%%%%%%%%%%%%%%%%%%%%%%%%%%%%%%%%%%%%%%%%%%%%%%
\end{document}

