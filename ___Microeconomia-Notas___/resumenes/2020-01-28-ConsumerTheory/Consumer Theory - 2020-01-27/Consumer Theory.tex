\documentclass{article}
\title{Consumer Theory}
\author{David Gabriel Corzo Mcmath}
\date{2020-Jan-28 22:04:01}
%%%%%%%%%%%%%%%%%%%%%%%%%%%%%%%%%%%%%%%%%%%%%%%%%%%%%%%%%%%%%%%%%%%%%%%%%%%%%%%%%%%%%%%%%%%%%%%%%%%%%%%%%%%%%%%%%%%%%%%%%%%%%%%%%%%%%%%%%%%%%%%
\usepackage[margin = 1in]{geometry}
\usepackage{graphicx}
\usepackage{fontenc}
\usepackage{pdfpages}
\usepackage[spanish]{babel}
\usepackage{amsmath}
\usepackage{amsthm}
\usepackage[utf8]{inputenc}
\usepackage{enumitem}
\usepackage{mathtools}
\usepackage{import}
\usepackage{xifthen}
\usepackage{pdfpages}
\usepackage{transparent}
\usepackage{color}
\usepackage{fancyhdr}
\usepackage{lipsum}
\usepackage{sectsty}
\usepackage{titlesec}
\usepackage{calc}
\usepackage{lmodern}
\usepackage{xpatch}
\usepackage{blindtext}
\usepackage{bookmark}
\usepackage{fancyhdr}
\usepackage{xcolor}
\usepackage{tikz}
\usepackage{blindtext}
\usepackage{hyperref}
\usepackage{listing}
\usepackage{spverbatim}
\usepackage{fancyvrb}
\usepackage{fvextra}
\usepackage{amssymb}
\usepackage{pifont}
\usepackage{longtable}
%%%%%%%%%%%%%%%%%%%%%%%%%%%%%%%%%%%%%%%%%%%%%%%%%%%%%%%%%%%%%%%%%%%%%%%%%%%%%%%%%%%%%%%%%%%%%%%%%%%%%%%%%%%%%%%%%%%%%%%%%%%%%%%%%%%%%%%%%%%%%%%
\begin{document}
\maketitle

\section{How does the consumer make choices}

\subsection{Consumer Choice}
\begin{itemize}
    \item All the variations that the consumer have allow us to make choices out of millions out of everyday.
    \item Marginal utility: all the happiness that a good can give you in comparison with the cost. The consumer chooses on a marginal utility bases.
\end{itemize}


\subsection{Budget Constraints}
\begin{itemize}
    \item Elements of consumer decisions: 
        \begin{itemize}
            \item Example: the division of labor.
            \item Example: your salary.
            \item Example: your productivity.
            \item There are so many choices! 
        \end{itemize}
    
    \item The budget constraint is all the diferent combnations that you can buy of two certain goods, the slope that results is always sloped downwards, this is the budget constraints.
    \item This takes in to account the oportunity cost.
    \item Changes in your income don't affect the market prices, the tradeofs remains the same. It changes if the relative price of the goods changes.
\end{itemize}

\subsection{Indiferencie curves}
\begin{itemize}
    \item Preferences are important.
    \item The more volume of the product the more preference it has. 
    \item Combinations that make no diference, the indiference curve represents all the combinations that could give you the same utility.
    \item The slope changes, the slope is called the marginal rate of substitution. The way to find it is draw a straight line of the tangent in that point. Its a hiperbola, tha more you have the marginal utility decreases.
    \item The marginal rate of substitution is an extreme case, in this case the indiference curve will be a straight line.
    \item The indiference curve of complementary goods are right rectangles. 
    \item Assume that more are beter, the further from the origin the better, for the ``bads'' the closer the better.
\end{itemize}

\section{Episode 538: Is a Stradivarius just a violin?}
\begin{itemize}
    \item 1790 Stradivari violin made.
    \item \$45,000,000 at the tops.
    \item Blinded test 
    \item Dificult to describe somethings as they are.
\end{itemize}

\section{Consumer optimization}
\begin{itemize}
    \item choices $\rightarrow$ dreams \& wants
    \item when the budget constraint intersects one of the indiference curves that is the optimal point.
    \[
      \frac{P_{\text{  pizza  }}}{P_{\text{  coffe  }   }} = MRS  
    \]

    
    \item 
\end{itemize} 



























\end{document}
