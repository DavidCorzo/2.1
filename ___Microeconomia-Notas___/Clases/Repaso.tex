\section{Elasticidad}
\begin{itemize}
    \item Si importan los signos.
    \item 
\end{itemize}


%----------------------------------------------------------------------------------------
\section{Efecto ingreso y sustitución}
\begin{itemize}
    \item Cambia en ingreso:
        \[
          \Delta Y = q_{p_1} \p{p_2-p_1} 
        \]
    
    \item Fórmula de restricción presupuestaria:
        \[
          Y = Q_1P_1+Q_2P_2
        \]
    
    \item Fórmula de nuevo ingreso:
        \[
          Y_2 = Y_1 + \Delta Y
        \]
    
    \item Efecto sustitución: 
        \begin{itemize}
            \item Efecto sustitución es sobre la misma curva.
            \item Se hace una recta paralela a la nueva recta, usar esta para graficar una curva de indiferencia que intersecte dos puntos sobre esta recta.
            \item Si el precio en x disminuye se pivotea a la derecha, si el precio sube se pivotea a la izquierda.
            \item Si el precio y disminuye 
        \end{itemize}
        \[
          E_s = q_2(P_2,Y_2) - q_1(P_2,Y_1)
        \]
    
    \item Efecto ingreso: tengo más poder adquisitivo para comprar.
        \begin{itemize}
            \item El efecto sustitución debe de ser mayor al efecto ingreso, para ser bien inferior.
            \item Bien es inferiores cuando los efectos sustitución y efecto ingreso se mueven en direcciónes opuestas.
            \item Es un bien normal si se mueven en la misma dirección y el punto C está a la derecha.
        \end{itemize}
        \[
          E_y = q_2(P_2,Y_1)-q_1(P_2,Y_2)
        \]
    
    \item Efecto total, Slutsky: 
        \[
          E_T = E_s + E_y
        \]
    
    \item Graficar:
        \begin{itemize}
            \item Hacer una línea paralela a $q_2$
            \item hacer una curva de indiferencia que intersecte a esa paralela en dos puntos
            \item Evaluar si el es un bien normal, inferior o superior.
            \item Un bien giffen es un bien demasiado inferior.
        \end{itemize}
    
    \item Para comprobar si el efecto sustitución:
        \[
          q(P_2) - q(P_1) = \underbrace{E_s + E_y }_{\text{Slutsky} }
        \]
\end{itemize}


%----------------------------------------------------------------------------------------
\section{Teoría del productor \& teoría de consumidor}
\begin{itemize}
    \item Isocoste = Restricción presupuestaria
    \item Isocuanta = Curvas de indiferencia 
\end{itemize}


%----------------------------------------------------------------------------------------
\section{Rendimientos de escala}
\begin{itemize}
    \item La variación debe de ser proporcional y simultánea.
\end{itemize}

%----------------------------------------------------------------------------------------
\section{Costos}
\begin{itemize}
    \item Minimizamos el costo cuando el costo marginal corta el costo medio total. Ó cuando el $C_M = C_{TMe}$ 
    \item Elementos del costo:
        \begin{itemize}
            \item $q$ la función de cantidad.
            
            \item $C_F$: Costo fijo 
                \begin{itemize}
                    \item Ejemplo: $q(x) = 10x + 5$ el $5$ es el costo fijo
                \end{itemize}
            
            \item $C_T$: Costo total
                \begin{itemize}
                    \item Es una función, la derivada es el costo marginal. 
                \end{itemize}
            
            \item Costo medio fijo 
                \begin{itemize}
                    \item Dividir la función de costo total entre $q$:
                        \[
                          C_{CFMe} = \frac{C_F}{q} 
                        \]
                \end{itemize}
            
            \item Costo medio variable 
                \begin{itemize}
                    \item Dividir el costo variable entre $q$
                        \[
                          C_{VMe} = \frac{C_V }{q} 
                        \]
                \end{itemize}
            
            \item Costo medio total 
                \begin{itemize}
                    \item Dividir el costo total entre $q$:
                        \[
                            C_{TMe} = \frac{C_T}{q} 
                        \]
                \end{itemize}
            
            \item Costo marginal:
                \begin{itemize}
                    \item La derivada de la función de costo total, derivar .
                        \[
                          C_M = \frac{\Delta C_T}{\Delta q} 
                        \]
                \end{itemize}
        \end{itemize}
\end{itemize}


%----------------------------------------------------------------------------------------
\section{Dudas}
\begin{itemize}
    \item Lab\#3 Costo de oportunidad (no está claro, solo memorizar el concepto):
        \[
          \text{ Costo de oportunidad } = \text{ Costo de valoración } - \text{ Costo incurrido }
        \]
        \begin{center}
           \begin{tabular}{ | p{5cm} | p{5cm} | }
               \hline
                    Costo & Valoración subjetiva   \\
                \hline
                    \$125 & \$200 \\ 
                    \$50 & \$100 \\ 
               \hline
           \end{tabular}
        \end{center}
\end{itemize}
