\section{Tasa marginal de sustitución}
\begin{itemize}
    \item \[
        TMS = \frac{MU_x}{MU_y} 
      \]
    
    \item TMS: Puede ser expresada en términos de utilidades marginales.
    \item Derivadas parciales: ejemplo, derivada parcial con respecto a x es :     
        \begin{align*}
            f(x) = 3x y \quad f'(x) = 3\cancel{x}y \quad \therefore f'(x) = 3y \\ 
        \end{align*}
\end{itemize}

%%%%%%%%%%%%%%%%%%%%%%%%%%%%%%%%%%%%%%%%%%%%%%%%%%%%%%%%%%%%%%%%%%%%%%%%%%%%%%%%%%%%%%%%%%%%%%%%

\section{Tasa de transformación}
\begin{itemize}
    \item La pendiente de la restricción presupuestaria.
    \item TMT:
        \begin{align*}
            TMT = -\frac{P_x}{P_y} \\ 
        \end{align*}
\end{itemize}


%%%%%%%%%%%%%%%%%%%%%%%%%%%%%%%%%%%%%%%%%%%%%%%%%%%%%%%%%%%%%%%%%%%%%%%%%%%%%%%%%%%%%%%%%%%%%%%%

\section{Efectos por el cambio de precio}
\begin{itemize}
    \item Todas las combinaciones óptimos de las curvas de indiferencia forma la \emph{curva de demanda}.
    \item Curva de precio consumo: la línea que traza las combinaciones óptimas como respuesta a un cambio de precio, manteniendo el ingrso constante. 
    \item La unión de las combinciones óptimas de las curvas de indiferencia respecto de los cabios de precio.
\end{itemize}

%%%%%%%%%%%%%%%%%%%%%%%%%%%%%%%%%%%%%%%%%%%%%%%%%%%%%%%%%%%%%%%%%%%%%%%%%%%%%%%%%%%%%%%%%%%%%%%%

\section{Efectos por el cambio en el ingreso}
\begin{itemize}
    \item Curva de ingreso consumo: es la unión de los puntos óptimos respecto a los cambios en nuestros ingresos.
\end{itemize}


%%%%%%%%%%%%%%%%%%%%%%%%%%%%%%%%%%%%%%%%%%%%%%%%%%%%%%%%%%%%%%%%%%%%%%%%%%%%%%%%%%%%%%%%%%%%%%%%

\section{Curva de ingreso consumo}
\begin{itemize}
    \item Tomar en cuenta:
        \begin{enumerate}
            \item Bien normal: sube el ingreso, sube la demanda de un bien.
            \item Bien inferior: sube el ingreso, dejo de demandar ese bien.
        \end{enumerate}
    
    \item Considerar lo siguiente: puede ser que cuando aumente el ingreso se mantenga constante con la alternativa de \textbf{ahorrar}, \emph{\textbf{Interesante:} Ahorrar se considera como un bien normal}.
    \item Un bien no siempre va a ser inferior o normal.
    \item La unión de los puntos óptimos de las curvas de indiferencia, la curva de ingreso consumo es equivalente a decir la \emph{curva de Engel}.
\end{itemize}

%%%%%%%%%%%%%%%%%%%%%%%%%%%%%%%%%%%%%%%%%%%%%%%%%%%%%%%%%%%%%%%%%%%%%%%%%%%%%%%%%%%%%%%%%%%%%%%%

\section{Fórmulas}
\begin{itemize}
    \item Restricción presupuestaria:
        \[
          Y = Q_vP_v \times Q_qP_q
        \]
\end{itemize}






