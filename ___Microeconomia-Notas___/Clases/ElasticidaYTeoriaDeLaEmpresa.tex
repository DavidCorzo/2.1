\section{Resolución de corto}
\begin{enumerate}
    
    \item Los tres factores de producción:
        \begin{itemize}
            \item Tecnología de producción. 
            \item Restricciones de costos.
            \item Elecciones de los factores.
        \end{itemize}
    
    \item $Q = 94-2ps+0.2pt+0.4Y$
        \begin{align*}
            \begin{matrix}
                ps=8 \\ 
                pt=10 \\ 
                Y=50 \\ 
            \end{matrix}
        \end{align*}
        \begin{itemize}
            \item Recordar la fórmula de Elasticidad de precio
        \end{itemize}
        \[
            E_P = \frac{\Delta Q}{\Delta P} \times \frac{P}{Q} 
        \]
        \begin{align*}
            \frac{D_{ps}}{D_Q} = -2 \\ 
            E_p = -2 \times \frac{8}{100}  \\ 
            E_p = -0.16 \\ 
        \end{align*}
        \begin{itemize}
            \item $\therefore $ Es una demanda inelástica.
        \end{itemize}
    
    \item $Q=003Y-2p$
        \begin{align*}
            \begin{matrix}
                Y=500 \\ 
                P=\text{  \$  } 5 \\ 
            \end{matrix}
        \end{align*}
        \begin{itemize}
            \item Encontrar la cantidad:
                \begin{align*}
                    q_1 = 0.03(500)-2(5) \\ 
                    q_1 = 5 \\ 
                    \\ 
                    5 = 0.03Y - 2(7) \\ 
                    19 = 0.03Y \\ 
                    \Delta Y = Q_1 \times \Delta p = 5 \times  (7-5) = 10 \\ 
                    \text{  Calcular el nuevo ingreso  } \\ 
                    Y_2 = Y_1 + \Delta Y \\ 
                    Y_2 = 510 \\ 
                    \text{  Calcular la cantidad dos  } \\ 
                    q_2 = 0.03(\underbrace{510}_{Y_2})-s(\underbrace{7}_{p_2}) = -1.3 \\
                    \text{  Efecto sustitución:  } \\ 
                    E_s = q(p_2,y_2)-q(p_1,y_1) \\  
                    E_s = 1.3-5 = -3.7 \\ 
                    \text{  Efecto ingreso:  } \\ 
                    E_i = q(p_2,y_1) - q(p_2,y_2) = 1 -1.3= -0.3 \\ 
                \end{align*}
        \end{itemize}
\end{enumerate}



%%%%%%%%%%%%%%%%%%%%%%%%%%%%%%%%%%%%%%%%%%%%%%%%%%%%%%%%%%%%%%%%%%%%%%%%%%%%%%%%%%%%%%%%%%%%%%%%
\section{Teoría de la empresa}
\begin{itemize}
    \item \textbf{Nos preguntamos:} ¿Es mejor más productividad?
        \begin{itemize}
            \item No siempre, a veces exceden la demanda.
        \end{itemize}    
\end{itemize}


%%%%%%%%%%%%%%%%%%%%%%%%%%%%%%%%%%%%%%%%%%%%%%%%%%%%%%%%%%%%%%%%%%%%%%%%%%%%%%%%%%%%%%%%%%%%%%%%
\section{Las decisiones de producción de empresas}
\begin{itemize}
    \item Las decisiones de producción de una empresa:
    \begin{itemize}
        \item El punto de la teoría del consumidor era maximizar lo que quiere el consumidor.
    \end{itemize}

\end{itemize}
