\section{\textbf{Nos preguntamos:} ¿Qué significa el bien?}
\begin{enumerate}
    \item Normal:
        \begin{itemize}
            \item Si el bien es lo normal entonces es manipulable
        \end{itemize}

    \item Ausencia del mal: 
        \begin{itemize}
            \item Subjetivo 
            \item ¿ Cómo lo justifico ?
            \item El problema es que hay que definir el mal, por ende tenemos el mismo caso de indefinición.
            \item Entonces se da el caso que el que gana no es el que piensa si no el más fuerte, ley de los salvajes.
        \end{itemize}

    \item Algo que me hace mejor:
        \begin{itemize}
            \item \textbf{Nos preguntamos:} ¿qué significa que algo es mejor que alguna otra cosa o persona?
        \end{itemize}

    \item Generar un beneficio:
        \begin{itemize}
            \item \textbf{Nos preguntamos:} ¿qué es un beneficio?
            \item \emph{\textbf{Interesante:} Seminario de Vermon Smith, ``no todo se resumen en se resume en máxima utilidad.''}
        \end{itemize}

    \item Lo que está aceptado 
    \item Lo que no me duele:
        \begin{itemize}
            \item Hay veces que no me doy cuenta que me esta doliendo.
            \item Uno no se puede acostumbrar.
        \end{itemize}

    \item Lo que considero que es correcto: 
        \begin{itemize}
            \item \textbf{Nos preguntamos:} ¿qué es lo correcto?
            \item \textbf{Nos preguntamos:} ¿dónde está el norte?
            \item \emph{\textbf{Recordar lo siguiente: }Alicia en el país de las maravillas, ``¿a dónde querés ir? entonces no importa''}
            \item Si no tenés un norte no importa a dónde ir.
        \end{itemize}

    \item Lo que no afecta mi consciencia:
        \begin{itemize}
            \item Que me asegura que hay algo no afectará mi consciencia en el futuro.
        \end{itemize}

    \item La mejor opción:
        \begin{itemize}
            \item \textbf{Nos preguntamos:} ¿Cuál es la mejor opción?
        \end{itemize}
\end{enumerate}

%%%%%%%%%%%%%%%%%%%%%%%%%%%%%%%%%%%%%%%%%%%%%%%%%%%%%%%%%%%%%%%%%%%%%%%%%%%%%%%%%%%%%%%%%%%%%%%%

\subsection{Democracia}
\begin{center}
   \begin{tabular}{ | p{5cm} | p{5cm} | p{5cm} | }
       \hline
            & Democracia & \\ 
            
            \begin{itemize} 
                \item Aceptado 
                \item Normal 
            \end{itemize} & 
            \begin{itemize}
                \item Cuando las decisiones las toma el 51\%.
            \end{itemize} & 
             \begin{itemize}
                 \item \[
                   \text{Bondad} = \underbrace{\text{Belleza}}_{\text{Sentimientos externos}}
                 \]
                 
                 \item \textbf{Nos preguntamos:} ¿Entonces el bien es un fin?
             \end{itemize}\\ 
            
       \hline
   \end{tabular}
\end{center}

%%%%%%%%%%%%%%%%%%%%%%%%%%%%%%%%%%%%%%%%%%%%%%%%%%%%%%%%%%%%%%%%%%%%%%%%%%%%%%%%%%%%%%%%%%%%%%%%

\subsection{\textbf{Nos preguntamos:} ¿podemos predicar la palabra bien a lo que no tiene razón? Aristóteles}
\begin{itemize}
    \item \textbf{Nos preguntamos:} ¿una planta tiende a un bien?
    \item \begin{center}
       \begin{tabular}{ | p{5cm} | p{5cm} | }
           \hline
               Si & No    \\
           \hline
           \begin{enumerate} % si
               \item Son buenos por que cumplen una función. (hombre)
               \item Función $\rightarrow$ cumple una función.
               \item Entonces cada uno cumple su función, y entonces cada uno tiende a un bien.
               \item ``Antes que nosotros lo humanos existieramos existían los animales''
               \item \emph{\textbf{Interesante:} Lion King: \textbf{Nos preguntamos:} ¿por que nos comemos a los antílopes? \emph{\textbf{Respuesta:} ``it's the circle of life''}}.
               \item La belleza no es funcional, al menos que hay alguien que las pueda apreciar.
               \item \emph{Citación:``La belleza es algo que me cuativa por lo inútil que es, pero sería deprimente vivir en un mundo sin belleza, hay belleza en las cosas funcionan también."}
               \item \emph{\textbf{Ejemplo: }Alan Turin, se envenenó por una manzana, por eso Apple tiene una manzanita como logo.}
               \item \textbf{Nos preguntamos:} ¿Entonces la belleza para quién es?
               \item \emph{\textbf{Ejemplo: }Cuando un hombre le da un regalo a una mujer uno no le da una rata de alcntarilla, las cosas entonces además de funcionales tienen que ser bellas. }
               \item \emph{\textbf{Ejemplo: }yo necesito saber que es una silla para agarrar y delegar los recursos y construir una.}
           \end{enumerate} & 
           \begin{itemize} % no
               \item Simplemente no tendería ningun bien.
           \end{itemize} \\ 

       \end{tabular}
    \end{center}
\end{itemize}

%%%%%%%%%%%%%%%%%%%%%%%%%%%%%%%%%%%%%%%%%%%%%%%%%%%%%%%%%%%%%%%%%%%%%%%%%%%%%%%%%%%%%%%%%%%%%%%%
\subsection{Video:}
\begin{itemize}
    \item Lady Antebellum - Run To You Love 
    \item Lady Antebellum - Need You Now 
\end{itemize}

%%%%%%%%%%%%%%%%%%%%%%%%%%%%%%%%%%%%%%%%%%%%%%%%%%%%%%%%%%%%%%%%%%%%%%%%%%%%%%%%%%%%%%%%%%%%%%%%

%%%%%%%%%%%%%%%%%%%%%%%%%%%%%%%%%%%%%%%%%%%%%%%%%%%%%%%%%%%%%%%%%%%%%%%%%%%%%%%%%%%%%%%%%%%%%%%%

\section{Reflexión propia}
\begin{itemize}
    \item El bien es lo que no me duele hoy, o lo que me va a disminuir o exonerar dolor en el futuro, lo que me va a a beneficiar en cambio de dolor.
    \item Edomista, una manera de pensar.
\end{itemize}
