\section{\textbf{Nos preguntamos:} ¿La verdad tiene algo que ver con el bien?}
\begin{enumerate}
    \item \textbf{Nos preguntamos:} ¿qué es la verdad?
        \begin{itemize}
            \item Definir es saber los límites. Describir no es definir, definir es decir ``yo sé lo que es'', \emph{\textbf{Ejemplo: }La definición de Dios es lo indefinible.}
            \item Describamos la verdad, no puedo decir que algo está enfermo sin conocer la enfermedad, entonces propuestas: \textbf{Nos preguntamos:} ¿verdad = real?
                \begin{itemize}
                    \item La verdad es posible que no transcienda.
                    \item Definamos la tristeza, Harry Potter los dementores y causan que las almas se sientan pesadas, una ausencia total de la alegría. Alma es lo que anima, el dementor causa efectos similares a la tristeza.
                    \item \emph{Citación:``El poeta describe lo indescriptible"}
                    \item Virgina Wolf: oía cantar a los pájaros en griego, le encantaba el licor; se intentó suicidar y lo logró a la tercera, cómo una ``ola que la cubre totalmente'' esto era tristeza.
                    \item Dostoyevsky: sabía cuando venía un ataque de epilepsia, solo sabía que pronto vendría.
                \end{itemize}
        \end{itemize}

    \item \textbf{Nos preguntamos:} ¿qué es el bien?
    \item \textbf{Nos preguntamos:} ¿todo lo que es verdad es bien?
    \item \textbf{Nos preguntamos:} ¿la verdad puede causar mal?
    \item si al verdad lastima, \textbf{Nos preguntamos:} ¿por qué se considera bien?
    \item A largo plazo, \textbf{Nos preguntamos:} ¿la verdad puede cambiar de bien a mal?
\end{enumerate}

%%%%%%%%%%%%%%%%%%%%%%%%%%%%%%%%%%%%%%%%%%%%%%%%%%%%%%%%%%%%%%%%%%%%%%%%%%%%%%%%%%%%%%%%%%%%%%%%

\subsection{Verdad}
\begin{itemize}
    \item Lo que no corrompe el tiempo.
    \item Lo que no provoca cargo de consciencia. Énfasis: con\textbf{S}ciencia, con ``s''.
    \item Lo que libera el alma.
\end{itemize}

\section{Deducciones personales}
\begin{itemize}
    \item La verdad es subjetiva por que nunca se puede porbar con exactitud, entonces lo que yo considero como verdad es por que me es más útil.
    tipos de verdad, algunas que son mutuamente excluyentes. El interior conuna religión que requiera un mandato de dios a tomar lo más que se pueda.
    \item Hay una verdad universal? una justicia universal? en mi opinión no, en el humano lo que hay en común son los incentivos no la verdad, que a todos nos parezca atractivo no matar es por que tenemos un incentivo por naturaleza a no hacerlo, este incentivo puede cambiar.
    \item Hay líderes antiguos que cometieron genocidios mayores que a los de lo nazis y no tiene tanto impacto sentimental por que lo consideramos ``normal en ese tiempo''.
\end{itemize}



%%%%%%%%%%%%%%%%%%%%%%%%%%%%%%%%%%%%%%%%%%%%%%%%%%%%%%%%%%%%%%%%%%%%%%%%%%%%%%%%%%%%%%%%%%%%%%%%

\section{Entonces... \textbf{Nos preguntamos:} ¿La verdad tiene algo que ver con el bien?}
\begin{enumerate}
    \item \textbf{Nos preguntamos:} ¿Hay un juicio que hacemos?
    \item \textbf{Nos preguntamos:} ¿Cuál es l norte?
        \begin{itemize}
            \item El norte es la dirección que orienta.
            \item \textbf{Nos preguntamos:} ¿qué es el norte?
            \item En ese sentido el norte enlaza con el bien.
            \item La verdad es el norte al que nos dirigimos.
            \item La primera misión humana es buscar la verdad.
            \item \textbf{Nos preguntamos:} ¿por qué no es la primera misión sobrevivir?
                \begin{itemize}
                    \item Si el hombre no tiene ninguna verdad a conseguir sería lo más inútil de existencia.
                \end{itemize}
            
            \item El derecho 31, los derecho humanos son 30, el derecho 31 es una variante que se pretende tomar en cuenta a veces por que es la realidad, es el derecho ``la estupidez''.
        \end{itemize}

    \item Ejemplo de la carretera: 
        \begin{itemize}
            \item 
        \end{itemize}

    \item \textbf{Nos preguntamos:} ¿Hay una misión o yo la hago?
        \begin{itemize}
            \item El fenómeno de masa: producir todo con el objetivo de las masas, esto es bueno pero retira la individualidad.
            \item Lo más importante de la persona es lo que no se ve.
            \item 
        \end{itemize}
\end{enumerate}
 