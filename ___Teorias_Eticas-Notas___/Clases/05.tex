\section{\textbf{Nos preguntamos:} ¿Hay en el hombre algo que se le pueda llamar naturaleza?}
\begin{itemize}
    \item \begin{center}
       \begin{tabular}{ | p{8cm} | p{8cm} | }
           \hline
               Si & No     \\
           \hline
                \underline{Instinto} = Naturaleza & \\ 
                \begin{itemize}
                    \item Reacción
                    \item Algo con lo que nacemos 
                    \item Impulso 
                    \item Supervivencia 
                \end{itemize} & 
                \begin{itemize}
                    \item Conjunto de incentivos 
                    \item Ser vivo diferentes formas de actuar o sentir
                \end{itemize} \\ 
            \hline
                \multicolumn{2}{c}{$\overbrace{\text{Intinto = Naturaleza}}^{\text{Ser vivo}}$} \\
            \hline
                \multicolumn{2}{c}{Para los griegos es aquello que conozco que puedo conocer pero no puedo modificar, la naturaleza es aquello que tiene orden sin que el humano lo haga. }
       \end{tabular}
    \end{center}
    
    \item \textbf{Nos preguntamos:} ¿Hay cosas en el hombre que puedan influenciarlo inconscientemente?
\end{itemize}

%%%%%%%%%%%%%%%%%%%%%%%%%%%%%%%%%%%%%%%%%%%%%%%%%%%%%%%%%%%%%%%%%%%%%%%%%%%%%%%%%%%%%%%%%%%%%%%%

\subsection{Video \& Análisis}
\begin{itemize}
    \item \url{https://www.youtube.com/watch?v=LjCzPp-MK48} Flores
    \item \url{https://www.youtube.com/watch?v=PyMBI0ee1qs} Caballo
    \item \url{https://www.youtube.com/watch?v=zpARvJoO4aY} Nacimiento de orca en SeaWorld
    \item \url{https://www.youtube.com/watch?v=h82ltr84_Yg} Nacimiento de bebé humano
    \item Franco de Vita: Si tu no estas ft. Amaia Montero.
    \item Franco de Vita: Será
\end{itemize}

%%%%%%%%%%%%%%%%%%%%%%%%%%%%%%%%%%%%%%%%%%%%%%%%%%%%%%%%%%%%%%%%%%%%%%%%%%%%%%%%%%%%%%%%%%%%%%%%

\subsection{La felicidad}
\begin{itemize}
    \item $\underbrace{\text{Alma}}_{\text{De par en par}}$ $\rightarrow$ Nada $\rightarrow$ Esperanza 
    \item \emph{\textbf{Interesante:} La friendzone ``La felicidad depende de la decisión de otra persona''.}
    \item Abrir el alma es lo más difícil del ser humano. 
\end{itemize}

\section{Continuación de clase}
\begin{itemize}
    \item Naturaleza:
        \begin{enumerate}
            \item Ciclo de vida 
            \item Crecimiento 
            \item Reproducción 
            \item Vida  
            \item Involuntario 
        \end{enumerate}
    
    \item La naturaleza, y análisis: 
        \begin{itemize}
            \item Esto lo estudia la biología
            \item Los perros son cazadores por \emph{Naturaleza}.
        \end{itemize}
    
    \item \textbf{Nos preguntamos:} ¿Entonces hay naturaleza en el humano?
        \begin{itemize}
            \item Hay un orden dado, que se recibe, no se crea.
            \item Hay comportamiento humano que es natural.
            \item Tenemos un temperamento diferente.
            \item Lo dado de la psyche es la psicología.
            \item Si algo tiene alma $\rightarrow$ Está vivo.
            \item \textbf{Nos preguntamos:} ¿Cualquier comportamiento raro es por que está loco? 
                \begin{itemize}
                    \item \emph{Citación:``no se puede hacer nada si así es <tal persona>."} Acudiendo a la locura para explicar la conducta rara.
                    \item Nos afecta la \emph{natura}, hay altas tazas de depresión en países donde no sale el sol.
                    \item \emph{\textbf{Ejemplo: }En Holanda cuando sale el sol todos dejan lo que están haciendo y salen a recibir el sol, alegra el alma. }
                \end{itemize}
        \end{itemize}
    
    \item \emph{\textbf{Respuesta:} \textbf{Nos preguntamos:} ¿Hay algo en el hombre que se cataloga como naturaleza?} Sí hay una parte natural del hombre.
    \item \textbf{Nos preguntamos:} ¿Cómo explicamos la depresión?  
        \begin{itemize}
            \item Es enfermedad de la psyche, cómo estamos dominados por fuerzas inconscientes.
        \end{itemize}
\end{itemize}

