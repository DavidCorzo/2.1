\section{\textbf{Nos preguntamos:} ¿Qué significa que el hombre es además de naturaleza?}
\begin{itemize}
    \item $\underbrace{\text{Además}}_{\text{Adverbio}}$ 
        \begin{itemize}
            \item Un adverbio \textbf{Nos preguntamos:} ¿Qué es? 
            \item Pensar que cuando yo hago dos acciones y las \underline{uno} uso el adverbio además.
            \item Los hombres hacemos dos acciones al mismo tiempo para describir esto usamos ``además''.
        \end{itemize}
    
    \item Además de naturaleza en el hombre:
        \begin{enumerate}
            \item Racional? \textbf{Nos preguntamos:} ¿no es natural ser racional?
                \begin{itemize}
                    \item \textbf{Nos preguntamos:} ¿Alguien en coma está pensando o no? \textbf{Nos preguntamos:} ¿que es razonar? \textbf{Nos preguntamos:} ¿qué pasa si hay gente tarada (tara)?
                    \item \[
                      \text{Ser racional} \leftarrow \text{No tiene nada que ver} \rightarrow \text{Poder expresarse}
                    \]
                    \item \emph{\textbf{Ejemplo: }Descríbanme el sentimiento de tristeza, ¿se puede?}
                    \item La naturaleza, en su definición dada en clase como lo que tenemos en común.
                    \item Estudiar hace que la razón se desarrolle.
                    \item No es válido ser irracional, es que no decidiste seguir la razón.
                    \item \emph{\textbf{Interesante:} Cupido es el dios del enamoramiento, no del amor. El enamoramiento no es decisión.}
                    \item Diferencia entre emociones y sentimientos:
                        \begin{itemize}
                            \item Los sentimientos son emociones a largo plazo.
                            \item Las emociones son reacciones biológicas inmediatas.
                        \end{itemize}
                    
                    \item $\underbrace{\overbrace{\text{Naturaleza}}^{\text{Reacción o dada}}}_{Elegibes}$
                    \item Pensar: \textbf{Nos preguntamos:} ¿Qué penasría un extraterrestre al ver una biblioteca? \textbf{Nos preguntamos:} ¿por qué tiene que haber silencio? \textbf{Nos preguntamos:} ¿es talvés por que uno necesita silencio para leer, por que necesitamos silencio para atender con inteligencia algo?
                \end{itemize}
        \end{enumerate}
\end{itemize}

%%%%%%%%%%%%%%%%%%%%%%%%%%%%%%%%%%%%%%%%%%%%%%%%%%%%%%%%%%%%%%%%%%%%%%%%%%%%%%%%%%%%%%%%%%%%%%%%

\section{Video \& Análisis}
\begin{itemize}
    \item Eros Ramazzotti 
    \item Película ``The intern''
\end{itemize}

%%%%%%%%%%%%%%%%%%%%%%%%%%%%%%%%%%%%%%%%%%%%%%%%%%%%%%%%%%%%%%%%%%%%%%%%%%%%%%%%%%%%%%%%%%%%%%%%

\section{Continuación}
\begin{itemize}
    \item $\text{   Trabajo   } \leftrightarrow \text{  Remuneración  }$
    \item \textbf{Nos preguntamos:} ¿Cuál es el sentido de la conducta humana? Una gran tarea humana es buscar sentido.
        \begin{itemize}
            \item El hombre busca sentido 
            \item Puede no encontrar sentido 
        \end{itemize}
    
    \item Viktor Frankl: 
        \begin{itemize}
            \item Mientras estaba en el campo de concentración se plantea ``por qué vivir en el sufrimiento''.
            \item Una de las cosas que hace que el hombre busque sentido es el sufrimiento. 
            \item \textbf{Nos preguntamos:} ¿Por qué si descubrimos un sentido y hacemos acciones contrarias no sentimos sin sentido?
        \end{itemize}
    
    \item \textbf{Nos preguntamos:} ¿Los animales trabajan?
        \begin{itemize}
            \item \textbf{Nos preguntamos:} ¿Puedo estar feliz en una vida sin sentido?
            \item \textbf{Nos preguntamos:} ¿Se puede ser happy sin worry?
            \item \textbf{Nos preguntamos:} ¿Una frustrante es sentirse inútil?
        \end{itemize}
\end{itemize}


