\section{\textbf{Nos preguntamos:} ¿Qué se necesita para poder estudiar algo?}
%%%%%%%%%%%%%%%%%%%%%%%%%%%%%%%%%%%%%%%%%%%%%%%%%%%%%%%%%%%%%%%%%%%%%%%%%%%%%%%%%%%%%%%%%%%%%%%%
\subsection{Video de Alzheimer}
\begin{itemize}
    \item 4sec una persona se le diagnostica Alzheimer.
    \item El científico Alemán se dió cuenta de que se debe a un daño en la estructura celular, donde se afecta la comunicación celular.
    \item Se provoca la muerte cuando no se puede distribuir las moléculas.
    \item Cada parte del cerebro se ve afectada, cada una con síntomas exorbitantemente destructivo al cerebro.
    \item No hay cura al Alzheimer actualmente.
\end{itemize}
%%%%%%%%%%%%%%%%%%%%%%%%%%%%%%%%%%%%%%%%%%%%%%%%%%%%%%%%%%%%%%%%%%%%%%%%%%%%%%%%%%%%%%%%%%%%%%%%
\subsection{Se estudia por que:}
\begin{enumerate}
    \item Voluntad 
    \item Dedicación 
    \item Interés
\end{enumerate}

\begin{enumerate}
    \item Para que podamos dar información de algo ese algo debe de existir.
        \begin{itemize}
            \item Ante la probabilidad no estoy viendo las cosas, el hombre tiene un problema, no conoce la parte del tiempo que es el futuro.
            \item Hay una enorme cantidad de dinero invertido en intentar especular el futuro, por que el futuro es incierto.
            \item Entonces se guía por la percepción. Problema con decir ``no se''.
        \end{itemize}

    \item La segunda cosa que tiene que existir son hechos: 
        \begin{itemize}
            \item por ejemplo el Alzheimer: tiene que haber una causa de enfermedad.
            \item Función $\rightarrow$ saludable.
            \item Para conocer la enfermedad implica que tuve que haber conocido la salud una vez.
            \item \[
              \underbrace{\text{Salud}}_{Subjetivo} \rightarrow \underbrace{\text{Enfermo}}_{\text{Mysterioso}}
            \]
            \textbf{Nos preguntamos:} ¿La salud es lo que tenemos en común? \textbf{Nos preguntamos:} ¿y la enfermedad lo que no?
            
            \item Lo corrupto es lo que no es normal, lo que era saludable pero no.
        \end{itemize}
    
        
    \item* \emph{\textbf{Ejemplo: }cuadro, \textbf{Nos preguntamos:} ¿es bello?} representación de la mitología griega de Cronos.
        \begin{itemize}
            \item \textbf{Nos preguntamos:} ¿por qué no es bello? para poder opinar esto hay que \textbf{conocer} lo bello.
                \begin{enumerate}
                    \item Canibalismo 
                    \item Limpio 
                    \item Emite y transmite algo feo
                \end{enumerate}
            
            \item \textbf{Nos preguntamos:} ¿por qué es bello? 
                \begin{itemize}
                    \item Por que transmite un ¿mensaje? $\rightarrow$ ¿que todo lo que transmite un mensaje es bello?
                \end{itemize}
            
            \item Gea produce todo lo que vemos, Gea pare a los hijos y Cronos se los come, al único que no se come Cronos es a Zeus.
                \begin{itemize}
                    \item Todo está sujeto al tiempo.
                    \item \emph{Citación:``cuando un quiere que no pase el tiempo"}.
                \end{itemize}
            
            \item Nos parecen mejor las cosas que no están corruptas.
        \end{itemize}
    
        
    \item* \emph{\textbf{Ejemplo: }Mona Lisa \textbf{Nos preguntamos:} ¿es bella?}
        \begin{itemize}
            \item \textbf{Nos preguntamos:} ¿La belleza tiene que ver con la moral?
            \item Platón: el filósofo de la belleza, decía en lugar de es malo decía ``no es bello''.
            \item Chester: ``Para decir que está mal $\rightarrow$ tengo que conocer lo bueno''.
            \item La belleza no es igual a la lujuria.
        \end{itemize}
\end{enumerate}

%%%%%%%%%%%%%%%%%%%%%%%%%%%%%%%%%%%%%%%%%%%%%%%%%%%%%%%%%%%%%%%%%%%%%%%%%%%%%%%%%%%%%%%%%%%%%%%%
\subsection{Receso}
\begin{itemize}
    \item Ricardo Montaner 
\end{itemize}

%%%%%%%%%%%%%%%%%%%%%%%%%%%%%%%%%%%%%%%%%%%%%%%%%%%%%%%%%%%%%%%%%%%%%%%%%%%%%%%%%%%%%%%%%%%%%%%%
\subsection{Oscar Wilde}
\begin{itemize}
    \item \textbf{Nos preguntamos:} ¿qué personaje es usted? \emph{\textbf{La respuesta a esta pregunta es: }uno es el que soy, otro es el que quiero ser.}
\end{itemize}

%%%%%%%%%%%%%%%%%%%%%%%%%%%%%%%%%%%%%%%%%%%%%%%%%%%%%%%%%%%%%%%%%%%%%%%%%%%%%%%%%%%%%%%%%%%%%%%%
\section{Entonces \textbf{Nos preguntamos:} ¿qué se necesita para estudiar algo?}
\begin{itemize}
    \item Que exista lo bueno (enfermedad - salud)
    \item Que el hombre sea capaz de su conocimiento. Puede ser que exista pero que yo sea incapaz de poder verlo. \emph{\textbf{Ejemplo: }El ciego no puede ver el color rojo eso no implica que no exista.}
    \item Hay cosas que no puedo ver pero no $\rightarrow$ que no exista.
    \item Para que exista tiene que ser capaz de su conocimiento.
\end{itemize}
