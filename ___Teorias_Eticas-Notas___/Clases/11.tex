\section{Análisis de Hildebrand}
% --------------------------------------------------------------------------------------------
Text, Pedro Salinos:
\subsection{Actitud}
\emph{\textbf{Definición de ``actitud":} es la manera que la persona se para frente la realidad.}
% --------------------------------------------------------------------------------------------
\subsection{Text, Pedro Salinas:}
\begin{Verbatim}[breaklines=true, breakanywhere=true]
    ¡Qué alegría, vivir
    sintiéndose vivido!
    Rendirse
    a la gran certidumbre, oscuramente,
    de que otro ser, fuera de mí, muy lejos,
    me está viviendo.
    Que cuando los espejos, los espías
    -azogues, almas cortas-, aseguran
    que estoy aquí, yo inmóvil,
    con los ojos cerrados y los labios,
    negándome al amor
    de la luz, de la flor y de los hombres,
    la verdad trasvisible es que camino
    sin mis pasos, con otros,
    allá lejos, y allí
    estoy buscando flores, luces, hablo.
    Que hay otro ser por el que miro el mundo
    porque me está queriendo con sus ojos.
    Que hay otra voz con la que digo cosas
    no sospechadas por mi gran silencio;
    y es que también me quiere con su voz.
    La vida -¡qué transporte ya!-, ignorancia
    de lo que son mis actos, que ella hace,
    En que ella vive, doble, suya y mía.
    Y cuando ella me hable
    de un cielo oscuro, de un paisaje blanco,
    recordaré
    estrellas que no vi, que ella miraba,
    y nieve que nevaba allá en su cielo.
    Con la extraña delicia de acordarse
    De haber tocado lo que no toqué
    sino con esas manos que no alcanzo
    a coger con las mías, tan distantes.
    y todo enajenado podrá el cuerpo
    descansar, quieto, muerto ya. Morirse
    en la alta confianza
    de que este vivir mío no era sólo
    mi vivir: era el nuestro. Y que me vive
    otro ser por detrás de la no muerte.
\end{Verbatim}

% --------------------------------------------------------------------------------------------
\subsection{Actitudes $\biconditional$ Reverencia:}
\begin{itemize}
    \item Crece 
    \item Útil para dar 
    \item Finita 
    \item Ciclo 
\end{itemize} 




%%%%%%%%%%%%%%%%%%%%%%%%%%%%%%%%%%%%%%%%%%%%%%%%%%%%%%%%%%%%%%%%%%%%%%%%%%%%%%%%%%%%%%%%%%%%%%%%
\section{Analogía de los ciegos de Hildebrand}
\begin{itemize}
    \item \begin{center}
        \begin{tabular}{ | p{5cm} | p{5cm} | }
            \hline
                 Ciego & Ciego ( intelectual )     \\
            \hline
                 \begin{itemize}
                     \item No \underline{puedo} ver.
                     \item No \underline{quiero} ver, puedo cerrar los ojos.
                     \item Incertidumbre. 
                     \item Oscuridad.
                 \end{itemize}
                 \begin{itemize}[label=\#]
                    \item Estoy dispuesto a confiarle al chucho mi visión.
                    \item Muchas veces queremos le delegamos más confianza al ``cuate de la par'', es el equivalente muchas veces a preguntarle a otro ciego.
                    \item Para Hildebrant \textbf{``Blindly disregrards''} es referente a una actitud irreverente.
                 \end{itemize} & 
                 \begin{itemize}
                     \item Ignorar 
                 \end{itemize} 
                 \begin{itemize}[label=\#]
                     \item Deliberadamente saber y renunciar a la curiosidad de saber.
                     \item Cuando vemos algo y queremos difundirlo (y no es casaca), no es recibida de la mejor manera, el mito de la caverna ilustra que las masas son demasiado tontas como para gobernarse a sí mismos.
                 \end{itemize}
                 \\ 
        \end{tabular}
     \end{center}     
     
     \item Persistencia de la memoria:
        \begin{itemize}
            \item Recuerdo 
            \item Identidad 
            \item 
        \end{itemize}
        \begin{itemize}[label=\#]
            \item 
        \end{itemize}
        \begin{itemize}
            \item Cuadro, el reloj, \emph{Citación:``Si no existiese el tiempo, no existiera la memoria. Si no existiese el presente no existieran recuerdos."}
        \end{itemize}
    
    \item Mientras uno más conoce algo, lo trata mejor:
    
    \item La inmediatez:
        \begin{itemize}[label=\#]
            \item La inmediatez en un compromiso no es deseable, si me dejo llevar por el corto-placismo y no tengo una persistencia de la memoria tomo malas decisiones.
            \item Si sólo me pregunto ``¿Me satisface hoy?'' caigo en inmediatez.
            \item Tratar de no ser ciego implica pararse frente a la realidad y preguntarle ¿quién eres?
            \item Preguntas:
                \begin{enumerate}
                    \item ¿Quién es en general? (si no automáticamente la agarro como útil para mi)
                \end{enumerate}
            
            \item Si yo no conzco al otro ser, puedo cometer injusticias ignorantemente si no es algo peor como injusticias por que estoy ciego intelectualmente.
        \end{itemize}
\end{itemize}


%%%%%%%%%%%%%%%%%%%%%%%%%%%%%%%%%%%%%%%%%%%%%%%%%%%%%%%%%%%%%%%%%%%%%%%%%%%%%%%%%%%%%%%%%%%%%%%%%%%
\section{Videos}
\begin{itemize}
    \item \url{https://www.youtube.com/watch?v=1RWOpQXTltA}
\end{itemize}


%%%%%%%%%%%%%%%%%%%%%%%%%%%%%%%%%%%%%%%%%%%%%%%%%%%%%%%%%%%%%%%%%%%%%%%%%%%%%%%%%%%%%%%%%%%%%%%%%%%
