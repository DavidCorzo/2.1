\section{Análisis de Hildebrant}
\begin{itemize}
    \item ``Only he who understands that there exists things ``important in themselves,'' that there are things which are beautiful and good in themselves, only the man who grasps the sublime demand of values ''.
    \item Hacer las preguntas:
        \begin{enumerate}
            \item \textbf{Nos preguntamos:} ¿Qué preguntas se harían si ven a alguien caminando por ahí?

            \item \textbf{Nos preguntamos:} ¿Quién es? 
                \begin{itemize}
                    \item ¿Me sirve saber quién es? dgcm 
                    \item \emph{\textbf{Ejemplo: }¿es útil ver al catedrático y al alúmno como que se benefician, el catedrático ?} dgcm
                    \item Lo que oigo ahí es que es más útil pesar en los demás como ``qué hace o cómo me sirve''
                    \item En términos estrictamente utilitarios no es útil pensar de una manera utalitaria por que uno es estereotipado como egoísta.
                    \item ¿Lo correcto yo lo pienso como lo correcto sin la utilidad?
                    \item Hay relaciones en las que no rigen las reglas de la economía.
                \end{itemize}
        \end{enumerate}
\end{itemize}



%%%%%%%%%%%%%%%%%%%%%%%%%%%%%%%%%%%%%%%%%%%%%%%%%%%%%%%%%%%%%%%%%%%%%%%%%%%%%%%%%%%%%%%%%%%%%%%%%%%
\section{Palabras clave}
\begin{itemize}
    \item Sócrates: es mejor que le hagan la injusticia a cometerla.
\end{itemize}
