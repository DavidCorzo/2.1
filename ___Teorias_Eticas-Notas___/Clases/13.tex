\section{Análisis de película}
\begin{itemize}
    \item Personaje:
        \begin{center}
           \begin{tabular}{ | p{5cm} | p{5cm} | }
               \hline
                    Personajes & Actos   \\
               \hline
                    \begin{itemize}
                        \item Auddie 
                    \end{itemize}
                    & 
                    \begin{itemize}
                        \item Odio a su apariencia 
                        \item Miedo a la no aceptación 
                        \item Is a la escuela 
                        \item Usa el casco 
                        \item Estudiar 
                        \item Is al campamento 
                        \item Agarrarse a cuentasos 
                        \item Reacciones de enojo 
                        \item Llorar 
                        \item Dejar de hablar 
                        \item Comer 
                        \item Jugar Minecraft 
                    \end{itemize}
               \hline
           \end{tabular}
        \end{center}
    
    \item Las etapas de aceptación.
        \begin{enumerate}
            \item Negación 
            \item Negociación 
            \item Depresión 
            \item Aceptación 
        \end{enumerate}
    
    \item Inseguridad:
        \begin{itemize}
            \item Confía en mí 
            \item Cómoda 
            \item No se quiere 
            \item Importancia a lo que ven los demás 
            \item Igual 
        \end{itemize}

    
    \item Comprobación de nociones:
        \begin{itemize}
            \item Para yo decirme gordo debo compararme con una persona muy delgada.
            \item Compararme a mí mismo con las demás personas.
        \end{itemize}
    
    \item En la cabeza humana se intuye:
        \begin{itemize}
            \item \textbf{¿}por qué pensamos que las demás personas piensan en nosotros\textbf{?} 
            \item Hay actos que no son malos en sí, pero por la no aceptación de los demás lo hacen malos. Influencia de las masas.
        \end{itemize}
    
    \item Lo paradójico de la belleza:
        \begin{itemize}
            \item La bella y la bestia: cuando uno se ve como una bestia es difícil poder aceptar eso, poder lograr que una persona te quiera no por tu belleza si no por lo que sos realmente.
            \item Conocer lo que sos realmente tiene un costo, tiempo y dinero, costos sociales; es por eso que es tan difícil conocer a tantas personas.
        \end{itemize}
\end{itemize}
