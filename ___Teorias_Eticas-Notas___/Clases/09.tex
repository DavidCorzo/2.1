\section{Dudas - Ditrich Von Hildebrand} 
\begin{itemize}
    \item Alemán, es del siglo 20.
    \item Inicio de la fenomenología: \emph{\textbf{Definición de ``fenomenología":} trata de volver a los fenómenos afuera de la mente, todo hasta la fenomenología pensaba que todo era una construcción mental, decían básicamente que sin la mente no hay una filosofía, la fenomenología intenta refutar eso y exteriorizar a la filosofía del humano y considerarlo como un fenómeno externo.}
\end{itemize}


%%%%%%%%%%%%%%%%%%%%%%%%%%%%%%%%%%%%%%%%%%%%%%%%%%%%%%%%%%%%%%%%%%%%%%%%%%%%%%%%%%%%%%%%%%%%%%%%
\section{\textbf{Nos preguntamos:} ¿Qué son los valores morales?, \textbf{Nos preguntamos:} ¿Qué significa el \textbf{ser personal} para Hildebrand?}
\begin{itemize}
    \item Dietrich Von Hildebrand ``Fundamental moral attitudes''
        \begin{enumerate}
            \item \textbf{Nos preguntamos:} ¿Descubrimiento (¿no moral?)?
            \item \textbf{Nos preguntamos:} ¿Libertad y moral están relacionadas?
            \item \textbf{Nos preguntamos:} ¿Valores personales y morales?
            \item \textbf{Nos preguntamos:} ¿Fieles, humildes y amorosos?
            \item \textbf{Nos preguntamos:} ¿Por qué son sólo las opciones de Bien y Mal y no una alterna?
        \end{enumerate}
    
    \item Preguntas a Hildebrand:
        \begin{enumerate}
            \item \textbf{Nos preguntamos:} ¿Qué le preguntaría?
        \end{enumerate}

    \item Inhere:
        \begin{tikzpicture}[node distance = 2.5cm, auto]
            \node [block] (1) {valor}; 
            \node [block,below of=1] (2) {Habita}; 
            \node [block,right of=2] (3) {Hogar}; 
            \node [block,right of=3] (4) {Alguien}; 
            \path [line] (2) -- (1);
            \path [line] (2) -- (3);
            \path [line] (3) -- (4);
            \path [line] (1) -- (3);
        \end{tikzpicture}
    
    \item El secreto:
        \begin{itemize}
            \item Tenemos semejanza al perro por que podemos hacer cosas similares.
            \item Pero el secreto nos hace diferentes, el secreto sólo sale cuando voluntariamente quiere que salga.
            \item El secreto se manifiesta en el corazón y los pensamientos en la mente.
            \item El enamoramiento se manifiesta en el corazón.
            \item \emph{Citación:``Nosotros somos personas por que tenemos ese ondón de alma, ese espíritu. Si nos meten a un laboratorio a estudiar las neuronas van a poder descubrir muchas cosas menos lo que pensamos; no pueden saber por más que estudien las neuronas a quién le he dado mi corazón, esta es la \textbf{intimidad}"}.
            \item La intimidad, 
            \item El cuerpo le sirve al hombre para expresar algo que no puedo ver.
            \item Cuando digo ``tú no eres así'' estoy acudiendo a el corazón.
        \end{itemize}
    
    \item Para ser fiel \textbf{Nos preguntamos:} ¿qué se necesita?
        \begin{itemize}
            \item Empieza con una promesa; Hildebrand dice que la promesa proviene del corazón, solo promete alguien que es capaz de iniciar un proceso, por que la promesa no es necesaria.
            \item \emph{Citación:``tenemos capacidad de proponer hacer carreteras a otras personas, la idea de estar con alguien aunque se ponga gorda, tenga carácter horrible. Los celos es pensar que esa carretera se va a romper."} 
            \item \emph{Citación:``El único capaz de hacer una carretera es alguien comprometido ``con promesa", esta es la persona humana, la promesa es inherente al hombre, si alguien ya hizo una carretera no puede hacer otra por que no puede pasar por las dos al mismo tiempo ``costo de oportunidad''}.
            \item \emph{Citación:``Propósito de un anillo, es un recordatorio a la promesa"}.
            \item \emph{Citación:``El enamoramiento no es decidido pero puedo decidir si amar o no, en ese sentido el amor también nace de el corazón."} 
            \item \emph{\textbf{Ejemplo: }``Hannah Arendt''; \emph{Citación:``El acto del perdon parte del ondón del alma, en la naturaleza solo hay vendetta, decir perdón no es necsario "}}.
        \end{itemize}
    
    
    \item La humildad:
        \begin{itemize}
            \item \emph{Citación:``Consiste en no pensar que por un valor que adopté soy mejor que los demás."}
            \item \emph{Citación:``La única persona que sabe las intenciones es uno en la intimidad, el hombre es el único capaz de mentir con sus acciones."}
            \item El hombre es capaz de falsear su identidad, su actuar, etcétera.
            \item En este sentido la maldad sale del corazón también, \emph{Citación:``El sentirse usado es lo peor, por que en el fondo me mintió, y a nadie le gusta ser engañado"}, \emph{Citación:``Él no dice es malo mentir, dice que el hombre deliberadamente prefiere no ser engañada"}. 
        \end{itemize}
\end{itemize}


%%%%%%%%%%%%%%%%%%%%%%%%%%%%%%%%%%%%%%%%%%%%%%%%%%%%%%%%%%%%%%%%%%%%%%%%%%%%%%%%%%%%%%%%%%%%%%%%
\section{Continuación}
\begin{itemize}
    \item \emph{Citación:``no vivir en una mentira, ser hinesto"}.
    \item Los valores personales morales son:
        \begin{itemize}
            \item Todo lo que pude haber \textbf{evitado}, sólo la persona pudo haberlo no hecho.
            \item Libertad y moral están relacionadas.
            \item Un descubrimiento no es moral: entender algo no viene deliberadamente, uno viene un ``aja moment'' inconsciente.
            \item En el sentido de verdad o falso, sólo en esos términos se puede poner el bien y el mal.
        \end{itemize}
\end{itemize}

%%%%%%%%%%%%%%%%%%%%%%%%%%%%%%%%%%%%%%%%%%%%%%%%%%%%%%%%%%%%%%%%%%%%%%%%%%%%%%%%%%%%%%%%%%%%%%%%
\section{Deducciones personales}
\begin{itemize}
    \item \textbf{Nos preguntamos:} ¿Un descubrimiento, libro científico, arte; es bueno o mal?
        \begin{itemize}
            \item A veces los descubrimientos científicos son malos para una persona, cuando se descubra que hay alternativas más efectivas que Quimioterapia para tratar el cáncer muchas personas perderán y considerarán malo el descubrimiento.
            \item Que un libro sea malo o bueno es subjetivo.
            \item Por qué no catalogar en una categoría \emph{neutral} los adjetivos que usamos para estos objetos inherentes, no son ni buenos ni malos.
        \end{itemize}
    
    \item Opino que no todos sabemos del todo nuestras intenciones.
\end{itemize}


%%%%%%%%%%%%%%%%%%%%%%%%%%%%%%%%%%%%%%%%%%%%%%%%%%%%%%%%%%%%%%%%%%%%%%%%%%%%%%%%%%%%%%%%%%%%%%%%
\section{The beauty and the beast}
\begin{itemize}
    \item \emph{Citación:``No tiene amor en su corazón"}, toda las fuerzas del alma las tenía en la apariencia.
    \item \emph{\textbf{Interesante:} Cómo la Bestia logra que Bella se enamore de él.}
\end{itemize}

%%%%%%%%%%%%%%%%%%%%%%%%%%%%%%%%%%%%%%%%%%%%%%%%%%%%%%%%%%%%%%%%%%%%%%%%%%%%%%%%%%%%%%%%%%%%%%%%
\section{Recursos}
\begin{itemize}
    \item Orígenes del totalitarismo, Arendt.
\end{itemize}





