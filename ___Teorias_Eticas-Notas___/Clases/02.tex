\section{Preguntas}
\subsection{\textbf{Nos preguntamos:} ¿Por qué estudiamos ética?}
\begin{itemize}
    \item Requisito 
    \item Conocerse 
    \item Mejor percepción de lo bueno 
    \item Vida con valores 
    \item Decisiones de negocios  
\end{itemize}

\subsection*{La ética es ... }
\begin{itemize}
    \item Valores 
    \item Moral 
    \item Correcto 
\end{itemize}

\subsection{\textbf{Nos preguntamos:} ¿Qué se necesita para poder estudiar algo?}
\begin{itemize}
    \item tender con la inteligencia algo para tratar de comprenderlo.
    \item Video de Jordan Peterson \url{https://www.youtube.com/watch?v=IFmQ5waavJY}
        \begin{itemize}
            \item ``Your atention, is mediated by unconscious forces -- and you know that perfectly well.''
            \item Cuando yo tengo una distracción y la atiendo las fuerzas inconscientes van a atenderla.
            \item Las fuerzas inconscientes tienden a no querer sentarse aburridamente a estudiar.
            \item Memory function: these are not in conscious control; 
            \item The idea that you are in control of them is egotistical.
            \item For examples, The OCD disorder is a manifestations; all of this is autnomous, they don0t control.
        \end{itemize}
    
    \item En este video el obstáculo para estudiar es el subconsciente; la cabeza cuando uno pone a estudiar uno tiene que tener control; cuando yo domino no las fuerzas inconscientes no les gusta y me \textbf{encuentro haciendo cosas que no quiero hacer}, dejarse llevar por esas fuerzas inconscientes es fácil y sin esfuerzo, lo difícil es intentar esforzarse en no hacer cosas que no quiero hacer.
    \item \textbf{Nos preguntamos:} ¿Cómo podemos atender con la inteligencia un tema?
        \begin{itemize}
            \item \textbf{Nos preguntamos:} ¿por qué estudiamos integrales triples? \emph{\textbf{La respuesta a esta pregunta es: }por que será útil un día (esperanza)}
            \item Lo propio del hombre tiene un impulso a querer conocer, es interesante, 
        \end{itemize}

        \begin{center}
           \begin{tabular}{ | p{5cm} | p{5cm} p{5cm} | }
               \hline
                A. Esperanza      &                       &      \\
                B. Útil           & Conocer $\Rightarrow$ & Interesante \\ 
                C. Tiempo y lugar &                       &         \\ 
               \hline
           \end{tabular}
        \end{center}

        \begin{figure}[htbp]
            \centering
            %\includegraphics[width=6cm]{2020-01-09_01.jpg}
            %\includegraphics[width=6cm]{2020-01-09_02.jpg}
            %\includegraphics[width=6cm]{2020-01-09_03.jpg}
            \caption{Diagramas}
            \label{}--
        \end{figure} 
    
    \item \textbf{Nos preguntamos:} ¿Por qué elegimos la carrera, atender con la inteligencia varios años? 
        \begin{itemize}
            \item El problema actual es que tenemos que elegir, por que no podemos estar en dos lugares a la vez, ni al mismo tiempo.
            \item Entre todas estas decisiones estudiar al hombre y cierto tipo de acciones y comportamientos humanos, vamos a atender con inteligencia eso.
        \end{itemize}

\end{itemize}


\subsection{\textbf{Nos preguntamos:} ¿Qué vamos a estudiar?}
\begin{itemize}
    \item Énfasis: el comportamiento
    \item \url{https://www.youtube.com/watch?v=WNWVzg8ZfHg} -  pelea entre leones.
        \begin{itemize}
            \item \textbf{Nos preguntamos:} ¿Por qué no se puede decir que los leones no les enseñaron que está mal pelear?
            \item \textbf{Nos preguntamos:} ¿Es lo mismo la naturaleza y los instintos?
            \item \textbf{Nos preguntamos:} ¿Es la diferencia entre las personas y los leones, qué hubiera pasado si un hombre le hace eso a otro?
            \item Ejemplo del perro y la reproducción.
        \end{itemize}
    \item Joker 00:00 - 04:00 minutos: 
        \begin{itemize}
            \item Si tomamos la proposición que los leones no actúan por voluntad entonces $\rightarrow$ el humano es el único que es realmente capaz de ser malvado.
            \item A finalizar la WWII surge el existencialismo.
            \item \textbf{Nos preguntamos:} ¿Por que no es lo mismo comerse a la abuelita que comerse al cerdo?
            \item Vamos a tratar en este curso de tratar de atender con inteligencia estos comportamientos.
        \end{itemize}
\end{itemize}

%%%%%%%%%%%%%%%%%%%%%%%%%%%%%%%%%%%%%%%%%%%%%%%%%%%%%%%%%%%%%%%%%%%%%%%%%%%%%%%%%%%%%%%%%%%%%%%%
\section{Notas finales}
\begin{itemize}
    \item \textbf{Nos preguntamos:} ¿Cómo justificamos nuestro comportamiento?
    \item \textbf{Nos preguntamos:} ¿Cual es la diferencia entre instinto y naturaleza?
    \item \textbf{Nos preguntamos:} ¿Problemas epistemológicos de lo que sabemos?
    \item \textbf{Nos preguntamos:} ¿Es por problemas de instinto que no controlamos las fuerzas inconscientes?
\end{itemize}
