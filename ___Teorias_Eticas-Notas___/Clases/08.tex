\section{Análisis de Oscar Wilde}
\begin{itemize}
    \item \emph{\textbf{Recordar lo siguiente: }Describir $\neq $ definir, definir encapsula.} 
    \item \textbf{Nos preguntamos:} ¿Qué significa que algo es bello?
        \begin{enumerate}
            \item Genera paz 
            \item Asombro 
            \item Felicidad
        \end{enumerate}
    
    \item Entonces lo bello $\rightarrow$ defectos $\rightarrow$ Envidia.
    \item Conclusión personal de MF: La belleza genera después de todo eso genera gracia, agradecimiento, ``Gracias por existir'', la consecuencia de la gracia es cuidar la existencia del otro.
    \item \textbf{Nos preguntamos:} ¿por que Oscar Wilde dice que los elegidos para quienes las cosas son bellas significan sólo belleza? \emph{\textbf{Respuesta:} \emph{Citación:``Yo no quiero usarte"}, estas personas son las elegidas por que se siguen asombrando por ver belleza.}
    \item \emph{\textbf{Ejemplo: }Personas que ya no le conmueve nada...  }
        \begin{itemize}
            \item Ditrich Von Hindermat
        \end{itemize}
\end{itemize}


%%%%%%%%%%%%%%%%%%%%%%%%%%%%%%%%%%%%%%%%%%%%%%%%%%%%%%%%%%%%%%%%%%%%%%%%%%%%%%%%%%%%%%%%%%%%%%%%

\subsection{\textbf{Nos preguntamos:} ¿Lo que se acerca más a la verdad sobre el hombre?, \textbf{Nos preguntamos:} ¿Entonces la verdad tiene algo que ver con el bien?}
\begin{itemize}
    \item \emph{Citación:``Nadie ama lo que no conoce"}; \textbf{Nos preguntamos:} ¿Cómo conozco? Tiene el asunto de que para conocer uno tiene que manifestar su alma al otro.
    \item \textbf{Nos preguntamos:} ¿Entonces existe el amor a primera vista? 
    \item \textbf{Nos preguntamos:} ¿No es el amor a primera vista la atracción biológica del uno al otro? si no me abres el alma no te conozco, no abres tu corazón no te puedo conocer. Si sólo fuera por atracción a la primera deformación física se acabó.
    \item No puedo amar algo, hacer algo, o vivir de algo si no lo conozco.
    \item Entonces la verdad es el norte.
    \item \emph{Citación:``Por eso no me gusta el reggaetón, por que sólo describe la atracción, lo superficial, la lujuria."}
    \item \emph{Citación:``El hombre tiene que conocer si no se vuelve en chucho."}
    \item \emph{Citación:``La persona que está enfrente de algo grande y no se da cuenta, es de las personas que dice Oscar Wilde, algo pasa"}. \emph{Citación:``Que una persona cuenta un secreto, es como se estuviera compartiendo parte de su alma"}. 
    \item \emph{Citación:``La persona no puede abrir el alma si no hay una persona digna de su confianza."}
    \item \emph{Citación:``por eso conocer al hombre tiene muchas repercusiones"}.
    \item Lo que se acerca mas a la verdad sobre el hombre, por eso tenemos que conoces al hombre.
    \item 
\end{itemize}
\begin{center}
   \begin{center}
      \begin{tabular}{ | p{9cm} | p{9cm} | }
        \hline
            Actitud & Actos   \\
        \hline
            \begin{itemize}
                \item Indiferencia
                \item $\overbrace{\text{  Positivo  }}^{\text{  ¿Feliz? / ¿optimista? }}$
            \end{itemize} & 
            \begin{itemize}
                \item Desinterés
            \end{itemize}\\

            \begin{itemize}
                \item La única manera que tenemos de conocer la actitud es conocernos a nosotros.
                \item Muchas de las cosas antes de salir por la boca se cuese en el corazón. Todas las cosas malas yacen del corazón
            \end{itemize} & 
            \begin{itemize}
                \item \textbf{Suponer} actitudes a partir de desinterés.
                \item Es una gran ignorancia pretender asumir que se puede decir qué pasa en la intimidad del otro.
            \end{itemize} \\ 
            \multicolumn{2}{|c|}{Las actitudes propuestas por Hindebrand} \\ 
            \begin{itemize}
                \item Los hombres podemos mentir existencialmente.
                \item Cuando uno está solo en su habitación, ahí no hay engaño.
                \item \emph{\textbf{Ejemplo: }Chava que escoge de 50 fotos, la que escoge la photoshopea...}
            \end{itemize} & 
            \begin{itemize}
                \item Mientras más se adentre uno en la realidad más es la esperanza.
            \end{itemize} \\ 
            \multicolumn{2}{|c|}{Las dos, actitudes y los actos necesita limpieza  } \\ 
        \hline
      \end{tabular}
   \end{center}
\end{center}




%%%%%%%%%%%%%%%%%%%%%%%%%%%%%%%%%%%%%%%%%%%%%%%%%%%%%%%%%%%%%%%%%%%%%%%%%%%%%%%%%%%%%%%%%%%%%%%%
\section{\textbf{Consultar el siguiente recurso:}}
\begin{itemize}
    \item Hidebrant
\end{itemize}

