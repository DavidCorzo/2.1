\documentclass{article}

\usepackage{generalsnips}
\usepackage{calculussnips}
\usepackage[margin = 1in, top=0.5in]{geometry}
\usepackage{pdfpages}
\usepackage[spanish]{babel}
\usepackage{amsmath}
\usepackage{amsthm}
\usepackage[utf8]{inputenc}
\usepackage{titlesec}
\usepackage{xpatch}
\usepackage{fancyhdr}
\usepackage{tikz}
\usepackage{hyperref}
\title{Parcial \#1}
\date{2020 Marzo 17}
\author{Carné: 20190432}

\begin{document}
\maketitle
%%%%%%%%%%%%%%%%%%%%%%%%%%%%%%%%%%%%%%%%%%%%%%%%%%%%%%%%%%%%%%%%%%%%%%%%%%%%%%%%%%%%%%%%%%%%%%%%%%%%%%%%%%%%%%%%%%%%%%%%%%%%%%%%%%%%%%%%%%%%%%
\section{Introducción}
\begin{center}
    \begin{tabular}{ |p{14cm}| }
        \hline
            ``The said truth is that it is the greatest happiness of the greatest number that is the measure of right and wrong.'' \\
        \hline
    \end{tabular}
\end{center}
\begin{center}
    \begin{tabular}{ |p{14cm}| }
        \hline
        Afirmación: ``Las personas construyen sus propias verdades de acuerdo a cuánta felicidad les brinda, las verdades generales son las que se aceptan en mayores números de acuerdo a qué tan justo sea para todos. '' \\ 
        \hline
    \end{tabular}
\end{center}
% Una de las cosas que más me sorprendió cuando empecé a estudiar el concepto de verdad en las clases de filosofía es qué tanto desacuerdo hay en el concepto
Una de las cosas que es más sorprendente cuando se empieza a estudiar el concepto de \emph{verdad} en clases de filosofía es qué tanto desacuerdo hay en el concepto, está la versión cristiana de la verdad, la versión atea de la verdad, etcétera; Al uno tener tantas formas de definir un concepto al que se le refiere con el mismo nombre pero que el concepto en sí tiene tan amplias y mutuamente excluyentes interpretaciones es sólo lógico empezar a formular un argumento que sostenga que la verdad es una construcción individual y social. Las personas tendrán diferentes interpretaciones de las instancias físicas de conceptos mentales en el mundo. Entonces cuando se menciona en la frase inicial que la verdad es aquello que es la mayor felicidad de la mayor cantidad y es una medida de el bien y el mal se torna ambigüo conceptualizar lo que se intenta describir. Por un lado la verdad que todas las personas al estudiarla de cierta manera pueden llegar a concluir casi sin desacuerdo alguno, por otro lado las verdades que la gente construye individualmente, por ejemplo la gente está dispuesta a construir verdades como las de intentar explicar con conocimientos aposteriori fenómenos. La intuición es la clave aquí, las personas intuyen o asumen que puestos en $x$ situación o afrontado con $x$ problema la solución sería $y$. Pero... \pregunta{Qué podemos decir que es una construcción útil y qué no} \pregunta{Cómo se define la mayor cantidad de personas felices} \pregunta{Qué se puede hacer para no empezar a construir verdades en el aire} 


%----------------------------------------------------------------------------------------
\section{Razón \# 1: Las verdades pueden ser construidas y aceptadas o no, basado en qué tanta utilidad les brinde a la mayor cantidad de personas}
Cuando uno intenta usar una fórmula matemática, es increíble cómo una simbología y un sistema numérico puedan ilustrar conceptos que son tan precisos que nunca se logrará manifestar tal precisión en el mundo real. 
Cosas como $\sqrt{2}$ cuyos decimales son infinitos y no hay patrón alguno que pueda seguirse para calcularlo. Este concepto que se sabe que nunca se va a poder ver empíricamente se considera una verdad, pero no siempre fue así, se llegó a aceptar hasta que se le vió la utilidad, anteriormente se veía la descripción de $\sqrt{2}$ como blasfemia de hecho. 
Antiguamente en los tiempos de Pitágoras, se tenía una religión, su religión era de los número enteros, es decir que él y sus discípulos creían que sólamente existían los números enteros, cosas como fracciónes y decimales eran consideradas indeseables pero uno de los conceptos cuya existencia en sí hacía fuerte frente a la religión de los números enteros eran los números irracionales tales como $\sqrt{2}, \pi, \sqrt{3}$ etcétera. 
Un día un discípulo de Pitágoras llamado Hippasus llegó a descubrir la existencia de los números irracionales y al Pitágoras encontrar esta irreverencia al culto decidió mandar a matar a Hippasus, tanto así que Hippasus murió por haber descubierto la verdad. 
Este ejemplo es increíblemente recursivo pasó lo mismo con Galileo Galilei cuando descubrió que la tierra es la que orbita al rededor del sol, hasta 300 años pasaron hasta que la iglesia católica le diera la razón a Galileo, estos ejemplos de cosas que todos aceptamos como ciertas no siempre fueron aceptadas.
Si aceptar una verdad no me brinda utilidad lo más probable es que no la vaya aceptar, en este caso el culto de Pitágoras no le brindaba utilidad la aceptación de los números irracionales y a la iglesia católica no le convenía ni le brindaba utilidad aceptar las verdades propuestas por Galileo.

\subsection{Posible refutación}
El hecho que los ejemplos tomados en el argumento son derivados de una verdad general en el mundo de matemáticas y física pueden ser acusados a haber sido escogidos especialmente (cherry picked). En este sentido al aplicar este concepto a todas las sociedades humanas es peligroso generalizar, la cantidad de cosas dadas y tomadas como premisas es grandísimo y con la posibilidad de sólamente una de esas cosas tomadas como dadas torne a ser falsa esta teoría se vuelve un desastre. El hecho que este sea el estado de las cosas nos pone a decidir entre un trade off, por un lado si uno la acepta está la incertidumbre que una o más de las muchas premisas puedan resultar siendo falsas a través del tiempo, también el costo de oportunidad de aceptarlo implica que tengo que renunciar a muchas otras cosas porque la aceptación de esta nueva forma de pensar implica que cosas que he aceptado en el pasado se vuelven un desastre y colapsan, la aceptación de esta nueva teoría corresponde a una verdad que es mutuamente excluyente de las demás que existen y es por eso que hay un punto que se puede decir que la teoría truena.

%----------------------------------------------------------------------------------------
\section{Razón \# 2: Las verdades por tener que ser aceptadas son una construcción social que se construye a partir de una individual}
Mucha gente encuentra ciertos conceptos inútiles, es por eso que no los acepta, mucha gente el día de hoy que se sabe que no tienen educación ni personal ni profesional dicen saber que el mundo es plano y no redondo, sin embargo hoy en día la gente ya aceptó la construcción conceptual del individuo que formuló el argumento que la tierra es redonda. 
Sin embargo antes de eso se creía que la tierra era un plato, plana y que si uno no tenía cuidado se podía caer al espacio. 
Las verdades en ese aspecto son una construcción social que se aceptan a partir de ciertas evidencias por una comunidad que comparten ciertos criterios para aceptar nuevas tésis e hipótesis, cabe entonces pensar que por eso es la comunidad científica (la que acepta primero sus descubrimientos y después los acepta la demás gente) son los que construyen la verdad y después nosotros evaluamos lo siguiente: \begin{itemize}
    \item \pregunta{Me es útil aceptar esta verdad} 
    \item \pregunta{A partir de esta verdad qué pierdo}   
    \item \pregunta{Qué implicaciones sociales inmediatas tiene la adopción de esta nueva \emph{verdad} a mi cosmovisión  a mi forma de ver el mundo} 
\end{itemize} etcétera.
El humano es un ser que únicamente acepta las verdades si les serán útil, un ejemplo contemporáneo es cuando una persona empieza a sentir atracción a una personas del mismo sexo, típicamente lo primero que hace una persona que se empieza a dar cuenta de esto es intentar suprimirlo, \pregunta{por qué} precisamente por que saben que en la sociedad no son fácilmente aceptadas esas conductas y por ende no les son útiles al individuo. Es por eso también que la gente sigue una vasta cantidad de (diferentes y mutuamente excluyentes) religiones, uno acepta las verdades según le sean útil, por ejemplo a uno le puede resultar útil creer en el dios judío Yahweh, mientras a un hindú le puede ser útil creer en varios dioses, y a un budista le puede ser útil seguir la filosofía del budismo y a un islámico le es útil seguir a Allah y a Muhamed, pero sería incorrecto decir que una sóla religión es la verdadera verdad universal por que no la es, es una construcción que le es útil al individuo pero no al entorno social universal, en este sentido dudo de la existencia de una verdad verdaderamente universal. 
Esta conceptualización de la verdad es interesante, \pregunta{y entonces cómo construimos la verdad de la belleza} Es una muy interesante pregunta por que ningún concepto es universalmente aplicable por ser una construcción. 
La interpretación conceptual de la belleza es entonces una construcción también, pero ojo que es una construcción conceptual que sólo me es útil a mi, es por eso que los gustos físicos, emocionales y de carácter de las personas son tan variados, a todos nos es util un concepto diferente. 
Teniendo en cuenta que entonces nosotros efectivamente hacemos que algo sea bello y que el hecho que sea bello o no es verdad de acuerdo a nuestra subjetividad ilustra el concepto que la verdad es una construcción individual que si le es útil a los demás se adopta como una construcción social.

\subsection{Posible refutación}
Dado a que ningún concepto por sí solo puede englobar o delegarse como la verdadera verdad universal, en algún punto falla, por ejemplo hay personas que no les parece la idea utilitarista, esto está bien, las personas tienen diferentes formas de aceptar sus verdades, y no hay una verdad universal que se englobe en un solo concepto.

%----------------------------------------------------------------------------------------
\section{Razón \# 3: Lo útil es lo justo y lo bueno}
\pregunta{Entonces cómo puede ser uno justo si cada quien tiene una verdad diferente} es una pregunta difícil de responder, pero uno puede partir del hecho que todos tenemos cosas que consideramos en cierta medida inútiles, por ejemplo, creo que a ningún humano le gusta estar en peligro de muerte, o a ningún humano le gusta ser asesinado o que le asesinen a un ser querido, por otro lado todos queremos evadir el dolor, y la fuente del dolor no es universal, es decir que lo que le duele a uno puede ser que no le duela (o al menos no en la misma magnitud) a las demás personas, pero es irrefutable el hecho que en efecto le duele. Entonces de cierta manera la justicia emana de establecer prohibiciones, en cierta manera definir lo prohibido como lo que le es desútil a la mayoría es un poco tenebroso, el 51\% puede tiranizar al 49\%, hay que tener en cuenta que en este tipo de conceptos es bien fácil caer en estado de indefinición, es decir que uno intenta definir algo usando términos que también no tienen definición; \pregunta{entonces lo justo es lo bueno} esto realmente no nos dice nada porque estamos construyendo sobre dos indefiniciones, lo justo es ligeramente definible pero en sí indefinible y lo bueno es plenamente indefinible, pero se puede partir del hecho que en cierta manera lo bueno se puede poner en términos utilitarios, a mi me gusta ponerlo así: es lo que no me duele, me exonera dolor hoy o en el futuro. 
Me gusta definirlo así por que opino que es la manera más congruente de formular una construcción individual que se pueda adoptar como construcción social collectiva de muchas personas, pero entonces si lo que no me duele es lo bueno entonces lo justo es que las cosas no me duelan y como no a todos nos duelen las mismas cosas es una paradoja. 

\subsection{Posible refutación}
Las personas construyen sus verdades, la justicia y el bien, al igual que la belleza y qué me duele y qué no son en esencia conceptos indefinibles dado a la magnitud de cosas dadas que debemos tomar como premisas para considerar cualquier concepto como infinitesimalmente exacto o verdaderamente universal es un riesgo, puede ser que a lo largo de los años se tomen diferentes premisas y los conceptos truenen; el hecho que en sí uno no decide qué le duele y qué puede aceptar como verdad, ilustra el hecho que hay cierto punto en el que mis definiciones personales de belleza, verdad y justicia truenan. No existe un concepto que pretenda englobar la verdad en su plenitud por lo que esta explicación se debe de absorber sólo si le es útil al lector.
%----------------------------------------------------------------------------------------
\section{Conclusión}
En realidad la verdad, justicia, belleza y lo bueno son construcciones individuales que sólamente nos molestamos en considerar por que no son útiles, ya sea en conceptualizar el mundo o satisfacer el incentivo inato al hombre de querer descubrir. La verdad es como el compás del hombre, le da sentido a la vida humana y entretiene a los humanos en su existencia corta en el universo. La verdad es además una construcción individual que puede ser a lo largo del tiempo adoptada a ser una construcción social y collectiva. Finalmente la verdad, justicia, belleza y bien son términos que el humano individual formula y construye individualmente, después que el humano individualmente formula su verdad si le es útil a los demás los demás adoptarán su verdad y la verdad empieza su camino a poder ser ascendida a una construcción social o verdad social. Las verdades generales como las que se mencionaron y ejemplificaron con los descubrimientos de Hippasus y Galileo son construcciones sociales que describen una verdad construida y aceptada por un gran segmento de la sociedad por la utilidad que brinda para diversas tareas humanas. A pesar que se pueda absorber estas definiciones si les son útiles al lector, no hay una verdad universal que habite en un solo concepto, uno no puede volver el concepto de verdad en una generalización, es curioso por que decir la palabra ``verdad'' es una generalización en sí misma que se manifiesta en infinitos conceptos que el humano construirá y demolerá a través de su existencia en este universo.




%%%%%%%%%%%%%%%%%%%%%%%%%%%%%%%%%%%%%%%%%%%%%%%%%%%%%%%%%%%%%%%%%%%%%%%%%%%%%%%%%%%%%%%%%%%%%%%%%%%%%%%%%%%%%%%%%%%%%%%%%%%%%%%%%%%%%%%%%%%%%%
\end{document}

