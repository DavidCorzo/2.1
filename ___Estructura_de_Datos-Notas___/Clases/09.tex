\section{Strings}
Los strings se almacenan de manera continua en memoria, se separan por medio de un \textbf{null byte} o \\0:
\begin{itemize}
    \item \[
        \text{  \textbackslash  }0 = \text{  NULL  } = \text{  NIL  }
    \]
    
    \item Hay varias maneras de inicializar un strings:
        \begin{itemize}
            \item char str[10]; // Lo declaro como un array y pide 10 bytes en memoria.
            \item char str[]; // Lo declaro como una array de indefinida cantidad de posiciones.
            \item char str[] = \{'h','e','l','l','o',\textbackslash0\}
        \end{itemize}
    
    \item Cómo el compilador funciona: 
        \begin{tikzpicture}[node distance = 2cm, auto]
            \node [block] (1) {Código}; 
            \node [block,below of=1] (2) {Compilado};
            \node [block,below of=2] (3) {Assembler};
            \node [block,below of=3] (4) {Lenguaje de máquina}; 
            \path [line] (1) -- (2);
            \path [line] (2) -- (3);
            \path [line] (3) -- (4);
        \end{tikzpicture}
        \newline 
    
    \item Los strings son considerados como un array:
        \begin{itemize}
            \item Para sacar el lenght de un array:
                \begin{Verbatim}[breaklines=true, breakanywhere=true]
                    function len(){
                        str = "hola";
                        i = 0; 
                        while (str[i] != NIL){
                            i++; 
                        }
                        return i;
                    }
                \end{Verbatim}
            
            \item Estas operaciones de los strings tienen sus abstracciones en código de asembler.
        \end{itemize}
        
    
    \item \url{https://repl.it/languages/c} Para probar código de c en línea.
        \begin{Verbatim}[breaklines=true, breakanywhere=true]
            #include <stdio.h>

            int main(void) {
            char second_string[6] = {'h','e','l','l','o'};
            printf("Second string: %s\n", second_string);

            char third_string[6] = "Hello";
            printf("Third string: %s\n", third_string);
            printf("This is an element is one before NIL => \"%c\"\n",third_string[5]);

            char string[] = "hello this doesn't delimit the memory size";
            printf("String: %s\n",string);
            
            return 0;
            }
        \end{Verbatim}
\end{itemize}
