\section{Datos del auxiliar}
\begin{itemize}
    \item tortola@ufm.edu 
    \item 56904805 
\end{itemize}

%%%%%%%%%%%%%%%%%%%%%%%%%%%%%%%%%%%%%%%%%%%%%%%%%%%%%%%%%%%%%%%%%%%%%%%%%%%%%%%%%%%%%%%%%%%%%%%%

\section{Información preliminar}
\begin{itemize}
    \item Sugerencia de lenguaje: Java 
    \item Se puede utilizar el lenguaje que quiera. Ideal Java por el aspecto de ser orientado a objetos.
\end{itemize}

%%%%%%%%%%%%%%%%%%%%%%%%%%%%%%%%%%%%%%%%%%%%%%%%%%%%%%%%%%%%%%%%%%%%%%%%%%%%%%%%%%%%%%%%%%%%%%%%

\section{Dos formas de pasar parámetros a una función}
\begin{itemize}
    \item \begin{Verbatim}[breaklines=true, breakanywhere=true]
        num = 0;
        func SumOne(n){
            return n + 1; 
        }
        num_2 = SumOne(num) 
        print(num)
        print(num_2)
    \end{Verbatim}
    \begin{center}
       \begin{tabular}{ | p{5cm} | p{5cm} |}
           \hline
         Por valor & Por referencia \\
         > 0 & > 1 \\ 
         > 1 & > 1 \\ 
           \hline
       \end{tabular}
    \end{center}
    Los de valor no modifican num, los de por referencia no lo modifican y otra lugar en memoria.  
\end{itemize}

\section{Ver:}
\begin{itemize}
    \item Postman para API's
    \item Jmeter
    \item Spring: para API's en Java 
\end{itemize}
