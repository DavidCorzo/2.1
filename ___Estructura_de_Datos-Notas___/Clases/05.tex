\section{Data structure}
\begin{itemize}
    \item Es simplemente cómo en memoria tengo esa información disponible mientras estoy ejecutando esa información.
    \item \emph{\textbf{Ejemplo: }Google, si google hubiera guardado una llave primaria por todo lo que existe estaríamos aún esperando, separan el software y las clasifican en diferentes estructuras de datos; es diferente transformar la data que manipularla. }
\end{itemize}

%%%%%%%%%%%%%%%%%%%%%%%%%%%%%%%%%%%%%%%%%%%%%%%%%%%%%%%%%%%%%%%%%%%%%%%%%%%%%%%%%%%%%%%%%%%%%%%%

\subsection{SOLID}
\begin{enumerate}
    \item \textbf{S}ingle responsability:
        \begin{itemize}
            \item No hacer dos o tres métodos en una clase llamada ``tarea1''.
        \end{itemize}
    
    \item \textbf{O}pen - Close 
        \begin{itemize}
            \item 
        \end{itemize}
    
    \item \textbf{L} is for substitution 
    \item \textbf{I}
    \item \textbf{D}
\end{enumerate}

%%%%%%%%%%%%%%%%%%%%%%%%%%%%%%%%%%%%%%%%%%%%%%%%%%%%%%%%%%%%%%%%%%%%%%%%%%%%%%%%%%%%%%%%%%%%%%%%

\subsection{The 3 steps}
\begin{itemize}
    \item Analyse the problem to determine basic operations that must be suported to interact with the data.
    \item Quantify the resource constraints for each operation.
    \item Select the data structure that best meets these requirements.
\end{itemize}

%%%%%%%%%%%%%%%%%%%%%%%%%%%%%%%%%%%%%%%%%%%%%%%%%%%%%%%%%%%%%%%%%%%%%%%%%%%%%%%%%%%%%%%%%%%%%%%%

\subsubsection{Recomendaciones}
\begin{itemize}
    \item ``Todo trabajo tiene su estructura de datos perfecta.''
\end{itemize}

%%%%%%%%%%%%%%%%%%%%%%%%%%%%%%%%%%%%%%%%%%%%%%%%%%%%%%%%%%%%%%%%%%%%%%%%%%%%%%%%%%%%%%%%%%%%%%%%

\section{Elementary data structure organization}
\begin{itemize}
    \item Data 
    \item Record 
    \item File
    \item Key 
    \item Values 
\end{itemize}

%%%%%%%%%%%%%%%%%%%%%%%%%%%%%%%%%%%%%%%%%%%%%%%%%%%%%%%%%%%%%%%%%%%%%%%%%%%%%%%%%%%%%%%%%%%%%%%%

\subsection{Clasificaciones}
\begin{itemize}
    \item Fundamental data types: Linear, Non-linear.
    \item Primitive:  int, char, float, double.
    \item Non-primitive: linked list, array.
    \item Data structures: \textbf{Nos preguntamos:} ¿una clase es una estructura de datos? depende, para el compilador sí; en memoria sí es una estructura de datos; Para nosotros no es una estructura de datos.
\end{itemize}

%%%%%%%%%%%%%%%%%%%%%%%%%%%%%%%%%%%%%%%%%%%%%%%%%%%%%%%%%%%%%%%%%%%%%%%%%%%%%%%%%%%%%%%%%%%%%%%%

\section{Abstract data types}
\begin{itemize}
    \item Solo la voy a usar.
\end{itemize}

%%%%%%%%%%%%%%%%%%%%%%%%%%%%%%%%%%%%%%%%%%%%%%%%%%%%%%%%%%%%%%%%%%%%%%%%%%%%%%%%%%%%%%%%%%%%%%%%
\section{Tarea}
\begin{itemize}
    \item Integration $\overbrace{\neq}^{Testing}$ Unit
    \item Prueba unitaria va directo a la clase. En este probas tu API
    \item Prueba de integración va al medio de exposure.
\end{itemize}

%%%%%%%%%%%%%%%%%%%%%%%%%%%%%%%%%%%%%%%%%%%%%%%%%%%%%%%%%%%%%%%%%%%%%%%%%%%%%%%%%%%%%%%%%%%%%%%%

\subsection{Asert - es parte de la prueba de integración}
\begin{itemize}
    \item Es una prueba de confirmación, se puede colocar que algo está incluído.
    \item Es pasarle los aserts a Jmeter.
\end{itemize}
