\section{¿Qué es el marketing?}
\begin{itemize}
    \item Todo aquello que una empresa u organización realiza con el objeto de identificar, conocer, cultivar y satisfacer el mercado que sirve y ser retribuido por él de manera sostenida.
    \item \emph{\textbf{Ejemplo: }Problema acerca de tarjeta de crédito. Las marcas problemáticas afectan su posicionamiento.}
    \item Se necesita hacer énfasis que el target tiene que ser investigado, conocido, entendido a profundidad.
    \item \emph{\textbf{Ejemplo: }Krispy Cream: la noción de ``tal empresa va a hacer que otra quibre''}.
    \item Analizar: \textbf{Nos preguntamos:} ¿por qué compramos donas? Es fácil saber por qué compran, \textbf{lo difícil es saber por NO compran.}
    \item Historia de McDonald's: 
        \begin{itemize}
            \item Cuando se fundó la franquicia, se inició la idea de la cajita feliz con juguete; se pudieron deducir estas conclusiones por que la Sr. Cofiño estaba en el restaurante dando la comida; ``el peor lugar para tomar decisiones es el escritorio, uno tiene que estar en la jugada''.
            \item Ahora Mc está teniendo muchos detractores ya que la gente valora más los saludable por ejemplo. 
        \end{itemize}    
    \item Ejemplo de madre e hija:
        \begin{itemize}
            \item Aprender a ver la escena.
            \item Aprender a absorber todo el contexto.
        \end{itemize}
\end{itemize}

\section{Ejemplo de señor leyendo:}
\begin{itemize}
    \item El ambiente se adecua al target, sería mal si se propone poner música a alto volúmen.
    \item Todos nosotros vamos a necesitar alguna cosa, estas son oportunidades de negocios.
    \item Se analiza que el target necesita un ambiente para poder juntarse a leer, \textbf{identificar las necesidades del target}; resumir el target a una necesidad, conocer los intereses del target.  
\end{itemize}

\section{Ejemplo de la moto:}
\begin{itemize}
    \item \textbf{Nos preguntamos:} ¿Qué pensamos?
    \item \textbf{Nos preguntamos:} ¿Qué necesidades tendrá esta familia? 
    \item \textbf{Nos preguntamos:} ¿Será la cultura?
    \item Hay que entender los targets, por qué quieren usar una moto con 3 pasajeros y el piloto.
    \item Aspecto relevante: tener en cuenta que una de las cosas más importantes es investigar el target.
    \item \textbf{Nos preguntamos:} ¿Será porque se vuelve económicamente imposible tener carro por la distancia entre la casa y el trabajo? \emph{\textbf{La respuesta a esta problemática es: }tenemos que entender cuál es el problema.}
\end{itemize}

%%%%%%%%%%%%%%%%%%%%%%%%%%%%%%%%%%%%%%%%%%%%%%%%%%%%%%%%%%%%%%%%%%%%%%%%%%%%%%%%%%%%%%%%%%%%%%%%
\section{Ejemplos del marketing (buenos / malos)}
\subsection{Anécdota colgate:}
\begin{itemize}
    \item Insight: Darnos cuenta que había un niño que llevaba cuatro veces al día por cositas, esto no se hubiera podido analizar en el escritorio.
    \item Marketing es un poco como trabajo de detective, como trabajo de investigador.
    \item Hay que conocer el mercado; a partir del ejemplo de la moto se puede intentar coordinar el mercado para ejercer funciones empresariales para resolver esta problemática.
\end{itemize}

\subsection{Ejemplo de maratón:}
\begin{itemize}
    \item Especulamos que es por una buena causa, les interesa su salud, les interesa el reto, les interesa para ver que podemos hacer.
    \item Ejemplo en Chile: el bicicross;  así van y vienen las tendencias.
\end{itemize}

\subsection{Ejemplo de Donald Trump:}
\begin{itemize}
    \item Él ganó su candidatura en EEUU por que supo identificar su target.
    \item En política se utiliza mucho esto de identificar su target.
\end{itemize}

\subsection{Ejemplo de Madonna:}
\begin{itemize}
    \item Madonna se ha mantenido por décadas, ella como artista genera más dinero que las grandes marcas en GT.
    \item Otros ejemplos como Shakira, generan mucho dinero.
    \item Walkman: es el abuelito del iPod, identificaron que los adolecentes llevaban su boom box e innovó a crear el walkman, donde se guardaba música, ahora no tenían que cargar la grabadora con sus baterías, el walkman ue por mucho tiempo el aparato para escuchar música. Esto un ejemplo clásico de identificar las necesidades del target.
    \item Los targets van cambiando sus intereses a través del tiempo.
\end{itemize}

\subsection{Ejemplo DudeWipes:}
\begin{itemize}
    \item Muchas veces con que se identifique la necesidad del target si no que diseñar un producto con el target en mente. Agarraron el mismo producto que era para bebés solo que lo orientaron para hombres.
    \item Las toallas humedas las modificaron un poco, como no tienen olor, son más gruesas, son más grandes, etcétera; este es un producto \textbf{muy bien enfocado} desde el color, hasta el logos, hasta el nombre de la marca.
    \item Todo el modelo de negocios está orientado al target.
    \item Se pueden introducir a nuevos sectores del mercado; posiblemente no podía usar toallitas de bebé se introduce estas nuevas toallitas, lugares como barbería.
    \item Otro ejemplo es el shampoo Ego; Se enfocó en el target de los hombres; se posiciona de tal manera que resuenan mucho mejor con el target.
    \item Video promoción de DudeWipes - \url{https://www.youtube.com/watch?v=4jMgM0pKEUw}
    \item Dude wipes se separó de los otros productos casi identicos.
    \item Otro ejemplo puntual es - \url{https://www.youtube.com/watch?v=vdWHucg900U}
    \item Carreer builder es un sitio para conseguir trabajo, en el comercial sale el mismo actor que en marketing es el target que trata de conseguir que los targets de la empresa visiten el sitio; Este video tiene ese problema, implica que todos los demas ajenos al target es un mono.
    \item Regla básica en marketing es \textbf{No hablar mal de nadie}.
    \item Ejemplo de Coca~Cola y pepsi; su estrategia de marketing es riesgosa por sacar anuncios en contra de pepsi.
    \item Ejemplo de Beneton: El ejemplo de un Sueco con un negro de África.
\end{itemize}

\subsection{\textbf{Nos preguntamos:} ¿Cómo evadir que las personas no encuentren cosas ajenas al marketing a la intención?}
\begin{itemize}
    \item Una solución es leer de todo. 
    \item Entender si ya se había intentado y qué tal funcionó.
    \item Que en tu mesa de trabajo hayan expertos; experimentar.
\end{itemize}

\subsection{Analizar: El debate presidencial de Donald Trump}
\begin{itemize}
    \item El audio que salió acerca de la charla en el bus.
\end{itemize}
