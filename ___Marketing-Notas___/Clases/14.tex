\section{Ejercicios de mapas percentuales}

%%%%%%%%%%%%%%%%%%%%%%%%%%%%%%%%%%%%%%%%%%%%%%%%%%%%%%%%%%%%%%%%%%%%%%%%%%%%%%%%%%%%%%%%%%
\section{Introducción al análisis de FODA}
\begin{itemize}
    \item Ejemplo de una FODA:  
    \begin{center}
       \begin{tabular}{  p{5cm} | p{5cm} }
            Fortalezas: \begin{itemize}
                \item Líderes 
                \item Equipos 
            \end{itemize}            
            & 
            Oportunidades: \begin{itemize}
                \item Capacitación 
                \item Nuevos mercados 
            \end{itemize}    
            \\
            \hline
            \\ 
            Debilidades: \begin{itemize}
                \item Costos 
                \item Procesos burocráticos
            \end{itemize}
            & 
            Amenazas: \begin{itemize}
                \item Competencia 
                \item Nuevos productos
            \end{itemize}
            \\ 
       \end{tabular}
    \end{center}
    
    \item Ojo:
        \begin{itemize}
            \item Las amenazas no son internas, las debilidades son internas pero las amenzas por definición no.
        \end{itemize}
\end{itemize}
