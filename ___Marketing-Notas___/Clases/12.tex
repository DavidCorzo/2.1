\section{Posicionamiento}   
\begin{itemize}
    \item Posicionamiento por valor: posicionarme por  un valor universal por ejemplo Apple con la idea de ``pensar diferente''.
    \item Posicionamiento por beneficio: qué beneficio me deriva el producto, \emph{\textbf{Ejemplo: }bateria de nikel no me importa sólo quiero que me dure mucho.}
    \item Posicionamiento por característica o atributo: Una distinción o diferencia que distingue la marca respecto a las demás.
\end{itemize}
%----------------------------------------------------------------------------------------
\begin{itemize}[label=\#]
    \item Si una marca no hace posicionamiento, nosotros como consumidores la vamos a posicionar. 
    \item La marca he así no invierta nada en publicidad con sus haceres diarios se posiciona. Que una marca exista y labore en el mercado implica posicionamiento.
    \item Concepto de private labels es cuando una marca saca una marca sólo para un producto, por ejemplo great value. Cuando busco una marca de confianza tiendo a querer comprar de alguna marca privada.
\end{itemize}

%----------------------------------------------------------------------------------------
\subsection{Ejemplos de posicionamiento por característica o atributo}
\begin{enumerate}
    \item \emph{\textbf{Ejemplo: }I can't believe it's not butter}        
        \begin{itemize}
            \item I can't believe it's not butter: en el anuncio repite ``I can't believe it's not butter", estrategia de una historia, las narrativas son valiosísimas, pueden ser cortas o largas pero los humanos siempre les gustan las narrativas y es válido hacerlo para publicidad también.
        \end{itemize}
    
    \item \emph{\textbf{Ejemplo: }Beggin' Strips}
        \begin{itemize}
            \item Lo que quiero que el consumidor se recuerde es que es una marca que tiene producto que ``parece tocino pero no lo es''.
        \end{itemize}
        \begin{itemize}[label=\#]
            \item Los consumidores van a recordarse de muy pocas cosas, entoces en ese poquito de cosas tengo que meter algo bueno.
            \item \emph{\textbf{Recordar lo siguiente: }Swatch, como se distingue por diferencia}.
        \end{itemize}
\end{enumerate}

%----------------------------------------------------------------------------------------
\subsection{Posicionamiento por beneficio}  
\begin{enumerate}
    \item \emph{\textbf{Ejemplo: }Leche y Britney Spears}   
        \begin{itemize}
            \item ¿Si tomo leche $\rightarrow$ me vuelvo en Britney Spears? 
        \end{itemize}
    
    \item Julio Cortazar: El preámbulo a las instrucciones para dar cuerda al reloj
        \begin{itemize}
            \item \textbf{Preámbulo a las instrucciones para dar cuerda al reloj}: \newline 
            Piensa en esto: cuando te regalan un reloj te regalan un pequeño infierno florido, una cadena de rosas, un calabozo de aire. No te dan solamente el reloj, que los cumplas muy felices y esperamos que te dure porque es de buena marca, suizo con áncora de rubíes; no te regalan solamente ese menudo picapedrero que te atarás a la muñeca y pasearás contigo. Te regalan -no lo saben, lo terrible es que no lo saben-, te regalan un nuevo pedazo frágil y precario de ti mismo, algo que es tuyo pero no es tu cuerpo, que hay que atar a tu cuerpo con su correa como un bracito desesperado colgándose de tu muñeca. Te regalan la necesidad de darle cuerda todos los días, la obligación de darle cuerda para que siga siendo un reloj; te regalan la obsesión de atender a la hora exacta en las vitrinas de las joyerías, en el anuncio por la radio, en el servicio telefónico. Te regalan el miedo de perderlo, de que te lo roben, de que se te caiga al suelo y se rompa. Te regalan su marca, y la seguridad de que es una marca mejor que las otras, te regalan la tendencia de comparar tu reloj con los demás relojes. No te regalan un reloj, tú eres el regalado, a ti te ofrecen para el cumpleaños del reloj.            
        \end{itemize}
        \begin{itemize}[label=\#]
            \item Esta poesía es sólo para extraer el fragmento que después se declara en la declaración de posicionamiento.
            \item Recomendación: Leer todo lo que se pueda, conocer ayuda a posicionar una marca.
        \end{itemize}
    
    \item \emph{\textbf{Ejemplo: }Acción, jabón}:
        \begin{itemize}
            \item Apela a la necesidad de las personas a poder lavar los platos. 
        \end{itemize}

    \item \emph{\textbf{Ejemplo: }Uber}
        \begin{itemize}
            \item Apela a la necesidad de la gente que maneje en la plataforma de Uber.
            \item Usan cosas como sonidos de máquinas, usan la narrativa universal de ruido de máquina. Apela a la desesperación y frustración del trabajo y ofrece que trabajen para Uber como piloto y todos los beneficios.
        \end{itemize}
    
    \item \emph{\textbf{Ejemplo: }imágenes de Literacy Foundation}
        \begin{itemize}
            \item Apela a las historias infantiles como las de Peter Pan por ejemplo.
        \end{itemize}
        \begin{itemize}[label=\#]
            \item Hay anuncios que son un hit pero que no dicen nada, es básicamente una pérdida ya que a pesar de ser un anuncio llamativo desvía la atención del consumidor a cosas ajenas a la marca.
        \end{itemize}
\end{enumerate}
