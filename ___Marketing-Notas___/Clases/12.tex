\section{Decisiones que toman las empresas}


% --------------------------------------------------------------------------------------------

\subsection{Los tres factores de producción}
\begin{itemize}
    \item Capital 
    \item Trabajo
    \item Materia Prima 
\end{itemize}

% --------------------------------------------------------------------------------------------

\subsection{Función de producción}
\begin{itemize}
    \item Función de producción: Muestra el nivel de producción máximo que puede obtener la empresa con cada combinación especificada de factores. 
    \item 
\end{itemize}

% --------------------------------------------------------------------------------------------

\subsection{Tiempo}
\begin{itemize}
    \item Corto plazo: asumimos que por lo menos hay un factor que no se pueden alterar, permanece fijo.
    \item Largo plazo: asumimos que todos los factores son variables.
\end{itemize}


%----------------------------------------------------------------------------------------

\subsection{Producción en el corto plazo}
\begin{itemize}
    \item Nivel de producción por unidad de trabajo:
        \[
          \frac{\overbrace{q}^{\text{  Cantidad  }}}{\underbrace{L}_{\text{  Trabajo  }}} 
        \]
    
    \item Producto marginal:
        \begin{itemize}
            \item Marginalmente cuánta productividad deriva aumentar una unidad más de trabajo.
            \item Producto marginal es decreciente.
            \item Hay un punto donde la productividad es óptimo.
        \end{itemize}
        \begin{itemize}[label=\#]
            \item La función de producción en la ida real 
        \end{itemize}
    
    \item Producto medio:
        \begin{itemize}
            \item Siempre que el producto medio esté en aumento el producto marginal estará en aumento, cuando empieza a bajar el producto medio empieza también a bajar el producto marginal.
        \end{itemize}
    
\end{itemize}


%----------------------------------------------------------------------------------------
\subsection{ley de rendimientos marginales decrecientes}
\begin{itemize}
    \item \emph{\textbf{Interesante:} Puede que una empresa esté produciendo más y a pesar en el mismo punto empezar a tener rendimientos marginales decrecientes.}
    \item En el corto plazo los rendimientos productivos son variables: $(K,L)$ , 
\end{itemize}


%----------------------------------------------------------------------------------------
\subsection{Isocuantas}
\begin{itemize}
    \item Es el equivalente a las curvas de indiferencia, estas ilustran indiferencia que tengo al combinar el capital y trabajo y me producen en este caso el mismo output.
    \item La pendiente indica la disposición a intercambiar un factor de producción por otro (TMST).
    \item La isocuanta nos dice que disposición tenemos a sustituir, recordar a la tasa marginal de sustitución, el equvalente en isocuanta es la TMST (Tasa Marginal de Sustitución Técnica).
    \item Recordar el cálculo de la TMS:
        \[
          TMS = -\frac{UM_x}{UM_y} 
        \]
    
    \item El cálculo de TMST:
        \[
          TMST = -\frac{\overbrace{PML}^{\text{  Eje x  }}}{\underbrace{PMK}_{\text{  Eje y  }}} =  
          -\frac{\frac{\Delta q}{\Delta L}}{\frac{\Delta q}{\Delta k} } 
          = -\frac{\Delta q \Delta K }{\Delta q \Delta L} 
          = -\frac{\Delta K}{\Delta L} 
        \]
        \begin{itemize}[label=\#]
            \item L = Trabajo 
            \item K = Capital 
        \end{itemize}
    
    \item Rendimiento de escala, maneras de aumentar la producción:
        \begin{enumerate}
            \item Aumentar en factor y mantener el otro constante ( movimiento de la curva ) 
            \item Disminuir uno de los factores 
            \item Aumentar los dos factores
        \end{enumerate}
    
    \item Cuando se aplican las maneras de aumentar la producción:
        \begin{itemize}
            \item Rendimientos constantes de escala: si duplica los factores se duplica la productividad. 
            \item Rendimientos crecientes de escala: si duplica los factores aumenta más del doble en la productividad.
            \item Rendimientos decrecientes de escala: si duplica los factores ni siquiera llega a la mitad de productividad de más.
        \end{itemize}
\end{itemize}


%----------------------------------------------------------------------------------------
