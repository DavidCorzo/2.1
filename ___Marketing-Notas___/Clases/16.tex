\section{Estacionalidad}
\begin{itemize}
    \item Cuando un producto tiene demandas sólo en ciertas épocas del año.
    \item Puedo hacer cosas para mover la estacionalidad.
    \item Considérese:
        \begin{itemize}
            \item Casi todos los productos o servicios experimentan estacionalidad.
            \item Las fuentes de estacionalidad pueden ser externas.
            \item Provocadas internamente.  
        \end{itemize}
\end{itemize}


\section{Marketing Mix}
\begin{itemize}
    \item 4 P's: producto, precio, plaza, promoción, *extra: portafolio (evolución normal del producto).
    
    \item Producto: 
        \begin{itemize}
            \item Dr. Jerome McCarthy, propuso las 4P's del marketing.
            \item Producto es la más importante. No siempre es un producto puede ser un servicio pero es el mismo concepto.
            \item En la gran mayoría de industrias el producto define quién soy.
            \item El producto se vuelve en la parte central o lo que me define.
            \item Las 4 P's difieren de las P's de las posicionamiento.
            \item El producto requiere muchas actividades en el backofice, cuando tengo un portafolio de productos las actividades internas van a ser exponenciales.
        \end{itemize}

    
    \item Precio: 
        \begin{itemize}
            \item Tener precios accesibles y justos en la mente del consumidor, esto es ambigüo por que hay productos de status en los que se necesita un precio alto.
        \end{itemize}

    \item Plaza: 
        \begin{itemize}
            \item La traducción es confusa pero plaza es el lugar o local en donde estas.
        \end{itemize}
        
    \item Promoción: 
        \begin{itemize}
            \item Lealtad del cliente produce promoción de ``boca en boca''.
        \end{itemize}
\end{itemize}


\begin{center}
    \begin{tikzpicture}[node distance = 2cm, auto]
        \node[draw, circle] at (0,0) (1) {\begin{tabular}{|c|c|}
            \hline
            Producto & Price \\
            \hline
            Place & Promotion \\  
            \hline
        \end{tabular}};
    
        \node[] at (0,4) (2) {Panograma};
        \node[] at (0,-4) (3) {Propaganda};

        \node[] at (4,0) (4) {Publicidad};
        \node[] at (-4,0) (5) {People};

        \node[] at (4,1.8) (6) {Point of purchase POP};
        \node[] at (-3.8,1.8) (7) {Physical evidence};
        \node[] at (2.25,3) (8) {Planning};
        \node[] at (-2.5,3) (9) {Public relations};
        \node[] at (-2.5,-3) (10) {Planing};
        \node[] at (2.25,-3) (11) {Public relations};
        \node[] at (4,-1.8) (12) {Procesos};
        \node[] at (-3.8,-1.8) (13) {Packaging};
        
        \path [line] (1) -- (2);
        \path [line] (1) -- (3);
        \path [line] (1) -- (4);
        \path [line] (1) -- (5);
        \path [line] (1) -- (6);
        \path [line] (1) -- (7);
        \path [line] (1) -- (8);
        \path [line] (1) -- (9);
        \path [line] (1) -- (10);
        \path [line] (1) -- (11);
        \path [line] (1) -- (12);
        \path [line] (1) -- (13);

    \end{tikzpicture}    
\end{center}

\begin{itemize}
    \item Panograma: el plano de la góndola vertical es como la vitrina.
    \item Propaganda: es más orientada a la política.
    \item Publicidad: pagar por puublicidad.
    \item People: las personas de la organización y los clientes que se necesitan (profundamente leales).
    \item Point of purchase POP: puntos de venta.
    \item Physical evidence: \pregunta{cual es la evidencia que se utiliza para probar que mi producto no es una estafa y si aporta valor} 
    \item Planning: planear el marketing por medio de la especulación.
    \item Public relations: alguien que tiene buenas relaciones públicas tiene muchos contactos, trabajan para crear esa red de contactos.
    \item Procesos: para todo hay procedimientos, desde cómo lidear con un cliente enojado hasta qué decir en una entrevista.
    \item Packaging: es algo falsa.
\end{itemize}

%----------------------------------------------------------------------------------------
\subsection{\pregunta{por qué se llama marketing mix}}
\begin{itemize}
    \item Al final marketing tiene muchos componentes y no una esencia, funciona como una mezcla balanceada de decisiones.
    \item La analogía del café: el marketing es una mezcla, tiene componentes en diferentes cantidades y con diferentes combinaciones que todas son ricas, estas diferentes combinaciones funcionan.
\end{itemize}


%----------------------------------------------------------------------------------------
\subsection{Producto}
\begin{itemize}
    \item \termdefinition{Producto $\rightarrow$ portafolio}{Es un conjunto de productos que llenan espacios o necesidades específicas de consumidores.}
    \item Portafolio: lo importante es que no choquen los productos entre sí. 
    \item The paradox of choice:
        \begin{itemize}
            \item Cuando el consumidor tiene demasiadas opciones no procesa bien en su mente qué quiere.
        \end{itemize}
    
    \item El portafolio es cuando hablo de mis productos, la categoría es todos los productos y marcas que conforman el ambientes.
\end{itemize}

