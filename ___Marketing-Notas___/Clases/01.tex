\section{Introducción a marketing}
\begin{itemize}
    \item Target: grupo objetivo, grupo meta, es a quien vamos a servir.
    \item Marketing vs. publicidad:
        \begin{itemize}
            \item Marketing tiene que ver con el entorno, con el target, la publicidad es un herramienta en marketing; marketing está relacionado con finanzas, ver la contabilidad de los productos, \emph{\textbf{Ejemplo: }si se pretende vender un combo se debe de verificar si se está produciendo bien.}; marketing se puede volver muy numérico, especialmente en excel. 
            \item La publicidad es una herramienta del marketing. 
        \end{itemize}

    \item Ejemplos de marketing: 
        \begin{itemize}
            \item El uso del call center, como \emph{todólogos}.
        \end{itemize}

    \item Marketing ayuda a mejorar la competitividad de las empresas, no veremos el marketing de monopolios, etcétera.
    \item Modelo de negocio:
        \begin{itemize}
            \item La forma en la cual genera valor.
            \item La forma  en la cual la empresa se desarrolla para el cumplimiento de sus objetivos.
            \item Es encontrar qué somos, las empresas que lo encuentran de mejor forma pueden acertar mejor con su target, la clave es tener claro que somos.
        \end{itemize}
    
    \item \emph{\textbf{(Paréntesis ``las cinco fuerzas de porter'':} es relevante para saber cómo estamos compitiendo, es un mercado agresivo\textbf{)}}
    \item \textbf{Nos preguntamos:} ¿qué es \textbf{posicionamiento}?
        \begin{itemize}
            \item Es cómo me recuerda la gente como empresa.
            \item \emph{\textbf{Ejemplo: }Meicos, se está metiendo en meterse en el seguro, ``expertos en seguros'' se va a recordar en la mente de sus consumidores como la empresa que cuando yo necesite comprar algo que sea de mucha plata iré con los ``expertos en seguros''}. A comparación Galeno no se nos viene a la mente seguros.
            \item Los consumidores \textbf{perciben}, esa percepción puede ser verdadera o falsa. 
            \item A veces los productos son tan fuertes que anclan en la mente del consumidor.
        \end{itemize}
    
    \item \emph{\textbf{Recordar lo siguiente: }La gente es muy \textbf{ingrata}, eso nos incluye.}; muchas veces un día que esté malo el producto o al menos no a las expectativa del consumidor no vamos, alegamos, nos quejamos, etcétera. Esto trae serias consecuencias para la marca ya que anclan a otros consumidores malas asociaciones mentales acerca de la marca; inclusive gente que quería ir y nunca ha ido, no irá por la mala fama.
    \item El consumidor no es buena onda; sería excelente si el consumidor fuera empático pero con pocas excepciones nunca son empáticos, la más mínima imperfección en el producto implica problemas serios; no digamos cuando de veras son imperfecciónes.
    \item Solventar problemas para no anclar en la mente del consumidor malas cosas:
        \begin{itemize}
            \item Intentar solventar haciendo cosas como ``si tuvo problemas le damos el producto gratis''.
            \item ``Le damos un cupón la próxima vez que venga''.
        \end{itemize}
    
    \item \emph{\textbf{Recordar lo siguiente: } el dinero es propiedad privada, entonces cuando pagamos a alguien por un servicio y le quedan mal el consumidor se sentirá como que si le quebraron la ventana de su casa.}
    \item \emph{\textbf{Interesante:} Los animales se parecen al modelo de negocio, \emph{\textbf{Ejemplo: }el águila no es así como es solo por que sí, es por que funciona, es un modelo de un ser que \textbf{funciona};  Amazon, facebook, ebay son organismos que funcionan.}} 
        \begin{itemize}
            \item El modelo de negocio ballena podría ser como el IGGS, la ballena no tiene nada que ver con la hormiga, \textbf{funcionan diferente}, mismo asunto con los negocios \&Café funciona diferente que Apple. 
        \end{itemize}
\end{itemize}
