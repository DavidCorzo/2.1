\documentclass{article}

\usepackage{generalsnips}
\usepackage{calculussnips}
\usepackage[margin = 1in]{geometry}
\usepackage{pdfpages}
\usepackage[spanish]{babel}
\usepackage{amsmath}
\usepackage{amsthm}
\usepackage[utf8]{inputenc}
\usepackage{titlesec}
\usepackage{xpatch}
\usepackage{fancyhdr}
\usepackage{tikz}
\usepackage{hyperref}
\title{Marketing P's in análisis}
\date{2020 April 18}
\author{David Corzo}

\begin{document}
\maketitle
%%%%%%%%%%%%%%%%%%%%%%%%%%%%%%%%%%%%%%%%%%%%%%%%%%%%%%%%%%%%%%%%%%%%%%%%%%%%%%%%%%%%%%%%%%%%%%%%%%%%%%%%%%%%%%%%%%%%%%%%%%%%%%%%%%%%%%%%%%%%%%
%----------------------------------------------------------------------------------------
\section{Precio}
\begin{itemize}
    \item ``Las tiendas de superdescuentos (hard discount) operan con bajos costos, pero con una experiencia austera y digna que se traduce en productos de calidad al mejor precio.''
    \item ``Las marcas privadas «invaden» un espacio que era exclusivo de grandes marcas y fabricantes; ahora se ofrecen productos con la misma o superior calidad, más margen comercial y menor precio final.''
\end{itemize}

%----------------------------------------------------------------------------------------
\section{Promoción}
\begin{itemize}
    \item ``Cuando se entra a la tienda de Samsung impacta el ambiente que permite interactuar con todos sus productos.'' Reflexionar en instancias de promoción Point Of Purchase. 
    \item ``Lo primero que se ve es un DJ, luego sigue una visita con realidad aumentada y finaliza con la posibilidad de usar los teléfonos, las computadoras y los electrodomésticos.'' Instancia de una activación. 
\end{itemize}

%----------------------------------------------------------------------------------------
\section{Producto}
\begin{itemize}
    \item `` Las marcas privadas se han convertido en opciones preferidas por el consumidor, especialmente en categorías de alta rotación: leche, alimentos congelados, bebidas, embutidos, granos, aceite, snacks, bolsas de basura, detergentes, jabón y champú. '' Una tendencia que determina el portafolio de una marca privada.
    \item ``Una gran marca como Kellogg’s ha creado su propio restaurante con recetas basadas en sus productos, que le permite ofrecer una visión más amplia de su marca, atributos y versatilidad.'' respuesta a una pregunta del producto. 
\end{itemize}

%----------------------------------------------------------------------------------------
\section{Plaza}
\begin{itemize}
    \item ``Esa transformación es meritoria, pero muchas empresas obvian lo más básico de los procesos de transformación digital'' apela al hecho que el mundo físico debe estar íntimamente relacionado con el virtual, el lugar no es tan relevante como lo fue alguna vez, ahora el lugar habita también en lo virtual. 
    \item ``Estar a poca distancia de un punto de venta dejó de ser relevante en muchos comercios'' intenta responder a la tendencia de la tecnología y el lugar virtual de comercio. 
    \item ``Esta es la base de la omnicanalidad, que permite al consumidor comprar donde quiera y como quiera, y recibir una atención homologada caracterizada por el buen servicio y el conocimiento de sus gustos.'' apela al hecho que se pueden tener muchos canales de distribución y el lugar habita en algo más que sólo lo físico. 
\end{itemize}



%%%%%%%%%%%%%%%%%%%%%%%%%%%%%%%%%%%%%%%%%%%%%%%%%%%%%%%%%%%%%%%%%%%%%%%%%%%%%%%%%%%%%%%%%%%%%%%%%%%%%%%%%%%%%%%%%%%%%%%%%%%%%%%%%%%%%%%%%%%%%%
\end{document}

